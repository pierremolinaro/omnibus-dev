%!TEX encoding = UTF-8 Unicode
%!TEX root = ../doc-omnibus.tex





\chapterLabel{Le type \texttt{\$ctInt}}{chapitreTypeEntierStatique}

Ce chapitre est consacré au type \omnibus!$ctInt!, \emph{compile time integer}. Ce type est celui de valeurs entières signées évaluées lors de la compilation.

C'est le type de toute constante entière littérale. Mais ce type est aussi applicable à des constantes \emph{entières statiques}, c'est-à-dire dont la valeur est calculée à la compilation.

Le type \omnibus!$ctInt! n'est pas applicable à une variable : une variable entière doit être déclarée du type \omnibus!$uintN! (entier non signé de $N$ bits) ou \omnibus!$intN! (entier signé de $N$ bits), décrit au \refChapterPage{chapitreTypesEntiers}.


Le langage accepte des constantes littérales entières d'une valeur quelconque.

Il est valide de déclarer des constantes globales (c'est-à-dire définies en dehors de toute construction) de type \omnibus!$ctInt! :
\begin{OMNIBUS}
let N $ctInt = 1_000_000 // Ok, N a pour type $ctInt
\end{OMNIBUS}

L'annotation de type peut être omise :
\begin{OMNIBUS}
let N = 1_000_000 // Ok, N a pour type $ctInt
\end{OMNIBUS}



Il est possible d'utiliser une constante entière statique pour définir une autre constante :
\begin{OMNIBUS}
let P = N + 1 // Ok, P a pour type $ctInt, et vaut 1_000_001
\end{OMNIBUS}

Le type \omnibus!$ctInt! n'est pas acceptable pour une variable, aussi la déclaration suivante provoque une erreur de compilation :
\begin{OMNIBUS}
var N = 1_000_000 // Erreur, $ctInt invalide pour une variable
\end{OMNIBUS}

Il faut une annotation de type qui nomme un type \omnibus!$uintN! (entier non signé de $N$ bits) ou \omnibus!$intN! (entier signé de $N$ bits) :
\begin{OMNIBUS}
var v : $uint32 = 1_000_000 // Ok
\end{OMNIBUS}

Une constante entière statique de type \omnibus!$ctInt! est silencieusement convertie en un type entier \omnibus!$uintN! ou \omnibus!$intN!, en vérifiant si la conversion est possible ; par exemple, l'écriture suivante déclenche une erreur de compilation :
\begin{OMNIBUS}
var w $uint8 = 1_000 // Erreur, 1_000 ne peut pas être représenté
                     // par un entier non signé de 8 bits
\end{OMNIBUS}

Comme les constantes entières statiques sont calculées à la compilation, l'écriture suivante est correcte~:
\begin{OMNIBUS}
let A = 1_000_000_000_000_000_000_000_000_000_000_000_000_000
let B = 1_000_000_000_000_000_000_000_000_000_000_000_000_001
var z $uint1 = B - A
\end{OMNIBUS}

En effet, la valeur initiale de \omnibus!z! est $1$, représentable par un entier non signé de $1$ bit.


\section{Opérateurs préfixés}

Trois opérateurs sont définis~:
\begin{itemize}
\item  \omnibus!-! et \omnibus!-%! renvoient l'opposé de l'argument (ces deux opérateurs sont équivalents lorsqu'ils s'appliquent à un \omnibus!$ctInt!)~;
  \item \omnibus!~! renvoie le complément bit-à-bit de l'argument.
\end{itemize}



\section{Opérateurs infixés}

Les opérateurs suivants sont définis~:
\begin{itemize}
\item \omnibus!==! et \omnibus!≠! testent l'égalité et l'inégalité, et renvoient un \omnibus!$ctBool!~;
\item \omnibus!<!, \omnibus!≤!, \omnibus!>! et \omnibus!≥! comparent leurs arguments, et renvoient un \omnibus!$ctBool!~;
\item  \omnibus!&! effectue le \emph{et bit-à-bit} entre ses arguments~;
\item  \omnibus!|! effectue le \emph{ou bit-à-bit} entre ses arguments~;
\item  \omnibus!^! effectue le \emph{ou exclusif bit-à-bit} entre ses arguments~;
\item  \omnibus!<<! effectue un décalage logique à gauche, une erreur de compilation est déclenchée si l'argument de droite est strictement négatif~;
\item  \omnibus!>>! effectue un décalage arithmétique à droite, une erreur de compilation est déclenchée si l'argument de droite est strictement négatif~;
\item  \omnibus!+! et \omnibus!+%! effectuent la somme de leurs arguments (ces deux opérateurs sont équivalents lorsqu'ils s'appliquent à un \omnibus!$ctInt!)~;
\item  \omnibus!-! et \omnibus!-%! effectuent la différence de leurs arguments (ces deux opérateurs sont équivalents lorsqu'ils s'appliquent à un \omnibus!$ctInt!)~;
\item  \omnibus!*! et \omnibus!*%! effectuent le produit de leurs arguments (ces deux opérateurs sont équivalents lorsqu'ils s'appliquent à un \omnibus!$ctInt!)~;
\item \omnibus!/! effectue la division entière de l'argument de gauche par l'argument de droite, une erreur de compilation est déclenchée si l'argument de droite est zéro~;
\item \omnibus+!/+ effectue la division entière de l'argument de gauche par l'argument de droite, le résultat est zéro si l'argument de droite est zéro~;
\item \omnibus!%! retourne le reste de la division entière de l'argument de gauche par l'argument de droite, une erreur de compilation est déclenchée si l'argument de droite est zéro~;
\item \omnibus+!%+ retourne le reste de la division entière de l'argument de gauche par l'argument de droite, le résultat est zéro si l'argument de droite est zéro~;

\end{itemize}



