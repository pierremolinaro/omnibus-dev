%!TEX encoding = UTF-8 Unicode
%!TEX root = ../doc-omnibus.tex





\chapter{Les types opaque}

Un \emph{type opaque} est un type dont la définition est externe à OMNIBUS, dans le code C associé. Par défaut, les objets de ce type ne peuvent pas être copiés, ni comparés, ni instanciés et leur contenu est inaccessible. La déclaration d'attributs associés modifie ces propriétés par défaut (\refSectionPage{attributTypeOpaque}).

\section{Déclaration d'un type opaque}

La déclaration d'un type opaque est introduite par le mot réservé \omnibus!newtype! :

\begin{OMNIBUS}
newtype MonTypeOpaque : [[32]]
\end{OMNIBUS}

La séquence \omnibus![[expression]]! caractérise la déclaration d'un type opaque. Les deux niveaux de parenthèses sont obligatoires. Le nombre associé (ici \omnibus!32!) est le nombre de bits pour représenter un objet de ce type.

L'\omnibus!expression! doit être une expression statique (c'est-à-dire calculée à la compilation), de type entier. On peut donc faire appel à des constantes globales, comme par exemple :

\begin{OMNIBUS}
let TAILLE = 32
newtype MonTypeOpaque : [[TAILLE]] // 32 bits
newtype AutreTypeOpaque : [[TAILLE * 2]] // 64 bits
\end{OMNIBUS}





\sectionLabel{Attributs d'un type opaque}{attributTypeOpaque}

La déclaration d'un type opaque accepte deux attributs :
\begin{itemize}
\item \omnibus=@instantiable=, qui rend instanciable un objet d'un type opaque (par défaut, il n'est pas instanciable) ;
\item \omnibus=@copyable=, qui rend copiable un objet d'un type opaque (par défaut, il n'est pas copiable).
\end{itemize}

Ces deux attributs sont cumulables.

\subsection{Attribut \texttt{@instantiable}}\index{"@instantiable}

L'attribut \omnibus=@instantiable= est spécifié après l'indication de taille du type :

\begin{OMNIBUS}
newtype MonTypeOpaque : [[32]] @instantiable
\end{OMNIBUS}


Par défaut, un type opaque n'est pas instanciable ; l'attribut \omnibus=@instantiable= le rend instanciable, c'est-à-dire que l'on peut écrire :

\begin{OMNIBUS}
var t MonTypeOpaque = MonTypeOpaque ()
\end{OMNIBUS}

Ou encore, en supprimant l'annotation de type :

\begin{OMNIBUS}
var t = MonTypeOpaque ()
\end{OMNIBUS}

L'expression \omnibus=MonTypeOpaque ()= est à une instance de \omnibus=MonTypeOpaque= dont la valeur correspond à des zéros binaires.






\subsection{Attribut \texttt{@copyable}}\index{"@copyable}

L'attribut \omnibus=@copyable= est spécifié après l'indication de taille du type :

\begin{OMNIBUS}
newtype MonTypeOpaque : [[32]] @copyable
\end{OMNIBUS}


Par défaut, un type opaque n'est pas copiable ; l'attribut \omnibus=@copyable= le rend copiable, c'est-à-dire que l'on peut écrire :

\begin{OMNIBUS}
var t = MonTypeOpaque ()
var u = t // Copie d'une instance de MonTypeOpaque
\end{OMNIBUS}


