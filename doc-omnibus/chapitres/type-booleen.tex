%!TEX encoding = UTF-8 Unicode
%!TEX root = ../doc-omnibus.tex





\chapter{Le type booléen}


Le type booléen \omnibus=Bool=\index{bool!\omnibus=Bool=} est prédéfini par le langage.



\section{Les mots réservés \texttt{yes} et \texttt{no}}

Les mots réservés \omnibus+yes+ et \omnibus+no+ dénotent respectivement la valeur logique \emph{vraie} et la valeur logique \emph{fausse}.

\section{Les opérateurs infix de comparaison}

Les valeurs booléennes sont comparables, les six opérateurs \omnibus+==+, \omnibus+≠+, \omnibus+≥+, \omnibus+>+, \omnibus+≤+ et \omnibus+<+ sont acceptés, avec \omnibus=no < yes=.

\section{Les opérateurs infixes \texttt{and}, \texttt{or} et \texttt{xor}}

Les opérateurs infixes \omnibus=and=, \omnibus=or= et \omnibus=xor= implémentent respectivement le \emph{et} logique, \emph{ou} logique, \emph{ou exclusif} logique. Les deux premiers évaluent les opérandes en \emph{court-circuit}, c'est-à-dire que si la valeur de l'opérande de gauche détermine la valeur de l'expression, alors l'opérande de droite n'est pas évalué.

Noter que les opérateurs infixes \omnibus=&=, \omnibus=|= et \omnibus=^= sont des opérateurs bit-à-bit sur les entiers non signés, et ne peuvent pas être appliqués à des valeurs booléennes.


\section{L'opérateur préfixé \texttt{not}}

L'opérateur préfixé \omnibus=not= est la complémentation booléenne. Noter que l'opérateur préfixé \omnibus=~= effectue la complémentation bit-à-bit d'un entier non signé et ne peut pas être appliqué à une valeur booléenne.

\section{Conversion en une valeur entière}

La conversion d'une valeur booléenne en une valeur entière s'effectue par l'intermédiaire d'une expression \omnibus=if=. Par exemple :

\begin{OMNIBUS}
let x Bool = ...
let result UInt8 = if x { 4 }else{ 2 }
\end{OMNIBUS}


\section{Conversion d'une valeur entière en booléen}

Il n'y a pas d'opérateur dédié à la conversion d'une valeur entière vers un booléen. Il suffit d'utiliser des opérateurs entre entiers comme \omnibus+==+ ou \omnibus+≠+ pour réaliser une conversion :

\begin{OMNIBUS}
let result Bool = x ≠ 0 // x est une expression entière
\end{OMNIBUS}


