%!TEX encoding = UTF-8 Unicode
%!TEX root = ../doc-omnibus.tex





\chapter{Expressions}


\begin{table}[htbp]
\centering
\begin{tabular}{llll}
  \textbf{Priorité} & \textbf{Opérateur} & \textbf{Commentaire}\\
   0 & \omnibus+-+, \omnibus+-%+ & \emph{moins} unaire \\
   0 & \omnibus+~+, \omnibus+not+ & \emph{complémentation} binaire et \emph{non} logique \\
   1 & \omnibus=convert= & Conversion \\
   2 & \omnibus+*+, \omnibus+*%+, \omnibus+/+, \omnibus+!/+, \omnibus-%-, \omnibus-!%- & Multiplication, division, modulo \\
   3 & \omnibus-+-, \omnibus-+%-, \omnibus+-+, \omnibus+-%+ & Addition, soustraction \\
   4 & \omnibus+<<+, \omnibus+>>+ & Décalage à gauche et à droite \\
   5 & \omnibus+≤+, \omnibus+<+, \omnibus+>=+, \omnibus+>+ & Comparaison \\
   6 & \omnibus+==+, \omnibus+≠+ & Test d'égalité, d'inégalité \\
   7 & \omnibus+&+ & \emph{et} binaire \\
   8 & \omnibus+^+ & \emph{ou exclusif} binaire \\
   9 & \omnibus+|+ & \emph{ou} binaire \\
   10 & \omnibus+and+ & \emph{et} logique \\
   11 & \omnibus+xor+ & \emph{ou exclusif} logique \\
   12 & \omnibus+or+ & \emph{ou} logique \\
\end{tabular}
\caption{Priorité des opérateurs}\index{Operateur@Opérateur!Priorite@Priorité}
\labelTableau{tableauPrioriteOperateurs}
\end{table}



\section{Opérateur $\sim$}

L'opérateur $\sim$ renvoie la complémentation bit-à-bit d'un entier non signé. Une erreur de compilation est déclenchée si l'opérateur est appliqué à un entier signé~:
\begin{OMNIBUS}
let x Int8 = 3
let y = ~ x // Erreur, x est signé
\end{OMNIBUS}

Le nombre de bits complémentés dépend du nombre de bits du type entier non signé~:
\begin{OMNIBUS}
let x UInt8 = 1
let y = ~ x // y est égal à 0xFE
let z UInt16 = 1
let t = ~ z // t est égal à 0xFFFE
\end{OMNIBUS}

L'opérateur $\sim$ ne peut s'appliquer à une constante entière statique uniquement si le type du résultat peut être inféré, et que ce type est un entier non signé~:
\begin{OMNIBUS}
let x = ~ 1 // Erreur, le type du résultat ne peut pas être inféré
let y Int8 = ~ 1 // Erreur, le type inféré est signé
let z UInt8 = ~ 1 // Ok, z = 0xFE
let t UInt16 = ~ 1 // Ok, z = 0xFFFE
\end{OMNIBUS}





\section{Expression \texttt{if}}

\begin{OMNIBUS}
let x = if expression_1 { expression_2 } else { expression_3 }
\end{OMNIBUS}

L'expression \omnibus=if= fonctionne comme suit~:
\begin{itemize}
  \item l'\omnibus=expression_1= est une expression booléenne~;
  \item si l'\omnibus=expression_1= est vraie, l'\omnibus=expression_2= est calculée et sa valeur est celle renvoyée par l'expression \omnibus=if=~;
  \item si l'\omnibus=expression_1= est fausse, l'\omnibus=expression_3= est calculée et sa valeur est celle renvoyée par l'expression \omnibus=if=.
\end{itemize}

Les expressions \omnibus=if= peuvent se succéder, dans tous les cas il faut terminer par une clause \omnibus=else=~:

\begin{OMNIBUS}
let x =
  if expression_1 {
    expression_2
  }else if expression_3 {
    expression_4
  }else{
    expression_5
  }
\end{OMNIBUS}



\section{Expression \texttt{addressof}}


L'expression \omnibus=addressof= permet d'obtenir l'adresse de toute \emph{lvalue}. La valeur retournée a pour type l'entier non signé de la taille d'un pointeur (sur Cortex, c'est donc \omnibus=UInt32=).

Par exemple~:
\begin{OMNIBUS}
var x = yes // x est booléen
let adresse UInt32 = addressof (x)
let adresse_registre_controle UInt32 = addressof (PORTA_PCR0)
\end{OMNIBUS}




\section{Expression \texttt{sizeof}}


L'expression \omnibus=sizeof= permet d'obtenir la taille (en nombre d'octets) de toute \emph{lvalue}, ou de tout type. La valeur retournée a pour type l'entier non signé de la taille d'un pointeur (sur Cortex, c'est donc \omnibus=UInt32=).

Par exemple~:
\begin{OMNIBUS}
var x = yes // x est booléen
let s1 UInt32 = sizeof (x) // 1
let s2 UInt32 = sizeof (Bool) // 1
\end{OMNIBUS}

