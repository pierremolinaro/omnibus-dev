%!TEX encoding = UTF-8 Unicode
%!TEX root = ../doc-omnibus.tex


\chapter{Contrôle d'accès}



\section{Routines \texttt{boot}}

Les routines \omnibus=boot= n'ont pas le droit d'appeler des pilotes ou de solliciter les points d'entrée d'une tâche.

En effet, les routines \omnibus=boot= sont exécutées avant l'initialisation des propriétés des pilotes, et les tâches ne sont pas démarrées.





\section{Propriétés}

Les propriétés des structures sont par défaut privées~; le qualificatif \omnibus=public= les rend publiques.

Les propriétés des pilotes et des tâches sont toujours privées.







\section{Méthodes}

Le contrôle d'accès suivant s'applique aux~:
\begin{itemize}
  \item méthodes de structure ;
  \item méthodes de pilote~;
  \item méthode de tâche.
\end{itemize}

Une méthode déclarée sans mode ou avec le mode \omnibus=user= ne peut accéder qu'aux propriétés constantes.


\section{Registres de contrôles}

Si un registre de contrôle est déclaré avec l'attribut \omnibus=@ro=, il ne peut pas être modifié.

Si un registre de contrôle est déclaré avec l'attribut \omnibus=@user=, il peut être accédé en mode \omnibus=user=.

