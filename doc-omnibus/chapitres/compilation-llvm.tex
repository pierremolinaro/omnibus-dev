%!TEX encoding = UTF-8 Unicode
%!TEX root = ../doc-omnibus.tex





\chapterLabel{Compilation LLVM}{chapitreCompilationLLVM}

Ce chapitre contient différentes notes sur la compilation en LLVM des sources OMNIBUS. Sa lecture n'est pas nécessaire pour comprendre l'utilisation de OMNIBUS.


Le point le plus délicat est l'analyse sémantique et la génération de code des constructions implicant \omnibus=self=, l'accès aux propriétés, aux éléments de tableaux, à l'appel de fonction. Ces constructions sont :
\begin{itemize}
  \item l'expression primaire (qui peut apparaître par exemple en partie droite de l'instruction d'affectation) ;
  \item la cible d'affection (partie gauche de l'instruction d'affectation) ;
  \item l'instruction d'appel de procédure (c'est-à-dire l'appel d'une fonction qui ne retourne aucune valeur).
\end{itemize}

\section{Diagrammes syntaxiques}

\tikzset{
  nonterminal/.style={
    % The shape:
    rectangle,
    % The size:
    minimum size=6mm,
    % The border:
    very thick,
    draw=red!50!black!50,         % 50% red and 50% black,
                                  % and that mixed with 50% white
    % The filling:
    top color=white,              % a shading that is white at the top...
    bottom color=red!50!black!20, % and something else at the bottom
    % Font
    font=\itshape,
    text height=1.5ex,
    text depth=.25ex
  },
  terminal/.style={
    % The shape:
    rounded rectangle,
    minimum size=6mm,
    % The rest
    very thick,draw=black!50,
    top color=white,bottom color=black!20,
    font=\ttfamily,
    text height=1.5ex,
    text depth=.25ex
  },
  point/.style={coordinate, >=stealth', thick, draw=black!50},
  tip/.style={->, shorten >=0.007pt,every join/.style={rounded corners}},
  hv path/.style={to path={-| (\tikztotarget)}},
  vh path/.style={to path={|- (\tikztotarget)}},
%  skip loop/.style={to path={-- ++(0,#1) -| (\tikztotarget)}}
}

\subsection{Expression primaire}

\begin{tikzpicture}
  \matrix[column sep=4mm, row sep=1mm] {
  % rows:
    & & & & & & & & \node (loopleft) [point] {}; & \node (startloopup2) [point] {}; & & \\
    & & & & & & & & & & & \node (startloopup) [point] {}; & \\
    & & & & & & & & \node (parouv) [terminal] {(}; & \node (args) [nonterminal] {arguments}; & \node (parfer) [terminal] {)};  & \\
    & & & & & & & & \node (croouv) [terminal] {\verb+[+}; & \node (exparray) [nonterminal] {expression}; & \node (crofer) [terminal] {\verb+]+};  & \\
    & & \node (self) [terminal] {self}; & \node (selfdot) [terminal] {.}; & & & & &  \node (propdot) [terminal] {.}; & \node (prop) [terminal] {idf};  & \\
    \node (pstart) [point] {}; & \node (selfstart) [point] {}; & & & \node (selfend) [point] {}; & \node (idf) [terminal] {idf}; & \node (optionloop) [point] {};  & \node (optionstart) [point] {}; & & & & & \node (pend) [point] {};\\
  };

  { [start chain]
    \chainin (pstart);
    \chainin (selfstart)     [join];
    { [start branch=selfstart]
      \chainin (self)  [join=by {vh path,tip}];
      \chainin (selfdot)  [join=by tip];
      \chainin (selfend)  [join=by {hv path,tip}];
    }
    \chainin (selfend)     [join];
    \chainin (idf)         [join=by tip];
    \chainin (optionloop) [join];
    \chainin (optionstart)  [join];
    { [start branch=optionstart]
      \chainin (propdot)  [join=by {vh path,tip}];
      \chainin (prop)     [join=by tip];
      \chainin (startloopup)  [join=by {hv path,tip}];
    }
    \chainin (pend)        [join=by tip];
  }
  { [start chain]
    \chainin (startloopup) ;
    \chainin (startloopup2) [join=by {vh path}];
    \chainin (loopleft)     [join] ;
    \chainin (optionloop) [join=by {hv path}];
  }
  { [start chain]
    \chainin (optionstart) ;
    \chainin (croouv)     [join=by {vh path, tip}];
    \chainin (exparray)   [join=by tip] ;
    \chainin (crofer)     [join=by tip];
      \chainin (startloopup) [join=by {hv path}];
  }
  { [start chain]
    \chainin (optionstart) ;
    \chainin (parouv)  [join=by {vh path, tip}];
    \chainin (args)   [join=by tip] ;
    \chainin (parfer)     [join=by tip];
    \chainin (startloopup) [join=by {hv path}];
  }
\end{tikzpicture}

Le préfixe \omnibus=self.= est obligatoire pour accéder à une propriété ou fonction locale.

\subsection{Cible d'affectation}

\begin{tikzpicture}
  \matrix[column sep=4mm, row sep=1mm] {
  % rows:
    & & & & & & & & \node (loopleft) [point] {}; & \node (startloopup2) [point] {}; & & \\
    & & & & & & & & & & & \node (startloopup) [point] {}; & \\
    & & & & & & & & \node (croouv) [terminal] {\verb+[+}; & \node (exparray) [nonterminal] {expression}; & \node (crofer) [terminal] {\verb+]+};  & \\
    & & \node (self) [terminal] {self}; & \node (selfdot) [terminal] {.}; & & & & &  \node (propdot) [terminal] {.}; & \node (prop) [terminal] {idf};  & \\
    \node (pstart) [point] {}; & \node (selfstart) [point] {}; & & & \node (selfend) [point] {}; & \node (idf) [terminal] {idf}; & \node (optionloop) [point] {};  & \node (optionstart) [point] {}; & & & & & \node (pend) [point] {};\\
  };

  { [start chain]
    \chainin (pstart);
    \chainin (selfstart)     [join];
    { [start branch=selfstart]
      \chainin (self)  [join=by {vh path,tip}];
      \chainin (selfdot)  [join=by tip];
      \chainin (selfend)  [join=by {hv path,tip}];
    }
    \chainin (selfend)     [join];
    \chainin (idf)         [join=by tip];
    \chainin (optionloop) [join];
    \chainin (optionstart)  [join];
    { [start branch=optionstart]
      \chainin (propdot)  [join=by {vh path,tip}];
      \chainin (prop)     [join=by tip];
      \chainin (startloopup)  [join=by {hv path,tip}];
    }
    \chainin (pend)        [join=by tip];
  }
  { [start chain]
    \chainin (startloopup) ;
    \chainin (startloopup2) [join=by {vh path}];
    \chainin (loopleft)     [join] ;
    \chainin (optionloop) [join=by {hv path}];
  }
  { [start chain]
    \chainin (optionstart) ;
    \chainin (croouv)     [join=by {vh path, tip}];
    \chainin (exparray)   [join=by tip] ;
    \chainin (crofer)     [join=by tip];
      \chainin (startloopup) [join=by {hv path}];
  }
\end{tikzpicture}



\subsection{Appel de procédure}

\begin{tikzpicture}
  \matrix[column sep=4mm, row sep=1mm] {
  % rows:
    & & & & & & & & \node (loopleft) [point] {}; & \node (startloopup2) [point] {}; & & \\
    & & & & & & & & & & & \node (startloopup) [point] {}; & \\
    & & & & & & & & \node (croouv) [terminal] {\verb+[+}; & \node (exparray) [nonterminal] {expression}; & \node (crofer) [terminal] {\verb+]+};  & \\
    & & \node (self) [terminal] {self}; & \node (selfdot) [terminal] {.}; & & & & &  \node (propdot) [terminal] {.}; & \node (prop) [terminal] {idf};  & \\
    \node (pstart) [point] {}; & \node (selfstart) [point] {}; & & & \node (selfend) [point] {}; & \node (idf) [terminal] {idf}; & \node (optionloop) [point] {};  & \node (optionstart) [point] {}; & \node (parouv) [terminal] {(}; & \node (args) [nonterminal] {arguments}; & \node (parfer) [terminal] {)}; & \node (pend) [point] {};\\
  };

  { [start chain]
    \chainin (pstart);
    \chainin (selfstart)     [join];
    { [start branch=selfstart]
      \chainin (self)  [join=by {vh path,tip}];
      \chainin (selfdot)  [join=by tip];
      \chainin (selfend)  [join=by {hv path,tip}];
    }
    \chainin (selfend)     [join];
    \chainin (idf)         [join=by tip];
    \chainin (optionloop) [join];
    \chainin (optionstart)  [join];
    { [start branch=optionstart]
      \chainin (propdot)  [join=by {vh path,tip}];
      \chainin (prop)     [join=by tip];
      \chainin (startloopup)  [join=by {hv path,tip}];
    }
    \chainin (parouv)      [join=by tip];
    \chainin (args)        [join=by tip];
    \chainin (parfer)      [join=by tip];
    \chainin (pend)        [join=by tip];
  }
  { [start chain]
    \chainin (startloopup) ;
    \chainin (startloopup2) [join=by {vh path}];
    \chainin (loopleft)     [join] ;
    \chainin (optionloop) [join=by {hv path}];
  }
  { [start chain]
    \chainin (optionstart) ;
    \chainin (croouv)      [join=by {vh path, tip}];
    \chainin (exparray)    [join=by tip] ;
    \chainin (crofer)      [join=by tip];
    \chainin (startloopup) [join=by {hv path}];
  }
\end{tikzpicture}


