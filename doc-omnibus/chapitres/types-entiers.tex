%!TEX encoding = UTF-8 Unicode
%!TEX root = ../doc-omnibus.tex





\chapterLabel{Les types entiers}{chapitreTypesEntiers}

Sont définis implicitement les types entiers signés et non signés d'une taille variant entre $1$ bit et $32768$ bits, et sont notés :
\begin{itemize}
  \item \omnibus!UInt1! à \omnibus!UInt32768! pour les types entiers non signés de $1$ à $32768$ bits~;
  \item \omnibus!Int1! à \omnibus!Int32768! pour les types entiers signés de $1$ à $32768$ bits.
\end{itemize}

OMNIBUS définit aussi le type \omnibus!LiteralInt!, qui est le type de toute constante entière littérale. Mais ce type est aussi applicable à des constantes entières \emph{statiques}, c'est-à-dire dont la valeur est calculée à la compilation. Ce type est décrit au \refChapterPage{chapitreTypeEntierStatique}.

\section{Constante littérale entière}

Le langage accepte des constantes littérales entières d'une taille quelconque. Une constante est convertie dans le type entier requis par le contexte sémantique, et une erreur est déclenchée à la compilation en cas d'impossibilité. Par exemple :

\begin{OMNIBUS}
var v Int8 = 128  // Erreur de compilation : 128 non
                   // représentable par un entier signé 8 bits
var v Int8 = -128 // Ok
\end{OMNIBUS}

Une constante littérale entière a pour type \omnibus!LiteralInt!, or ce type n'est pas acceptable pour une variable. Par exemple, si on écrit :
\begin{OMNIBUS}
var v = 28 // Erreur, le type LiteralInt n'est pas valide pour une variable
\end{OMNIBUS}

Dans ce cas, il faut que la déclaration contienne l'annotation de type :
\begin{OMNIBUS}
var v Int32 = 28 // Ok
\end{OMNIBUS}




\section{Conversion entre objets de type entier}

Il y a trois types de conversion entre objets de type entier :
\begin{itemize}
  \item les conversions toujours possibles \omnibus+extend Type (exp)+ (\refSubsectionPage{conversionsToujoursPossibles})~;
  \item les conversions pouvant échouer \omnibus+convert Type (exp)+ (\refSubsectionPage{conversionsPouvantEchouer})~;
  \item les troncatures \omnibus+truncate Type (exp)+  (\refSubsectionPage{conversionsTroncature}).
\end{itemize}


\subsectionLabel{Conversions toujours possibles : \texttt{extend}}{conversionsToujoursPossibles}

Les conversions qui sont toujours possibles sont exprimées par le mot réservé \omnibus=extend=. Par exemple :
\begin{OMNIBUS}
let v UInt8 = ...
let x UInt9 = extend UInt9 (v)
let y Int9 =  extend Int9 (v)
let z Int10 = extend Int10 (y)
\end{OMNIBUS}

D'une manière générale :
\begin{itemize}
\item un entier non signé peut être étendu en un entier non signé de taille strictement supérieure~;
\item un entier non signé peut être étendu en un entier signé de taille strictement supérieure~;
\item un entier signé peut être étendu en un entier signé de taille strictement supérieure.
\end{itemize}

Par contre, une conversion pouvant provoquer un débordement est rejetée à la compilation :
\begin{OMNIBUS}
let s Int8 = ...
let x UInt16 = x // Erreur de compilation
\end{OMNIBUS}

L'annotation de type après le mot-clé \omnibus=extend= est optionnel~: par défaut, le type est inféré par le contexte. Par exemple, on peut écrire~:
\begin{OMNIBUS}
let x Int9 =  ...
let y Int10 = extend (x) // Le type Int10 est inféré du contexte
let z = extend (x) // Invalide : aucun type ne peut être inféré.
\end{OMNIBUS}


\subsectionLabel{Conversions pouvant échouer : \texttt{convert}}{conversionsPouvantEchouer}

Les conversions pouvant échouer sont exprimées par le mot réservé \omnibus=convert=. Par exemple :

\begin{OMNIBUS}
let s Int8 = ...
let x UInt16 = convert UInt16 (s)
\end{OMNIBUS}

L'opérateur \omnibus+convert+ engendre un code qui vérifie à l'exécution que l'expression source (ici \omnibus+x+) peut être convertie dans le type cible (ici \omnibus+UInt16+) sans débordement. En cas de débordement détecté à l'exécution, la panique dont le code est donné dans le \refTableauPage{tableauCodePanique} est déclenchée. L'opérateur \omnibus+convert+ est donc interdit dans les constructions où la panique ne peut être déclenchée : il faut alors utiliser l'opérateur \omnibus=truncate=.

L'annotation de type après le mot-clé \omnibus=convert= est optionnel~: par défaut, le type est inféré par le contexte. Par exemple, on peut écrire~:
\begin{OMNIBUS}
let s Int8 = ...
let x UInt16 = convert (s) // Le type UInt16 est inféré du contexte
let y = convert (s) // Invalide : aucun type ne peut être inféré
\end{OMNIBUS}

L'opérateur \omnibus+convert+ ne peut pas apparaître dans une expression statique.

De plus, une erreur de compilation est déclenchée si l'opérateur \omnibus+convert+ est utilisé alors que la conversion est toujours possible :
\begin{OMNIBUS}
let v UInt8 = ...
let y = convert Int16 (v) // Erreur, conversion toujours possible
\end{OMNIBUS}

\subsectionLabel{Troncatures : \texttt{truncate}}{conversionsTroncature}

L'opérateur \omnibus+truncate+ permet de spécifier une conversion explicite silencieuse, qui ne déclenche aucune panique. La valeur de l'expression source est tronquée en cas de débordement\footnote{L'opérateur \texttt{truncate} est équivalent au \emph{type cast} entre entiers du langage C.}. Par exemple :

\begin{OMNIBUS}
let s Int8 = -10
let x UInt16 = truncate UInt16 (x)
\end{OMNIBUS}

L'annotation de type après le mot-clé \omnibus=truncate= est optionnel~: par défaut, le type est inféré par le contexte. Par exemple, on peut écrire~:
\begin{OMNIBUS}
let s Int8 = -10
let x UInt16 = truncate (s) // Le type UInt16 est inféré du contexte
let y = truncate (s) // Invalide : aucun type ne peut être inféré
\end{OMNIBUS}


L'opérateur \omnibus+truncate+ ne peut pas apparaître dans une expression statique.

De plus, une erreur de compilation est déclenchée si l'opérateur \omnibus+truncate+ est utilisé alors qu'une conversion implicite est possible :
\begin{OMNIBUS}
let v UInt8 = ...
let y = truncate Int16 (v) // Erreur, conversion toujours possible
\end{OMNIBUS}

\section{Opérateurs infixes de comparaison}

Les valeurs entières sont comparables, les six opérateurs \omnibus+==+, \omnibus+≠+, \omnibus+≥+, \omnibus+>+, \omnibus+≤+ et \omnibus+<+ sont acceptés.

La comparaison ne peut s'effectuer qu'entre objets du même type entier, ou entre un objet de type entier et une constante littérale entière.









\sectionLabel{Opérateurs infixes arithmétiques}{operateursInfixArithmétiques}


Les opérateurs infixes arithmétiques sont listés dans le \refTableau{operateursInfixesArithmetiques} avec leur signification. Ils ne peuvent opérer qu'entre objets du même type entier, ou entre un objet de type entier et une constante littérale entière.


\begin{table}[!ht]
\centering
\begin{tabular}{lllll}
  \textbf{Opérateur} & \textbf{Signification} \\
  \omnibus=+= & Addition avec détection de débordement\\
  \omnibus=-= & Soustraction avec détection de débordement\\
  \omnibus=*= & Multiplication avec détection de débordement\\
  \omnibus=/= & Division avec détection de débordement\\
  \omnibus=%= & Modulo avec détection de division par zéro\\
  \omnibus=+%= & Addition sans détection de débordement\\
  \omnibus=-%= & Soustraction sans détection de débordement\\
  \omnibus=*%= & Multiplication sans détection de débordement\\
  \omnibus=!/= & Division sans détection de débordement\\
  \omnibus=!%= & Modulo sans détection de division par zéro\\
\end{tabular}
\caption{Opérateurs infixes arithmétiques}
\labelTableau{operateursInfixesArithmetiques}
\ligne
\end{table}




\section{Opérateurs préfixés de négation arithmétique}

\subsectionLabel{Opérateur \texttt{-}}{negationOvf}

L'opérateur préfixé \omnibus=-= est la négation arithmétique avec détection de débordement. Il n'est accepté que sur les types signés. La négation de la borne inférieure d'un type signé (\omnibus+-128+ pour \omnibus+Int8+, \omnibus+-32768+ pour \omnibus+Int16+, ...) entraîne un débordement arithmétique qui déclenche une panique dont le code est donné dans le \refTableau{tableauCodePanique}.


\subsectionLabel{Opérateur \texttt{-\%}}{negationNoOvf}

L'opérateur préfixé \omnibus=-%= est la négation arithmétique sans détection de débordement. Il n'est accepté que sur les types signés. La négation de la borne inférieure d'un type signé (\omnibus+-128+ pour \omnibus+Int8+, \omnibus+-32768+ pour \omnibus+Int16+, ...) retourne cette même valeur. Cet opérateur ne déclenche jamais de panique.




\sectionLabel{Opérateurs infixes bit-à-bit}{operateurBitABitEntier}
\index{"\&!Entier}
\index{\textbar!Entier}
\index{\^!Entier}

Les opérateurs infixes bit-à-bit acceptent les types entiers non signés (\refTableau{operateursInfixesBitABit}).

\begin{table}[h]
\centering
\begin{tabular}{lllll}
  \textbf{Opérateur} & \textbf{Signification} \\
  \omnibus=|= & \emph{ou} bit-à-bit\\
  \omnibus=&= & \emph{et} bit-à-bit\\
  \omnibus=^= & \emph{ou exclusif} bit-à-bit\\
\end{tabular}
\caption{Opérateurs infixes bit-à-bit sur les entiers non signés}
\labelTableau{operateursInfixesBitABit}
\ligne
\end{table}





\section{Opérateur préfixé bit-à-bit}

L'opérateur préfixé \omnibus=~= retourne la complémentation bit-à-bit d'une valeur entière non signée.




\section{Opérateurs infixes de décalage}

Les opérateurs infixes \omnibus=<<= et \omnibus=>>= réalisent respectivement le décalage à gauche et à droite de l'opérande de gauche. L'amplitude du décalage est spécifiée par la valeur de l'opérande droite (\refTableau{operateursInfixesDecalage}). \omnibus=a= est une expression entière signée ou non signée, et l'expression renvoie une valeur de même type que \omnibus=a=. L'expression \omnibus=b= est une expression entière non signée.

\begin{table}[h]
\centering
\begin{tabular}{lllll}
  \textbf{Expression} & \textbf{Signification} \\
  \omnibus=a << b= & Décalage à gauche de \omnibus=a= d'une amplitude de \omnibus=b= bits\\
  \omnibus=a >> b= & Décalage à droite de \omnibus=a= d'une amplitude de \omnibus=b= bits\\
\end{tabular}
\caption{Opérateurs infixes de décalage sur les entiers}
\labelTableau{operateursInfixesDecalage}
\ligne
\end{table}








\sectionLabel{Opérateurs combinées avec une affectation}{operateursCombinesAffectationEntier}
\index{\&=!Entier}
\index{\textbar=!Entier}
\index{\^{}=!Entier}

Les opérateurs suivants sont définis pour les entiers.

\omnibus!a &= b! est équivalent à \omnibus!a = a & b!.

\omnibus!a |= b! est équivalent à \omnibus!a = a | b!.

\omnibus!a ^= b! est équivalent à \omnibus!a = a ^ b!.

\omnibus!a += b! est équivalent à \omnibus!a = a + b!.

\omnibus!a +%= b! est équivalent à \omnibus!a = a +% b!.

\omnibus!a -= b! est équivalent à \omnibus!a = a - b!.

\omnibus!a -%= b! est équivalent à \omnibus!a = a -% b!.

\omnibus!a *= b! est équivalent à \omnibus!a = a * b!.

\omnibus!a *%= b! est équivalent à \omnibus!a = a *% b!.

Les opérateurs infixes \omnibus!&=!, \omnibus!|=! et \omnibus!^=! sont décrits à la \refSectionPage{operateurBitABitEntier}.

Les opérateurs infixes \omnibus!+!, \omnibus!+%!, \omnibus!-!, \omnibus!-%!, \omnibus!*! et \omnibus!*%! sont décrits à la \refSectionPage{operateursInfixArithmétiques}.








\sectionLabel{Accesseurs}{accesseursEntiers}


\subsection{Accesseur \texttt{bitReversed}}

L'accesseur \omnibus=bitReversed= est implémenté pour tout type entier. Pour un entier sur $n$ bits, il retourne une valeur dont le bit d'indice $i$ (avec $0 \leqslant i < n$) a la valeur du bit $n-i-1$ du récepteur. Par exemple~:

\begin{OMNIBUS}
let x UInt3 = 0x_3
let y = x.bitReversed () // 0x_6
let z UInt16 = 0x_1234
let t = z.bitReversed () // 0x_2C48
\end{OMNIBUS}



\subsection{Accesseur \texttt{byteSwapped}}

L'accesseur \omnibus=byteSwapped= est implémenté pour tout type entier dont la taille est un multiple de 16 bits, qu'il soit signé ou non (\omnibus=Int16=, \omnibus=UInt16=, \omnibus=Int32=, \omnibus=UInt32=, \omnibus=Int48=, \omnibus=UInt48=, \omnibus=Int64=, \omnibus=UInt64=, …). Il réalise la conversion \emph{big endian} $\leftrightarrow$ \emph{little endian}. Par exemple~:

\begin{OMNIBUS}
let x : UInt32 = 0x_1234_5678
let y = x.byteSwapped () // 0x_8756_3412
let z UInt48 = 0x_1234_5678_ABCD
let t = z.byteSwapped () // 0x_CDAB_8756_3412
\end{OMNIBUS}



\subsection{Accesseur \texttt{leadingZeroCount}}

L'accesseur \omnibus=leadingZeroCount= est implémenté pour tout type entier. Il retourne le nombre de bits à zéro à partir du bit le plus significatif. Par exemple~:

\begin{OMNIBUS}
let x : UInt32 = 0x_1234_5678
let y = x.leadingZeroCount () // 3
\end{OMNIBUS}



\subsection{Accesseur \texttt{setBitCount}}

L'accesseur \omnibus=setBitCount= est implémenté pour tout type entier. Il retourne le nombre de bits à un dans la valeur du récepteur. Par exemple~:

\begin{OMNIBUS}
let x UInt16 = 0x_1234
let y = x.setBitCount () // 5
\end{OMNIBUS}



\subsection{Accesseur \texttt{trainingZeroCount}}

L'accesseur \omnibus=trainingZeroCount= est implémenté pour tout type entier. Il retourne le nombre de bits à zéro à partir du bit le moins significatif. Par exemple~:

\begin{OMNIBUS}
let x UInt16 = 0x_1234
let y = x.trainingZeroCount () // 2
\end{OMNIBUS}




\section{Construction d'un entier non signé par tranches}

Cette expression permet de construire un entier non signé à partir d'entiers non signés ou de booléens. Par exemple~:
\begin{OMNIBUS}
let x = {UInt8 !1:1 !1:0 !6:12} // 0x8C
\end{OMNIBUS}

Dans l'exemple ci-dessus, le bit n°7 est à 1, le bit n°6 à 0, et les 6 bits de poids faibles à 12.

Les tranches sont désignés par des sélecteurs (\omnibus=!1:=, \omnibus=!6:=) qui indiquent le nombre de bits de la tranche. La description commence à partir du bit de poids fort. L'expression qui suit le sélecteur doit être du type entier non signé correspondant~: par exemple, \omnibus=!1:= $\rightarrow$ \omnibus=UInt1=, \omnibus=!6:= $\rightarrow$ \omnibus=UInt6=.

Attention, une tranche \omnibus=!1:= signifie que l'expression doit être un entier non signé sur un bit, et non pas un booléen. Si l'on veut initialiser une tranche de 1 bit à partir d'un booléen, il faut utiliser le sélecteur particulier \omnibus=!b:=. Par exemple~:
\begin{OMNIBUS}
let x = {UInt8 !b:yes !1:0 !6:12} // 0x8C
\end{OMNIBUS}

Si toutes les expressions associées aux sélecteurs sont statiques, alors l'expression de construction d'un entier non signé par tranches est aussi une expression statique~: on peut donc utiliser cette construction pour définir des constantes globales.



