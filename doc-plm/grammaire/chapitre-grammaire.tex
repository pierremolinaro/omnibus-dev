%!TEX encoding = UTF-8 Unicode
%!TEX root = ../doc-plm.tex





\chapter{Grammaires}

Le langage PLM définit deux grammaires :
\begin{itemize}
\item la grammaire des sources des programmes PLM (\refSectionPage{grammairePLM}) ;
\item la grammaire de description d'un cible (\refSectionPage{grammaireCible}).
\end{itemize}

Ce chapitre liste l'ensemble des règles de production de ces deux grammaires.


{

\tikzset{
  nonterminal/.style={
    % The shape:
    rectangle,
    % The size:
    minimum size=6mm,
    % The border:
    very thick,
    draw=red!50!black!50,         % 50% red and 50% black,
                                  % and that mixed with 50% white
    % The filling:
    top color=white,              % a shading that is white at the top...
    bottom color=red!50!black!20, % and something else at the bottom
    % Font
    font=\itshape\footnotesize
  },
  terminal/.style={
    % The shape:
    rounded rectangle,
    minimum size=6mm,
    % The rest
    very thick,draw=black!50,
    top color=white,bottom color=black!20,
    font=\ttfamily\footnotesize
  },
  firstPoint/.style={circle,>=stealth',thick,draw=black!50},
  point/.style={coordinate,>=stealth',thick,draw=black!50},
  tip/.style={->,shorten >=0.007pt},
  lastPoint/.style={rectangle,>=stealth',thick,draw=black!50},
  every join/.style={rounded corners}
}

\newcommand\ruleSubsection[3]{} % \subsection{Component \texttt{#1}, in file \texttt{#2}, line #3}}
\newcommand\ruleMatrixColumnSeparation{2mm}
\newcommand\ruleMatrixRowSeparation{1.5mm}
\newcommand\nonTerminalSummarySeparator{, }
\newcommand\nonTerminalSummaryEnd{.\\}
\newcommand\nonTerminalSummaryStart{Voici la liste alphabétique des non terminaux: }

\newcommand\nonTerminalSection[2]{\subsection{Non terminal \texttt{\it#1}}\label{nt:#2}}
\newcommand\nonTerminalSummary[2]{\hyperref[nt:#2]{#1}}
\newcommand\nonTerminalSymbol[2]{\hyperref[nt:#2]{#1}}
\newcommand\startSymbol[2]{L'axiome de la grammaire est \hyperref[nt:#2]{#1}.}

\sectionLabel{Grammaire du langage \texttt{PLM}}{grammairePLM}

\startSymbol{start\_symbol}{1}

\nonTerminalSummaryStart \nonTerminalSummary{assignment\_operator}{38}\nonTerminalSummarySeparator \nonTerminalSummary{declaration}{2}\nonTerminalSummarySeparator \nonTerminalSummary{declaration\_init}{15}\nonTerminalSummarySeparator \nonTerminalSummary{effective\_parameters}{43}\nonTerminalSummarySeparator \nonTerminalSummary{expression}{22}\nonTerminalSummarySeparator \nonTerminalSummary{expression\_access\_list}{35}\nonTerminalSummarySeparator \nonTerminalSummary{expression\_addition}{31}\nonTerminalSummarySeparator \nonTerminalSummary{expression\_bitwise\_and}{27}\nonTerminalSummarySeparator \nonTerminalSummary{expression\_bitwise\_or}{25}\nonTerminalSummarySeparator \nonTerminalSummary{expression\_bitwise\_xor}{26}\nonTerminalSummarySeparator \nonTerminalSummary{expression\_comparison}{29}\nonTerminalSummarySeparator \nonTerminalSummary{expression\_equality}{28}\nonTerminalSummarySeparator \nonTerminalSummary{expression\_if}{34}\nonTerminalSummarySeparator \nonTerminalSummary{expression\_logical\_and}{24}\nonTerminalSummarySeparator \nonTerminalSummary{expression\_logical\_xor}{23}\nonTerminalSummarySeparator \nonTerminalSummary{expression\_product}{32}\nonTerminalSummarySeparator \nonTerminalSummary{expression\_shift}{30}\nonTerminalSummarySeparator \nonTerminalSummary{function}{16}\nonTerminalSummarySeparator \nonTerminalSummary{function\_header}{18}\nonTerminalSummarySeparator \nonTerminalSummary{guard}{21}\nonTerminalSummarySeparator \nonTerminalSummary{guarded\_command}{40}\nonTerminalSummarySeparator \nonTerminalSummary{if\_instruction}{39}\nonTerminalSummarySeparator \nonTerminalSummary{import\_file}{0}\nonTerminalSummarySeparator \nonTerminalSummary{instruction}{37}\nonTerminalSummarySeparator \nonTerminalSummary{instructionList}{36}\nonTerminalSummarySeparator \nonTerminalSummary{isr}{20}\nonTerminalSummarySeparator \nonTerminalSummary{lvalue}{42}\nonTerminalSummarySeparator \nonTerminalSummary{mode}{17}\nonTerminalSummarySeparator \nonTerminalSummary{module\_property}{9}\nonTerminalSummarySeparator \nonTerminalSummary{primary}{33}\nonTerminalSummarySeparator \nonTerminalSummary{private\_or\_public\_struct\_property\_declaration}{4}\nonTerminalSummarySeparator \nonTerminalSummary{private\_struct\_property\_declaration}{5}\nonTerminalSummarySeparator \nonTerminalSummary{procedure\_call}{41}\nonTerminalSummarySeparator \nonTerminalSummary{procedure\_formal\_arguments}{19}\nonTerminalSummarySeparator \nonTerminalSummary{property\_in\_extension}{7}\nonTerminalSummarySeparator \nonTerminalSummary{registerDeclaration}{8}\nonTerminalSummarySeparator \nonTerminalSummary{start\_symbol}{1}\nonTerminalSummarySeparator \nonTerminalSummary{staticArrayProperty}{10}\nonTerminalSummarySeparator \nonTerminalSummary{staticArray\_exp}{11}\nonTerminalSummarySeparator \nonTerminalSummary{struct\_property\_declaration}{6}\nonTerminalSummarySeparator \nonTerminalSummary{system\_routine}{14}\nonTerminalSummarySeparator \nonTerminalSummary{task\_entry\_declaration}{12}\nonTerminalSummarySeparator \nonTerminalSummary{task\_guard\_declaration}{13}\nonTerminalSummarySeparator \nonTerminalSummary{type\_definition}{3}\nonTerminalSummaryEnd \nonTerminalSection{assignment\_operator}{38}

\ruleSubsection{plm\_syntax}{instruction-assignment-operator}{9}

\begin{tikzpicture}
  \matrix[column sep=\ruleMatrixColumnSeparation, row sep=\ruleMatrixRowSeparation] {
    & & & \node (p10-3) [terminal] {>>=}; & \\
    & & & \node (p9-3) [terminal] {<<=}; & \\
    & & & \node (p8-3) [terminal] {*\verb=%==}; & \\
    & & & \node (p7-3) [terminal] {*=}; & \\
    & & & \node (p6-3) [terminal] {-\verb=%==}; & \\
    & & & \node (p5-3) [terminal] {-=}; & \\
    & & & \node (p4-3) [terminal] {+\verb=%==}; & \\
    & & & \node (p3-3) [terminal] {+=}; & \\
    & & & \node (p2-3) [terminal] {\verb=^==}; & \\
    & & & \node (p1-3) [terminal] {\&=}; & \\
    \node (P0start) [firstPoint] {}; & & \node (p0-2) [point] {}; & \node (p0-3) [terminal] {|=}; & \node (p0-4) [point] {}; & \node (p0-5) [lastPoint] {}; & \\
  };
  \draw[->] (P0start) -- (p0-3) ;
  \draw[->] (p0-2) |- (p1-3) ;
  \draw[->] (p0-2) |- (p2-3) ;
  \draw[->] (p0-2) |- (p3-3) ;
  \draw[->] (p0-2) |- (p4-3) ;
  \draw[->] (p0-2) |- (p5-3) ;
  \draw[->] (p0-2) |- (p6-3) ;
  \draw[->] (p0-2) |- (p7-3) ;
  \draw[->] (p0-2) |- (p8-3) ;
  \draw[->] (p0-2) |- (p9-3) ;
  \draw[->] (p0-2) |- (p10-3) ;
  \draw (p0-3) -- (p0-4) ;
  \draw[->] (p1-3) -| (p0-4) ;
  \draw[->] (p2-3) -| (p0-4) ;
  \draw[->] (p3-3) -| (p0-4) ;
  \draw[->] (p4-3) -| (p0-4) ;
  \draw[->] (p5-3) -| (p0-4) ;
  \draw[->] (p6-3) -| (p0-4) ;
  \draw[->] (p7-3) -| (p0-4) ;
  \draw[->] (p8-3) -| (p0-4) ;
  \draw[->] (p9-3) -| (p0-4) ;
  \draw[->] (p10-3) -| (p0-4) ;
  \draw[->] (p0-4) -- (p0-5) ;
\end{tikzpicture}

\nonTerminalSection{declaration}{2}

\ruleSubsection{plm\_syntax}{declaration-type}{34}

\begin{tikzpicture}
  \matrix[column sep=\ruleMatrixColumnSeparation, row sep=\ruleMatrixRowSeparation] {
    \node (P0start) [firstPoint] {}; & & \node (p0-2) [terminal] {type}; & \node (p0-3) [terminal] {\$type}; & \node (p0-4) [terminal] {:}; & \node (p0-5) [nonterminal] {\nonTerminalSymbol{type\_definition}{3}}; & \node (p0-6) [lastPoint] {}; & \\
  };
  \draw[->] (P0start) -- (p0-2) ;
  \draw[->] (p0-2) -- (p0-3) ;
  \draw[->] (p0-3) -- (p0-4) ;
  \draw[->] (p0-4) -- (p0-5) ;
  \draw[->] (p0-5) -- (p0-6) ;
\end{tikzpicture}

\ruleSubsection{plm\_syntax}{type-enumeration-declaration}{25}

\begin{tikzpicture}
  \matrix[column sep=\ruleMatrixColumnSeparation, row sep=\ruleMatrixRowSeparation] {
    & & & & & & & & & \node (p1-9) [point] {}; & \\
    \node (P0start) [firstPoint] {}; & & \node (p0-2) [terminal] {enum}; & \node (p0-3) [terminal] {\$type}; & \node (p0-4) [terminal] {\{}; & \node (p0-5) [point] {}; & \node (p0-6) [terminal] {case}; & \node (p0-7) [terminal] {identifier}; & \node (p0-8) [point] {}; & & \node (p0-10) [terminal] {\}}; & \node (p0-11) [lastPoint] {}; & \\
  };
  \draw[->] (P0start) -- (p0-2) ;
  \draw[->] (p0-2) -- (p0-3) ;
  \draw[->] (p0-3) -- (p0-4) ;
  \draw[->] (p0-4) -- (p0-6) ;
  \draw[->] (p0-6) -- (p0-7) ;
  \draw[->] (p1-9) -| (p0-5) ;
  \draw[->] (p0-8) -| (p1-9) ;
  \draw[->] (p0-7) -- (p0-10) ;
  \draw[->] (p0-10) -- (p0-11) ;
\end{tikzpicture}

\ruleSubsection{plm\_syntax}{type-structure-declaration}{87}

\begin{tikzpicture}
  \matrix[column sep=\ruleMatrixColumnSeparation, row sep=\ruleMatrixRowSeparation] {
    & & & & & & & & & & & & & & \node (p6-14) [point] {}; & \\
    & & & & & & & & & & & & & \node (p5-13) [terminal] {;}; & \\
    & & & & & & & & & & & & & \node (p4-13) [nonterminal] {\nonTerminalSymbol{guard}{21}}; & \\
    & & & & & & & & & & & & & \node (p3-13) [nonterminal] {\nonTerminalSymbol{system\_routine}{14}}; & \\
    & & & & & & & & \node (p2-8) [point] {}; & & & & & \node (p2-13) [nonterminal] {\nonTerminalSymbol{function}{16}}; & \\
    & & & & & & & \node (p1-7) [terminal] {@attribute}; & & & & & & \node (p1-13) [nonterminal] {\nonTerminalSymbol{private\_or\_public\_struct\_property\_declaration}{4}}; & \\
    \node (P0start) [firstPoint] {}; & & \node (p0-2) [terminal] {struct}; & \node (p0-3) [terminal] {\$type}; & \node (p0-4) [point] {}; & \node (p0-5) [point] {}; & \node (p0-6) [point] {}; & & & \node (p0-9) [terminal] {\{}; & \node (p0-10) [point] {}; & \node (p0-11) [point] {}; & \node (p0-12) [point] {}; & & & \node (p0-15) [terminal] {\}}; & \node (p0-16) [lastPoint] {}; & \\
  };
  \draw[->] (P0start) -- (p0-2) ;
  \draw[->] (p0-2) -- (p0-3) ;
  \draw (p0-3) -- (p0-5) ;
  \draw[->] (p0-6) |- (p1-7) ;
  \draw[->] (p2-8) -| (p0-4) ;
  \draw[->] (p1-7) -| (p2-8) ;
  \draw[->] (p0-5) -- (p0-9) ;
  \draw (p0-9) -- (p0-11) ;
  \draw[->] (p0-12) |- (p1-13) ;
  \draw[->] (p0-12) |- (p2-13) ;
  \draw[->] (p0-12) |- (p3-13) ;
  \draw[->] (p0-12) |- (p4-13) ;
  \draw[->] (p0-12) |- (p5-13) ;
  \draw[->] (p6-14) -| (p0-10) ;
  \draw[->] (p1-13) -| (p6-14) ;
  \draw[->] (p2-13) -| (p6-14) ;
  \draw[->] (p3-13) -| (p6-14) ;
  \draw[->] (p4-13) -| (p6-14) ;
  \draw[->] (p5-13) -| (p6-14) ;
  \draw[->] (p0-11) -- (p0-15) ;
  \draw[->] (p0-15) -- (p0-16) ;
\end{tikzpicture}

\ruleSubsection{plm\_syntax}{type-extension-declaration}{21}

\begin{tikzpicture}
  \matrix[column sep=\ruleMatrixColumnSeparation, row sep=\ruleMatrixRowSeparation] {
    & & & & & & & & & \node (p6-9) [point] {}; & \\
    & & & & & & & & \node (p5-8) [terminal] {;}; & \\
    & & & & & & & & \node (p4-8) [nonterminal] {\nonTerminalSymbol{guard}{21}}; & \\
    & & & & & & & & \node (p3-8) [nonterminal] {\nonTerminalSymbol{system\_routine}{14}}; & \\
    & & & & & & & & \node (p2-8) [nonterminal] {\nonTerminalSymbol{function}{16}}; & \\
    & & & & & & & & \node (p1-8) [nonterminal] {\nonTerminalSymbol{property\_in\_extension}{7}}; & \\
    \node (P0start) [firstPoint] {}; & & \node (p0-2) [terminal] {extension}; & \node (p0-3) [terminal] {\$type}; & \node (p0-4) [terminal] {\{}; & \node (p0-5) [point] {}; & \node (p0-6) [point] {}; & \node (p0-7) [point] {}; & & & \node (p0-10) [terminal] {\}}; & \node (p0-11) [lastPoint] {}; & \\
  };
  \draw[->] (P0start) -- (p0-2) ;
  \draw[->] (p0-2) -- (p0-3) ;
  \draw[->] (p0-3) -- (p0-4) ;
  \draw (p0-4) -- (p0-6) ;
  \draw[->] (p0-7) |- (p1-8) ;
  \draw[->] (p0-7) |- (p2-8) ;
  \draw[->] (p0-7) |- (p3-8) ;
  \draw[->] (p0-7) |- (p4-8) ;
  \draw[->] (p0-7) |- (p5-8) ;
  \draw[->] (p6-9) -| (p0-5) ;
  \draw[->] (p1-8) -| (p6-9) ;
  \draw[->] (p2-8) -| (p6-9) ;
  \draw[->] (p3-8) -| (p6-9) ;
  \draw[->] (p4-8) -| (p6-9) ;
  \draw[->] (p5-8) -| (p6-9) ;
  \draw[->] (p0-6) -- (p0-10) ;
  \draw[->] (p0-10) -- (p0-11) ;
\end{tikzpicture}

\ruleSubsection{plm\_syntax}{declaration-control-register}{53}

\begin{tikzpicture}
  \matrix[column sep=\ruleMatrixColumnSeparation, row sep=\ruleMatrixRowSeparation] {
    & & & & & & & & & & & & & & & & & & & & & \node (p4-21) [point] {}; & \\
    & & & & & & & & & & & & & & \node (p3-14) [terminal] {[}; & \node (p3-15) [terminal] {integer}; & \node (p3-16) [terminal] {]}; & \\
    & & & & & & & & & & & & \node (p2-12) [terminal] {identifier}; & \node (p2-13) [point] {}; & \node (p2-14) [point] {}; & & & \node (p2-17) [point] {}; & & & \node (p2-20) [terminal] {,}; & \\
    & & & & & & \node (p1-6) [point] {}; & & & \node (p1-9) [terminal] {\{}; & \node (p1-10) [point] {}; & \node (p1-11) [point] {}; & \node (p1-12) [terminal] {integer}; & & & & & & \node (p1-18) [point] {}; & \node (p1-19) [point] {}; & & & \node (p1-22) [terminal] {\}}; & \\
    \node (P0start) [firstPoint] {}; & & \node (p0-2) [terminal] {register}; & \node (p0-3) [point] {}; & \node (p0-4) [nonterminal] {\nonTerminalSymbol{registerDeclaration}{8}}; & \node (p0-5) [point] {}; & & \node (p0-7) [terminal] {\$type}; & \node (p0-8) [point] {}; & \node (p0-9) [point] {}; & & & & & & & & & & & & & & \node (p0-23) [point] {}; & \node (p0-24) [lastPoint] {}; & \\
  };
  \draw[->] (P0start) -- (p0-2) ;
  \draw[->] (p0-2) -- (p0-4) ;
  \draw[->] (p1-6) -| (p0-3) ;
  \draw[->] (p0-5) -| (p1-6) ;
  \draw[->] (p0-4) -- (p0-7) ;
  \draw (p0-7) -- (p0-9) ;
  \draw[->] (p0-8) |- (p1-9) ;
  \draw[->] (p1-9) -- (p1-12) ;
  \draw[->] (p1-11) |- (p2-12) ;
  \draw (p2-12) -- (p2-14) ;
  \draw[->] (p2-13) |- (p3-14) ;
  \draw[->] (p3-14) -- (p3-15) ;
  \draw[->] (p3-15) -- (p3-16) ;
  \draw (p2-14) -- (p2-17) ;
  \draw[->] (p3-16) -| (p2-17) ;
  \draw (p1-12) -- (p1-18) ;
  \draw[->] (p2-17) -| (p1-18) ;
  \draw[->] (p1-19) |- (p2-20) ;
  \draw[->] (p4-21) -| (p1-10) ;
  \draw[->] (p2-20) -| (p4-21) ;
  \draw[->] (p1-18) -- (p1-22) ;
  \draw (p0-9) -- (p0-23) ;
  \draw[->] (p1-22) -| (p0-23) ;
  \draw[->] (p0-23) -- (p0-24) ;
\end{tikzpicture}

\ruleSubsection{plm\_syntax}{declaration-global-constant}{25}

\begin{tikzpicture}
  \matrix[column sep=\ruleMatrixColumnSeparation, row sep=\ruleMatrixRowSeparation] {
    & & & & & \node (p1-5) [nonterminal] {\nonTerminalSymbol{type\_definition}{3}}; & \\
    \node (P0start) [firstPoint] {}; & & \node (p0-2) [terminal] {let}; & \node (p0-3) [terminal] {identifier}; & \node (p0-4) [point] {}; & \node (p0-5) [point] {}; & \node (p0-6) [point] {}; & \node (p0-7) [terminal] {=}; & \node (p0-8) [nonterminal] {\nonTerminalSymbol{expression}{22}}; & \node (p0-9) [lastPoint] {}; & \\
  };
  \draw[->] (P0start) -- (p0-2) ;
  \draw[->] (p0-2) -- (p0-3) ;
  \draw (p0-3) -- (p0-5) ;
  \draw[->] (p0-4) |- (p1-5) ;
  \draw (p0-5) -- (p0-6) ;
  \draw[->] (p1-5) -| (p0-6) ;
  \draw[->] (p0-6) -- (p0-7) ;
  \draw[->] (p0-7) -- (p0-8) ;
  \draw[->] (p0-8) -- (p0-9) ;
\end{tikzpicture}

\ruleSubsection{plm\_syntax}{declaration-module}{86}

\begin{tikzpicture}
  \matrix[column sep=\ruleMatrixColumnSeparation, row sep=\ruleMatrixRowSeparation] {
    & & & & & & & & & \node (p8-9) [point] {}; & \\
    & & & & & & & & \node (p7-8) [terminal] {;}; & \\
    & & & & & & & & \node (p6-8) [nonterminal] {\nonTerminalSymbol{guard}{21}}; & \\
    & & & & & & & & \node (p5-8) [nonterminal] {\nonTerminalSymbol{system\_routine}{14}}; & \\
    & & & & & & & & \node (p4-8) [nonterminal] {\nonTerminalSymbol{function}{16}}; & \\
    & & & & & & & & \node (p3-8) [nonterminal] {\nonTerminalSymbol{module\_property}{9}}; & \\
    & & & & & & & & \node (p2-8) [nonterminal] {\nonTerminalSymbol{isr}{20}}; & \\
    & & & & & & & & \node (p1-8) [nonterminal] {\nonTerminalSymbol{declaration\_init}{15}}; & \\
    \node (P0start) [firstPoint] {}; & & \node (p0-2) [terminal] {module}; & \node (p0-3) [terminal] {identifier}; & \node (p0-4) [terminal] {\{}; & \node (p0-5) [point] {}; & \node (p0-6) [point] {}; & \node (p0-7) [point] {}; & & & \node (p0-10) [terminal] {\}}; & \node (p0-11) [lastPoint] {}; & \\
  };
  \draw[->] (P0start) -- (p0-2) ;
  \draw[->] (p0-2) -- (p0-3) ;
  \draw[->] (p0-3) -- (p0-4) ;
  \draw (p0-4) -- (p0-6) ;
  \draw[->] (p0-7) |- (p1-8) ;
  \draw[->] (p0-7) |- (p2-8) ;
  \draw[->] (p0-7) |- (p3-8) ;
  \draw[->] (p0-7) |- (p4-8) ;
  \draw[->] (p0-7) |- (p5-8) ;
  \draw[->] (p0-7) |- (p6-8) ;
  \draw[->] (p0-7) |- (p7-8) ;
  \draw[->] (p8-9) -| (p0-5) ;
  \draw[->] (p1-8) -| (p8-9) ;
  \draw[->] (p2-8) -| (p8-9) ;
  \draw[->] (p3-8) -| (p8-9) ;
  \draw[->] (p4-8) -| (p8-9) ;
  \draw[->] (p5-8) -| (p8-9) ;
  \draw[->] (p6-8) -| (p8-9) ;
  \draw[->] (p7-8) -| (p8-9) ;
  \draw[->] (p0-6) -- (p0-10) ;
  \draw[->] (p0-10) -- (p0-11) ;
\end{tikzpicture}

\ruleSubsection{plm\_syntax}{declaration-module}{126}

\begin{tikzpicture}
  \matrix[column sep=\ruleMatrixColumnSeparation, row sep=\ruleMatrixRowSeparation] {
    & & & & & & & & & & \node (p2-10) [point] {}; & \\
    & & & & & & & & \node (p1-8) [terminal] {!selector:}; & \node (p1-9) [nonterminal] {\nonTerminalSymbol{expression}{22}}; & \\
    \node (P0start) [firstPoint] {}; & & \node (p0-2) [terminal] {module}; & \node (p0-3) [terminal] {identifier}; & \node (p0-4) [terminal] {(}; & \node (p0-5) [point] {}; & \node (p0-6) [point] {}; & \node (p0-7) [point] {}; & & & & \node (p0-11) [terminal] {)}; & \node (p0-12) [lastPoint] {}; & \\
  };
  \draw[->] (P0start) -- (p0-2) ;
  \draw[->] (p0-2) -- (p0-3) ;
  \draw[->] (p0-3) -- (p0-4) ;
  \draw (p0-4) -- (p0-6) ;
  \draw[->] (p0-7) |- (p1-8) ;
  \draw[->] (p1-8) -- (p1-9) ;
  \draw[->] (p2-10) -| (p0-5) ;
  \draw[->] (p1-9) -| (p2-10) ;
  \draw[->] (p0-6) -- (p0-11) ;
  \draw[->] (p0-11) -- (p0-12) ;
\end{tikzpicture}

\ruleSubsection{plm\_syntax}{declaration-static-array}{60}

\begin{tikzpicture}
  \matrix[column sep=\ruleMatrixColumnSeparation, row sep=\ruleMatrixRowSeparation] {
    & & & & & & & & \node (p1-8) [point] {}; & \\
    \node (P0start) [firstPoint] {}; & & \node (p0-2) [terminal] {staticArray}; & \node (p0-3) [terminal] {identifier}; & \node (p0-4) [terminal] {\{}; & \node (p0-5) [point] {}; & \node (p0-6) [nonterminal] {\nonTerminalSymbol{staticArrayProperty}{10}}; & \node (p0-7) [point] {}; & & \node (p0-9) [terminal] {\}}; & \node (p0-10) [lastPoint] {}; & \\
  };
  \draw[->] (P0start) -- (p0-2) ;
  \draw[->] (p0-2) -- (p0-3) ;
  \draw[->] (p0-3) -- (p0-4) ;
  \draw[->] (p0-4) -- (p0-6) ;
  \draw[->] (p1-8) -| (p0-5) ;
  \draw[->] (p0-7) -| (p1-8) ;
  \draw[->] (p0-6) -- (p0-9) ;
  \draw[->] (p0-9) -- (p0-10) ;
\end{tikzpicture}

\ruleSubsection{plm\_syntax}{declaration-static-array}{100}

\begin{tikzpicture}
  \matrix[column sep=\ruleMatrixColumnSeparation, row sep=\ruleMatrixRowSeparation] {
    & & & & & & & & & & & & & & \node (p3-14) [point] {}; & \\
    & & & & & & & & & & & \node (p2-11) [point] {}; & \\
    & & & & & & & & & & \node (p1-10) [terminal] {,}; & & & \node (p1-13) [terminal] {;}; & \\
    \node (P0start) [firstPoint] {}; & & \node (p0-2) [terminal] {extend}; & \node (p0-3) [terminal] {staticArray}; & \node (p0-4) [terminal] {identifier}; & \node (p0-5) [terminal] {(}; & \node (p0-6) [point] {}; & \node (p0-7) [point] {}; & \node (p0-8) [nonterminal] {\nonTerminalSymbol{staticArray\_exp}{11}}; & \node (p0-9) [point] {}; & & & \node (p0-12) [point] {}; & & & \node (p0-15) [terminal] {)}; & \node (p0-16) [lastPoint] {}; & \\
  };
  \draw[->] (P0start) -- (p0-2) ;
  \draw[->] (p0-2) -- (p0-3) ;
  \draw[->] (p0-3) -- (p0-4) ;
  \draw[->] (p0-4) -- (p0-5) ;
  \draw[->] (p0-5) -- (p0-8) ;
  \draw[->] (p0-9) |- (p1-10) ;
  \draw[->] (p2-11) -| (p0-7) ;
  \draw[->] (p1-10) -| (p2-11) ;
  \draw[->] (p0-12) |- (p1-13) ;
  \draw[->] (p3-14) -| (p0-6) ;
  \draw[->] (p1-13) -| (p3-14) ;
  \draw[->] (p0-8) -- (p0-15) ;
  \draw[->] (p0-15) -- (p0-16) ;
\end{tikzpicture}

\ruleSubsection{plm\_syntax}{task-declaration}{45}

\begin{tikzpicture}
  \matrix[column sep=\ruleMatrixColumnSeparation, row sep=\ruleMatrixRowSeparation] {
    & & & & & & & & & & & & & & & & & & & & & & \node (p8-22) [point] {}; & \\
    & & & & & & & & & & & & \node (p7-12) [nonterminal] {\nonTerminalSymbol{task\_guard\_declaration}{13}}; & \\
    & & & & & & & & & & & & \node (p6-12) [nonterminal] {\nonTerminalSymbol{task\_entry\_declaration}{12}}; & \\
    & & & & & & & & & & & & \node (p5-12) [nonterminal] {\nonTerminalSymbol{guarded\_command}{40}}; & \node (p5-13) [terminal] {\{}; & \node (p5-14) [nonterminal] {\nonTerminalSymbol{instructionList}{36}}; & \node (p5-15) [terminal] {\}}; & \\
    & & & & & & & & & & & & \node (p4-12) [terminal] {setup}; & \node (p4-13) [terminal] {integer}; & \node (p4-14) [terminal] {\{}; & \node (p4-15) [nonterminal] {\nonTerminalSymbol{instructionList}{36}}; & \node (p4-16) [terminal] {\}}; & \\
    & & & & & & & & & & & & & & & & \node (p3-16) [terminal] {->}; & \node (p3-17) [nonterminal] {\nonTerminalSymbol{type\_definition}{3}}; & \\
    & & & & & & & & & & & & \node (p2-12) [terminal] {func}; & \node (p2-13) [terminal] {identifier}; & \node (p2-14) [nonterminal] {\nonTerminalSymbol{procedure\_formal\_arguments}{19}}; & \node (p2-15) [point] {}; & \node (p2-16) [point] {}; & & \node (p2-18) [point] {}; & \node (p2-19) [terminal] {\{}; & \node (p2-20) [nonterminal] {\nonTerminalSymbol{instructionList}{36}}; & \node (p2-21) [terminal] {\}}; & \\
    & & & & & & & & & & & & \node (p1-12) [nonterminal] {\nonTerminalSymbol{private\_struct\_property\_declaration}{5}}; & \\
    \node (P0start) [firstPoint] {}; & & \node (p0-2) [terminal] {task}; & \node (p0-3) [terminal] {identifier}; & \node (p0-4) [terminal] {priority}; & \node (p0-5) [terminal] {integer}; & \node (p0-6) [terminal] {stackSize}; & \node (p0-7) [terminal] {integer}; & \node (p0-8) [terminal] {\{}; & \node (p0-9) [point] {}; & \node (p0-10) [point] {}; & \node (p0-11) [point] {}; & & & & & & & & & & & & \node (p0-23) [terminal] {\}}; & \node (p0-24) [lastPoint] {}; & \\
  };
  \draw[->] (P0start) -- (p0-2) ;
  \draw[->] (p0-2) -- (p0-3) ;
  \draw[->] (p0-3) -- (p0-4) ;
  \draw[->] (p0-4) -- (p0-5) ;
  \draw[->] (p0-5) -- (p0-6) ;
  \draw[->] (p0-6) -- (p0-7) ;
  \draw[->] (p0-7) -- (p0-8) ;
  \draw (p0-8) -- (p0-10) ;
  \draw[->] (p0-11) |- (p1-12) ;
  \draw[->] (p0-11) |- (p2-12) ;
  \draw[->] (p2-12) -- (p2-13) ;
  \draw[->] (p2-13) -- (p2-14) ;
  \draw (p2-14) -- (p2-16) ;
  \draw[->] (p2-15) |- (p3-16) ;
  \draw[->] (p3-16) -- (p3-17) ;
  \draw (p2-16) -- (p2-18) ;
  \draw[->] (p3-17) -| (p2-18) ;
  \draw[->] (p2-18) -- (p2-19) ;
  \draw[->] (p2-19) -- (p2-20) ;
  \draw[->] (p2-20) -- (p2-21) ;
  \draw[->] (p0-11) |- (p4-12) ;
  \draw[->] (p4-12) -- (p4-13) ;
  \draw[->] (p4-13) -- (p4-14) ;
  \draw[->] (p4-14) -- (p4-15) ;
  \draw[->] (p4-15) -- (p4-16) ;
  \draw[->] (p0-11) |- (p5-12) ;
  \draw[->] (p5-12) -- (p5-13) ;
  \draw[->] (p5-13) -- (p5-14) ;
  \draw[->] (p5-14) -- (p5-15) ;
  \draw[->] (p0-11) |- (p6-12) ;
  \draw[->] (p0-11) |- (p7-12) ;
  \draw[->] (p8-22) -| (p0-9) ;
  \draw[->] (p1-12) -| (p8-22) ;
  \draw[->] (p2-21) -| (p8-22) ;
  \draw[->] (p4-16) -| (p8-22) ;
  \draw[->] (p5-15) -| (p8-22) ;
  \draw[->] (p6-12) -| (p8-22) ;
  \draw[->] (p7-12) -| (p8-22) ;
  \draw[->] (p0-10) -- (p0-23) ;
  \draw[->] (p0-23) -- (p0-24) ;
\end{tikzpicture}

\ruleSubsection{plm\_syntax}{panic}{20}

\begin{tikzpicture}
  \matrix[column sep=\ruleMatrixColumnSeparation, row sep=\ruleMatrixRowSeparation] {
    \node (P0start) [firstPoint] {}; & & \node (p5-2) [terminal] {panic}; & \\
    & & \node (p4-2) [terminal] {setup}; & \\
    & & \node (p3-2) [terminal] {integer}; & \\
    & & \node (p2-2) [terminal] {\{}; & \\
    & & \node (p1-2) [nonterminal] {\nonTerminalSymbol{instructionList}{36}}; & \\
    & & \node (p0-2) [terminal] {\}}; & \node (p0-3) [lastPoint] {}; & \\
  };
  \draw[->] (P0start) -- (p5-2) ;
  \draw[->] (p5-2) -- (p4-2) ;
  \draw[->] (p4-2) -- (p3-2) ;
  \draw[->] (p3-2) -- (p2-2) ;
  \draw[->] (p2-2) -- (p1-2) ;
  \draw[->] (p1-2) -- (p0-2) ;
  \draw[->] (p0-2) -- (p0-3) ;
\end{tikzpicture}

\ruleSubsection{plm\_syntax}{panic}{38}

\begin{tikzpicture}
  \matrix[column sep=\ruleMatrixColumnSeparation, row sep=\ruleMatrixRowSeparation] {
    \node (P0start) [firstPoint] {}; & & \node (p5-2) [terminal] {panic}; & \\
    & & \node (p4-2) [terminal] {loop}; & \\
    & & \node (p3-2) [terminal] {integer}; & \\
    & & \node (p2-2) [terminal] {\{}; & \\
    & & \node (p1-2) [nonterminal] {\nonTerminalSymbol{instructionList}{36}}; & \\
    & & \node (p0-2) [terminal] {\}}; & \node (p0-3) [lastPoint] {}; & \\
  };
  \draw[->] (P0start) -- (p5-2) ;
  \draw[->] (p5-2) -- (p4-2) ;
  \draw[->] (p4-2) -- (p3-2) ;
  \draw[->] (p3-2) -- (p2-2) ;
  \draw[->] (p2-2) -- (p1-2) ;
  \draw[->] (p1-2) -- (p0-2) ;
  \draw[->] (p0-2) -- (p0-3) ;
\end{tikzpicture}

\ruleSubsection{plm\_syntax}{declaration-boot}{19}

\begin{tikzpicture}
  \matrix[column sep=\ruleMatrixColumnSeparation, row sep=\ruleMatrixRowSeparation] {
    \node (P0start) [firstPoint] {}; & & \node (p4-2) [terminal] {boot}; & \\
    & & \node (p3-2) [terminal] {integer}; & \\
    & & \node (p2-2) [terminal] {\{}; & \\
    & & \node (p1-2) [nonterminal] {\nonTerminalSymbol{instructionList}{36}}; & \\
    & & \node (p0-2) [terminal] {\}}; & \node (p0-3) [lastPoint] {}; & \\
  };
  \draw[->] (P0start) -- (p4-2) ;
  \draw[->] (p4-2) -- (p3-2) ;
  \draw[->] (p3-2) -- (p2-2) ;
  \draw[->] (p2-2) -- (p1-2) ;
  \draw[->] (p1-2) -- (p0-2) ;
  \draw[->] (p0-2) -- (p0-3) ;
\end{tikzpicture}

\ruleSubsection{plm\_syntax}{declaration-init}{26}

\begin{tikzpicture}
  \matrix[column sep=\ruleMatrixColumnSeparation, row sep=\ruleMatrixRowSeparation] {
    \node (P0start) [firstPoint] {}; & & \node (p0-2) [nonterminal] {\nonTerminalSymbol{declaration\_init}{15}}; & \node (p0-3) [lastPoint] {}; & \\
  };
  \draw[->] (P0start) -- (p0-2) ;
  \draw[->] (p0-2) -- (p0-3) ;
\end{tikzpicture}

\ruleSubsection{plm\_syntax}{declaration-required-proc}{21}

\begin{tikzpicture}
  \matrix[column sep=\ruleMatrixColumnSeparation, row sep=\ruleMatrixRowSeparation] {
    \node (P0start) [firstPoint] {}; & & \node (p0-2) [terminal] {required}; & \node (p0-3) [nonterminal] {\nonTerminalSymbol{function\_header}{18}}; & \node (p0-4) [lastPoint] {}; & \\
  };
  \draw[->] (P0start) -- (p0-2) ;
  \draw[->] (p0-2) -- (p0-3) ;
  \draw[->] (p0-3) -- (p0-4) ;
\end{tikzpicture}

\ruleSubsection{plm\_syntax}{declaration-extern-proc}{22}

\begin{tikzpicture}
  \matrix[column sep=\ruleMatrixColumnSeparation, row sep=\ruleMatrixRowSeparation] {
    & & & & & \node (p1-5) [terminal] {->}; & \node (p1-6) [nonterminal] {\nonTerminalSymbol{type\_definition}{3}}; & \\
    \node (P0start) [firstPoint] {}; & & \node (p0-2) [terminal] {extern}; & \node (p0-3) [nonterminal] {\nonTerminalSymbol{function\_header}{18}}; & \node (p0-4) [point] {}; & \node (p0-5) [point] {}; & & \node (p0-7) [point] {}; & \node (p0-8) [terminal] {:}; & \node (p0-9) [terminal] {"string"}; & \node (p0-10) [lastPoint] {}; & \\
  };
  \draw[->] (P0start) -- (p0-2) ;
  \draw[->] (p0-2) -- (p0-3) ;
  \draw (p0-3) -- (p0-5) ;
  \draw[->] (p0-4) |- (p1-5) ;
  \draw[->] (p1-5) -- (p1-6) ;
  \draw (p0-5) -- (p0-7) ;
  \draw[->] (p1-6) -| (p0-7) ;
  \draw[->] (p0-7) -- (p0-8) ;
  \draw[->] (p0-8) -- (p0-9) ;
  \draw[->] (p0-9) -- (p0-10) ;
\end{tikzpicture}

\ruleSubsection{plm\_syntax}{target-generation}{9}

\begin{tikzpicture}
  \matrix[column sep=\ruleMatrixColumnSeparation, row sep=\ruleMatrixRowSeparation] {
    \node (P0start) [firstPoint] {}; & & \node (p0-2) [terminal] {target}; & \node (p0-3) [terminal] {"string"}; & \node (p0-4) [lastPoint] {}; & \\
  };
  \draw[->] (P0start) -- (p0-2) ;
  \draw[->] (p0-2) -- (p0-3) ;
  \draw[->] (p0-3) -- (p0-4) ;
\end{tikzpicture}

\ruleSubsection{plm\_syntax}{declaration-check-target}{9}

\begin{tikzpicture}
  \matrix[column sep=\ruleMatrixColumnSeparation, row sep=\ruleMatrixRowSeparation] {
    & & & & & & & & \node (p2-8) [point] {}; & \\
    & & & & & & & \node (p1-7) [terminal] {,}; & \\
    \node (P0start) [firstPoint] {}; & & \node (p0-2) [terminal] {check}; & \node (p0-3) [terminal] {target}; & \node (p0-4) [point] {}; & \node (p0-5) [terminal] {"string"}; & \node (p0-6) [point] {}; & & & \node (p0-9) [lastPoint] {}; & \\
  };
  \draw[->] (P0start) -- (p0-2) ;
  \draw[->] (p0-2) -- (p0-3) ;
  \draw[->] (p0-3) -- (p0-5) ;
  \draw[->] (p0-6) |- (p1-7) ;
  \draw[->] (p2-8) -| (p0-4) ;
  \draw[->] (p1-7) -| (p2-8) ;
  \draw[->] (p0-5) -- (p0-9) ;
\end{tikzpicture}

\nonTerminalSection{declaration\_init}{15}

\ruleSubsection{plm\_syntax}{declaration-init}{33}

\begin{tikzpicture}
  \matrix[column sep=\ruleMatrixColumnSeparation, row sep=\ruleMatrixRowSeparation] {
    \node (P0start) [firstPoint] {}; & & \node (p4-2) [terminal] {init}; & \\
    & & \node (p3-2) [terminal] {integer}; & \\
    & & \node (p2-2) [terminal] {\{}; & \\
    & & \node (p1-2) [nonterminal] {\nonTerminalSymbol{instructionList}{36}}; & \\
    & & \node (p0-2) [terminal] {\}}; & \node (p0-3) [lastPoint] {}; & \\
  };
  \draw[->] (P0start) -- (p4-2) ;
  \draw[->] (p4-2) -- (p3-2) ;
  \draw[->] (p3-2) -- (p2-2) ;
  \draw[->] (p2-2) -- (p1-2) ;
  \draw[->] (p1-2) -- (p0-2) ;
  \draw[->] (p0-2) -- (p0-3) ;
\end{tikzpicture}

\nonTerminalSection{effective\_parameters}{43}

\ruleSubsection{plm\_syntax}{effective-parameters}{40}

\begin{tikzpicture}
  \matrix[column sep=\ruleMatrixColumnSeparation, row sep=\ruleMatrixRowSeparation] {
    & & & & & & & & & & & & & & \node (p7-14) [point] {}; & \\
    & & & & & & & & \node (p6-8) [terminal] {let}; & & & & \node (p6-12) [nonterminal] {\nonTerminalSymbol{type\_definition}{3}}; & \\
    & & & & & & \node (p5-6) [terminal] {?selector:}; & \node (p5-7) [point] {}; & \node (p5-8) [terminal] {var}; & \node (p5-9) [point] {}; & \node (p5-10) [terminal] {identifier}; & \node (p5-11) [point] {}; & \node (p5-12) [point] {}; & \node (p5-13) [point] {}; & \\
    & & & & & & \node (p4-6) [terminal] {?selector:}; & \node (p4-7) [terminal] {identifier}; & \\
    & & & & & & \node (p3-6) [terminal] {!?selector:}; & \node (p3-7) [terminal] {self}; & \node (p3-8) [terminal] {.}; & \node (p3-9) [terminal] {identifier}; & \\
    & & & & & & \node (p2-6) [terminal] {!?selector:}; & \node (p2-7) [terminal] {identifier}; & \\
    & & & & & & \node (p1-6) [terminal] {!selector:}; & \node (p1-7) [nonterminal] {\nonTerminalSymbol{expression}{22}}; & \\
    \node (P0start) [firstPoint] {}; & & \node (p0-2) [terminal] {(}; & \node (p0-3) [point] {}; & \node (p0-4) [point] {}; & \node (p0-5) [point] {}; & & & & & & & & & & \node (p0-15) [terminal] {)}; & \node (p0-16) [lastPoint] {}; & \\
  };
  \draw[->] (P0start) -- (p0-2) ;
  \draw (p0-2) -- (p0-4) ;
  \draw[->] (p0-5) |- (p1-6) ;
  \draw[->] (p1-6) -- (p1-7) ;
  \draw[->] (p0-5) |- (p2-6) ;
  \draw[->] (p2-6) -- (p2-7) ;
  \draw[->] (p0-5) |- (p3-6) ;
  \draw[->] (p3-6) -- (p3-7) ;
  \draw[->] (p3-7) -- (p3-8) ;
  \draw[->] (p3-8) -- (p3-9) ;
  \draw[->] (p0-5) |- (p4-6) ;
  \draw[->] (p4-6) -- (p4-7) ;
  \draw[->] (p0-5) |- (p5-6) ;
  \draw[->] (p5-6) -- (p5-8) ;
  \draw[->] (p5-7) |- (p6-8) ;
  \draw (p5-8) -- (p5-9) ;
  \draw[->] (p6-8) -| (p5-9) ;
  \draw[->] (p5-9) -- (p5-10) ;
  \draw (p5-10) -- (p5-12) ;
  \draw[->] (p5-11) |- (p6-12) ;
  \draw (p5-12) -- (p5-13) ;
  \draw[->] (p6-12) -| (p5-13) ;
  \draw[->] (p7-14) -| (p0-3) ;
  \draw[->] (p1-7) -| (p7-14) ;
  \draw[->] (p2-7) -| (p7-14) ;
  \draw[->] (p3-9) -| (p7-14) ;
  \draw[->] (p4-7) -| (p7-14) ;
  \draw[->] (p5-13) -| (p7-14) ;
  \draw[->] (p0-4) -- (p0-15) ;
  \draw[->] (p0-15) -- (p0-16) ;
\end{tikzpicture}

\nonTerminalSection{expression}{22}

\ruleSubsection{plm\_syntax}{expression-operator-priority}{16}

\begin{tikzpicture}
  \matrix[column sep=\ruleMatrixColumnSeparation, row sep=\ruleMatrixRowSeparation] {
    & & & & & & & & \node (p2-8) [point] {}; & \\
    & & & & & & \node (p1-6) [terminal] {or}; & \node (p1-7) [nonterminal] {\nonTerminalSymbol{expression\_logical\_xor}{23}}; & \\
    \node (P0start) [firstPoint] {}; & & \node (p0-2) [nonterminal] {\nonTerminalSymbol{expression\_logical\_xor}{23}}; & \node (p0-3) [point] {}; & \node (p0-4) [point] {}; & \node (p0-5) [point] {}; & & & & \node (p0-9) [lastPoint] {}; & \\
  };
  \draw[->] (P0start) -- (p0-2) ;
  \draw (p0-2) -- (p0-4) ;
  \draw[->] (p0-5) |- (p1-6) ;
  \draw[->] (p1-6) -- (p1-7) ;
  \draw[->] (p2-8) -| (p0-3) ;
  \draw[->] (p1-7) -| (p2-8) ;
  \draw[->] (p0-4) -- (p0-9) ;
\end{tikzpicture}

\nonTerminalSection{expression\_access\_list}{35}

\ruleSubsection{plm\_syntax}{expression-primary}{49}

\begin{tikzpicture}
  \matrix[column sep=\ruleMatrixColumnSeparation, row sep=\ruleMatrixRowSeparation] {
    & & & & & & & & \node (p4-8) [point] {}; & \\
    & & & & & \node (p3-5) [terminal] {.}; & \node (p3-6) [terminal] {identifier}; & \node (p3-7) [nonterminal] {\nonTerminalSymbol{effective\_parameters}{43}}; & \\
    & & & & & \node (p2-5) [terminal] {[}; & \node (p2-6) [nonterminal] {\nonTerminalSymbol{expression}{22}}; & \node (p2-7) [terminal] {]}; & \\
    & & & & & \node (p1-5) [terminal] {.}; & \node (p1-6) [terminal] {identifier}; & \\
    \node (P0start) [firstPoint] {}; & & \node (p0-2) [point] {}; & \node (p0-3) [point] {}; & \node (p0-4) [point] {}; & & & & & \node (p0-9) [lastPoint] {}; & \\
  };
  \draw (P0start) -- (p0-3) ;
  \draw[->] (p0-4) |- (p1-5) ;
  \draw[->] (p1-5) -- (p1-6) ;
  \draw[->] (p0-4) |- (p2-5) ;
  \draw[->] (p2-5) -- (p2-6) ;
  \draw[->] (p2-6) -- (p2-7) ;
  \draw[->] (p0-4) |- (p3-5) ;
  \draw[->] (p3-5) -- (p3-6) ;
  \draw[->] (p3-6) -- (p3-7) ;
  \draw[->] (p4-8) -| (p0-2) ;
  \draw[->] (p1-6) -| (p4-8) ;
  \draw[->] (p2-7) -| (p4-8) ;
  \draw[->] (p3-7) -| (p4-8) ;
  \draw[->] (p0-3) -- (p0-9) ;
\end{tikzpicture}

\nonTerminalSection{expression\_addition}{31}

\ruleSubsection{plm\_syntax}{expression-operator-priority}{232}

\begin{tikzpicture}
  \matrix[column sep=\ruleMatrixColumnSeparation, row sep=\ruleMatrixRowSeparation] {
    & & & & & & & & \node (p5-8) [point] {}; & \\
    & & & & & & \node (p4-6) [terminal] {-\verb=%=}; & \node (p4-7) [nonterminal] {\nonTerminalSymbol{expression\_product}{32}}; & \\
    & & & & & & \node (p3-6) [terminal] {-}; & \node (p3-7) [nonterminal] {\nonTerminalSymbol{expression\_product}{32}}; & \\
    & & & & & & \node (p2-6) [terminal] {+\verb=%=}; & \node (p2-7) [nonterminal] {\nonTerminalSymbol{expression\_product}{32}}; & \\
    & & & & & & \node (p1-6) [terminal] {+}; & \node (p1-7) [nonterminal] {\nonTerminalSymbol{expression\_product}{32}}; & \\
    \node (P0start) [firstPoint] {}; & & \node (p0-2) [nonterminal] {\nonTerminalSymbol{expression\_product}{32}}; & \node (p0-3) [point] {}; & \node (p0-4) [point] {}; & \node (p0-5) [point] {}; & & & & \node (p0-9) [lastPoint] {}; & \\
  };
  \draw[->] (P0start) -- (p0-2) ;
  \draw (p0-2) -- (p0-4) ;
  \draw[->] (p0-5) |- (p1-6) ;
  \draw[->] (p1-6) -- (p1-7) ;
  \draw[->] (p0-5) |- (p2-6) ;
  \draw[->] (p2-6) -- (p2-7) ;
  \draw[->] (p0-5) |- (p3-6) ;
  \draw[->] (p3-6) -- (p3-7) ;
  \draw[->] (p0-5) |- (p4-6) ;
  \draw[->] (p4-6) -- (p4-7) ;
  \draw[->] (p5-8) -| (p0-3) ;
  \draw[->] (p1-7) -| (p5-8) ;
  \draw[->] (p2-7) -| (p5-8) ;
  \draw[->] (p3-7) -| (p5-8) ;
  \draw[->] (p4-7) -| (p5-8) ;
  \draw[->] (p0-4) -- (p0-9) ;
\end{tikzpicture}

\nonTerminalSection{expression\_bitwise\_and}{27}

\ruleSubsection{plm\_syntax}{expression-operator-priority}{110}

\begin{tikzpicture}
  \matrix[column sep=\ruleMatrixColumnSeparation, row sep=\ruleMatrixRowSeparation] {
    & & & & & & & & \node (p2-8) [point] {}; & \\
    & & & & & & \node (p1-6) [terminal] {\&}; & \node (p1-7) [nonterminal] {\nonTerminalSymbol{expression\_equality}{28}}; & \\
    \node (P0start) [firstPoint] {}; & & \node (p0-2) [nonterminal] {\nonTerminalSymbol{expression\_equality}{28}}; & \node (p0-3) [point] {}; & \node (p0-4) [point] {}; & \node (p0-5) [point] {}; & & & & \node (p0-9) [lastPoint] {}; & \\
  };
  \draw[->] (P0start) -- (p0-2) ;
  \draw (p0-2) -- (p0-4) ;
  \draw[->] (p0-5) |- (p1-6) ;
  \draw[->] (p1-6) -- (p1-7) ;
  \draw[->] (p2-8) -| (p0-3) ;
  \draw[->] (p1-7) -| (p2-8) ;
  \draw[->] (p0-4) -- (p0-9) ;
\end{tikzpicture}

\nonTerminalSection{expression\_bitwise\_or}{25}

\ruleSubsection{plm\_syntax}{expression-operator-priority}{74}

\begin{tikzpicture}
  \matrix[column sep=\ruleMatrixColumnSeparation, row sep=\ruleMatrixRowSeparation] {
    & & & & & & & & \node (p2-8) [point] {}; & \\
    & & & & & & \node (p1-6) [terminal] {|}; & \node (p1-7) [nonterminal] {\nonTerminalSymbol{expression\_bitwise\_xor}{26}}; & \\
    \node (P0start) [firstPoint] {}; & & \node (p0-2) [nonterminal] {\nonTerminalSymbol{expression\_bitwise\_xor}{26}}; & \node (p0-3) [point] {}; & \node (p0-4) [point] {}; & \node (p0-5) [point] {}; & & & & \node (p0-9) [lastPoint] {}; & \\
  };
  \draw[->] (P0start) -- (p0-2) ;
  \draw (p0-2) -- (p0-4) ;
  \draw[->] (p0-5) |- (p1-6) ;
  \draw[->] (p1-6) -- (p1-7) ;
  \draw[->] (p2-8) -| (p0-3) ;
  \draw[->] (p1-7) -| (p2-8) ;
  \draw[->] (p0-4) -- (p0-9) ;
\end{tikzpicture}

\nonTerminalSection{expression\_bitwise\_xor}{26}

\ruleSubsection{plm\_syntax}{expression-operator-priority}{92}

\begin{tikzpicture}
  \matrix[column sep=\ruleMatrixColumnSeparation, row sep=\ruleMatrixRowSeparation] {
    & & & & & & & & \node (p2-8) [point] {}; & \\
    & & & & & & \node (p1-6) [terminal] {\verb=^=}; & \node (p1-7) [nonterminal] {\nonTerminalSymbol{expression\_bitwise\_and}{27}}; & \\
    \node (P0start) [firstPoint] {}; & & \node (p0-2) [nonterminal] {\nonTerminalSymbol{expression\_bitwise\_and}{27}}; & \node (p0-3) [point] {}; & \node (p0-4) [point] {}; & \node (p0-5) [point] {}; & & & & \node (p0-9) [lastPoint] {}; & \\
  };
  \draw[->] (P0start) -- (p0-2) ;
  \draw (p0-2) -- (p0-4) ;
  \draw[->] (p0-5) |- (p1-6) ;
  \draw[->] (p1-6) -- (p1-7) ;
  \draw[->] (p2-8) -| (p0-3) ;
  \draw[->] (p1-7) -| (p2-8) ;
  \draw[->] (p0-4) -- (p0-9) ;
\end{tikzpicture}

\nonTerminalSection{expression\_comparison}{29}

\ruleSubsection{plm\_syntax}{expression-operator-priority}{156}

\begin{tikzpicture}
  \matrix[column sep=\ruleMatrixColumnSeparation, row sep=\ruleMatrixRowSeparation] {
    & & & & \node (p4-4) [terminal] {>}; & \node (p4-5) [nonterminal] {\nonTerminalSymbol{expression\_shift}{30}}; & \\
    & & & & \node (p3-4) [terminal] {<}; & \node (p3-5) [nonterminal] {\nonTerminalSymbol{expression\_shift}{30}}; & \\
    & & & & \node (p2-4) [terminal] {≥}; & \node (p2-5) [nonterminal] {\nonTerminalSymbol{expression\_shift}{30}}; & \\
    & & & & \node (p1-4) [terminal] {≤}; & \node (p1-5) [nonterminal] {\nonTerminalSymbol{expression\_shift}{30}}; & \\
    \node (P0start) [firstPoint] {}; & & \node (p0-2) [nonterminal] {\nonTerminalSymbol{expression\_shift}{30}}; & \node (p0-3) [point] {}; & \node (p0-4) [point] {}; & & \node (p0-6) [point] {}; & \node (p0-7) [lastPoint] {}; & \\
  };
  \draw[->] (P0start) -- (p0-2) ;
  \draw (p0-2) -- (p0-4) ;
  \draw[->] (p0-3) |- (p1-4) ;
  \draw[->] (p1-4) -- (p1-5) ;
  \draw[->] (p0-3) |- (p2-4) ;
  \draw[->] (p2-4) -- (p2-5) ;
  \draw[->] (p0-3) |- (p3-4) ;
  \draw[->] (p3-4) -- (p3-5) ;
  \draw[->] (p0-3) |- (p4-4) ;
  \draw[->] (p4-4) -- (p4-5) ;
  \draw (p0-4) -- (p0-6) ;
  \draw[->] (p1-5) -| (p0-6) ;
  \draw[->] (p2-5) -| (p0-6) ;
  \draw[->] (p3-5) -| (p0-6) ;
  \draw[->] (p4-5) -| (p0-6) ;
  \draw[->] (p0-6) -- (p0-7) ;
\end{tikzpicture}

\nonTerminalSection{expression\_equality}{28}

\ruleSubsection{plm\_syntax}{expression-operator-priority}{128}

\begin{tikzpicture}
  \matrix[column sep=\ruleMatrixColumnSeparation, row sep=\ruleMatrixRowSeparation] {
    & & & & \node (p2-4) [terminal] {≠}; & \node (p2-5) [nonterminal] {\nonTerminalSymbol{expression\_comparison}{29}}; & \\
    & & & & \node (p1-4) [terminal] {==}; & \node (p1-5) [nonterminal] {\nonTerminalSymbol{expression\_comparison}{29}}; & \\
    \node (P0start) [firstPoint] {}; & & \node (p0-2) [nonterminal] {\nonTerminalSymbol{expression\_comparison}{29}}; & \node (p0-3) [point] {}; & \node (p0-4) [point] {}; & & \node (p0-6) [point] {}; & \node (p0-7) [lastPoint] {}; & \\
  };
  \draw[->] (P0start) -- (p0-2) ;
  \draw (p0-2) -- (p0-4) ;
  \draw[->] (p0-3) |- (p1-4) ;
  \draw[->] (p1-4) -- (p1-5) ;
  \draw[->] (p0-3) |- (p2-4) ;
  \draw[->] (p2-4) -- (p2-5) ;
  \draw (p0-4) -- (p0-6) ;
  \draw[->] (p1-5) -| (p0-6) ;
  \draw[->] (p2-5) -| (p0-6) ;
  \draw[->] (p0-6) -- (p0-7) ;
\end{tikzpicture}

\nonTerminalSection{expression\_if}{34}

\ruleSubsection{plm\_syntax}{expression-if}{29}

\begin{tikzpicture}
  \matrix[column sep=\ruleMatrixColumnSeparation, row sep=\ruleMatrixRowSeparation] {
    & & & & & & & & & \node (p1-9) [nonterminal] {\nonTerminalSymbol{expression\_if}{34}}; & \\
    \node (P0start) [firstPoint] {}; & & \node (p0-2) [terminal] {if}; & \node (p0-3) [nonterminal] {\nonTerminalSymbol{expression}{22}}; & \node (p0-4) [terminal] {\{}; & \node (p0-5) [nonterminal] {\nonTerminalSymbol{expression}{22}}; & \node (p0-6) [terminal] {\}}; & \node (p0-7) [terminal] {else}; & \node (p0-8) [point] {}; & \node (p0-9) [terminal] {\{}; & \node (p0-10) [nonterminal] {\nonTerminalSymbol{expression}{22}}; & \node (p0-11) [terminal] {\}}; & \node (p0-12) [point] {}; & \node (p0-13) [lastPoint] {}; & \\
  };
  \draw[->] (P0start) -- (p0-2) ;
  \draw[->] (p0-2) -- (p0-3) ;
  \draw[->] (p0-3) -- (p0-4) ;
  \draw[->] (p0-4) -- (p0-5) ;
  \draw[->] (p0-5) -- (p0-6) ;
  \draw[->] (p0-6) -- (p0-7) ;
  \draw[->] (p0-7) -- (p0-9) ;
  \draw[->] (p0-9) -- (p0-10) ;
  \draw[->] (p0-10) -- (p0-11) ;
  \draw[->] (p0-8) |- (p1-9) ;
  \draw (p0-11) -- (p0-12) ;
  \draw[->] (p1-9) -| (p0-12) ;
  \draw[->] (p0-12) -- (p0-13) ;
\end{tikzpicture}

\nonTerminalSection{expression\_logical\_and}{24}

\ruleSubsection{plm\_syntax}{expression-operator-priority}{57}

\begin{tikzpicture}
  \matrix[column sep=\ruleMatrixColumnSeparation, row sep=\ruleMatrixRowSeparation] {
    & & & & & & & & \node (p2-8) [point] {}; & \\
    & & & & & & \node (p1-6) [terminal] {and}; & \node (p1-7) [nonterminal] {\nonTerminalSymbol{expression\_bitwise\_or}{25}}; & \\
    \node (P0start) [firstPoint] {}; & & \node (p0-2) [nonterminal] {\nonTerminalSymbol{expression\_bitwise\_or}{25}}; & \node (p0-3) [point] {}; & \node (p0-4) [point] {}; & \node (p0-5) [point] {}; & & & & \node (p0-9) [lastPoint] {}; & \\
  };
  \draw[->] (P0start) -- (p0-2) ;
  \draw (p0-2) -- (p0-4) ;
  \draw[->] (p0-5) |- (p1-6) ;
  \draw[->] (p1-6) -- (p1-7) ;
  \draw[->] (p2-8) -| (p0-3) ;
  \draw[->] (p1-7) -| (p2-8) ;
  \draw[->] (p0-4) -- (p0-9) ;
\end{tikzpicture}

\nonTerminalSection{expression\_logical\_xor}{23}

\ruleSubsection{plm\_syntax}{expression-operator-priority}{39}

\begin{tikzpicture}
  \matrix[column sep=\ruleMatrixColumnSeparation, row sep=\ruleMatrixRowSeparation] {
    & & & & & & & & \node (p2-8) [point] {}; & \\
    & & & & & & \node (p1-6) [terminal] {xor}; & \node (p1-7) [nonterminal] {\nonTerminalSymbol{expression\_logical\_and}{24}}; & \\
    \node (P0start) [firstPoint] {}; & & \node (p0-2) [nonterminal] {\nonTerminalSymbol{expression\_logical\_and}{24}}; & \node (p0-3) [point] {}; & \node (p0-4) [point] {}; & \node (p0-5) [point] {}; & & & & \node (p0-9) [lastPoint] {}; & \\
  };
  \draw[->] (P0start) -- (p0-2) ;
  \draw (p0-2) -- (p0-4) ;
  \draw[->] (p0-5) |- (p1-6) ;
  \draw[->] (p1-6) -- (p1-7) ;
  \draw[->] (p2-8) -| (p0-3) ;
  \draw[->] (p1-7) -| (p2-8) ;
  \draw[->] (p0-4) -- (p0-9) ;
\end{tikzpicture}

\nonTerminalSection{expression\_product}{32}

\ruleSubsection{plm\_syntax}{expression-operator-priority}{280}

\begin{tikzpicture}
  \matrix[column sep=\ruleMatrixColumnSeparation, row sep=\ruleMatrixRowSeparation] {
    & & & & & & & & \node (p7-8) [point] {}; & \\
    & & & & & & \node (p6-6) [terminal] {!/}; & \node (p6-7) [nonterminal] {\nonTerminalSymbol{primary}{33}}; & \\
    & & & & & & \node (p5-6) [terminal] {/}; & \node (p5-7) [nonterminal] {\nonTerminalSymbol{primary}{33}}; & \\
    & & & & & & \node (p4-6) [terminal] {!\verb=%=}; & \node (p4-7) [nonterminal] {\nonTerminalSymbol{primary}{33}}; & \\
    & & & & & & \node (p3-6) [terminal] {\verb=%=}; & \node (p3-7) [nonterminal] {\nonTerminalSymbol{primary}{33}}; & \\
    & & & & & & \node (p2-6) [terminal] {*\verb=%=}; & \node (p2-7) [nonterminal] {\nonTerminalSymbol{primary}{33}}; & \\
    & & & & & & \node (p1-6) [terminal] {*}; & \node (p1-7) [nonterminal] {\nonTerminalSymbol{primary}{33}}; & \\
    \node (P0start) [firstPoint] {}; & & \node (p0-2) [nonterminal] {\nonTerminalSymbol{primary}{33}}; & \node (p0-3) [point] {}; & \node (p0-4) [point] {}; & \node (p0-5) [point] {}; & & & & \node (p0-9) [lastPoint] {}; & \\
  };
  \draw[->] (P0start) -- (p0-2) ;
  \draw (p0-2) -- (p0-4) ;
  \draw[->] (p0-5) |- (p1-6) ;
  \draw[->] (p1-6) -- (p1-7) ;
  \draw[->] (p0-5) |- (p2-6) ;
  \draw[->] (p2-6) -- (p2-7) ;
  \draw[->] (p0-5) |- (p3-6) ;
  \draw[->] (p3-6) -- (p3-7) ;
  \draw[->] (p0-5) |- (p4-6) ;
  \draw[->] (p4-6) -- (p4-7) ;
  \draw[->] (p0-5) |- (p5-6) ;
  \draw[->] (p5-6) -- (p5-7) ;
  \draw[->] (p0-5) |- (p6-6) ;
  \draw[->] (p6-6) -- (p6-7) ;
  \draw[->] (p7-8) -| (p0-3) ;
  \draw[->] (p1-7) -| (p7-8) ;
  \draw[->] (p2-7) -| (p7-8) ;
  \draw[->] (p3-7) -| (p7-8) ;
  \draw[->] (p4-7) -| (p7-8) ;
  \draw[->] (p5-7) -| (p7-8) ;
  \draw[->] (p6-7) -| (p7-8) ;
  \draw[->] (p0-4) -- (p0-9) ;
\end{tikzpicture}

\nonTerminalSection{expression\_shift}{30}

\ruleSubsection{plm\_syntax}{expression-operator-priority}{204}

\begin{tikzpicture}
  \matrix[column sep=\ruleMatrixColumnSeparation, row sep=\ruleMatrixRowSeparation] {
    & & & & & & & & \node (p3-8) [point] {}; & \\
    & & & & & & \node (p2-6) [terminal] {>>}; & \node (p2-7) [nonterminal] {\nonTerminalSymbol{expression\_addition}{31}}; & \\
    & & & & & & \node (p1-6) [terminal] {<<}; & \node (p1-7) [nonterminal] {\nonTerminalSymbol{expression\_addition}{31}}; & \\
    \node (P0start) [firstPoint] {}; & & \node (p0-2) [nonterminal] {\nonTerminalSymbol{expression\_addition}{31}}; & \node (p0-3) [point] {}; & \node (p0-4) [point] {}; & \node (p0-5) [point] {}; & & & & \node (p0-9) [lastPoint] {}; & \\
  };
  \draw[->] (P0start) -- (p0-2) ;
  \draw (p0-2) -- (p0-4) ;
  \draw[->] (p0-5) |- (p1-6) ;
  \draw[->] (p1-6) -- (p1-7) ;
  \draw[->] (p0-5) |- (p2-6) ;
  \draw[->] (p2-6) -- (p2-7) ;
  \draw[->] (p3-8) -| (p0-3) ;
  \draw[->] (p1-7) -| (p3-8) ;
  \draw[->] (p2-7) -| (p3-8) ;
  \draw[->] (p0-4) -- (p0-9) ;
\end{tikzpicture}

\nonTerminalSection{function}{16}

\ruleSubsection{plm\_syntax}{declaration-func}{64}

\begin{tikzpicture}
  \matrix[column sep=\ruleMatrixColumnSeparation, row sep=\ruleMatrixRowSeparation] {
    & & & \node (p1-3) [terminal] {public}; & & & & \node (p1-7) [terminal] {->}; & \node (p1-8) [nonterminal] {\nonTerminalSymbol{type\_definition}{3}}; & \\
    \node (P0start) [firstPoint] {}; & & \node (p0-2) [point] {}; & \node (p0-3) [point] {}; & \node (p0-4) [point] {}; & \node (p0-5) [nonterminal] {\nonTerminalSymbol{function\_header}{18}}; & \node (p0-6) [point] {}; & \node (p0-7) [point] {}; & & \node (p0-9) [point] {}; & \node (p0-10) [terminal] {\{}; & \node (p0-11) [nonterminal] {\nonTerminalSymbol{instructionList}{36}}; & \node (p0-12) [terminal] {\}}; & \node (p0-13) [lastPoint] {}; & \\
  };
  \draw (P0start) -- (p0-3) ;
  \draw[->] (p0-2) |- (p1-3) ;
  \draw (p0-3) -- (p0-4) ;
  \draw[->] (p1-3) -| (p0-4) ;
  \draw[->] (p0-4) -- (p0-5) ;
  \draw (p0-5) -- (p0-7) ;
  \draw[->] (p0-6) |- (p1-7) ;
  \draw[->] (p1-7) -- (p1-8) ;
  \draw (p0-7) -- (p0-9) ;
  \draw[->] (p1-8) -| (p0-9) ;
  \draw[->] (p0-9) -- (p0-10) ;
  \draw[->] (p0-10) -- (p0-11) ;
  \draw[->] (p0-11) -- (p0-12) ;
  \draw[->] (p0-12) -- (p0-13) ;
\end{tikzpicture}

\nonTerminalSection{function\_header}{18}

\ruleSubsection{plm\_syntax}{declaration-func}{137}

\begin{tikzpicture}
  \matrix[column sep=\ruleMatrixColumnSeparation, row sep=\ruleMatrixRowSeparation] {
    & & & & & & & & & \node (p2-9) [point] {}; & \\
    & & & & & & & & \node (p1-8) [terminal] {@attribute}; & \\
    \node (P0start) [firstPoint] {}; & & \node (p0-2) [terminal] {func}; & \node (p0-3) [nonterminal] {\nonTerminalSymbol{mode}{17}}; & \node (p0-4) [terminal] {identifier}; & \node (p0-5) [point] {}; & \node (p0-6) [point] {}; & \node (p0-7) [point] {}; & & & \node (p0-10) [nonterminal] {\nonTerminalSymbol{procedure\_formal\_arguments}{19}}; & \node (p0-11) [lastPoint] {}; & \\
  };
  \draw[->] (P0start) -- (p0-2) ;
  \draw[->] (p0-2) -- (p0-3) ;
  \draw[->] (p0-3) -- (p0-4) ;
  \draw (p0-4) -- (p0-6) ;
  \draw[->] (p0-7) |- (p1-8) ;
  \draw[->] (p2-9) -| (p0-5) ;
  \draw[->] (p1-8) -| (p2-9) ;
  \draw[->] (p0-6) -- (p0-10) ;
  \draw[->] (p0-10) -- (p0-11) ;
\end{tikzpicture}

\nonTerminalSection{guard}{21}

\ruleSubsection{plm\_syntax}{declaration-guard}{30}

\begin{tikzpicture}
  \matrix[column sep=\ruleMatrixColumnSeparation, row sep=\ruleMatrixRowSeparation] {
    & & & & & & & & & & & \node (p2-11) [point] {}; & \\
    & & & \node (p1-3) [terminal] {public}; & & & & & & & \node (p1-10) [terminal] {@attribute}; & & & & \node (p1-14) [terminal] {:}; & \node (p1-15) [nonterminal] {\nonTerminalSymbol{procedure\_call}{41}}; & \\
    \node (P0start) [firstPoint] {}; & & \node (p0-2) [point] {}; & \node (p0-3) [point] {}; & \node (p0-4) [point] {}; & \node (p0-5) [terminal] {guard}; & \node (p0-6) [terminal] {identifier}; & \node (p0-7) [point] {}; & \node (p0-8) [point] {}; & \node (p0-9) [point] {}; & & & \node (p0-12) [nonterminal] {\nonTerminalSymbol{procedure\_formal\_arguments}{19}}; & \node (p0-13) [point] {}; & \node (p0-14) [point] {}; & & \node (p0-16) [point] {}; & \node (p0-17) [terminal] {\{}; & \node (p0-18) [nonterminal] {\nonTerminalSymbol{instructionList}{36}}; & \node (p0-19) [terminal] {\}}; & \node (p0-20) [lastPoint] {}; & \\
  };
  \draw (P0start) -- (p0-3) ;
  \draw[->] (p0-2) |- (p1-3) ;
  \draw (p0-3) -- (p0-4) ;
  \draw[->] (p1-3) -| (p0-4) ;
  \draw[->] (p0-4) -- (p0-5) ;
  \draw[->] (p0-5) -- (p0-6) ;
  \draw (p0-6) -- (p0-8) ;
  \draw[->] (p0-9) |- (p1-10) ;
  \draw[->] (p2-11) -| (p0-7) ;
  \draw[->] (p1-10) -| (p2-11) ;
  \draw[->] (p0-8) -- (p0-12) ;
  \draw (p0-12) -- (p0-14) ;
  \draw[->] (p0-13) |- (p1-14) ;
  \draw[->] (p1-14) -- (p1-15) ;
  \draw (p0-14) -- (p0-16) ;
  \draw[->] (p1-15) -| (p0-16) ;
  \draw[->] (p0-16) -- (p0-17) ;
  \draw[->] (p0-17) -- (p0-18) ;
  \draw[->] (p0-18) -- (p0-19) ;
  \draw[->] (p0-19) -- (p0-20) ;
\end{tikzpicture}

\nonTerminalSection{guarded\_command}{40}

\ruleSubsection{plm\_syntax}{instruction-sync}{42}

\begin{tikzpicture}
  \matrix[column sep=\ruleMatrixColumnSeparation, row sep=\ruleMatrixRowSeparation] {
    & & & & & & & & & & & & & & & & \node (p2-16) [point] {}; & \\
    & & & & \node (p1-4) [point] {}; & & & & & \node (p1-9) [terminal] {self}; & \node (p1-10) [terminal] {.}; & & & & & \node (p1-15) [terminal] {.}; & & & & \node (p1-19) [point] {}; & \\
    \node (P0start) [firstPoint] {}; & & \node (p0-2) [terminal] {on}; & \node (p0-3) [point] {}; & \node (p0-4) [terminal] {(}; & \node (p0-5) [nonterminal] {\nonTerminalSymbol{expression}{22}}; & \node (p0-6) [terminal] {)}; & \node (p0-7) [point] {}; & \node (p0-8) [point] {}; & \node (p0-9) [point] {}; & & \node (p0-11) [point] {}; & \node (p0-12) [point] {}; & \node (p0-13) [terminal] {identifier}; & \node (p0-14) [point] {}; & & & \node (p0-17) [nonterminal] {\nonTerminalSymbol{effective\_parameters}{43}}; & \node (p0-18) [point] {}; & \node (p0-19) [terminal] {exit}; & \node (p0-20) [point] {}; & \node (p0-21) [lastPoint] {}; & \\
  };
  \draw[->] (P0start) -- (p0-2) ;
  \draw[->] (p0-2) -- (p0-4) ;
  \draw[->] (p0-4) -- (p0-5) ;
  \draw[->] (p0-5) -- (p0-6) ;
  \draw (p0-3) |- (p1-4) ;
  \draw (p0-6) -- (p0-7) ;
  \draw[->] (p1-4) -| (p0-7) ;
  \draw (p0-7) -- (p0-9) ;
  \draw[->] (p0-8) |- (p1-9) ;
  \draw[->] (p1-9) -- (p1-10) ;
  \draw (p0-9) -- (p0-11) ;
  \draw[->] (p1-10) -| (p0-11) ;
  \draw[->] (p0-11) -- (p0-13) ;
  \draw[->] (p0-14) |- (p1-15) ;
  \draw[->] (p2-16) -| (p0-12) ;
  \draw[->] (p1-15) -| (p2-16) ;
  \draw[->] (p0-13) -- (p0-17) ;
  \draw[->] (p0-17) -- (p0-19) ;
  \draw (p0-18) |- (p1-19) ;
  \draw (p0-19) -- (p0-20) ;
  \draw[->] (p1-19) -| (p0-20) ;
  \draw[->] (p0-20) -- (p0-21) ;
\end{tikzpicture}

\ruleSubsection{plm\_syntax}{instruction-sync}{94}

\begin{tikzpicture}
  \matrix[column sep=\ruleMatrixColumnSeparation, row sep=\ruleMatrixRowSeparation] {
    & & & & & & & \node (p1-7) [point] {}; & \\
    \node (P0start) [firstPoint] {}; & & \node (p0-2) [terminal] {on}; & \node (p0-3) [terminal] {(}; & \node (p0-4) [nonterminal] {\nonTerminalSymbol{expression}{22}}; & \node (p0-5) [terminal] {)}; & \node (p0-6) [point] {}; & \node (p0-7) [terminal] {exit}; & \node (p0-8) [point] {}; & \node (p0-9) [lastPoint] {}; & \\
  };
  \draw[->] (P0start) -- (p0-2) ;
  \draw[->] (p0-2) -- (p0-3) ;
  \draw[->] (p0-3) -- (p0-4) ;
  \draw[->] (p0-4) -- (p0-5) ;
  \draw[->] (p0-5) -- (p0-7) ;
  \draw (p0-6) |- (p1-7) ;
  \draw (p0-7) -- (p0-8) ;
  \draw[->] (p1-7) -| (p0-8) ;
  \draw[->] (p0-8) -- (p0-9) ;
\end{tikzpicture}

\nonTerminalSection{if\_instruction}{39}

\ruleSubsection{plm\_syntax}{instruction-if}{31}

\begin{tikzpicture}
  \matrix[column sep=\ruleMatrixColumnSeparation, row sep=\ruleMatrixRowSeparation] {
    & & & & & & & & & & & \node (p2-11) [terminal] {else}; & \node (p2-12) [nonterminal] {\nonTerminalSymbol{if\_instruction}{39}}; & \\
    & & & & \node (p1-4) [point] {}; & & & & & & & \node (p1-11) [terminal] {else}; & \node (p1-12) [terminal] {\{}; & \node (p1-13) [nonterminal] {\nonTerminalSymbol{instructionList}{36}}; & \node (p1-14) [terminal] {\}}; & \\
    \node (P0start) [firstPoint] {}; & & \node (p0-2) [terminal] {if}; & \node (p0-3) [point] {}; & \node (p0-4) [terminal] {@attribute}; & \node (p0-5) [point] {}; & \node (p0-6) [nonterminal] {\nonTerminalSymbol{expression}{22}}; & \node (p0-7) [terminal] {\{}; & \node (p0-8) [nonterminal] {\nonTerminalSymbol{instructionList}{36}}; & \node (p0-9) [terminal] {\}}; & \node (p0-10) [point] {}; & \node (p0-11) [point] {}; & & & & \node (p0-15) [point] {}; & \node (p0-16) [lastPoint] {}; & \\
  };
  \draw[->] (P0start) -- (p0-2) ;
  \draw[->] (p0-2) -- (p0-4) ;
  \draw (p0-3) |- (p1-4) ;
  \draw (p0-4) -- (p0-5) ;
  \draw[->] (p1-4) -| (p0-5) ;
  \draw[->] (p0-5) -- (p0-6) ;
  \draw[->] (p0-6) -- (p0-7) ;
  \draw[->] (p0-7) -- (p0-8) ;
  \draw[->] (p0-8) -- (p0-9) ;
  \draw (p0-9) -- (p0-11) ;
  \draw[->] (p0-10) |- (p1-11) ;
  \draw[->] (p1-11) -- (p1-12) ;
  \draw[->] (p1-12) -- (p1-13) ;
  \draw[->] (p1-13) -- (p1-14) ;
  \draw[->] (p0-10) |- (p2-11) ;
  \draw[->] (p2-11) -- (p2-12) ;
  \draw (p0-11) -- (p0-15) ;
  \draw[->] (p1-14) -| (p0-15) ;
  \draw[->] (p2-12) -| (p0-15) ;
  \draw[->] (p0-15) -- (p0-16) ;
\end{tikzpicture}

\nonTerminalSection{import\_file}{0}

\ruleSubsection{plm\_syntax}{syntax-grammar}{9}

\begin{tikzpicture}
  \matrix[column sep=\ruleMatrixColumnSeparation, row sep=\ruleMatrixRowSeparation] {
    \node (P0start) [firstPoint] {}; & & \node (p0-2) [terminal] {import}; & \node (p0-3) [terminal] {"string"}; & \node (p0-4) [lastPoint] {}; & \\
  };
  \draw[->] (P0start) -- (p0-2) ;
  \draw[->] (p0-2) -- (p0-3) ;
  \draw[->] (p0-3) -- (p0-4) ;
\end{tikzpicture}

\nonTerminalSection{instruction}{37}

\ruleSubsection{plm\_syntax}{directive-check}{17}

\begin{tikzpicture}
  \matrix[column sep=\ruleMatrixColumnSeparation, row sep=\ruleMatrixRowSeparation] {
    \node (P0start) [firstPoint] {}; & & \node (p0-2) [terminal] {check}; & \node (p0-3) [nonterminal] {\nonTerminalSymbol{expression}{22}}; & \node (p0-4) [lastPoint] {}; & \\
  };
  \draw[->] (P0start) -- (p0-2) ;
  \draw[->] (p0-2) -- (p0-3) ;
  \draw[->] (p0-3) -- (p0-4) ;
\end{tikzpicture}

\ruleSubsection{plm\_syntax}{instruction-assignment}{18}

\begin{tikzpicture}
  \matrix[column sep=\ruleMatrixColumnSeparation, row sep=\ruleMatrixRowSeparation] {
    \node (P0start) [firstPoint] {}; & & \node (p0-2) [nonterminal] {\nonTerminalSymbol{lvalue}{42}}; & \node (p0-3) [terminal] {=}; & \node (p0-4) [nonterminal] {\nonTerminalSymbol{expression}{22}}; & \node (p0-5) [lastPoint] {}; & \\
  };
  \draw[->] (P0start) -- (p0-2) ;
  \draw[->] (p0-2) -- (p0-3) ;
  \draw[->] (p0-3) -- (p0-4) ;
  \draw[->] (p0-4) -- (p0-5) ;
\end{tikzpicture}

\ruleSubsection{plm\_syntax}{instruction-assignment-operator}{49}

\begin{tikzpicture}
  \matrix[column sep=\ruleMatrixColumnSeparation, row sep=\ruleMatrixRowSeparation] {
    \node (P0start) [firstPoint] {}; & & \node (p0-2) [nonterminal] {\nonTerminalSymbol{lvalue}{42}}; & \node (p0-3) [nonterminal] {\nonTerminalSymbol{assignment\_operator}{38}}; & \node (p0-4) [nonterminal] {\nonTerminalSymbol{expression}{22}}; & \node (p0-5) [lastPoint] {}; & \\
  };
  \draw[->] (P0start) -- (p0-2) ;
  \draw[->] (p0-2) -- (p0-3) ;
  \draw[->] (p0-3) -- (p0-4) ;
  \draw[->] (p0-4) -- (p0-5) ;
\end{tikzpicture}

\ruleSubsection{plm\_syntax}{instruction-var}{26}

\begin{tikzpicture}
  \matrix[column sep=\ruleMatrixColumnSeparation, row sep=\ruleMatrixRowSeparation] {
    & & & & & \node (p1-5) [nonterminal] {\nonTerminalSymbol{type\_definition}{3}}; & \\
    \node (P0start) [firstPoint] {}; & & \node (p0-2) [terminal] {var}; & \node (p0-3) [terminal] {identifier}; & \node (p0-4) [point] {}; & \node (p0-5) [point] {}; & \node (p0-6) [point] {}; & \node (p0-7) [terminal] {=}; & \node (p0-8) [nonterminal] {\nonTerminalSymbol{expression}{22}}; & \node (p0-9) [lastPoint] {}; & \\
  };
  \draw[->] (P0start) -- (p0-2) ;
  \draw[->] (p0-2) -- (p0-3) ;
  \draw (p0-3) -- (p0-5) ;
  \draw[->] (p0-4) |- (p1-5) ;
  \draw (p0-5) -- (p0-6) ;
  \draw[->] (p1-5) -| (p0-6) ;
  \draw[->] (p0-6) -- (p0-7) ;
  \draw[->] (p0-7) -- (p0-8) ;
  \draw[->] (p0-8) -- (p0-9) ;
\end{tikzpicture}

\ruleSubsection{plm\_syntax}{instruction-var}{47}

\begin{tikzpicture}
  \matrix[column sep=\ruleMatrixColumnSeparation, row sep=\ruleMatrixRowSeparation] {
    \node (P0start) [firstPoint] {}; & & \node (p0-2) [terminal] {var}; & \node (p0-3) [terminal] {identifier}; & \node (p0-4) [nonterminal] {\nonTerminalSymbol{type\_definition}{3}}; & \node (p0-5) [lastPoint] {}; & \\
  };
  \draw[->] (P0start) -- (p0-2) ;
  \draw[->] (p0-2) -- (p0-3) ;
  \draw[->] (p0-3) -- (p0-4) ;
  \draw[->] (p0-4) -- (p0-5) ;
\end{tikzpicture}

\ruleSubsection{plm\_syntax}{instruction-let}{19}

\begin{tikzpicture}
  \matrix[column sep=\ruleMatrixColumnSeparation, row sep=\ruleMatrixRowSeparation] {
    & & & & & \node (p1-5) [nonterminal] {\nonTerminalSymbol{type\_definition}{3}}; & \\
    \node (P0start) [firstPoint] {}; & & \node (p0-2) [terminal] {let}; & \node (p0-3) [terminal] {identifier}; & \node (p0-4) [point] {}; & \node (p0-5) [point] {}; & \node (p0-6) [point] {}; & \node (p0-7) [terminal] {=}; & \node (p0-8) [nonterminal] {\nonTerminalSymbol{expression}{22}}; & \node (p0-9) [lastPoint] {}; & \\
  };
  \draw[->] (P0start) -- (p0-2) ;
  \draw[->] (p0-2) -- (p0-3) ;
  \draw (p0-3) -- (p0-5) ;
  \draw[->] (p0-4) |- (p1-5) ;
  \draw (p0-5) -- (p0-6) ;
  \draw[->] (p1-5) -| (p0-6) ;
  \draw[->] (p0-6) -- (p0-7) ;
  \draw[->] (p0-7) -- (p0-8) ;
  \draw[->] (p0-8) -- (p0-9) ;
\end{tikzpicture}

\ruleSubsection{plm\_syntax}{instruction-nop}{16}

\begin{tikzpicture}
  \matrix[column sep=\ruleMatrixColumnSeparation, row sep=\ruleMatrixRowSeparation] {
    \node (P0start) [firstPoint] {}; & & \node (p0-2) [terminal] {nop}; & \node (p0-3) [lastPoint] {}; & \\
  };
  \draw[->] (P0start) -- (p0-2) ;
  \draw[->] (p0-2) -- (p0-3) ;
\end{tikzpicture}

\ruleSubsection{plm\_syntax}{instruction-assert}{17}

\begin{tikzpicture}
  \matrix[column sep=\ruleMatrixColumnSeparation, row sep=\ruleMatrixRowSeparation] {
    \node (P0start) [firstPoint] {}; & & \node (p0-2) [terminal] {assert}; & \node (p0-3) [nonterminal] {\nonTerminalSymbol{expression}{22}}; & \node (p0-4) [lastPoint] {}; & \\
  };
  \draw[->] (P0start) -- (p0-2) ;
  \draw[->] (p0-2) -- (p0-3) ;
  \draw[->] (p0-3) -- (p0-4) ;
\end{tikzpicture}

\ruleSubsection{plm\_syntax}{instruction-panic}{17}

\begin{tikzpicture}
  \matrix[column sep=\ruleMatrixColumnSeparation, row sep=\ruleMatrixRowSeparation] {
    \node (P0start) [firstPoint] {}; & & \node (p0-2) [terminal] {panic}; & \node (p0-3) [nonterminal] {\nonTerminalSymbol{expression}{22}}; & \node (p0-4) [lastPoint] {}; & \\
  };
  \draw[->] (P0start) -- (p0-2) ;
  \draw[->] (p0-2) -- (p0-3) ;
  \draw[->] (p0-3) -- (p0-4) ;
\end{tikzpicture}

\ruleSubsection{plm\_syntax}{instruction-if}{24}

\begin{tikzpicture}
  \matrix[column sep=\ruleMatrixColumnSeparation, row sep=\ruleMatrixRowSeparation] {
    \node (P0start) [firstPoint] {}; & & \node (p0-2) [nonterminal] {\nonTerminalSymbol{if\_instruction}{39}}; & \node (p0-3) [lastPoint] {}; & \\
  };
  \draw[->] (P0start) -- (p0-2) ;
  \draw[->] (p0-2) -- (p0-3) ;
\end{tikzpicture}

\ruleSubsection{plm\_syntax}{instruction-sync}{116}

\begin{tikzpicture}
  \matrix[column sep=\ruleMatrixColumnSeparation, row sep=\ruleMatrixRowSeparation] {
    & & & & & & & & & \node (p1-9) [point] {}; & \\
    \node (P0start) [firstPoint] {}; & & \node (p0-2) [terminal] {sync}; & \node (p0-3) [terminal] {\{}; & \node (p0-4) [point] {}; & \node (p0-5) [nonterminal] {\nonTerminalSymbol{guarded\_command}{40}}; & \node (p0-6) [terminal] {:}; & \node (p0-7) [nonterminal] {\nonTerminalSymbol{instructionList}{36}}; & \node (p0-8) [point] {}; & & \node (p0-10) [terminal] {\}}; & \node (p0-11) [lastPoint] {}; & \\
  };
  \draw[->] (P0start) -- (p0-2) ;
  \draw[->] (p0-2) -- (p0-3) ;
  \draw[->] (p0-3) -- (p0-5) ;
  \draw[->] (p0-5) -- (p0-6) ;
  \draw[->] (p0-6) -- (p0-7) ;
  \draw[->] (p1-9) -| (p0-4) ;
  \draw[->] (p0-8) -| (p1-9) ;
  \draw[->] (p0-7) -- (p0-10) ;
  \draw[->] (p0-10) -- (p0-11) ;
\end{tikzpicture}

\ruleSubsection{plm\_syntax}{instruction-while}{20}

\begin{tikzpicture}
  \matrix[column sep=\ruleMatrixColumnSeparation, row sep=\ruleMatrixRowSeparation] {
    \node (P0start) [firstPoint] {}; & & \node (p4-2) [terminal] {while}; & \\
    & & \node (p3-2) [nonterminal] {\nonTerminalSymbol{expression}{22}}; & \\
    & & \node (p2-2) [terminal] {\{}; & \\
    & & \node (p1-2) [nonterminal] {\nonTerminalSymbol{instructionList}{36}}; & \\
    & & \node (p0-2) [terminal] {\}}; & \node (p0-3) [lastPoint] {}; & \\
  };
  \draw[->] (P0start) -- (p4-2) ;
  \draw[->] (p4-2) -- (p3-2) ;
  \draw[->] (p3-2) -- (p2-2) ;
  \draw[->] (p2-2) -- (p1-2) ;
  \draw[->] (p1-2) -- (p0-2) ;
  \draw[->] (p0-2) -- (p0-3) ;
\end{tikzpicture}

\ruleSubsection{plm\_syntax}{instruction-for-in-do}{23}

\begin{tikzpicture}
  \matrix[column sep=\ruleMatrixColumnSeparation, row sep=\ruleMatrixRowSeparation] {
    & & & & & & & & & \node (p2-9) [point] {}; & \\
    & & & & & & & \node (p1-7) [terminal] {while}; & \node (p1-8) [point] {}; & \node (p1-9) [terminal] {@attribute}; & \node (p1-10) [point] {}; & \node (p1-11) [nonterminal] {\nonTerminalSymbol{expression}{22}}; & \\
    \node (P0start) [firstPoint] {}; & & \node (p0-2) [terminal] {for}; & \node (p0-3) [terminal] {identifier}; & \node (p0-4) [terminal] {in}; & \node (p0-5) [terminal] {identifier}; & \node (p0-6) [point] {}; & \node (p0-7) [point] {}; & & & & & \node (p0-12) [point] {}; & \node (p0-13) [terminal] {\{}; & \node (p0-14) [nonterminal] {\nonTerminalSymbol{instructionList}{36}}; & \node (p0-15) [terminal] {\}}; & \node (p0-16) [lastPoint] {}; & \\
  };
  \draw[->] (P0start) -- (p0-2) ;
  \draw[->] (p0-2) -- (p0-3) ;
  \draw[->] (p0-3) -- (p0-4) ;
  \draw[->] (p0-4) -- (p0-5) ;
  \draw (p0-5) -- (p0-7) ;
  \draw[->] (p0-6) |- (p1-7) ;
  \draw[->] (p1-7) -- (p1-9) ;
  \draw (p1-8) |- (p2-9) ;
  \draw (p1-9) -- (p1-10) ;
  \draw[->] (p2-9) -| (p1-10) ;
  \draw[->] (p1-10) -- (p1-11) ;
  \draw (p0-7) -- (p0-12) ;
  \draw[->] (p1-11) -| (p0-12) ;
  \draw[->] (p0-12) -- (p0-13) ;
  \draw[->] (p0-13) -- (p0-14) ;
  \draw[->] (p0-14) -- (p0-15) ;
  \draw[->] (p0-15) -- (p0-16) ;
\end{tikzpicture}

\ruleSubsection{plm\_syntax}{instruction-for-in-lower-upper-bounds}{24}

\begin{tikzpicture}
  \matrix[column sep=\ruleMatrixColumnSeparation, row sep=\ruleMatrixRowSeparation] {
    & & & & \node (p1-4) [terminal] {identifier}; & \\
    \node (P0start) [firstPoint] {}; & & \node (p0-2) [terminal] {for}; & \node (p0-3) [point] {}; & \node (p0-4) [terminal] {\_}; & \node (p0-5) [point] {}; & \node (p0-6) [nonterminal] {\nonTerminalSymbol{type\_definition}{3}}; & \node (p0-7) [terminal] {in}; & \node (p0-8) [nonterminal] {\nonTerminalSymbol{expression}{22}}; & \node (p0-9) [terminal] {..<}; & \node (p0-10) [nonterminal] {\nonTerminalSymbol{expression}{22}}; & \node (p0-11) [terminal] {\{}; & \node (p0-12) [nonterminal] {\nonTerminalSymbol{instructionList}{36}}; & \node (p0-13) [terminal] {\}}; & \node (p0-14) [lastPoint] {}; & \\
  };
  \draw[->] (P0start) -- (p0-2) ;
  \draw[->] (p0-2) -- (p0-4) ;
  \draw[->] (p0-3) |- (p1-4) ;
  \draw (p0-4) -- (p0-5) ;
  \draw[->] (p1-4) -| (p0-5) ;
  \draw[->] (p0-5) -- (p0-6) ;
  \draw[->] (p0-6) -- (p0-7) ;
  \draw[->] (p0-7) -- (p0-8) ;
  \draw[->] (p0-8) -- (p0-9) ;
  \draw[->] (p0-9) -- (p0-10) ;
  \draw[->] (p0-10) -- (p0-11) ;
  \draw[->] (p0-11) -- (p0-12) ;
  \draw[->] (p0-12) -- (p0-13) ;
  \draw[->] (p0-13) -- (p0-14) ;
\end{tikzpicture}

\ruleSubsection{plm\_syntax}{instruction-procedure-call}{31}

\begin{tikzpicture}
  \matrix[column sep=\ruleMatrixColumnSeparation, row sep=\ruleMatrixRowSeparation] {
    \node (P0start) [firstPoint] {}; & & \node (p0-2) [nonterminal] {\nonTerminalSymbol{procedure\_call}{41}}; & \node (p0-3) [lastPoint] {}; & \\
  };
  \draw[->] (P0start) -- (p0-2) ;
  \draw[->] (p0-2) -- (p0-3) ;
\end{tikzpicture}

\ruleSubsection{plm\_syntax}{instruction-switch}{27}

\begin{tikzpicture}
  \matrix[column sep=\ruleMatrixColumnSeparation, row sep=\ruleMatrixRowSeparation] {
    & & & & & & & & & & & & & & & \node (p3-15) [point] {}; & \\
    & & & & & & & & & & & \node (p2-11) [point] {}; & \\
    & & & & & & & & & & \node (p1-10) [terminal] {,}; & \\
    \node (P0start) [firstPoint] {}; & & \node (p0-2) [terminal] {switch}; & \node (p0-3) [nonterminal] {\nonTerminalSymbol{expression}{22}}; & \node (p0-4) [terminal] {\{}; & \node (p0-5) [point] {}; & \node (p0-6) [terminal] {case}; & \node (p0-7) [point] {}; & \node (p0-8) [terminal] {identifier}; & \node (p0-9) [point] {}; & & & \node (p0-12) [terminal] {:}; & \node (p0-13) [nonterminal] {\nonTerminalSymbol{instructionList}{36}}; & \node (p0-14) [point] {}; & & \node (p0-16) [terminal] {\}}; & \node (p0-17) [lastPoint] {}; & \\
  };
  \draw[->] (P0start) -- (p0-2) ;
  \draw[->] (p0-2) -- (p0-3) ;
  \draw[->] (p0-3) -- (p0-4) ;
  \draw[->] (p0-4) -- (p0-6) ;
  \draw[->] (p0-6) -- (p0-8) ;
  \draw[->] (p0-9) |- (p1-10) ;
  \draw[->] (p2-11) -| (p0-7) ;
  \draw[->] (p1-10) -| (p2-11) ;
  \draw[->] (p0-8) -- (p0-12) ;
  \draw[->] (p0-12) -- (p0-13) ;
  \draw[->] (p3-15) -| (p0-5) ;
  \draw[->] (p0-14) -| (p3-15) ;
  \draw[->] (p0-13) -- (p0-16) ;
  \draw[->] (p0-16) -- (p0-17) ;
\end{tikzpicture}

\nonTerminalSection{instructionList}{36}

\ruleSubsection{plm\_syntax}{instructionList}{23}

\begin{tikzpicture}
  \matrix[column sep=\ruleMatrixColumnSeparation, row sep=\ruleMatrixRowSeparation] {
    & & & & & & \node (p3-6) [point] {}; & \\
    & & & & & \node (p2-5) [terminal] {;}; & \\
    & & & & & \node (p1-5) [nonterminal] {\nonTerminalSymbol{instruction}{37}}; & \\
    \node (P0start) [firstPoint] {}; & & \node (p0-2) [point] {}; & \node (p0-3) [point] {}; & \node (p0-4) [point] {}; & & & \node (p0-7) [lastPoint] {}; & \\
  };
  \draw (P0start) -- (p0-3) ;
  \draw[->] (p0-4) |- (p1-5) ;
  \draw[->] (p0-4) |- (p2-5) ;
  \draw[->] (p3-6) -| (p0-2) ;
  \draw[->] (p1-5) -| (p3-6) ;
  \draw[->] (p2-5) -| (p3-6) ;
  \draw[->] (p0-3) -- (p0-7) ;
\end{tikzpicture}

\nonTerminalSection{isr}{20}

\ruleSubsection{plm\_syntax}{declaration-isr}{21}

\begin{tikzpicture}
  \matrix[column sep=\ruleMatrixColumnSeparation, row sep=\ruleMatrixRowSeparation] {
    & & & & \node (p2-4) [terminal] {safe}; & \\
    & & & & \node (p1-4) [terminal] {section}; & \\
    \node (P0start) [firstPoint] {}; & & \node (p0-2) [terminal] {isr}; & \node (p0-3) [point] {}; & \node (p0-4) [terminal] {service}; & \node (p0-5) [point] {}; & \node (p0-6) [terminal] {identifier}; & \node (p0-7) [terminal] {\{}; & \node (p0-8) [nonterminal] {\nonTerminalSymbol{instructionList}{36}}; & \node (p0-9) [terminal] {\}}; & \node (p0-10) [lastPoint] {}; & \\
  };
  \draw[->] (P0start) -- (p0-2) ;
  \draw[->] (p0-2) -- (p0-4) ;
  \draw[->] (p0-3) |- (p1-4) ;
  \draw[->] (p0-3) |- (p2-4) ;
  \draw (p0-4) -- (p0-5) ;
  \draw[->] (p1-4) -| (p0-5) ;
  \draw[->] (p2-4) -| (p0-5) ;
  \draw[->] (p0-5) -- (p0-6) ;
  \draw[->] (p0-6) -- (p0-7) ;
  \draw[->] (p0-7) -- (p0-8) ;
  \draw[->] (p0-8) -- (p0-9) ;
  \draw[->] (p0-9) -- (p0-10) ;
\end{tikzpicture}

\nonTerminalSection{lvalue}{42}

\ruleSubsection{plm\_syntax}{lvalue}{31}

\begin{tikzpicture}
  \matrix[column sep=\ruleMatrixColumnSeparation, row sep=\ruleMatrixRowSeparation] {
    & & & & & & & & & & & \node (p3-11) [point] {}; & \\
    & & & & & & & & \node (p2-8) [terminal] {[}; & \node (p2-9) [nonterminal] {\nonTerminalSymbol{expression}{22}}; & \node (p2-10) [terminal] {]}; & \\
    & & & \node (p1-3) [terminal] {self}; & & & & & \node (p1-8) [terminal] {.}; & \node (p1-9) [terminal] {identifier}; & \\
    \node (P0start) [firstPoint] {}; & & \node (p0-2) [point] {}; & \node (p0-3) [terminal] {identifier}; & \node (p0-4) [point] {}; & \node (p0-5) [point] {}; & \node (p0-6) [point] {}; & \node (p0-7) [point] {}; & & & & & \node (p0-12) [lastPoint] {}; & \\
  };
  \draw[->] (P0start) -- (p0-3) ;
  \draw[->] (p0-2) |- (p1-3) ;
  \draw (p0-3) -- (p0-4) ;
  \draw[->] (p1-3) -| (p0-4) ;
  \draw (p0-4) -- (p0-6) ;
  \draw[->] (p0-7) |- (p1-8) ;
  \draw[->] (p1-8) -- (p1-9) ;
  \draw[->] (p0-7) |- (p2-8) ;
  \draw[->] (p2-8) -- (p2-9) ;
  \draw[->] (p2-9) -- (p2-10) ;
  \draw[->] (p3-11) -| (p0-5) ;
  \draw[->] (p1-9) -| (p3-11) ;
  \draw[->] (p2-10) -| (p3-11) ;
  \draw[->] (p0-6) -- (p0-12) ;
\end{tikzpicture}

\nonTerminalSection{mode}{17}

\ruleSubsection{plm\_syntax}{declaration-func}{102}

\begin{tikzpicture}
  \matrix[column sep=\ruleMatrixColumnSeparation, row sep=\ruleMatrixRowSeparation] {
    & & & \node (p9-3) [terminal] {safe}; & \\
    & & & \node (p8-3) [terminal] {guard}; & \\
    & & & \node (p7-3) [terminal] {primitive}; & \\
    & & & \node (p6-3) [terminal] {service}; & \\
    & & & \node (p5-3) [terminal] {section}; & \\
    & & & \node (p4-3) [terminal] {init}; & \\
    & & & \node (p3-3) [terminal] {boot}; & \\
    & & & \node (p2-3) [terminal] {panic}; & \\
    & & & \node (p1-3) [terminal] {user}; & \\
    \node (P0start) [firstPoint] {}; & & \node (p0-2) [point] {}; & \node (p0-3) [point] {}; & \node (p0-4) [point] {}; & \node (p0-5) [lastPoint] {}; & \\
  };
  \draw (P0start) -- (p0-3) ;
  \draw[->] (p0-2) |- (p1-3) ;
  \draw[->] (p0-2) |- (p2-3) ;
  \draw[->] (p0-2) |- (p3-3) ;
  \draw[->] (p0-2) |- (p4-3) ;
  \draw[->] (p0-2) |- (p5-3) ;
  \draw[->] (p0-2) |- (p6-3) ;
  \draw[->] (p0-2) |- (p7-3) ;
  \draw[->] (p0-2) |- (p8-3) ;
  \draw[->] (p0-2) |- (p9-3) ;
  \draw (p0-3) -- (p0-4) ;
  \draw[->] (p1-3) -| (p0-4) ;
  \draw[->] (p2-3) -| (p0-4) ;
  \draw[->] (p3-3) -| (p0-4) ;
  \draw[->] (p4-3) -| (p0-4) ;
  \draw[->] (p5-3) -| (p0-4) ;
  \draw[->] (p6-3) -| (p0-4) ;
  \draw[->] (p7-3) -| (p0-4) ;
  \draw[->] (p8-3) -| (p0-4) ;
  \draw[->] (p9-3) -| (p0-4) ;
  \draw[->] (p0-4) -- (p0-5) ;
\end{tikzpicture}

\nonTerminalSection{module\_property}{9}

\ruleSubsection{plm\_syntax}{declaration-module}{58}

\begin{tikzpicture}
  \matrix[column sep=\ruleMatrixColumnSeparation, row sep=\ruleMatrixRowSeparation] {
    & & & & & & & \node (p2-7) [terminal] {=}; & \node (p2-8) [nonterminal] {\nonTerminalSymbol{expression}{22}}; & \\
    & & & \node (p1-3) [terminal] {let}; & & & & & & \node (p1-9) [terminal] {=}; & \node (p1-10) [nonterminal] {\nonTerminalSymbol{expression}{22}}; & \\
    \node (P0start) [firstPoint] {}; & & \node (p0-2) [point] {}; & \node (p0-3) [terminal] {var}; & \node (p0-4) [point] {}; & \node (p0-5) [terminal] {identifier}; & \node (p0-6) [point] {}; & \node (p0-7) [nonterminal] {\nonTerminalSymbol{type\_definition}{3}}; & \node (p0-8) [point] {}; & \node (p0-9) [point] {}; & & \node (p0-11) [point] {}; & \node (p0-12) [point] {}; & \node (p0-13) [lastPoint] {}; & \\
  };
  \draw[->] (P0start) -- (p0-3) ;
  \draw[->] (p0-2) |- (p1-3) ;
  \draw (p0-3) -- (p0-4) ;
  \draw[->] (p1-3) -| (p0-4) ;
  \draw[->] (p0-4) -- (p0-5) ;
  \draw[->] (p0-5) -- (p0-7) ;
  \draw (p0-7) -- (p0-9) ;
  \draw[->] (p0-8) |- (p1-9) ;
  \draw[->] (p1-9) -- (p1-10) ;
  \draw (p0-9) -- (p0-11) ;
  \draw[->] (p1-10) -| (p0-11) ;
  \draw[->] (p0-6) |- (p2-7) ;
  \draw[->] (p2-7) -- (p2-8) ;
  \draw (p0-11) -- (p0-12) ;
  \draw[->] (p2-8) -| (p0-12) ;
  \draw[->] (p0-12) -- (p0-13) ;
\end{tikzpicture}

\nonTerminalSection{primary}{33}

\ruleSubsection{plm\_syntax}{expression-operator-priority}{348}

\begin{tikzpicture}
  \matrix[column sep=\ruleMatrixColumnSeparation, row sep=\ruleMatrixRowSeparation] {
    \node (P0start) [firstPoint] {}; & & \node (p0-2) [terminal] {$\sim$}; & \node (p0-3) [nonterminal] {\nonTerminalSymbol{primary}{33}}; & \node (p0-4) [lastPoint] {}; & \\
  };
  \draw[->] (P0start) -- (p0-2) ;
  \draw[->] (p0-2) -- (p0-3) ;
  \draw[->] (p0-3) -- (p0-4) ;
\end{tikzpicture}

\ruleSubsection{plm\_syntax}{expression-operator-priority}{361}

\begin{tikzpicture}
  \matrix[column sep=\ruleMatrixColumnSeparation, row sep=\ruleMatrixRowSeparation] {
    \node (P0start) [firstPoint] {}; & & \node (p0-2) [terminal] {not}; & \node (p0-3) [nonterminal] {\nonTerminalSymbol{primary}{33}}; & \node (p0-4) [lastPoint] {}; & \\
  };
  \draw[->] (P0start) -- (p0-2) ;
  \draw[->] (p0-2) -- (p0-3) ;
  \draw[->] (p0-3) -- (p0-4) ;
\end{tikzpicture}

\ruleSubsection{plm\_syntax}{expression-operator-priority}{374}

\begin{tikzpicture}
  \matrix[column sep=\ruleMatrixColumnSeparation, row sep=\ruleMatrixRowSeparation] {
    \node (P0start) [firstPoint] {}; & & \node (p0-2) [terminal] {-}; & \node (p0-3) [nonterminal] {\nonTerminalSymbol{primary}{33}}; & \node (p0-4) [lastPoint] {}; & \\
  };
  \draw[->] (P0start) -- (p0-2) ;
  \draw[->] (p0-2) -- (p0-3) ;
  \draw[->] (p0-3) -- (p0-4) ;
\end{tikzpicture}

\ruleSubsection{plm\_syntax}{expression-operator-priority}{387}

\begin{tikzpicture}
  \matrix[column sep=\ruleMatrixColumnSeparation, row sep=\ruleMatrixRowSeparation] {
    \node (P0start) [firstPoint] {}; & & \node (p0-2) [terminal] {-\verb=%=}; & \node (p0-3) [nonterminal] {\nonTerminalSymbol{primary}{33}}; & \node (p0-4) [lastPoint] {}; & \\
  };
  \draw[->] (P0start) -- (p0-2) ;
  \draw[->] (p0-2) -- (p0-3) ;
  \draw[->] (p0-3) -- (p0-4) ;
\end{tikzpicture}

\ruleSubsection{plm\_syntax}{expression-operator-priority}{400}

\begin{tikzpicture}
  \matrix[column sep=\ruleMatrixColumnSeparation, row sep=\ruleMatrixRowSeparation] {
    \node (P0start) [firstPoint] {}; & & \node (p0-2) [terminal] {(}; & \node (p0-3) [nonterminal] {\nonTerminalSymbol{expression}{22}}; & \node (p0-4) [terminal] {)}; & \node (p0-5) [lastPoint] {}; & \\
  };
  \draw[->] (P0start) -- (p0-2) ;
  \draw[->] (p0-2) -- (p0-3) ;
  \draw[->] (p0-3) -- (p0-4) ;
  \draw[->] (p0-4) -- (p0-5) ;
\end{tikzpicture}

\ruleSubsection{plm\_syntax}{expression-convert}{19}

\begin{tikzpicture}
  \matrix[column sep=\ruleMatrixColumnSeparation, row sep=\ruleMatrixRowSeparation] {
    & & & & \node (p1-4) [nonterminal] {\nonTerminalSymbol{type\_definition}{3}}; & \\
    \node (P0start) [firstPoint] {}; & & \node (p0-2) [terminal] {convert}; & \node (p0-3) [point] {}; & \node (p0-4) [point] {}; & \node (p0-5) [point] {}; & \node (p0-6) [terminal] {(}; & \node (p0-7) [nonterminal] {\nonTerminalSymbol{expression}{22}}; & \node (p0-8) [terminal] {)}; & \node (p0-9) [lastPoint] {}; & \\
  };
  \draw[->] (P0start) -- (p0-2) ;
  \draw (p0-2) -- (p0-4) ;
  \draw[->] (p0-3) |- (p1-4) ;
  \draw (p0-4) -- (p0-5) ;
  \draw[->] (p1-4) -| (p0-5) ;
  \draw[->] (p0-5) -- (p0-6) ;
  \draw[->] (p0-6) -- (p0-7) ;
  \draw[->] (p0-7) -- (p0-8) ;
  \draw[->] (p0-8) -- (p0-9) ;
\end{tikzpicture}

\ruleSubsection{plm\_syntax}{expression-extend}{19}

\begin{tikzpicture}
  \matrix[column sep=\ruleMatrixColumnSeparation, row sep=\ruleMatrixRowSeparation] {
    & & & & \node (p1-4) [nonterminal] {\nonTerminalSymbol{type\_definition}{3}}; & \\
    \node (P0start) [firstPoint] {}; & & \node (p0-2) [terminal] {extend}; & \node (p0-3) [point] {}; & \node (p0-4) [point] {}; & \node (p0-5) [point] {}; & \node (p0-6) [terminal] {(}; & \node (p0-7) [nonterminal] {\nonTerminalSymbol{expression}{22}}; & \node (p0-8) [terminal] {)}; & \node (p0-9) [lastPoint] {}; & \\
  };
  \draw[->] (P0start) -- (p0-2) ;
  \draw (p0-2) -- (p0-4) ;
  \draw[->] (p0-3) |- (p1-4) ;
  \draw (p0-4) -- (p0-5) ;
  \draw[->] (p1-4) -| (p0-5) ;
  \draw[->] (p0-5) -- (p0-6) ;
  \draw[->] (p0-6) -- (p0-7) ;
  \draw[->] (p0-7) -- (p0-8) ;
  \draw[->] (p0-8) -- (p0-9) ;
\end{tikzpicture}

\ruleSubsection{plm\_syntax}{expression-truncate}{19}

\begin{tikzpicture}
  \matrix[column sep=\ruleMatrixColumnSeparation, row sep=\ruleMatrixRowSeparation] {
    & & & & \node (p1-4) [nonterminal] {\nonTerminalSymbol{type\_definition}{3}}; & \\
    \node (P0start) [firstPoint] {}; & & \node (p0-2) [terminal] {truncate}; & \node (p0-3) [point] {}; & \node (p0-4) [point] {}; & \node (p0-5) [point] {}; & \node (p0-6) [terminal] {(}; & \node (p0-7) [nonterminal] {\nonTerminalSymbol{expression}{22}}; & \node (p0-8) [terminal] {)}; & \node (p0-9) [lastPoint] {}; & \\
  };
  \draw[->] (P0start) -- (p0-2) ;
  \draw (p0-2) -- (p0-4) ;
  \draw[->] (p0-3) |- (p1-4) ;
  \draw (p0-4) -- (p0-5) ;
  \draw[->] (p1-4) -| (p0-5) ;
  \draw[->] (p0-5) -- (p0-6) ;
  \draw[->] (p0-6) -- (p0-7) ;
  \draw[->] (p0-7) -- (p0-8) ;
  \draw[->] (p0-8) -- (p0-9) ;
\end{tikzpicture}

\ruleSubsection{plm\_syntax}{expression-addressof}{17}

\begin{tikzpicture}
  \matrix[column sep=\ruleMatrixColumnSeparation, row sep=\ruleMatrixRowSeparation] {
    \node (P0start) [firstPoint] {}; & & \node (p0-2) [terminal] {addressof}; & \node (p0-3) [terminal] {(}; & \node (p0-4) [nonterminal] {\nonTerminalSymbol{lvalue}{42}}; & \node (p0-5) [terminal] {)}; & \node (p0-6) [lastPoint] {}; & \\
  };
  \draw[->] (P0start) -- (p0-2) ;
  \draw[->] (p0-2) -- (p0-3) ;
  \draw[->] (p0-3) -- (p0-4) ;
  \draw[->] (p0-4) -- (p0-5) ;
  \draw[->] (p0-5) -- (p0-6) ;
\end{tikzpicture}

\ruleSubsection{plm\_syntax}{expression-sizeof}{23}

\begin{tikzpicture}
  \matrix[column sep=\ruleMatrixColumnSeparation, row sep=\ruleMatrixRowSeparation] {
    \node (P0start) [firstPoint] {}; & & \node (p0-2) [terminal] {sizeof}; & \node (p0-3) [terminal] {(}; & \node (p0-4) [nonterminal] {\nonTerminalSymbol{lvalue}{42}}; & \node (p0-5) [terminal] {)}; & \node (p0-6) [lastPoint] {}; & \\
  };
  \draw[->] (P0start) -- (p0-2) ;
  \draw[->] (p0-2) -- (p0-3) ;
  \draw[->] (p0-3) -- (p0-4) ;
  \draw[->] (p0-4) -- (p0-5) ;
  \draw[->] (p0-5) -- (p0-6) ;
\end{tikzpicture}

\ruleSubsection{plm\_syntax}{expression-sizeof}{33}

\begin{tikzpicture}
  \matrix[column sep=\ruleMatrixColumnSeparation, row sep=\ruleMatrixRowSeparation] {
    \node (P0start) [firstPoint] {}; & & \node (p0-2) [terminal] {sizeof}; & \node (p0-3) [terminal] {(}; & \node (p0-4) [nonterminal] {\nonTerminalSymbol{type\_definition}{3}}; & \node (p0-5) [terminal] {)}; & \node (p0-6) [lastPoint] {}; & \\
  };
  \draw[->] (P0start) -- (p0-2) ;
  \draw[->] (p0-2) -- (p0-3) ;
  \draw[->] (p0-3) -- (p0-4) ;
  \draw[->] (p0-4) -- (p0-5) ;
  \draw[->] (p0-5) -- (p0-6) ;
\end{tikzpicture}

\ruleSubsection{plm\_syntax}{expression-constructor-call}{24}

\begin{tikzpicture}
  \matrix[column sep=\ruleMatrixColumnSeparation, row sep=\ruleMatrixRowSeparation] {
    & & & & & & & & & \node (p2-9) [point] {}; & \\
    & & & & & & & \node (p1-7) [terminal] {!selector:}; & \node (p1-8) [nonterminal] {\nonTerminalSymbol{expression}{22}}; & \\
    \node (P0start) [firstPoint] {}; & & \node (p0-2) [nonterminal] {\nonTerminalSymbol{type\_definition}{3}}; & \node (p0-3) [terminal] {(}; & \node (p0-4) [point] {}; & \node (p0-5) [point] {}; & \node (p0-6) [point] {}; & & & & \node (p0-10) [terminal] {)}; & \node (p0-11) [lastPoint] {}; & \\
  };
  \draw[->] (P0start) -- (p0-2) ;
  \draw[->] (p0-2) -- (p0-3) ;
  \draw (p0-3) -- (p0-5) ;
  \draw[->] (p0-6) |- (p1-7) ;
  \draw[->] (p1-7) -- (p1-8) ;
  \draw[->] (p2-9) -| (p0-4) ;
  \draw[->] (p1-8) -| (p2-9) ;
  \draw[->] (p0-5) -- (p0-10) ;
  \draw[->] (p0-10) -- (p0-11) ;
\end{tikzpicture}

\ruleSubsection{plm\_syntax}{expression-typed-constant}{19}

\begin{tikzpicture}
  \matrix[column sep=\ruleMatrixColumnSeparation, row sep=\ruleMatrixRowSeparation] {
    & & & \node (p1-3) [nonterminal] {\nonTerminalSymbol{type\_definition}{3}}; & \\
    \node (P0start) [firstPoint] {}; & & \node (p0-2) [point] {}; & \node (p0-3) [point] {}; & \node (p0-4) [point] {}; & \node (p0-5) [terminal] {.}; & \node (p0-6) [terminal] {identifier}; & \node (p0-7) [nonterminal] {\nonTerminalSymbol{expression\_access\_list}{35}}; & \node (p0-8) [lastPoint] {}; & \\
  };
  \draw (P0start) -- (p0-3) ;
  \draw[->] (p0-2) |- (p1-3) ;
  \draw (p0-3) -- (p0-4) ;
  \draw[->] (p1-3) -| (p0-4) ;
  \draw[->] (p0-4) -- (p0-5) ;
  \draw[->] (p0-5) -- (p0-6) ;
  \draw[->] (p0-6) -- (p0-7) ;
  \draw[->] (p0-7) -- (p0-8) ;
\end{tikzpicture}

\ruleSubsection{plm\_syntax}{expression-if}{22}

\begin{tikzpicture}
  \matrix[column sep=\ruleMatrixColumnSeparation, row sep=\ruleMatrixRowSeparation] {
    \node (P0start) [firstPoint] {}; & & \node (p0-2) [nonterminal] {\nonTerminalSymbol{expression\_if}{34}}; & \node (p0-3) [lastPoint] {}; & \\
  };
  \draw[->] (P0start) -- (p0-2) ;
  \draw[->] (p0-2) -- (p0-3) ;
\end{tikzpicture}

\ruleSubsection{plm\_syntax}{expression-literal-integer}{17}

\begin{tikzpicture}
  \matrix[column sep=\ruleMatrixColumnSeparation, row sep=\ruleMatrixRowSeparation] {
    \node (P0start) [firstPoint] {}; & & \node (p0-2) [terminal] {integer}; & \node (p0-3) [lastPoint] {}; & \\
  };
  \draw[->] (P0start) -- (p0-2) ;
  \draw[->] (p0-2) -- (p0-3) ;
\end{tikzpicture}

\ruleSubsection{plm\_syntax}{expression-literal-string}{17}

\begin{tikzpicture}
  \matrix[column sep=\ruleMatrixColumnSeparation, row sep=\ruleMatrixRowSeparation] {
    \node (P0start) [firstPoint] {}; & & \node (p0-2) [terminal] {"string"}; & \node (p0-3) [lastPoint] {}; & \\
  };
  \draw[->] (P0start) -- (p0-2) ;
  \draw[->] (p0-2) -- (p0-3) ;
\end{tikzpicture}

\ruleSubsection{plm\_syntax}{expression-true-false}{17}

\begin{tikzpicture}
  \matrix[column sep=\ruleMatrixColumnSeparation, row sep=\ruleMatrixRowSeparation] {
    \node (P0start) [firstPoint] {}; & & \node (p0-2) [terminal] {yes}; & \node (p0-3) [lastPoint] {}; & \\
  };
  \draw[->] (P0start) -- (p0-2) ;
  \draw[->] (p0-2) -- (p0-3) ;
\end{tikzpicture}

\ruleSubsection{plm\_syntax}{expression-true-false}{24}

\begin{tikzpicture}
  \matrix[column sep=\ruleMatrixColumnSeparation, row sep=\ruleMatrixRowSeparation] {
    \node (P0start) [firstPoint] {}; & & \node (p0-2) [terminal] {no}; & \node (p0-3) [lastPoint] {}; & \\
  };
  \draw[->] (P0start) -- (p0-2) ;
  \draw[->] (p0-2) -- (p0-3) ;
\end{tikzpicture}

\ruleSubsection{plm\_syntax}{expression-cst-registre}{26}

\begin{tikzpicture}
  \matrix[column sep=\ruleMatrixColumnSeparation, row sep=\ruleMatrixRowSeparation] {
    & & & & & & & & \node (p1-8) [point] {}; & \\
    \node (P0start) [firstPoint] {}; & & \node (p0-2) [terminal] {\{}; & \node (p0-3) [terminal] {identifier}; & \node (p0-4) [point] {}; & \node (p0-5) [terminal] {!selector:}; & \node (p0-6) [nonterminal] {\nonTerminalSymbol{expression}{22}}; & \node (p0-7) [point] {}; & & \node (p0-9) [terminal] {\}}; & \node (p0-10) [lastPoint] {}; & \\
  };
  \draw[->] (P0start) -- (p0-2) ;
  \draw[->] (p0-2) -- (p0-3) ;
  \draw[->] (p0-3) -- (p0-5) ;
  \draw[->] (p0-5) -- (p0-6) ;
  \draw[->] (p1-8) -| (p0-4) ;
  \draw[->] (p0-7) -| (p1-8) ;
  \draw[->] (p0-6) -- (p0-9) ;
  \draw[->] (p0-9) -- (p0-10) ;
\end{tikzpicture}

\ruleSubsection{plm\_syntax}{expression-primary}{32}

\begin{tikzpicture}
  \matrix[column sep=\ruleMatrixColumnSeparation, row sep=\ruleMatrixRowSeparation] {
    & & & \node (p1-3) [terminal] {self}; & \\
    \node (P0start) [firstPoint] {}; & & \node (p0-2) [point] {}; & \node (p0-3) [terminal] {identifier}; & \node (p0-4) [point] {}; & \node (p0-5) [nonterminal] {\nonTerminalSymbol{expression\_access\_list}{35}}; & \node (p0-6) [lastPoint] {}; & \\
  };
  \draw[->] (P0start) -- (p0-3) ;
  \draw[->] (p0-2) |- (p1-3) ;
  \draw (p0-3) -- (p0-4) ;
  \draw[->] (p1-3) -| (p0-4) ;
  \draw[->] (p0-4) -- (p0-5) ;
  \draw[->] (p0-5) -- (p0-6) ;
\end{tikzpicture}

\ruleSubsection{plm\_syntax}{expression-standalone-function-call}{19}

\begin{tikzpicture}
  \matrix[column sep=\ruleMatrixColumnSeparation, row sep=\ruleMatrixRowSeparation] {
    \node (P0start) [firstPoint] {}; & & \node (p0-2) [terminal] {identifier}; & \node (p0-3) [nonterminal] {\nonTerminalSymbol{effective\_parameters}{43}}; & \node (p0-4) [lastPoint] {}; & \\
  };
  \draw[->] (P0start) -- (p0-2) ;
  \draw[->] (p0-2) -- (p0-3) ;
  \draw[->] (p0-3) -- (p0-4) ;
\end{tikzpicture}

\nonTerminalSection{private\_or\_public\_struct\_property\_declaration}{4}

\ruleSubsection{plm\_syntax}{type-structure-declaration}{40}

\begin{tikzpicture}
  \matrix[column sep=\ruleMatrixColumnSeparation, row sep=\ruleMatrixRowSeparation] {
    & & & \node (p1-3) [terminal] {public}; & \\
    \node (P0start) [firstPoint] {}; & & \node (p0-2) [point] {}; & \node (p0-3) [point] {}; & \node (p0-4) [point] {}; & \node (p0-5) [nonterminal] {\nonTerminalSymbol{struct\_property\_declaration}{6}}; & \node (p0-6) [lastPoint] {}; & \\
  };
  \draw (P0start) -- (p0-3) ;
  \draw[->] (p0-2) |- (p1-3) ;
  \draw (p0-3) -- (p0-4) ;
  \draw[->] (p1-3) -| (p0-4) ;
  \draw[->] (p0-4) -- (p0-5) ;
  \draw[->] (p0-5) -- (p0-6) ;
\end{tikzpicture}

\nonTerminalSection{private\_struct\_property\_declaration}{5}

\ruleSubsection{plm\_syntax}{type-structure-declaration}{53}

\begin{tikzpicture}
  \matrix[column sep=\ruleMatrixColumnSeparation, row sep=\ruleMatrixRowSeparation] {
    \node (P0start) [firstPoint] {}; & & \node (p0-2) [nonterminal] {\nonTerminalSymbol{struct\_property\_declaration}{6}}; & \node (p0-3) [lastPoint] {}; & \\
  };
  \draw[->] (P0start) -- (p0-2) ;
  \draw[->] (p0-2) -- (p0-3) ;
\end{tikzpicture}

\nonTerminalSection{procedure\_call}{41}

\ruleSubsection{plm\_syntax}{instruction-procedure-call}{38}

\begin{tikzpicture}
  \matrix[column sep=\ruleMatrixColumnSeparation, row sep=\ruleMatrixRowSeparation] {
    \node (P0start) [firstPoint] {}; & & \node (p0-2) [nonterminal] {\nonTerminalSymbol{lvalue}{42}}; & \node (p0-3) [nonterminal] {\nonTerminalSymbol{effective\_parameters}{43}}; & \node (p0-4) [lastPoint] {}; & \\
  };
  \draw[->] (P0start) -- (p0-2) ;
  \draw[->] (p0-2) -- (p0-3) ;
  \draw[->] (p0-3) -- (p0-4) ;
\end{tikzpicture}

\nonTerminalSection{procedure\_formal\_arguments}{19}

\ruleSubsection{plm\_syntax}{declaration-func}{159}

\begin{tikzpicture}
  \matrix[column sep=\ruleMatrixColumnSeparation, row sep=\ruleMatrixRowSeparation] {
    & & & & & & & & & \node (p4-9) [point] {}; & \\
    & & & & & & \node (p3-6) [terminal] {?selector:}; & \node (p3-7) [terminal] {identifier}; & \node (p3-8) [nonterminal] {\nonTerminalSymbol{type\_definition}{3}}; & \\
    & & & & & & \node (p2-6) [terminal] {?!selector:}; & \node (p2-7) [terminal] {identifier}; & \node (p2-8) [nonterminal] {\nonTerminalSymbol{type\_definition}{3}}; & \\
    & & & & & & \node (p1-6) [terminal] {!selector:}; & \node (p1-7) [terminal] {identifier}; & \node (p1-8) [nonterminal] {\nonTerminalSymbol{type\_definition}{3}}; & \\
    \node (P0start) [firstPoint] {}; & & \node (p0-2) [terminal] {(}; & \node (p0-3) [point] {}; & \node (p0-4) [point] {}; & \node (p0-5) [point] {}; & & & & & \node (p0-10) [terminal] {)}; & \node (p0-11) [lastPoint] {}; & \\
  };
  \draw[->] (P0start) -- (p0-2) ;
  \draw (p0-2) -- (p0-4) ;
  \draw[->] (p0-5) |- (p1-6) ;
  \draw[->] (p1-6) -- (p1-7) ;
  \draw[->] (p1-7) -- (p1-8) ;
  \draw[->] (p0-5) |- (p2-6) ;
  \draw[->] (p2-6) -- (p2-7) ;
  \draw[->] (p2-7) -- (p2-8) ;
  \draw[->] (p0-5) |- (p3-6) ;
  \draw[->] (p3-6) -- (p3-7) ;
  \draw[->] (p3-7) -- (p3-8) ;
  \draw[->] (p4-9) -| (p0-3) ;
  \draw[->] (p1-8) -| (p4-9) ;
  \draw[->] (p2-8) -| (p4-9) ;
  \draw[->] (p3-8) -| (p4-9) ;
  \draw[->] (p0-4) -- (p0-10) ;
  \draw[->] (p0-10) -- (p0-11) ;
\end{tikzpicture}

\nonTerminalSection{property\_in\_extension}{7}

\ruleSubsection{plm\_syntax}{type-extension-declaration}{52}

\begin{tikzpicture}
  \matrix[column sep=\ruleMatrixColumnSeparation, row sep=\ruleMatrixRowSeparation] {
    & & & \node (p1-3) [terminal] {public}; & & & \node (p1-6) [terminal] {let}; & \\
    \node (P0start) [firstPoint] {}; & & \node (p0-2) [point] {}; & \node (p0-3) [point] {}; & \node (p0-4) [point] {}; & \node (p0-5) [point] {}; & \node (p0-6) [terminal] {var}; & \node (p0-7) [point] {}; & \node (p0-8) [terminal] {identifier}; & \node (p0-9) [nonterminal] {\nonTerminalSymbol{type\_definition}{3}}; & \node (p0-10) [terminal] {=}; & \node (p0-11) [nonterminal] {\nonTerminalSymbol{expression}{22}}; & \node (p0-12) [lastPoint] {}; & \\
  };
  \draw (P0start) -- (p0-3) ;
  \draw[->] (p0-2) |- (p1-3) ;
  \draw (p0-3) -- (p0-4) ;
  \draw[->] (p1-3) -| (p0-4) ;
  \draw[->] (p0-4) -- (p0-6) ;
  \draw[->] (p0-5) |- (p1-6) ;
  \draw (p0-6) -- (p0-7) ;
  \draw[->] (p1-6) -| (p0-7) ;
  \draw[->] (p0-7) -- (p0-8) ;
  \draw[->] (p0-8) -- (p0-9) ;
  \draw[->] (p0-9) -- (p0-10) ;
  \draw[->] (p0-10) -- (p0-11) ;
  \draw[->] (p0-11) -- (p0-12) ;
\end{tikzpicture}

\nonTerminalSection{registerDeclaration}{8}

\ruleSubsection{plm\_syntax}{declaration-control-register}{97}

\begin{tikzpicture}
  \matrix[column sep=\ruleMatrixColumnSeparation, row sep=\ruleMatrixRowSeparation] {
    & & & & & & & \node (p2-7) [point] {}; & \\
    & & & & & & \node (p1-6) [terminal] {@attribute}; & & & \node (p1-9) [terminal] {[}; & \node (p1-10) [nonterminal] {\nonTerminalSymbol{expression}{22}}; & \node (p1-11) [terminal] {]}; & \node (p1-12) [terminal] {at}; & \node (p1-13) [nonterminal] {\nonTerminalSymbol{expression}{22}}; & \node (p1-14) [terminal] {:}; & \node (p1-15) [nonterminal] {\nonTerminalSymbol{expression}{22}}; & \\
    \node (P0start) [firstPoint] {}; & & \node (p0-2) [terminal] {identifier}; & \node (p0-3) [point] {}; & \node (p0-4) [point] {}; & \node (p0-5) [point] {}; & & & \node (p0-8) [point] {}; & \node (p0-9) [terminal] {at}; & \node (p0-10) [nonterminal] {\nonTerminalSymbol{expression}{22}}; & & & & & & \node (p0-16) [point] {}; & \node (p0-17) [lastPoint] {}; & \\
  };
  \draw[->] (P0start) -- (p0-2) ;
  \draw (p0-2) -- (p0-4) ;
  \draw[->] (p0-5) |- (p1-6) ;
  \draw[->] (p2-7) -| (p0-3) ;
  \draw[->] (p1-6) -| (p2-7) ;
  \draw[->] (p0-4) -- (p0-9) ;
  \draw[->] (p0-9) -- (p0-10) ;
  \draw[->] (p0-8) |- (p1-9) ;
  \draw[->] (p1-9) -- (p1-10) ;
  \draw[->] (p1-10) -- (p1-11) ;
  \draw[->] (p1-11) -- (p1-12) ;
  \draw[->] (p1-12) -- (p1-13) ;
  \draw[->] (p1-13) -- (p1-14) ;
  \draw[->] (p1-14) -- (p1-15) ;
  \draw (p0-10) -- (p0-16) ;
  \draw[->] (p1-15) -| (p0-16) ;
  \draw[->] (p0-16) -- (p0-17) ;
\end{tikzpicture}

\nonTerminalSection{start\_symbol}{1}

\ruleSubsection{plm\_syntax}{syntax-grammar}{21}

\begin{tikzpicture}
  \matrix[column sep=\ruleMatrixColumnSeparation, row sep=\ruleMatrixRowSeparation] {
    & & & & & & \node (p5-6) [point] {}; & \\
    & & & & & \node (p4-5) [nonterminal] {\nonTerminalSymbol{import\_file}{0}}; & \\
    & & & & & \node (p3-5) [nonterminal] {\nonTerminalSymbol{system\_routine}{14}}; & \\
    & & & & & \node (p2-5) [nonterminal] {\nonTerminalSymbol{function}{16}}; & \\
    & & & & & \node (p1-5) [nonterminal] {\nonTerminalSymbol{declaration}{2}}; & \\
    \node (P0start) [firstPoint] {}; & & \node (p0-2) [point] {}; & \node (p0-3) [point] {}; & \node (p0-4) [point] {}; & & & \node (p0-7) [lastPoint] {}; & \\
  };
  \draw (P0start) -- (p0-3) ;
  \draw[->] (p0-4) |- (p1-5) ;
  \draw[->] (p0-4) |- (p2-5) ;
  \draw[->] (p0-4) |- (p3-5) ;
  \draw[->] (p0-4) |- (p4-5) ;
  \draw[->] (p5-6) -| (p0-2) ;
  \draw[->] (p1-5) -| (p5-6) ;
  \draw[->] (p2-5) -| (p5-6) ;
  \draw[->] (p3-5) -| (p5-6) ;
  \draw[->] (p4-5) -| (p5-6) ;
  \draw[->] (p0-3) -- (p0-7) ;
\end{tikzpicture}

\nonTerminalSection{staticArrayProperty}{10}

\ruleSubsection{plm\_syntax}{declaration-static-array}{75}

\begin{tikzpicture}
  \matrix[column sep=\ruleMatrixColumnSeparation, row sep=\ruleMatrixRowSeparation] {
    & & & & & & & & & \node (p2-9) [terminal] {->}; & \node (p2-10) [nonterminal] {\nonTerminalSymbol{type\_definition}{3}}; & \\
    & & & & & \node (p1-5) [terminal] {func}; & \node (p1-6) [nonterminal] {\nonTerminalSymbol{mode}{17}}; & \node (p1-7) [nonterminal] {\nonTerminalSymbol{procedure\_formal\_arguments}{19}}; & \node (p1-8) [point] {}; & \node (p1-9) [point] {}; & & \node (p1-11) [point] {}; & \\
    \node (P0start) [firstPoint] {}; & & \node (p0-2) [terminal] {let}; & \node (p0-3) [terminal] {identifier}; & \node (p0-4) [point] {}; & \node (p0-5) [nonterminal] {\nonTerminalSymbol{type\_definition}{3}}; & & & & & & & \node (p0-12) [point] {}; & \node (p0-13) [lastPoint] {}; & \\
  };
  \draw[->] (P0start) -- (p0-2) ;
  \draw[->] (p0-2) -- (p0-3) ;
  \draw[->] (p0-3) -- (p0-5) ;
  \draw[->] (p0-4) |- (p1-5) ;
  \draw[->] (p1-5) -- (p1-6) ;
  \draw[->] (p1-6) -- (p1-7) ;
  \draw (p1-7) -- (p1-9) ;
  \draw[->] (p1-8) |- (p2-9) ;
  \draw[->] (p2-9) -- (p2-10) ;
  \draw (p1-9) -- (p1-11) ;
  \draw[->] (p2-10) -| (p1-11) ;
  \draw (p0-5) -- (p0-12) ;
  \draw[->] (p1-11) -| (p0-12) ;
  \draw[->] (p0-12) -- (p0-13) ;
\end{tikzpicture}

\nonTerminalSection{staticArray\_exp}{11}

\ruleSubsection{plm\_syntax}{declaration-static-array}{121}

\begin{tikzpicture}
  \matrix[column sep=\ruleMatrixColumnSeparation, row sep=\ruleMatrixRowSeparation] {
    & & & \node (p1-3) [terminal] {func}; & \node (p1-4) [terminal] {identifier}; & \node (p1-5) [nonterminal] {\nonTerminalSymbol{procedure\_formal\_arguments}{19}}; & \\
    \node (P0start) [firstPoint] {}; & & \node (p0-2) [point] {}; & \node (p0-3) [nonterminal] {\nonTerminalSymbol{expression}{22}}; & & & \node (p0-6) [point] {}; & \node (p0-7) [lastPoint] {}; & \\
  };
  \draw[->] (P0start) -- (p0-3) ;
  \draw[->] (p0-2) |- (p1-3) ;
  \draw[->] (p1-3) -- (p1-4) ;
  \draw[->] (p1-4) -- (p1-5) ;
  \draw (p0-3) -- (p0-6) ;
  \draw[->] (p1-5) -| (p0-6) ;
  \draw[->] (p0-6) -- (p0-7) ;
\end{tikzpicture}

\nonTerminalSection{struct\_property\_declaration}{6}

\ruleSubsection{plm\_syntax}{type-structure-declaration}{59}

\begin{tikzpicture}
  \matrix[column sep=\ruleMatrixColumnSeparation, row sep=\ruleMatrixRowSeparation] {
    & & & & & & & \node (p2-7) [terminal] {=}; & \node (p2-8) [nonterminal] {\nonTerminalSymbol{expression}{22}}; & \\
    & & & \node (p1-3) [terminal] {let}; & & & & & & \node (p1-9) [terminal] {=}; & \node (p1-10) [nonterminal] {\nonTerminalSymbol{expression}{22}}; & \\
    \node (P0start) [firstPoint] {}; & & \node (p0-2) [point] {}; & \node (p0-3) [terminal] {var}; & \node (p0-4) [point] {}; & \node (p0-5) [terminal] {identifier}; & \node (p0-6) [point] {}; & \node (p0-7) [nonterminal] {\nonTerminalSymbol{type\_definition}{3}}; & \node (p0-8) [point] {}; & \node (p0-9) [point] {}; & & \node (p0-11) [point] {}; & \node (p0-12) [point] {}; & \node (p0-13) [lastPoint] {}; & \\
  };
  \draw[->] (P0start) -- (p0-3) ;
  \draw[->] (p0-2) |- (p1-3) ;
  \draw (p0-3) -- (p0-4) ;
  \draw[->] (p1-3) -| (p0-4) ;
  \draw[->] (p0-4) -- (p0-5) ;
  \draw[->] (p0-5) -- (p0-7) ;
  \draw (p0-7) -- (p0-9) ;
  \draw[->] (p0-8) |- (p1-9) ;
  \draw[->] (p1-9) -- (p1-10) ;
  \draw (p0-9) -- (p0-11) ;
  \draw[->] (p1-10) -| (p0-11) ;
  \draw[->] (p0-6) |- (p2-7) ;
  \draw[->] (p2-7) -- (p2-8) ;
  \draw (p0-11) -- (p0-12) ;
  \draw[->] (p2-8) -| (p0-12) ;
  \draw[->] (p0-12) -- (p0-13) ;
\end{tikzpicture}

\nonTerminalSection{system\_routine}{14}

\ruleSubsection{plm\_syntax}{declaration-svc}{24}

\begin{tikzpicture}
  \matrix[column sep=\ruleMatrixColumnSeparation, row sep=\ruleMatrixRowSeparation] {
    & & & & & & & \node (p3-7) [terminal] {service}; & \\
    & & & & & & & \node (p2-7) [terminal] {primitive}; & & & & & & & \node (p2-14) [point] {}; & \\
    & & & \node (p1-3) [terminal] {public}; & & & & \node (p1-7) [terminal] {safe}; & & & & & & \node (p1-13) [terminal] {@attribute}; & & & & \node (p1-17) [terminal] {->}; & \node (p1-18) [nonterminal] {\nonTerminalSymbol{type\_definition}{3}}; & \\
    \node (P0start) [firstPoint] {}; & & \node (p0-2) [point] {}; & \node (p0-3) [point] {}; & \node (p0-4) [point] {}; & \node (p0-5) [terminal] {system}; & \node (p0-6) [point] {}; & \node (p0-7) [terminal] {section}; & \node (p0-8) [point] {}; & \node (p0-9) [terminal] {identifier}; & \node (p0-10) [point] {}; & \node (p0-11) [point] {}; & \node (p0-12) [point] {}; & & & \node (p0-15) [nonterminal] {\nonTerminalSymbol{procedure\_formal\_arguments}{19}}; & \node (p0-16) [point] {}; & \node (p0-17) [point] {}; & & \node (p0-19) [point] {}; & \node (p0-20) [terminal] {\{}; & \node (p0-21) [nonterminal] {\nonTerminalSymbol{instructionList}{36}}; & \node (p0-22) [terminal] {\}}; & \node (p0-23) [lastPoint] {}; & \\
  };
  \draw (P0start) -- (p0-3) ;
  \draw[->] (p0-2) |- (p1-3) ;
  \draw (p0-3) -- (p0-4) ;
  \draw[->] (p1-3) -| (p0-4) ;
  \draw[->] (p0-4) -- (p0-5) ;
  \draw[->] (p0-5) -- (p0-7) ;
  \draw[->] (p0-6) |- (p1-7) ;
  \draw[->] (p0-6) |- (p2-7) ;
  \draw[->] (p0-6) |- (p3-7) ;
  \draw (p0-7) -- (p0-8) ;
  \draw[->] (p1-7) -| (p0-8) ;
  \draw[->] (p2-7) -| (p0-8) ;
  \draw[->] (p3-7) -| (p0-8) ;
  \draw[->] (p0-8) -- (p0-9) ;
  \draw (p0-9) -- (p0-11) ;
  \draw[->] (p0-12) |- (p1-13) ;
  \draw[->] (p2-14) -| (p0-10) ;
  \draw[->] (p1-13) -| (p2-14) ;
  \draw[->] (p0-11) -- (p0-15) ;
  \draw (p0-15) -- (p0-17) ;
  \draw[->] (p0-16) |- (p1-17) ;
  \draw[->] (p1-17) -- (p1-18) ;
  \draw (p0-17) -- (p0-19) ;
  \draw[->] (p1-18) -| (p0-19) ;
  \draw[->] (p0-19) -- (p0-20) ;
  \draw[->] (p0-20) -- (p0-21) ;
  \draw[->] (p0-21) -- (p0-22) ;
  \draw[->] (p0-22) -- (p0-23) ;
\end{tikzpicture}

\nonTerminalSection{task\_entry\_declaration}{12}

\ruleSubsection{plm\_syntax}{task-entry-declaration}{21}

\begin{tikzpicture}
  \matrix[column sep=\ruleMatrixColumnSeparation, row sep=\ruleMatrixRowSeparation] {
    & & & & & & & & & & & \node (p2-11) [point] {}; & \\
    & & & & & & & & & & \node (p1-10) [terminal] {.}; & & & & \node (p1-14) [terminal] {->}; & \node (p1-15) [nonterminal] {\nonTerminalSymbol{type\_definition}{3}}; & \\
    \node (P0start) [firstPoint] {}; & & \node (p0-2) [terminal] {entry}; & \node (p0-3) [terminal] {identifier}; & \node (p0-4) [terminal] {:}; & \node (p0-5) [terminal] {self}; & \node (p0-6) [terminal] {.}; & \node (p0-7) [point] {}; & \node (p0-8) [terminal] {identifier}; & \node (p0-9) [point] {}; & & & \node (p0-12) [nonterminal] {\nonTerminalSymbol{procedure\_formal\_arguments}{19}}; & \node (p0-13) [point] {}; & \node (p0-14) [point] {}; & & \node (p0-16) [point] {}; & \node (p0-17) [lastPoint] {}; & \\
  };
  \draw[->] (P0start) -- (p0-2) ;
  \draw[->] (p0-2) -- (p0-3) ;
  \draw[->] (p0-3) -- (p0-4) ;
  \draw[->] (p0-4) -- (p0-5) ;
  \draw[->] (p0-5) -- (p0-6) ;
  \draw[->] (p0-6) -- (p0-8) ;
  \draw[->] (p0-9) |- (p1-10) ;
  \draw[->] (p2-11) -| (p0-7) ;
  \draw[->] (p1-10) -| (p2-11) ;
  \draw[->] (p0-8) -- (p0-12) ;
  \draw (p0-12) -- (p0-14) ;
  \draw[->] (p0-13) |- (p1-14) ;
  \draw[->] (p1-14) -- (p1-15) ;
  \draw (p0-14) -- (p0-16) ;
  \draw[->] (p1-15) -| (p0-16) ;
  \draw[->] (p0-16) -- (p0-17) ;
\end{tikzpicture}

\nonTerminalSection{task\_guard\_declaration}{13}

\ruleSubsection{plm\_syntax}{task-guard-declaration}{20}

\begin{tikzpicture}
  \matrix[column sep=\ruleMatrixColumnSeparation, row sep=\ruleMatrixRowSeparation] {
    & & & & & & & & & & & \node (p2-11) [point] {}; & \\
    & & & & & & & & & & \node (p1-10) [terminal] {.}; & \\
    \node (P0start) [firstPoint] {}; & & \node (p0-2) [terminal] {guard}; & \node (p0-3) [terminal] {identifier}; & \node (p0-4) [terminal] {:}; & \node (p0-5) [terminal] {self}; & \node (p0-6) [terminal] {.}; & \node (p0-7) [point] {}; & \node (p0-8) [terminal] {identifier}; & \node (p0-9) [point] {}; & & & \node (p0-12) [nonterminal] {\nonTerminalSymbol{procedure\_formal\_arguments}{19}}; & \node (p0-13) [lastPoint] {}; & \\
  };
  \draw[->] (P0start) -- (p0-2) ;
  \draw[->] (p0-2) -- (p0-3) ;
  \draw[->] (p0-3) -- (p0-4) ;
  \draw[->] (p0-4) -- (p0-5) ;
  \draw[->] (p0-5) -- (p0-6) ;
  \draw[->] (p0-6) -- (p0-8) ;
  \draw[->] (p0-9) |- (p1-10) ;
  \draw[->] (p2-11) -| (p0-7) ;
  \draw[->] (p1-10) -| (p2-11) ;
  \draw[->] (p0-8) -- (p0-12) ;
  \draw[->] (p0-12) -- (p0-13) ;
\end{tikzpicture}

\nonTerminalSection{type\_definition}{3}

\ruleSubsection{plm\_syntax}{declaration-type}{47}

\begin{tikzpicture}
  \matrix[column sep=\ruleMatrixColumnSeparation, row sep=\ruleMatrixRowSeparation] {
    \node (P0start) [firstPoint] {}; & & \node (p0-2) [terminal] {\$type}; & \node (p0-3) [lastPoint] {}; & \\
  };
  \draw[->] (P0start) -- (p0-2) ;
  \draw[->] (p0-2) -- (p0-3) ;
\end{tikzpicture}

\ruleSubsection{plm\_syntax}{type-array}{20}

\begin{tikzpicture}
  \matrix[column sep=\ruleMatrixColumnSeparation, row sep=\ruleMatrixRowSeparation] {
    \node (P0start) [firstPoint] {}; & & \node (p0-2) [terminal] {\$type}; & \node (p0-3) [terminal] {[}; & \node (p0-4) [nonterminal] {\nonTerminalSymbol{expression}{22}}; & \node (p0-5) [terminal] {]}; & \node (p0-6) [lastPoint] {}; & \\
  };
  \draw[->] (P0start) -- (p0-2) ;
  \draw[->] (p0-2) -- (p0-3) ;
  \draw[->] (p0-3) -- (p0-4) ;
  \draw[->] (p0-4) -- (p0-5) ;
  \draw[->] (p0-5) -- (p0-6) ;
\end{tikzpicture}

\ruleSubsection{plm\_syntax}{type-opaque-declaration}{20}

\begin{tikzpicture}
  \matrix[column sep=\ruleMatrixColumnSeparation, row sep=\ruleMatrixRowSeparation] {
    & & & & & & & & & & & \node (p2-11) [point] {}; & \\
    & & & & & & & & & & \node (p1-10) [terminal] {@attribute}; & \\
    \node (P0start) [firstPoint] {}; & & \node (p0-2) [terminal] {[}; & \node (p0-3) [terminal] {[}; & \node (p0-4) [nonterminal] {\nonTerminalSymbol{expression}{22}}; & \node (p0-5) [terminal] {]}; & \node (p0-6) [terminal] {]}; & \node (p0-7) [point] {}; & \node (p0-8) [point] {}; & \node (p0-9) [point] {}; & & & \node (p0-12) [lastPoint] {}; & \\
  };
  \draw[->] (P0start) -- (p0-2) ;
  \draw[->] (p0-2) -- (p0-3) ;
  \draw[->] (p0-3) -- (p0-4) ;
  \draw[->] (p0-4) -- (p0-5) ;
  \draw[->] (p0-5) -- (p0-6) ;
  \draw (p0-6) -- (p0-8) ;
  \draw[->] (p0-9) |- (p1-10) ;
  \draw[->] (p2-11) -| (p0-7) ;
  \draw[->] (p1-10) -| (p2-11) ;
  \draw[->] (p0-8) -- (p0-12) ;
\end{tikzpicture}




\renewcommand\nonTerminalSection[2]{\subsection{Non terminal \texttt{\it#1}}\label{nt1:#2}}
\renewcommand\nonTerminalSummary[2]{\hyperref[nt1:#2]{#1}}
\renewcommand\nonTerminalSymbol[2]{\hyperref[nt1:#2]{#1}}
\renewcommand\startSymbol[2]{L'axiome de la grammaire est \hyperref[nt1:#2]{#1}.}

\sectionLabel{Grammaire du langage de description de cible}{grammaireCible}

\startSymbol{configuration\_start\_symbol}{1}

\nonTerminalSummaryStart \nonTerminalSummary{configuration\_start\_symbol}{1}\nonTerminalSummarySeparator \nonTerminalSummary{interruptConfigList}{2}\nonTerminalSummarySeparator \nonTerminalSummary{key}{0}\nonTerminalSummaryEnd \nonTerminalSection{configuration\_start\_symbol}{1}

\ruleSubsection{plm\_target\_specific\_syntax}{configuration}{95}

\begin{tikzpicture}
  \matrix[column sep=\ruleMatrixColumnSeparation, row sep=\ruleMatrixRowSeparation] {
    & & & & & & & & & & \node (p3-10) [point] {}; & & & & & & & & & & & & & & & & & & & & & & & & & & & & & & & & & & & & & & & & & & & & & & & & & & & & & & & & & & & & & & \node (p3-72) [point] {}; & \\
    & & & & & & & & & \node (p2-9) [terminal] {,}; & & & & & & & & & & & & & & & & & & & & & & & & & & & & & & & & & & & & & & & & & & & & \node (p2-53) [point] {}; & & & & & & \node (p2-59) [point] {}; & & & & & & \node (p2-65) [point] {}; & & & & & & \node (p2-71) [terminal] {,}; & \\
    & & & & \node (p1-4) [point] {}; & \node (p1-5) [terminal] {"string"}; & \node (p1-6) [terminal] {->}; & \node (p1-7) [terminal] {"string"}; & \node (p1-8) [point] {}; & & & & & & & & & & & & & & & & & & & & & & & & & & & & & & \node (p1-38) [terminal] {"string"}; & \node (p1-39) [terminal] {;}; & \node (p1-40) [terminal] {"string"}; & & & & & & & & & & & & \node (p1-52) [terminal] {,}; & & & & & & \node (p1-58) [terminal] {,}; & & & & & & \node (p1-64) [terminal] {,}; & & & & \node (p1-68) [point] {}; & \node (p1-69) [terminal] {"string"}; & \node (p1-70) [point] {}; & \\
    \node (P0start) [firstPoint] {}; & & \node (p0-2) [nonterminal] {\nonTerminalSymbol{key}{0}}; & \node (p0-3) [point] {}; & \node (p0-4) [point] {}; & & & & & & & \node (p0-11) [point] {}; & \node (p0-12) [nonterminal] {\nonTerminalSymbol{key}{0}}; & \node (p0-13) [terminal] {"string"}; & \node (p0-14) [nonterminal] {\nonTerminalSymbol{key}{0}}; & \node (p0-15) [terminal] {"string"}; & \node (p0-16) [nonterminal] {\nonTerminalSymbol{key}{0}}; & \node (p0-17) [terminal] {\$type}; & \node (p0-18) [terminal] {;}; & \node (p0-19) [terminal] {\$type}; & \node (p0-20) [terminal] {;}; & \node (p0-21) [terminal] {"string"}; & \node (p0-22) [nonterminal] {\nonTerminalSymbol{key}{0}}; & \node (p0-23) [terminal] {integer}; & \node (p0-24) [nonterminal] {\nonTerminalSymbol{key}{0}}; & \node (p0-25) [terminal] {integer}; & \node (p0-26) [nonterminal] {\nonTerminalSymbol{key}{0}}; & \node (p0-27) [terminal] {"string"}; & \node (p0-28) [terminal] {;}; & \node (p0-29) [terminal] {integer}; & \node (p0-30) [terminal] {;}; & \node (p0-31) [terminal] {"string"}; & \node (p0-32) [terminal] {;}; & \node (p0-33) [terminal] {"string"}; & \node (p0-34) [terminal] {;}; & \node (p0-35) [terminal] {"string"}; & \node (p0-36) [nonterminal] {\nonTerminalSymbol{key}{0}}; & \node (p0-37) [point] {}; & \node (p0-38) [terminal] {"string"}; & \node (p0-39) [terminal] {;}; & \node (p0-40) [terminal] {integer}; & \node (p0-41) [terminal] {;}; & \node (p0-42) [terminal] {"string"}; & \node (p0-43) [terminal] {;}; & \node (p0-44) [terminal] {"string"}; & \node (p0-45) [terminal] {;}; & \node (p0-46) [terminal] {"string"}; & \node (p0-47) [point] {}; & \node (p0-48) [nonterminal] {\nonTerminalSymbol{key}{0}}; & \node (p0-49) [point] {}; & \node (p0-50) [terminal] {"string"}; & \node (p0-51) [point] {}; & & & \node (p0-54) [nonterminal] {\nonTerminalSymbol{key}{0}}; & \node (p0-55) [point] {}; & \node (p0-56) [terminal] {"string"}; & \node (p0-57) [point] {}; & & & \node (p0-60) [nonterminal] {\nonTerminalSymbol{key}{0}}; & \node (p0-61) [point] {}; & \node (p0-62) [terminal] {"string"}; & \node (p0-63) [point] {}; & & & \node (p0-66) [nonterminal] {\nonTerminalSymbol{key}{0}}; & \node (p0-67) [point] {}; & \node (p0-68) [point] {}; & & & & & \node (p0-73) [point] {}; & \node (p0-74) [nonterminal] {\nonTerminalSymbol{key}{0}}; & \node (p0-75) [terminal] {"string"}; & \node (p0-76) [terminal] {;}; & \node (p0-77) [terminal] {"string"}; & \node (p0-78) [terminal] {;}; & \node (p0-79) [terminal] {integer}; & \node (p0-80) [nonterminal] {\nonTerminalSymbol{interruptConfigList}{2}}; & \node (p0-81) [lastPoint] {}; & \\
  };
  \draw[->] (P0start) -- (p0-2) ;
  \draw (p0-2) -- (p0-4) ;
  \draw[->] (p0-3) |- (p1-5) ;
  \draw[->] (p1-5) -- (p1-6) ;
  \draw[->] (p1-6) -- (p1-7) ;
  \draw[->] (p1-8) |- (p2-9) ;
  \draw[->] (p3-10) -| (p1-4) ;
  \draw[->] (p2-9) -| (p3-10) ;
  \draw (p0-4) -- (p0-11) ;
  \draw[->] (p1-7) -| (p0-11) ;
  \draw[->] (p0-11) -- (p0-12) ;
  \draw[->] (p0-12) -- (p0-13) ;
  \draw[->] (p0-13) -- (p0-14) ;
  \draw[->] (p0-14) -- (p0-15) ;
  \draw[->] (p0-15) -- (p0-16) ;
  \draw[->] (p0-16) -- (p0-17) ;
  \draw[->] (p0-17) -- (p0-18) ;
  \draw[->] (p0-18) -- (p0-19) ;
  \draw[->] (p0-19) -- (p0-20) ;
  \draw[->] (p0-20) -- (p0-21) ;
  \draw[->] (p0-21) -- (p0-22) ;
  \draw[->] (p0-22) -- (p0-23) ;
  \draw[->] (p0-23) -- (p0-24) ;
  \draw[->] (p0-24) -- (p0-25) ;
  \draw[->] (p0-25) -- (p0-26) ;
  \draw[->] (p0-26) -- (p0-27) ;
  \draw[->] (p0-27) -- (p0-28) ;
  \draw[->] (p0-28) -- (p0-29) ;
  \draw[->] (p0-29) -- (p0-30) ;
  \draw[->] (p0-30) -- (p0-31) ;
  \draw[->] (p0-31) -- (p0-32) ;
  \draw[->] (p0-32) -- (p0-33) ;
  \draw[->] (p0-33) -- (p0-34) ;
  \draw[->] (p0-34) -- (p0-35) ;
  \draw[->] (p0-35) -- (p0-36) ;
  \draw[->] (p0-36) -- (p0-38) ;
  \draw[->] (p0-38) -- (p0-39) ;
  \draw[->] (p0-39) -- (p0-40) ;
  \draw[->] (p0-40) -- (p0-41) ;
  \draw[->] (p0-41) -- (p0-42) ;
  \draw[->] (p0-42) -- (p0-43) ;
  \draw[->] (p0-43) -- (p0-44) ;
  \draw[->] (p0-44) -- (p0-45) ;
  \draw[->] (p0-45) -- (p0-46) ;
  \draw[->] (p0-37) |- (p1-38) ;
  \draw[->] (p1-38) -- (p1-39) ;
  \draw[->] (p1-39) -- (p1-40) ;
  \draw (p0-46) -- (p0-47) ;
  \draw[->] (p1-40) -| (p0-47) ;
  \draw[->] (p0-47) -- (p0-48) ;
  \draw[->] (p0-48) -- (p0-50) ;
  \draw[->] (p0-51) |- (p1-52) ;
  \draw[->] (p2-53) -| (p0-49) ;
  \draw[->] (p1-52) -| (p2-53) ;
  \draw[->] (p0-50) -- (p0-54) ;
  \draw[->] (p0-54) -- (p0-56) ;
  \draw[->] (p0-57) |- (p1-58) ;
  \draw[->] (p2-59) -| (p0-55) ;
  \draw[->] (p1-58) -| (p2-59) ;
  \draw[->] (p0-56) -- (p0-60) ;
  \draw[->] (p0-60) -- (p0-62) ;
  \draw[->] (p0-63) |- (p1-64) ;
  \draw[->] (p2-65) -| (p0-61) ;
  \draw[->] (p1-64) -| (p2-65) ;
  \draw[->] (p0-62) -- (p0-66) ;
  \draw (p0-66) -- (p0-68) ;
  \draw[->] (p0-67) |- (p1-69) ;
  \draw[->] (p1-70) |- (p2-71) ;
  \draw[->] (p3-72) -| (p1-68) ;
  \draw[->] (p2-71) -| (p3-72) ;
  \draw (p0-68) -- (p0-73) ;
  \draw[->] (p1-69) -| (p0-73) ;
  \draw[->] (p0-73) -- (p0-74) ;
  \draw[->] (p0-74) -- (p0-75) ;
  \draw[->] (p0-75) -- (p0-76) ;
  \draw[->] (p0-76) -- (p0-77) ;
  \draw[->] (p0-77) -- (p0-78) ;
  \draw[->] (p0-78) -- (p0-79) ;
  \draw[->] (p0-79) -- (p0-80) ;
  \draw[->] (p0-80) -- (p0-81) ;
\end{tikzpicture}

\nonTerminalSection{interruptConfigList}{2}

\ruleSubsection{plm\_target\_specific\_syntax}{configuration}{239}

\begin{tikzpicture}
  \matrix[column sep=\ruleMatrixColumnSeparation, row sep=\ruleMatrixRowSeparation] {
    & & & & & & & & & & & & \node (p3-12) [point] {}; & \\
    & & & & & & & \node (p2-7) [terminal] {->}; & \node (p2-8) [terminal] {integer}; & & & \node (p2-11) [terminal] {,}; & \\
    & & & & \node (p1-4) [point] {}; & \node (p1-5) [terminal] {identifier}; & \node (p1-6) [point] {}; & \node (p1-7) [point] {}; & & \node (p1-9) [point] {}; & \node (p1-10) [point] {}; & \\
    \node (P0start) [firstPoint] {}; & & \node (p0-2) [nonterminal] {\nonTerminalSymbol{key}{0}}; & \node (p0-3) [point] {}; & \node (p0-4) [point] {}; & & & & & & & & & \node (p0-13) [point] {}; & \node (p0-14) [lastPoint] {}; & \\
  };
  \draw[->] (P0start) -- (p0-2) ;
  \draw (p0-2) -- (p0-4) ;
  \draw[->] (p0-3) |- (p1-5) ;
  \draw (p1-5) -- (p1-7) ;
  \draw[->] (p1-6) |- (p2-7) ;
  \draw[->] (p2-7) -- (p2-8) ;
  \draw (p1-7) -- (p1-9) ;
  \draw[->] (p2-8) -| (p1-9) ;
  \draw[->] (p1-10) |- (p2-11) ;
  \draw[->] (p3-12) -| (p1-4) ;
  \draw[->] (p2-11) -| (p3-12) ;
  \draw (p0-4) -- (p0-13) ;
  \draw[->] (p1-9) -| (p0-13) ;
  \draw[->] (p0-13) -- (p0-14) ;
\end{tikzpicture}

\nonTerminalSection{key}{0}

\ruleSubsection{plm\_target\_specific\_syntax}{configuration}{85}

\begin{tikzpicture}
  \matrix[column sep=\ruleMatrixColumnSeparation, row sep=\ruleMatrixRowSeparation] {
    \node (P0start) [firstPoint] {}; & & \node (p0-2) [terminal] {identifier}; & \node (p0-3) [terminal] {:}; & \node (p0-4) [lastPoint] {}; & \\
  };
  \draw[->] (P0start) -- (p0-2) ;
  \draw[->] (p0-2) -- (p0-3) ;
  \draw[->] (p0-3) -- (p0-4) ;
\end{tikzpicture}


}
