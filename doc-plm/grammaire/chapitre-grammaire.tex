%!TEX encoding = UTF-8 Unicode
%!TEX root = ../doc-plm.tex





\chapter{Grammaire}

{

\tikzset{
  nonterminal/.style={
    % The shape:
    rectangle,
    % The size:
    minimum size=6mm,
    % The border:
    very thick,
    draw=red!50!black!50,         % 50% red and 50% black,
                                  % and that mixed with 50% white
    % The filling:
    top color=white,              % a shading that is white at the top...
    bottom color=red!50!black!20, % and something else at the bottom
    % Font
    font=\itshape\footnotesize
  },
  terminal/.style={
    % The shape:
    rounded rectangle,
    minimum size=6mm,
    % The rest
    very thick,draw=black!50,
    top color=white,bottom color=black!20,
    font=\ttfamily\footnotesize
  },
  firstPoint/.style={circle,>=stealth',thick,draw=black!50},
  point/.style={coordinate,>=stealth',thick,draw=black!50},
  tip/.style={->,shorten >=0.007pt},
  lastPoint/.style={rectangle,>=stealth',thick,draw=black!50},
  every join/.style={rounded corners}
}

\newcommand\nonTerminalSection[2]{\section{Non terminal \texttt{\it#1}}\label{nt:#2}}
\newcommand\ruleSubsection[3]{} % \subsection{Component \texttt{#1}, in file \texttt{#2}, line #3}}
\newcommand\ruleMatrixColumnSeparation{2mm}
\newcommand\ruleMatrixRowSeparation{1.5mm}
\newcommand\nonTerminalSymbol[2]{\hyperref[nt:#2]{#1}}
\newcommand\startSymbol[2]{L'axiome de la grammaire est \hyperref[nt:#2]{#1}.}

\newcommand\nonTerminalSummaryStart{Voici la liste alphabétique des non terminaux: }
\newcommand\nonTerminalSummary[2]{\hyperref[nt:#2]{#1}}
\newcommand\nonTerminalSummarySeparator{, }
\newcommand\nonTerminalSummaryEnd{.\\}

\startSymbol{start\_symbol}{6}

\nonTerminalSummaryStart \nonTerminalSummary{assignment\_operator}{33}\nonTerminalSummarySeparator \nonTerminalSummary{assignment\_target}{37}\nonTerminalSummarySeparator \nonTerminalSummary{declaration}{7}\nonTerminalSummarySeparator \nonTerminalSummary{declaration\_init}{12}\nonTerminalSummarySeparator \nonTerminalSummary{declaration\_struct\_var}{9}\nonTerminalSummarySeparator \nonTerminalSummary{declaration\_type}{8}\nonTerminalSummarySeparator \nonTerminalSummary{effective\_parameters}{30}\nonTerminalSummarySeparator \nonTerminalSummary{expression}{16}\nonTerminalSummarySeparator \nonTerminalSummary{expression\_1}{28}\nonTerminalSummarySeparator \nonTerminalSummary{expression\_10}{19}\nonTerminalSummarySeparator \nonTerminalSummary{expression\_11}{18}\nonTerminalSummarySeparator \nonTerminalSummary{expression\_12}{17}\nonTerminalSummarySeparator \nonTerminalSummary{expression\_2}{27}\nonTerminalSummarySeparator \nonTerminalSummary{expression\_3}{26}\nonTerminalSummarySeparator \nonTerminalSummary{expression\_4}{25}\nonTerminalSummarySeparator \nonTerminalSummary{expression\_5}{24}\nonTerminalSummarySeparator \nonTerminalSummary{expression\_6}{23}\nonTerminalSummarySeparator \nonTerminalSummary{expression\_7}{22}\nonTerminalSummarySeparator \nonTerminalSummary{expression\_8}{21}\nonTerminalSummarySeparator \nonTerminalSummary{expression\_9}{20}\nonTerminalSummarySeparator \nonTerminalSummary{global\_variable\_declaration}{10}\nonTerminalSummarySeparator \nonTerminalSummary{guard}{15}\nonTerminalSummarySeparator \nonTerminalSummary{guarded\_command}{35}\nonTerminalSummarySeparator \nonTerminalSummary{if\_instruction}{34}\nonTerminalSummarySeparator \nonTerminalSummary{import\_file}{5}\nonTerminalSummarySeparator \nonTerminalSummary{instruction}{32}\nonTerminalSummarySeparator \nonTerminalSummary{instructionList}{31}\nonTerminalSummarySeparator \nonTerminalSummary{isr}{4}\nonTerminalSummarySeparator \nonTerminalSummary{module\_variable}{11}\nonTerminalSummarySeparator \nonTerminalSummary{primary}{29}\nonTerminalSummarySeparator \nonTerminalSummary{primitive}{3}\nonTerminalSummarySeparator \nonTerminalSummary{procedure}{0}\nonTerminalSummarySeparator \nonTerminalSummary{procedure\_call}{36}\nonTerminalSummarySeparator \nonTerminalSummary{procedure\_formal\_arguments}{14}\nonTerminalSummarySeparator \nonTerminalSummary{procedure\_header}{13}\nonTerminalSummarySeparator \nonTerminalSummary{section}{1}\nonTerminalSummarySeparator \nonTerminalSummary{service}{2}\nonTerminalSummarySeparator \nonTerminalSummary{start\_symbol}{6}\nonTerminalSummaryEnd \nonTerminalSection{assignment\_operator}{33}

\ruleSubsection{plm\_syntax}{instruction-assignment-operator}{37}

\begin{tikzpicture}
  \matrix[column sep=\ruleMatrixColumnSeparation, row sep=\ruleMatrixRowSeparation] {
    & & & \node (p8-3) [terminal] {*\verb=%==}; & \\
    & & & \node (p7-3) [terminal] {*=}; & \\
    & & & \node (p6-3) [terminal] {-\verb=%==}; & \\
    & & & \node (p5-3) [terminal] {-=}; & \\
    & & & \node (p4-3) [terminal] {+\verb=%==}; & \\
    & & & \node (p3-3) [terminal] {+=}; & \\
    & & & \node (p2-3) [terminal] {\verb=^==}; & \\
    & & & \node (p1-3) [terminal] {\&=}; & \\
    \node (P0start) [firstPoint] {}; & & \node (p0-2) [point] {}; & \node (p0-3) [terminal] {|=}; & \node (p0-4) [point] {}; & \node (p0-5) [lastPoint] {}; & \\
  };
  \draw[->] (P0start) -- (p0-3) ;
  \draw[->] (p0-2) |- (p1-3) ;
  \draw[->] (p0-2) |- (p2-3) ;
  \draw[->] (p0-2) |- (p3-3) ;
  \draw[->] (p0-2) |- (p4-3) ;
  \draw[->] (p0-2) |- (p5-3) ;
  \draw[->] (p0-2) |- (p6-3) ;
  \draw[->] (p0-2) |- (p7-3) ;
  \draw[->] (p0-2) |- (p8-3) ;
  \draw (p0-3) -- (p0-4) ;
  \draw[->] (p1-3) -| (p0-4) ;
  \draw[->] (p2-3) -| (p0-4) ;
  \draw[->] (p3-3) -| (p0-4) ;
  \draw[->] (p4-3) -| (p0-4) ;
  \draw[->] (p5-3) -| (p0-4) ;
  \draw[->] (p6-3) -| (p0-4) ;
  \draw[->] (p7-3) -| (p0-4) ;
  \draw[->] (p8-3) -| (p0-4) ;
  \draw[->] (p0-4) -- (p0-5) ;
\end{tikzpicture}

\nonTerminalSection{assignment\_target}{37}

\ruleSubsection{plm\_syntax}{assignment-target}{39}

\begin{tikzpicture}
  \matrix[column sep=\ruleMatrixColumnSeparation, row sep=\ruleMatrixRowSeparation] {
    & & & & & & & & & & & & & \node (p3-13) [point] {}; & \\
    & & & & & & & & & & \node (p2-10) [terminal] {[}; & \node (p2-11) [nonterminal] {\nonTerminalSymbol{expression}{16}}; & \node (p2-12) [terminal] {]}; & \\
    & & & \node (p1-3) [terminal] {self}; & \node (p1-4) [terminal] {.}; & & & & & & \node (p1-10) [terminal] {.}; & \node (p1-11) [terminal] {identifier}; & \\
    \node (P0start) [firstPoint] {}; & & \node (p0-2) [point] {}; & \node (p0-3) [point] {}; & & \node (p0-5) [point] {}; & \node (p0-6) [terminal] {identifier}; & \node (p0-7) [point] {}; & \node (p0-8) [point] {}; & \node (p0-9) [point] {}; & & & & & \node (p0-14) [lastPoint] {}; & \\
  };
  \draw (P0start) -- (p0-3) ;
  \draw[->] (p0-2) |- (p1-3) ;
  \draw[->] (p1-3) -- (p1-4) ;
  \draw (p0-3) -- (p0-5) ;
  \draw[->] (p1-4) -| (p0-5) ;
  \draw[->] (p0-5) -- (p0-6) ;
  \draw (p0-6) -- (p0-8) ;
  \draw[->] (p0-9) |- (p1-10) ;
  \draw[->] (p1-10) -- (p1-11) ;
  \draw[->] (p0-9) |- (p2-10) ;
  \draw[->] (p2-10) -- (p2-11) ;
  \draw[->] (p2-11) -- (p2-12) ;
  \draw[->] (p3-13) -| (p0-7) ;
  \draw[->] (p1-11) -| (p3-13) ;
  \draw[->] (p2-12) -| (p3-13) ;
  \draw[->] (p0-8) -- (p0-14) ;
\end{tikzpicture}

\nonTerminalSection{declaration}{7}

\ruleSubsection{plm\_syntax}{declaration-type}{9}

\begin{tikzpicture}
  \matrix[column sep=\ruleMatrixColumnSeparation, row sep=\ruleMatrixRowSeparation] {
    \node (P0start) [firstPoint] {}; & & \node (p0-2) [terminal] {type}; & \node (p0-3) [terminal] {\$type}; & \node (p0-4) [terminal] {:}; & \node (p0-5) [nonterminal] {\nonTerminalSymbol{declaration\_type}{8}}; & \node (p0-6) [lastPoint] {}; & \\
  };
  \draw[->] (P0start) -- (p0-2) ;
  \draw[->] (p0-2) -- (p0-3) ;
  \draw[->] (p0-3) -- (p0-4) ;
  \draw[->] (p0-4) -- (p0-5) ;
  \draw[->] (p0-5) -- (p0-6) ;
\end{tikzpicture}

\ruleSubsection{plm\_syntax}{type-enumeration-declaration}{25}

\begin{tikzpicture}
  \matrix[column sep=\ruleMatrixColumnSeparation, row sep=\ruleMatrixRowSeparation] {
    & & & & & & & & & & \node (p2-10) [point] {}; & \\
    & & & & & & & & & \node (p1-9) [point] {}; & \\
    \node (P0start) [firstPoint] {}; & & \node (p0-2) [terminal] {enum}; & \node (p0-3) [terminal] {\$type}; & \node (p0-4) [terminal] {\{}; & \node (p0-5) [point] {}; & \node (p0-6) [terminal] {case}; & \node (p0-7) [terminal] {identifier}; & \node (p0-8) [point] {}; & & & \node (p0-11) [terminal] {\}}; & \node (p0-12) [lastPoint] {}; & \\
  };
  \draw[->] (P0start) -- (p0-2) ;
  \draw[->] (p0-2) -- (p0-3) ;
  \draw[->] (p0-3) -- (p0-4) ;
  \draw[->] (p0-4) -- (p0-6) ;
  \draw[->] (p0-6) -- (p0-7) ;
  \draw (p0-8) |- (p1-9) ;
  \draw[->] (p2-10) -| (p0-5) ;
  \draw[->] (p1-9) -| (p2-10) ;
  \draw[->] (p0-7) -- (p0-11) ;
  \draw[->] (p0-11) -- (p0-12) ;
\end{tikzpicture}

\ruleSubsection{plm\_syntax}{type-structure-declaration}{74}

\begin{tikzpicture}
  \matrix[column sep=\ruleMatrixColumnSeparation, row sep=\ruleMatrixRowSeparation] {
    & & & & & & & & & & & & & & \node (p8-14) [point] {}; & \\
    & & & & & & & & & & & & & \node (p7-13) [terminal] {;}; & \\
    & & & & & & & & & & & & & \node (p6-13) [nonterminal] {\nonTerminalSymbol{primitive}{3}}; & \\
    & & & & & & & & & & & & & \node (p5-13) [nonterminal] {\nonTerminalSymbol{guard}{15}}; & \\
    & & & & & & & & & & & & & \node (p4-13) [nonterminal] {\nonTerminalSymbol{service}{2}}; & \\
    & & & & & & & & & & & & & \node (p3-13) [nonterminal] {\nonTerminalSymbol{section}{1}}; & \\
    & & & & & & & & \node (p2-8) [point] {}; & & & & & \node (p2-13) [nonterminal] {\nonTerminalSymbol{procedure}{0}}; & \\
    & & & & & & & \node (p1-7) [terminal] {@attribute}; & & & & & & \node (p1-13) [nonterminal] {\nonTerminalSymbol{declaration\_struct\_var}{9}}; & \\
    \node (P0start) [firstPoint] {}; & & \node (p0-2) [terminal] {struct}; & \node (p0-3) [terminal] {\$type}; & \node (p0-4) [point] {}; & \node (p0-5) [point] {}; & \node (p0-6) [point] {}; & & & \node (p0-9) [terminal] {\{}; & \node (p0-10) [point] {}; & \node (p0-11) [point] {}; & \node (p0-12) [point] {}; & & & \node (p0-15) [terminal] {\}}; & \node (p0-16) [lastPoint] {}; & \\
  };
  \draw[->] (P0start) -- (p0-2) ;
  \draw[->] (p0-2) -- (p0-3) ;
  \draw (p0-3) -- (p0-5) ;
  \draw[->] (p0-6) |- (p1-7) ;
  \draw[->] (p2-8) -| (p0-4) ;
  \draw[->] (p1-7) -| (p2-8) ;
  \draw[->] (p0-5) -- (p0-9) ;
  \draw (p0-9) -- (p0-11) ;
  \draw[->] (p0-12) |- (p1-13) ;
  \draw[->] (p0-12) |- (p2-13) ;
  \draw[->] (p0-12) |- (p3-13) ;
  \draw[->] (p0-12) |- (p4-13) ;
  \draw[->] (p0-12) |- (p5-13) ;
  \draw[->] (p0-12) |- (p6-13) ;
  \draw[->] (p0-12) |- (p7-13) ;
  \draw[->] (p8-14) -| (p0-10) ;
  \draw[->] (p1-13) -| (p8-14) ;
  \draw[->] (p2-13) -| (p8-14) ;
  \draw[->] (p3-13) -| (p8-14) ;
  \draw[->] (p4-13) -| (p8-14) ;
  \draw[->] (p5-13) -| (p8-14) ;
  \draw[->] (p6-13) -| (p8-14) ;
  \draw[->] (p7-13) -| (p8-14) ;
  \draw[->] (p0-11) -- (p0-15) ;
  \draw[->] (p0-15) -- (p0-16) ;
\end{tikzpicture}

\ruleSubsection{plm\_syntax}{type-extension-declaration}{23}

\begin{tikzpicture}
  \matrix[column sep=\ruleMatrixColumnSeparation, row sep=\ruleMatrixRowSeparation] {
    & & & & & & & & & & & & & & & & \node (p9-16) [point] {}; & \\
    & & & & & & & & \node (p8-8) [terminal] {;}; & \\
    & & & & & & & & \node (p7-8) [nonterminal] {\nonTerminalSymbol{primitive}{3}}; & \\
    & & & & & & & & \node (p6-8) [nonterminal] {\nonTerminalSymbol{guard}{15}}; & \\
    & & & & & & & & \node (p5-8) [nonterminal] {\nonTerminalSymbol{service}{2}}; & \\
    & & & & & & & & \node (p4-8) [nonterminal] {\nonTerminalSymbol{section}{1}}; & \\
    & & & & & & & & \node (p3-8) [nonterminal] {\nonTerminalSymbol{procedure}{0}}; & \\
    & & & & & & & & & \node (p2-9) [terminal] {public}; & \\
    & & & & & & & & \node (p1-8) [point] {}; & \node (p1-9) [point] {}; & \node (p1-10) [point] {}; & \node (p1-11) [terminal] {var}; & \node (p1-12) [terminal] {identifier}; & \node (p1-13) [terminal] {\$type}; & \node (p1-14) [terminal] {=}; & \node (p1-15) [nonterminal] {\nonTerminalSymbol{expression}{16}}; & \\
    \node (P0start) [firstPoint] {}; & & \node (p0-2) [terminal] {extension}; & \node (p0-3) [terminal] {\$type}; & \node (p0-4) [terminal] {\{}; & \node (p0-5) [point] {}; & \node (p0-6) [point] {}; & \node (p0-7) [point] {}; & & & & & & & & & & \node (p0-17) [terminal] {\}}; & \node (p0-18) [lastPoint] {}; & \\
  };
  \draw[->] (P0start) -- (p0-2) ;
  \draw[->] (p0-2) -- (p0-3) ;
  \draw[->] (p0-3) -- (p0-4) ;
  \draw (p0-4) -- (p0-6) ;
  \draw (p0-7) |- (p1-9) ;
  \draw[->] (p1-8) |- (p2-9) ;
  \draw (p1-9) -- (p1-10) ;
  \draw[->] (p2-9) -| (p1-10) ;
  \draw[->] (p1-10) -- (p1-11) ;
  \draw[->] (p1-11) -- (p1-12) ;
  \draw[->] (p1-12) -- (p1-13) ;
  \draw[->] (p1-13) -- (p1-14) ;
  \draw[->] (p1-14) -- (p1-15) ;
  \draw[->] (p0-7) |- (p3-8) ;
  \draw[->] (p0-7) |- (p4-8) ;
  \draw[->] (p0-7) |- (p5-8) ;
  \draw[->] (p0-7) |- (p6-8) ;
  \draw[->] (p0-7) |- (p7-8) ;
  \draw[->] (p0-7) |- (p8-8) ;
  \draw[->] (p9-16) -| (p0-5) ;
  \draw[->] (p1-15) -| (p9-16) ;
  \draw[->] (p3-8) -| (p9-16) ;
  \draw[->] (p4-8) -| (p9-16) ;
  \draw[->] (p5-8) -| (p9-16) ;
  \draw[->] (p6-8) -| (p9-16) ;
  \draw[->] (p7-8) -| (p9-16) ;
  \draw[->] (p8-8) -| (p9-16) ;
  \draw[->] (p0-6) -- (p0-17) ;
  \draw[->] (p0-17) -- (p0-18) ;
\end{tikzpicture}

\ruleSubsection{plm\_syntax}{declaration-control-register}{54}

\begin{tikzpicture}
  \matrix[column sep=\ruleMatrixColumnSeparation, row sep=\ruleMatrixRowSeparation] {
    & & & & & & & & & & & & & & & & & & & & & & & & & & & & & & & & & & & & \node (p4-36) [point] {}; & \\
    & & & & & & & & & & & & & & & & & & & & & \node (p3-21) [point] {}; & & & & & & & & \node (p3-29) [terminal] {[}; & \node (p3-30) [terminal] {integer}; & \node (p3-31) [terminal] {]}; & \\
    & & & & & & & & & \node (p2-9) [point] {}; & & & & & & & & & & & & & & & & & & \node (p2-27) [terminal] {identifier}; & \node (p2-28) [point] {}; & \node (p2-29) [point] {}; & & & \node (p2-32) [point] {}; & & & \node (p2-35) [terminal] {,}; & \\
    & & & & & & & & \node (p1-8) [terminal] {@attribute}; & & & \node (p1-11) [terminal] {[}; & \node (p1-12) [nonterminal] {\nonTerminalSymbol{expression}{16}}; & \node (p1-13) [terminal] {]}; & \node (p1-14) [terminal] {at}; & \node (p1-15) [nonterminal] {\nonTerminalSymbol{expression}{16}}; & \node (p1-16) [terminal] {:}; & \node (p1-17) [nonterminal] {\nonTerminalSymbol{expression}{16}}; & & & \node (p1-20) [point] {}; & & & & \node (p1-24) [terminal] {\{}; & \node (p1-25) [point] {}; & \node (p1-26) [point] {}; & \node (p1-27) [terminal] {integer}; & & & & & & \node (p1-33) [point] {}; & \node (p1-34) [point] {}; & & & \node (p1-37) [terminal] {\}}; & \\
    \node (P0start) [firstPoint] {}; & & \node (p0-2) [terminal] {register}; & \node (p0-3) [point] {}; & \node (p0-4) [terminal] {identifier}; & \node (p0-5) [point] {}; & \node (p0-6) [point] {}; & \node (p0-7) [point] {}; & & & \node (p0-10) [point] {}; & \node (p0-11) [terminal] {at}; & \node (p0-12) [nonterminal] {\nonTerminalSymbol{expression}{16}}; & & & & & & \node (p0-18) [point] {}; & \node (p0-19) [point] {}; & & & \node (p0-22) [terminal] {\$type}; & \node (p0-23) [point] {}; & \node (p0-24) [point] {}; & & & & & & & & & & & & & & \node (p0-38) [point] {}; & \node (p0-39) [lastPoint] {}; & \\
  };
  \draw[->] (P0start) -- (p0-2) ;
  \draw[->] (p0-2) -- (p0-4) ;
  \draw (p0-4) -- (p0-6) ;
  \draw[->] (p0-7) |- (p1-8) ;
  \draw[->] (p2-9) -| (p0-5) ;
  \draw[->] (p1-8) -| (p2-9) ;
  \draw[->] (p0-6) -- (p0-11) ;
  \draw[->] (p0-11) -- (p0-12) ;
  \draw[->] (p0-10) |- (p1-11) ;
  \draw[->] (p1-11) -- (p1-12) ;
  \draw[->] (p1-12) -- (p1-13) ;
  \draw[->] (p1-13) -- (p1-14) ;
  \draw[->] (p1-14) -- (p1-15) ;
  \draw[->] (p1-15) -- (p1-16) ;
  \draw[->] (p1-16) -- (p1-17) ;
  \draw (p0-12) -- (p0-18) ;
  \draw[->] (p1-17) -| (p0-18) ;
  \draw (p0-19) |- (p1-20) ;
  \draw[->] (p3-21) -| (p0-3) ;
  \draw[->] (p1-20) -| (p3-21) ;
  \draw[->] (p0-18) -- (p0-22) ;
  \draw (p0-22) -- (p0-24) ;
  \draw[->] (p0-23) |- (p1-24) ;
  \draw[->] (p1-24) -- (p1-27) ;
  \draw[->] (p1-26) |- (p2-27) ;
  \draw (p2-27) -- (p2-29) ;
  \draw[->] (p2-28) |- (p3-29) ;
  \draw[->] (p3-29) -- (p3-30) ;
  \draw[->] (p3-30) -- (p3-31) ;
  \draw (p2-29) -- (p2-32) ;
  \draw[->] (p3-31) -| (p2-32) ;
  \draw (p1-27) -- (p1-33) ;
  \draw[->] (p2-32) -| (p1-33) ;
  \draw[->] (p1-34) |- (p2-35) ;
  \draw[->] (p4-36) -| (p1-25) ;
  \draw[->] (p2-35) -| (p4-36) ;
  \draw[->] (p1-33) -- (p1-37) ;
  \draw (p0-24) -- (p0-38) ;
  \draw[->] (p1-37) -| (p0-38) ;
  \draw[->] (p0-38) -- (p0-39) ;
\end{tikzpicture}

\ruleSubsection{plm\_syntax}{declaration-global-constant}{25}

\begin{tikzpicture}
  \matrix[column sep=\ruleMatrixColumnSeparation, row sep=\ruleMatrixRowSeparation] {
    & & & & & \node (p1-5) [terminal] {\$type}; & \\
    \node (P0start) [firstPoint] {}; & & \node (p0-2) [terminal] {let}; & \node (p0-3) [terminal] {identifier}; & \node (p0-4) [point] {}; & \node (p0-5) [point] {}; & \node (p0-6) [point] {}; & \node (p0-7) [terminal] {=}; & \node (p0-8) [nonterminal] {\nonTerminalSymbol{expression}{16}}; & \node (p0-9) [lastPoint] {}; & \\
  };
  \draw[->] (P0start) -- (p0-2) ;
  \draw[->] (p0-2) -- (p0-3) ;
  \draw (p0-3) -- (p0-5) ;
  \draw[->] (p0-4) |- (p1-5) ;
  \draw (p0-5) -- (p0-6) ;
  \draw[->] (p1-5) -| (p0-6) ;
  \draw[->] (p0-6) -- (p0-7) ;
  \draw[->] (p0-7) -- (p0-8) ;
  \draw[->] (p0-8) -- (p0-9) ;
\end{tikzpicture}

\ruleSubsection{plm\_syntax}{declaration-global-variable}{71}

\begin{tikzpicture}
  \matrix[column sep=\ruleMatrixColumnSeparation, row sep=\ruleMatrixRowSeparation] {
    \node (P0start) [firstPoint] {}; & & \node (p0-2) [nonterminal] {\nonTerminalSymbol{global\_variable\_declaration}{10}}; & \node (p0-3) [lastPoint] {}; & \\
  };
  \draw[->] (P0start) -- (p0-2) ;
  \draw[->] (p0-2) -- (p0-3) ;
\end{tikzpicture}

\ruleSubsection{plm\_syntax}{declaration-module}{33}

\begin{tikzpicture}
  \matrix[column sep=\ruleMatrixColumnSeparation, row sep=\ruleMatrixRowSeparation] {
    & & & & & & & & & \node (p10-9) [point] {}; & \\
    & & & & & & & & \node (p9-8) [terminal] {;}; & \\
    & & & & & & & & \node (p8-8) [nonterminal] {\nonTerminalSymbol{primitive}{3}}; & \\
    & & & & & & & & \node (p7-8) [nonterminal] {\nonTerminalSymbol{guard}{15}}; & \\
    & & & & & & & & \node (p6-8) [nonterminal] {\nonTerminalSymbol{section}{1}}; & \\
    & & & & & & & & \node (p5-8) [nonterminal] {\nonTerminalSymbol{service}{2}}; & \\
    & & & & & & & & \node (p4-8) [nonterminal] {\nonTerminalSymbol{procedure}{0}}; & \\
    & & & & & & & & \node (p3-8) [nonterminal] {\nonTerminalSymbol{module\_variable}{11}}; & \\
    & & & & & & & & \node (p2-8) [nonterminal] {\nonTerminalSymbol{isr}{4}}; & \\
    & & & & & & & & \node (p1-8) [nonterminal] {\nonTerminalSymbol{declaration\_init}{12}}; & \\
    \node (P0start) [firstPoint] {}; & & \node (p0-2) [terminal] {module}; & \node (p0-3) [terminal] {identifier}; & \node (p0-4) [terminal] {\{}; & \node (p0-5) [point] {}; & \node (p0-6) [point] {}; & \node (p0-7) [point] {}; & & & \node (p0-10) [terminal] {\}}; & \node (p0-11) [lastPoint] {}; & \\
  };
  \draw[->] (P0start) -- (p0-2) ;
  \draw[->] (p0-2) -- (p0-3) ;
  \draw[->] (p0-3) -- (p0-4) ;
  \draw (p0-4) -- (p0-6) ;
  \draw[->] (p0-7) |- (p1-8) ;
  \draw[->] (p0-7) |- (p2-8) ;
  \draw[->] (p0-7) |- (p3-8) ;
  \draw[->] (p0-7) |- (p4-8) ;
  \draw[->] (p0-7) |- (p5-8) ;
  \draw[->] (p0-7) |- (p6-8) ;
  \draw[->] (p0-7) |- (p7-8) ;
  \draw[->] (p0-7) |- (p8-8) ;
  \draw[->] (p0-7) |- (p9-8) ;
  \draw[->] (p10-9) -| (p0-5) ;
  \draw[->] (p1-8) -| (p10-9) ;
  \draw[->] (p2-8) -| (p10-9) ;
  \draw[->] (p3-8) -| (p10-9) ;
  \draw[->] (p4-8) -| (p10-9) ;
  \draw[->] (p5-8) -| (p10-9) ;
  \draw[->] (p6-8) -| (p10-9) ;
  \draw[->] (p7-8) -| (p10-9) ;
  \draw[->] (p8-8) -| (p10-9) ;
  \draw[->] (p9-8) -| (p10-9) ;
  \draw[->] (p0-6) -- (p0-10) ;
  \draw[->] (p0-10) -- (p0-11) ;
\end{tikzpicture}

\ruleSubsection{plm\_syntax}{declaration-task}{51}

\begin{tikzpicture}
  \matrix[column sep=\ruleMatrixColumnSeparation, row sep=\ruleMatrixRowSeparation] {
    & & & & & & & & & & & & & & & & & & & & & & \node (p7-22) [point] {}; & \\
    & & & & & & & & & & & & \node (p6-12) [nonterminal] {\nonTerminalSymbol{guarded\_command}{35}}; & \node (p6-13) [terminal] {\{}; & \node (p6-14) [nonterminal] {\nonTerminalSymbol{instructionList}{31}}; & \node (p6-15) [terminal] {\}}; & \\
    & & & & & & & & & & & & \node (p5-12) [terminal] {init}; & \node (p5-13) [terminal] {integer}; & \node (p5-14) [terminal] {\{}; & \node (p5-15) [nonterminal] {\nonTerminalSymbol{instructionList}{31}}; & \node (p5-16) [terminal] {\}}; & \\
    & & & & & & & & & & & & & & & & \node (p4-16) [terminal] {->}; & \node (p4-17) [terminal] {\$type}; & \\
    & & & & & & & & & & & & \node (p3-12) [terminal] {func}; & \node (p3-13) [terminal] {identifier}; & \node (p3-14) [nonterminal] {\nonTerminalSymbol{procedure\_formal\_arguments}{14}}; & \node (p3-15) [point] {}; & \node (p3-16) [point] {}; & & \node (p3-18) [point] {}; & \node (p3-19) [terminal] {\{}; & \node (p3-20) [nonterminal] {\nonTerminalSymbol{instructionList}{31}}; & \node (p3-21) [terminal] {\}}; & \\
    & & & & & & & & & & & & & & & \node (p2-15) [terminal] {\$type}; & \\
    & & & & & & & & & & & & \node (p1-12) [terminal] {var}; & \node (p1-13) [terminal] {identifier}; & \node (p1-14) [point] {}; & \node (p1-15) [point] {}; & \node (p1-16) [point] {}; & \node (p1-17) [terminal] {=}; & \node (p1-18) [nonterminal] {\nonTerminalSymbol{expression}{16}}; & \\
    \node (P0start) [firstPoint] {}; & & \node (p0-2) [terminal] {task}; & \node (p0-3) [terminal] {identifier}; & \node (p0-4) [terminal] {priority}; & \node (p0-5) [terminal] {integer}; & \node (p0-6) [terminal] {stackSize}; & \node (p0-7) [terminal] {integer}; & \node (p0-8) [terminal] {\{}; & \node (p0-9) [point] {}; & \node (p0-10) [point] {}; & \node (p0-11) [point] {}; & & & & & & & & & & & & \node (p0-23) [terminal] {\}}; & \node (p0-24) [lastPoint] {}; & \\
  };
  \draw[->] (P0start) -- (p0-2) ;
  \draw[->] (p0-2) -- (p0-3) ;
  \draw[->] (p0-3) -- (p0-4) ;
  \draw[->] (p0-4) -- (p0-5) ;
  \draw[->] (p0-5) -- (p0-6) ;
  \draw[->] (p0-6) -- (p0-7) ;
  \draw[->] (p0-7) -- (p0-8) ;
  \draw (p0-8) -- (p0-10) ;
  \draw[->] (p0-11) |- (p1-12) ;
  \draw[->] (p1-12) -- (p1-13) ;
  \draw (p1-13) -- (p1-15) ;
  \draw[->] (p1-14) |- (p2-15) ;
  \draw (p1-15) -- (p1-16) ;
  \draw[->] (p2-15) -| (p1-16) ;
  \draw[->] (p1-16) -- (p1-17) ;
  \draw[->] (p1-17) -- (p1-18) ;
  \draw[->] (p0-11) |- (p3-12) ;
  \draw[->] (p3-12) -- (p3-13) ;
  \draw[->] (p3-13) -- (p3-14) ;
  \draw (p3-14) -- (p3-16) ;
  \draw[->] (p3-15) |- (p4-16) ;
  \draw[->] (p4-16) -- (p4-17) ;
  \draw (p3-16) -- (p3-18) ;
  \draw[->] (p4-17) -| (p3-18) ;
  \draw[->] (p3-18) -- (p3-19) ;
  \draw[->] (p3-19) -- (p3-20) ;
  \draw[->] (p3-20) -- (p3-21) ;
  \draw[->] (p0-11) |- (p5-12) ;
  \draw[->] (p5-12) -- (p5-13) ;
  \draw[->] (p5-13) -- (p5-14) ;
  \draw[->] (p5-14) -- (p5-15) ;
  \draw[->] (p5-15) -- (p5-16) ;
  \draw[->] (p0-11) |- (p6-12) ;
  \draw[->] (p6-12) -- (p6-13) ;
  \draw[->] (p6-13) -- (p6-14) ;
  \draw[->] (p6-14) -- (p6-15) ;
  \draw[->] (p7-22) -| (p0-9) ;
  \draw[->] (p1-18) -| (p7-22) ;
  \draw[->] (p3-21) -| (p7-22) ;
  \draw[->] (p5-16) -| (p7-22) ;
  \draw[->] (p6-15) -| (p7-22) ;
  \draw[->] (p0-10) -- (p0-23) ;
  \draw[->] (p0-23) -- (p0-24) ;
\end{tikzpicture}

\ruleSubsection{plm\_syntax}{panic}{23}

\begin{tikzpicture}
  \matrix[column sep=\ruleMatrixColumnSeparation, row sep=\ruleMatrixRowSeparation] {
    \node (P0start) [firstPoint] {}; & & \node (p0-2) [terminal] {panic}; & \node (p0-3) [terminal] {func}; & \node (p0-4) [terminal] {identifier}; & \node (p0-5) [terminal] {integer}; & \node (p0-6) [terminal] {\{}; & \node (p0-7) [nonterminal] {\nonTerminalSymbol{instructionList}{31}}; & \node (p0-8) [terminal] {\}}; & \node (p0-9) [lastPoint] {}; & \\
  };
  \draw[->] (P0start) -- (p0-2) ;
  \draw[->] (p0-2) -- (p0-3) ;
  \draw[->] (p0-3) -- (p0-4) ;
  \draw[->] (p0-4) -- (p0-5) ;
  \draw[->] (p0-5) -- (p0-6) ;
  \draw[->] (p0-6) -- (p0-7) ;
  \draw[->] (p0-7) -- (p0-8) ;
  \draw[->] (p0-8) -- (p0-9) ;
\end{tikzpicture}

\ruleSubsection{plm\_syntax}{declaration-boot}{23}

\begin{tikzpicture}
  \matrix[column sep=\ruleMatrixColumnSeparation, row sep=\ruleMatrixRowSeparation] {
    \node (P0start) [firstPoint] {}; & & \node (p0-2) [terminal] {boot}; & \node (p0-3) [terminal] {integer}; & \node (p0-4) [terminal] {\{}; & \node (p0-5) [nonterminal] {\nonTerminalSymbol{instructionList}{31}}; & \node (p0-6) [terminal] {\}}; & \node (p0-7) [lastPoint] {}; & \\
  };
  \draw[->] (P0start) -- (p0-2) ;
  \draw[->] (p0-2) -- (p0-3) ;
  \draw[->] (p0-3) -- (p0-4) ;
  \draw[->] (p0-4) -- (p0-5) ;
  \draw[->] (p0-5) -- (p0-6) ;
  \draw[->] (p0-6) -- (p0-7) ;
\end{tikzpicture}

\ruleSubsection{plm\_syntax}{declaration-init}{25}

\begin{tikzpicture}
  \matrix[column sep=\ruleMatrixColumnSeparation, row sep=\ruleMatrixRowSeparation] {
    \node (P0start) [firstPoint] {}; & & \node (p0-2) [nonterminal] {\nonTerminalSymbol{declaration\_init}{12}}; & \node (p0-3) [lastPoint] {}; & \\
  };
  \draw[->] (P0start) -- (p0-2) ;
  \draw[->] (p0-2) -- (p0-3) ;
\end{tikzpicture}

\ruleSubsection{plm\_syntax}{declaration-required-proc}{20}

\begin{tikzpicture}
  \matrix[column sep=\ruleMatrixColumnSeparation, row sep=\ruleMatrixRowSeparation] {
    \node (P0start) [firstPoint] {}; & & \node (p0-2) [terminal] {required}; & \node (p0-3) [nonterminal] {\nonTerminalSymbol{procedure\_header}{13}}; & \node (p0-4) [lastPoint] {}; & \\
  };
  \draw[->] (P0start) -- (p0-2) ;
  \draw[->] (p0-2) -- (p0-3) ;
  \draw[->] (p0-3) -- (p0-4) ;
\end{tikzpicture}

\ruleSubsection{plm\_syntax}{declaration-extern-proc}{22}

\begin{tikzpicture}
  \matrix[column sep=\ruleMatrixColumnSeparation, row sep=\ruleMatrixRowSeparation] {
    & & & & & \node (p1-5) [terminal] {->}; & \node (p1-6) [terminal] {\$type}; & \\
    \node (P0start) [firstPoint] {}; & & \node (p0-2) [terminal] {extern}; & \node (p0-3) [nonterminal] {\nonTerminalSymbol{procedure\_header}{13}}; & \node (p0-4) [point] {}; & \node (p0-5) [point] {}; & & \node (p0-7) [point] {}; & \node (p0-8) [terminal] {:}; & \node (p0-9) [terminal] {"string"}; & \node (p0-10) [lastPoint] {}; & \\
  };
  \draw[->] (P0start) -- (p0-2) ;
  \draw[->] (p0-2) -- (p0-3) ;
  \draw (p0-3) -- (p0-5) ;
  \draw[->] (p0-4) |- (p1-5) ;
  \draw[->] (p1-5) -- (p1-6) ;
  \draw (p0-5) -- (p0-7) ;
  \draw[->] (p1-6) -| (p0-7) ;
  \draw[->] (p0-7) -- (p0-8) ;
  \draw[->] (p0-8) -- (p0-9) ;
  \draw[->] (p0-9) -- (p0-10) ;
\end{tikzpicture}

\ruleSubsection{plm\_syntax}{declaration-guard}{70}

\begin{tikzpicture}
  \matrix[column sep=\ruleMatrixColumnSeparation, row sep=\ruleMatrixRowSeparation] {
    \node (P0start) [firstPoint] {}; & & \node (p0-2) [nonterminal] {\nonTerminalSymbol{guard}{15}}; & \node (p0-3) [lastPoint] {}; & \\
  };
  \draw[->] (P0start) -- (p0-2) ;
  \draw[->] (p0-2) -- (p0-3) ;
\end{tikzpicture}

\ruleSubsection{plm\_syntax}{target-generation}{9}

\begin{tikzpicture}
  \matrix[column sep=\ruleMatrixColumnSeparation, row sep=\ruleMatrixRowSeparation] {
    \node (P0start) [firstPoint] {}; & & \node (p0-2) [terminal] {target}; & \node (p0-3) [terminal] {"string"}; & \node (p0-4) [lastPoint] {}; & \\
  };
  \draw[->] (P0start) -- (p0-2) ;
  \draw[->] (p0-2) -- (p0-3) ;
  \draw[->] (p0-3) -- (p0-4) ;
\end{tikzpicture}

\nonTerminalSection{declaration\_init}{12}

\ruleSubsection{plm\_syntax}{declaration-init}{31}

\begin{tikzpicture}
  \matrix[column sep=\ruleMatrixColumnSeparation, row sep=\ruleMatrixRowSeparation] {
    \node (P0start) [firstPoint] {}; & & \node (p0-2) [terminal] {init}; & \node (p0-3) [terminal] {integer}; & \node (p0-4) [terminal] {\{}; & \node (p0-5) [nonterminal] {\nonTerminalSymbol{instructionList}{31}}; & \node (p0-6) [terminal] {\}}; & \node (p0-7) [lastPoint] {}; & \\
  };
  \draw[->] (P0start) -- (p0-2) ;
  \draw[->] (p0-2) -- (p0-3) ;
  \draw[->] (p0-3) -- (p0-4) ;
  \draw[->] (p0-4) -- (p0-5) ;
  \draw[->] (p0-5) -- (p0-6) ;
  \draw[->] (p0-6) -- (p0-7) ;
\end{tikzpicture}

\nonTerminalSection{declaration\_struct\_var}{9}

\ruleSubsection{plm\_syntax}{type-structure-declaration}{46}

\begin{tikzpicture}
  \matrix[column sep=\ruleMatrixColumnSeparation, row sep=\ruleMatrixRowSeparation] {
    & & & & & & & & \node (p2-8) [terminal] {=}; & \node (p2-9) [nonterminal] {\nonTerminalSymbol{expression}{16}}; & \\
    & & & \node (p1-3) [terminal] {public}; & & & & & & & \node (p1-10) [terminal] {=}; & \node (p1-11) [nonterminal] {\nonTerminalSymbol{expression}{16}}; & \\
    \node (P0start) [firstPoint] {}; & & \node (p0-2) [point] {}; & \node (p0-3) [point] {}; & \node (p0-4) [point] {}; & \node (p0-5) [terminal] {var}; & \node (p0-6) [terminal] {identifier}; & \node (p0-7) [point] {}; & \node (p0-8) [terminal] {\$type}; & \node (p0-9) [point] {}; & \node (p0-10) [point] {}; & & \node (p0-12) [point] {}; & \node (p0-13) [point] {}; & \node (p0-14) [lastPoint] {}; & \\
  };
  \draw (P0start) -- (p0-3) ;
  \draw[->] (p0-2) |- (p1-3) ;
  \draw (p0-3) -- (p0-4) ;
  \draw[->] (p1-3) -| (p0-4) ;
  \draw[->] (p0-4) -- (p0-5) ;
  \draw[->] (p0-5) -- (p0-6) ;
  \draw[->] (p0-6) -- (p0-8) ;
  \draw (p0-8) -- (p0-10) ;
  \draw[->] (p0-9) |- (p1-10) ;
  \draw[->] (p1-10) -- (p1-11) ;
  \draw (p0-10) -- (p0-12) ;
  \draw[->] (p1-11) -| (p0-12) ;
  \draw[->] (p0-7) |- (p2-8) ;
  \draw[->] (p2-8) -- (p2-9) ;
  \draw (p0-12) -- (p0-13) ;
  \draw[->] (p2-9) -| (p0-13) ;
  \draw[->] (p0-13) -- (p0-14) ;
\end{tikzpicture}

\nonTerminalSection{declaration\_type}{8}

\ruleSubsection{plm\_syntax}{type-array}{26}

\begin{tikzpicture}
  \matrix[column sep=\ruleMatrixColumnSeparation, row sep=\ruleMatrixRowSeparation] {
    \node (P0start) [firstPoint] {}; & & \node (p0-2) [terminal] {\$type}; & \node (p0-3) [terminal] {[}; & \node (p0-4) [nonterminal] {\nonTerminalSymbol{expression}{16}}; & \node (p0-5) [terminal] {]}; & \node (p0-6) [lastPoint] {}; & \\
  };
  \draw[->] (P0start) -- (p0-2) ;
  \draw[->] (p0-2) -- (p0-3) ;
  \draw[->] (p0-3) -- (p0-4) ;
  \draw[->] (p0-4) -- (p0-5) ;
  \draw[->] (p0-5) -- (p0-6) ;
\end{tikzpicture}

\ruleSubsection{plm\_syntax}{type-alias}{24}

\begin{tikzpicture}
  \matrix[column sep=\ruleMatrixColumnSeparation, row sep=\ruleMatrixRowSeparation] {
    \node (P0start) [firstPoint] {}; & & \node (p0-2) [terminal] {\$type}; & \node (p0-3) [lastPoint] {}; & \\
  };
  \draw[->] (P0start) -- (p0-2) ;
  \draw[->] (p0-2) -- (p0-3) ;
\end{tikzpicture}

\ruleSubsection{plm\_syntax}{type-opaque-declaration}{26}

\begin{tikzpicture}
  \matrix[column sep=\ruleMatrixColumnSeparation, row sep=\ruleMatrixRowSeparation] {
    & & & & & & & & & & & \node (p2-11) [point] {}; & \\
    & & & & & & & & & & \node (p1-10) [terminal] {@attribute}; & \\
    \node (P0start) [firstPoint] {}; & & \node (p0-2) [terminal] {(}; & \node (p0-3) [terminal] {(}; & \node (p0-4) [nonterminal] {\nonTerminalSymbol{expression}{16}}; & \node (p0-5) [terminal] {)}; & \node (p0-6) [terminal] {)}; & \node (p0-7) [point] {}; & \node (p0-8) [point] {}; & \node (p0-9) [point] {}; & & & \node (p0-12) [lastPoint] {}; & \\
  };
  \draw[->] (P0start) -- (p0-2) ;
  \draw[->] (p0-2) -- (p0-3) ;
  \draw[->] (p0-3) -- (p0-4) ;
  \draw[->] (p0-4) -- (p0-5) ;
  \draw[->] (p0-5) -- (p0-6) ;
  \draw (p0-6) -- (p0-8) ;
  \draw[->] (p0-9) |- (p1-10) ;
  \draw[->] (p2-11) -| (p0-7) ;
  \draw[->] (p1-10) -| (p2-11) ;
  \draw[->] (p0-8) -- (p0-12) ;
\end{tikzpicture}

\nonTerminalSection{effective\_parameters}{30}

\ruleSubsection{plm\_syntax}{expression-primary}{104}

\begin{tikzpicture}
  \matrix[column sep=\ruleMatrixColumnSeparation, row sep=\ruleMatrixRowSeparation] {
    & & & & & & & & & & & & & & \node (p7-14) [point] {}; & \\
    & & & & & & & & \node (p6-8) [terminal] {let}; & & & & \node (p6-12) [terminal] {\$type}; & \\
    & & & & & & \node (p5-6) [terminal] {?}; & \node (p5-7) [point] {}; & \node (p5-8) [terminal] {var}; & \node (p5-9) [point] {}; & \node (p5-10) [terminal] {identifier}; & \node (p5-11) [point] {}; & \node (p5-12) [point] {}; & \node (p5-13) [point] {}; & \\
    & & & & & & \node (p4-6) [terminal] {?}; & \node (p4-7) [terminal] {identifier}; & \\
    & & & & & & \node (p3-6) [terminal] {!?}; & \node (p3-7) [terminal] {self}; & \node (p3-8) [terminal] {.}; & \node (p3-9) [terminal] {identifier}; & \\
    & & & & & & \node (p2-6) [terminal] {!?}; & \node (p2-7) [terminal] {identifier}; & \\
    & & & & & & \node (p1-6) [terminal] {!}; & \node (p1-7) [nonterminal] {\nonTerminalSymbol{expression}{16}}; & \\
    \node (P0start) [firstPoint] {}; & & \node (p0-2) [terminal] {(}; & \node (p0-3) [point] {}; & \node (p0-4) [point] {}; & \node (p0-5) [point] {}; & & & & & & & & & & \node (p0-15) [terminal] {)}; & \node (p0-16) [lastPoint] {}; & \\
  };
  \draw[->] (P0start) -- (p0-2) ;
  \draw (p0-2) -- (p0-4) ;
  \draw[->] (p0-5) |- (p1-6) ;
  \draw[->] (p1-6) -- (p1-7) ;
  \draw[->] (p0-5) |- (p2-6) ;
  \draw[->] (p2-6) -- (p2-7) ;
  \draw[->] (p0-5) |- (p3-6) ;
  \draw[->] (p3-6) -- (p3-7) ;
  \draw[->] (p3-7) -- (p3-8) ;
  \draw[->] (p3-8) -- (p3-9) ;
  \draw[->] (p0-5) |- (p4-6) ;
  \draw[->] (p4-6) -- (p4-7) ;
  \draw[->] (p0-5) |- (p5-6) ;
  \draw[->] (p5-6) -- (p5-8) ;
  \draw[->] (p5-7) |- (p6-8) ;
  \draw (p5-8) -- (p5-9) ;
  \draw[->] (p6-8) -| (p5-9) ;
  \draw[->] (p5-9) -- (p5-10) ;
  \draw (p5-10) -- (p5-12) ;
  \draw[->] (p5-11) |- (p6-12) ;
  \draw (p5-12) -- (p5-13) ;
  \draw[->] (p6-12) -| (p5-13) ;
  \draw[->] (p7-14) -| (p0-3) ;
  \draw[->] (p1-7) -| (p7-14) ;
  \draw[->] (p2-7) -| (p7-14) ;
  \draw[->] (p3-9) -| (p7-14) ;
  \draw[->] (p4-7) -| (p7-14) ;
  \draw[->] (p5-13) -| (p7-14) ;
  \draw[->] (p0-4) -- (p0-15) ;
  \draw[->] (p0-15) -- (p0-16) ;
\end{tikzpicture}

\nonTerminalSection{expression}{16}

\ruleSubsection{plm\_syntax}{expression-operator-priority}{16}

\begin{tikzpicture}
  \matrix[column sep=\ruleMatrixColumnSeparation, row sep=\ruleMatrixRowSeparation] {
    \node (P0start) [firstPoint] {}; & & \node (p0-2) [nonterminal] {\nonTerminalSymbol{expression\_12}{17}}; & \node (p0-3) [lastPoint] {}; & \\
  };
  \draw[->] (P0start) -- (p0-2) ;
  \draw[->] (p0-2) -- (p0-3) ;
\end{tikzpicture}

\nonTerminalSection{expression\_1}{28}

\ruleSubsection{plm\_syntax}{expression-operator-priority}{354}

\begin{tikzpicture}
  \matrix[column sep=\ruleMatrixColumnSeparation, row sep=\ruleMatrixRowSeparation] {
    \node (P0start) [firstPoint] {}; & & \node (p0-2) [nonterminal] {\nonTerminalSymbol{primary}{29}}; & \node (p0-3) [lastPoint] {}; & \\
  };
  \draw[->] (P0start) -- (p0-2) ;
  \draw[->] (p0-2) -- (p0-3) ;
\end{tikzpicture}

\nonTerminalSection{expression\_10}{19}

\ruleSubsection{plm\_syntax}{expression-operator-priority}{63}

\begin{tikzpicture}
  \matrix[column sep=\ruleMatrixColumnSeparation, row sep=\ruleMatrixRowSeparation] {
    & & & & & & & & \node (p2-8) [point] {}; & \\
    & & & & & & \node (p1-6) [terminal] {and}; & \node (p1-7) [nonterminal] {\nonTerminalSymbol{expression\_9}{20}}; & \\
    \node (P0start) [firstPoint] {}; & & \node (p0-2) [nonterminal] {\nonTerminalSymbol{expression\_9}{20}}; & \node (p0-3) [point] {}; & \node (p0-4) [point] {}; & \node (p0-5) [point] {}; & & & & \node (p0-9) [lastPoint] {}; & \\
  };
  \draw[->] (P0start) -- (p0-2) ;
  \draw (p0-2) -- (p0-4) ;
  \draw[->] (p0-5) |- (p1-6) ;
  \draw[->] (p1-6) -- (p1-7) ;
  \draw[->] (p2-8) -| (p0-3) ;
  \draw[->] (p1-7) -| (p2-8) ;
  \draw[->] (p0-4) -- (p0-9) ;
\end{tikzpicture}

\nonTerminalSection{expression\_11}{18}

\ruleSubsection{plm\_syntax}{expression-operator-priority}{45}

\begin{tikzpicture}
  \matrix[column sep=\ruleMatrixColumnSeparation, row sep=\ruleMatrixRowSeparation] {
    & & & & & & & & \node (p2-8) [point] {}; & \\
    & & & & & & \node (p1-6) [terminal] {xor}; & \node (p1-7) [nonterminal] {\nonTerminalSymbol{expression\_10}{19}}; & \\
    \node (P0start) [firstPoint] {}; & & \node (p0-2) [nonterminal] {\nonTerminalSymbol{expression\_10}{19}}; & \node (p0-3) [point] {}; & \node (p0-4) [point] {}; & \node (p0-5) [point] {}; & & & & \node (p0-9) [lastPoint] {}; & \\
  };
  \draw[->] (P0start) -- (p0-2) ;
  \draw (p0-2) -- (p0-4) ;
  \draw[->] (p0-5) |- (p1-6) ;
  \draw[->] (p1-6) -- (p1-7) ;
  \draw[->] (p2-8) -| (p0-3) ;
  \draw[->] (p1-7) -| (p2-8) ;
  \draw[->] (p0-4) -- (p0-9) ;
\end{tikzpicture}

\nonTerminalSection{expression\_12}{17}

\ruleSubsection{plm\_syntax}{expression-operator-priority}{22}

\begin{tikzpicture}
  \matrix[column sep=\ruleMatrixColumnSeparation, row sep=\ruleMatrixRowSeparation] {
    & & & & & & & & \node (p2-8) [point] {}; & \\
    & & & & & & \node (p1-6) [terminal] {or}; & \node (p1-7) [nonterminal] {\nonTerminalSymbol{expression\_11}{18}}; & \\
    \node (P0start) [firstPoint] {}; & & \node (p0-2) [nonterminal] {\nonTerminalSymbol{expression\_11}{18}}; & \node (p0-3) [point] {}; & \node (p0-4) [point] {}; & \node (p0-5) [point] {}; & & & & \node (p0-9) [lastPoint] {}; & \\
  };
  \draw[->] (P0start) -- (p0-2) ;
  \draw (p0-2) -- (p0-4) ;
  \draw[->] (p0-5) |- (p1-6) ;
  \draw[->] (p1-6) -- (p1-7) ;
  \draw[->] (p2-8) -| (p0-3) ;
  \draw[->] (p1-7) -| (p2-8) ;
  \draw[->] (p0-4) -- (p0-9) ;
\end{tikzpicture}

\nonTerminalSection{expression\_2}{27}

\ruleSubsection{plm\_syntax}{expression-operator-priority}{286}

\begin{tikzpicture}
  \matrix[column sep=\ruleMatrixColumnSeparation, row sep=\ruleMatrixRowSeparation] {
    & & & & & & & & \node (p7-8) [point] {}; & \\
    & & & & & & \node (p6-6) [terminal] {!/}; & \node (p6-7) [nonterminal] {\nonTerminalSymbol{expression\_1}{28}}; & \\
    & & & & & & \node (p5-6) [terminal] {/}; & \node (p5-7) [nonterminal] {\nonTerminalSymbol{expression\_1}{28}}; & \\
    & & & & & & \node (p4-6) [terminal] {!\verb=%=}; & \node (p4-7) [nonterminal] {\nonTerminalSymbol{expression\_1}{28}}; & \\
    & & & & & & \node (p3-6) [terminal] {\verb=%=}; & \node (p3-7) [nonterminal] {\nonTerminalSymbol{expression\_1}{28}}; & \\
    & & & & & & \node (p2-6) [terminal] {*\verb=%=}; & \node (p2-7) [nonterminal] {\nonTerminalSymbol{expression\_1}{28}}; & \\
    & & & & & & \node (p1-6) [terminal] {*}; & \node (p1-7) [nonterminal] {\nonTerminalSymbol{expression\_1}{28}}; & \\
    \node (P0start) [firstPoint] {}; & & \node (p0-2) [nonterminal] {\nonTerminalSymbol{expression\_1}{28}}; & \node (p0-3) [point] {}; & \node (p0-4) [point] {}; & \node (p0-5) [point] {}; & & & & \node (p0-9) [lastPoint] {}; & \\
  };
  \draw[->] (P0start) -- (p0-2) ;
  \draw (p0-2) -- (p0-4) ;
  \draw[->] (p0-5) |- (p1-6) ;
  \draw[->] (p1-6) -- (p1-7) ;
  \draw[->] (p0-5) |- (p2-6) ;
  \draw[->] (p2-6) -- (p2-7) ;
  \draw[->] (p0-5) |- (p3-6) ;
  \draw[->] (p3-6) -- (p3-7) ;
  \draw[->] (p0-5) |- (p4-6) ;
  \draw[->] (p4-6) -- (p4-7) ;
  \draw[->] (p0-5) |- (p5-6) ;
  \draw[->] (p5-6) -- (p5-7) ;
  \draw[->] (p0-5) |- (p6-6) ;
  \draw[->] (p6-6) -- (p6-7) ;
  \draw[->] (p7-8) -| (p0-3) ;
  \draw[->] (p1-7) -| (p7-8) ;
  \draw[->] (p2-7) -| (p7-8) ;
  \draw[->] (p3-7) -| (p7-8) ;
  \draw[->] (p4-7) -| (p7-8) ;
  \draw[->] (p5-7) -| (p7-8) ;
  \draw[->] (p6-7) -| (p7-8) ;
  \draw[->] (p0-4) -- (p0-9) ;
\end{tikzpicture}

\nonTerminalSection{expression\_3}{26}

\ruleSubsection{plm\_syntax}{expression-operator-priority}{238}

\begin{tikzpicture}
  \matrix[column sep=\ruleMatrixColumnSeparation, row sep=\ruleMatrixRowSeparation] {
    & & & & & & & & \node (p5-8) [point] {}; & \\
    & & & & & & \node (p4-6) [terminal] {-\verb=%=}; & \node (p4-7) [nonterminal] {\nonTerminalSymbol{expression\_2}{27}}; & \\
    & & & & & & \node (p3-6) [terminal] {-}; & \node (p3-7) [nonterminal] {\nonTerminalSymbol{expression\_2}{27}}; & \\
    & & & & & & \node (p2-6) [terminal] {+\verb=%=}; & \node (p2-7) [nonterminal] {\nonTerminalSymbol{expression\_2}{27}}; & \\
    & & & & & & \node (p1-6) [terminal] {+}; & \node (p1-7) [nonterminal] {\nonTerminalSymbol{expression\_2}{27}}; & \\
    \node (P0start) [firstPoint] {}; & & \node (p0-2) [nonterminal] {\nonTerminalSymbol{expression\_2}{27}}; & \node (p0-3) [point] {}; & \node (p0-4) [point] {}; & \node (p0-5) [point] {}; & & & & \node (p0-9) [lastPoint] {}; & \\
  };
  \draw[->] (P0start) -- (p0-2) ;
  \draw (p0-2) -- (p0-4) ;
  \draw[->] (p0-5) |- (p1-6) ;
  \draw[->] (p1-6) -- (p1-7) ;
  \draw[->] (p0-5) |- (p2-6) ;
  \draw[->] (p2-6) -- (p2-7) ;
  \draw[->] (p0-5) |- (p3-6) ;
  \draw[->] (p3-6) -- (p3-7) ;
  \draw[->] (p0-5) |- (p4-6) ;
  \draw[->] (p4-6) -- (p4-7) ;
  \draw[->] (p5-8) -| (p0-3) ;
  \draw[->] (p1-7) -| (p5-8) ;
  \draw[->] (p2-7) -| (p5-8) ;
  \draw[->] (p3-7) -| (p5-8) ;
  \draw[->] (p4-7) -| (p5-8) ;
  \draw[->] (p0-4) -- (p0-9) ;
\end{tikzpicture}

\nonTerminalSection{expression\_4}{25}

\ruleSubsection{plm\_syntax}{expression-operator-priority}{210}

\begin{tikzpicture}
  \matrix[column sep=\ruleMatrixColumnSeparation, row sep=\ruleMatrixRowSeparation] {
    & & & & & & & & \node (p3-8) [point] {}; & \\
    & & & & & & \node (p2-6) [terminal] {>>}; & \node (p2-7) [nonterminal] {\nonTerminalSymbol{expression\_3}{26}}; & \\
    & & & & & & \node (p1-6) [terminal] {<<}; & \node (p1-7) [nonterminal] {\nonTerminalSymbol{expression\_3}{26}}; & \\
    \node (P0start) [firstPoint] {}; & & \node (p0-2) [nonterminal] {\nonTerminalSymbol{expression\_3}{26}}; & \node (p0-3) [point] {}; & \node (p0-4) [point] {}; & \node (p0-5) [point] {}; & & & & \node (p0-9) [lastPoint] {}; & \\
  };
  \draw[->] (P0start) -- (p0-2) ;
  \draw (p0-2) -- (p0-4) ;
  \draw[->] (p0-5) |- (p1-6) ;
  \draw[->] (p1-6) -- (p1-7) ;
  \draw[->] (p0-5) |- (p2-6) ;
  \draw[->] (p2-6) -- (p2-7) ;
  \draw[->] (p3-8) -| (p0-3) ;
  \draw[->] (p1-7) -| (p3-8) ;
  \draw[->] (p2-7) -| (p3-8) ;
  \draw[->] (p0-4) -- (p0-9) ;
\end{tikzpicture}

\nonTerminalSection{expression\_5}{24}

\ruleSubsection{plm\_syntax}{expression-operator-priority}{162}

\begin{tikzpicture}
  \matrix[column sep=\ruleMatrixColumnSeparation, row sep=\ruleMatrixRowSeparation] {
    & & & & \node (p4-4) [terminal] {>}; & \node (p4-5) [nonterminal] {\nonTerminalSymbol{expression\_4}{25}}; & \\
    & & & & \node (p3-4) [terminal] {<}; & \node (p3-5) [nonterminal] {\nonTerminalSymbol{expression\_4}{25}}; & \\
    & & & & \node (p2-4) [terminal] {>=}; & \node (p2-5) [nonterminal] {\nonTerminalSymbol{expression\_4}{25}}; & \\
    & & & & \node (p1-4) [terminal] {<=}; & \node (p1-5) [nonterminal] {\nonTerminalSymbol{expression\_4}{25}}; & \\
    \node (P0start) [firstPoint] {}; & & \node (p0-2) [nonterminal] {\nonTerminalSymbol{expression\_4}{25}}; & \node (p0-3) [point] {}; & \node (p0-4) [point] {}; & & \node (p0-6) [point] {}; & \node (p0-7) [lastPoint] {}; & \\
  };
  \draw[->] (P0start) -- (p0-2) ;
  \draw (p0-2) -- (p0-4) ;
  \draw[->] (p0-3) |- (p1-4) ;
  \draw[->] (p1-4) -- (p1-5) ;
  \draw[->] (p0-3) |- (p2-4) ;
  \draw[->] (p2-4) -- (p2-5) ;
  \draw[->] (p0-3) |- (p3-4) ;
  \draw[->] (p3-4) -- (p3-5) ;
  \draw[->] (p0-3) |- (p4-4) ;
  \draw[->] (p4-4) -- (p4-5) ;
  \draw (p0-4) -- (p0-6) ;
  \draw[->] (p1-5) -| (p0-6) ;
  \draw[->] (p2-5) -| (p0-6) ;
  \draw[->] (p3-5) -| (p0-6) ;
  \draw[->] (p4-5) -| (p0-6) ;
  \draw[->] (p0-6) -- (p0-7) ;
\end{tikzpicture}

\nonTerminalSection{expression\_6}{23}

\ruleSubsection{plm\_syntax}{expression-operator-priority}{134}

\begin{tikzpicture}
  \matrix[column sep=\ruleMatrixColumnSeparation, row sep=\ruleMatrixRowSeparation] {
    & & & & \node (p2-4) [terminal] {!=}; & \node (p2-5) [nonterminal] {\nonTerminalSymbol{expression\_5}{24}}; & \\
    & & & & \node (p1-4) [terminal] {==}; & \node (p1-5) [nonterminal] {\nonTerminalSymbol{expression\_5}{24}}; & \\
    \node (P0start) [firstPoint] {}; & & \node (p0-2) [nonterminal] {\nonTerminalSymbol{expression\_5}{24}}; & \node (p0-3) [point] {}; & \node (p0-4) [point] {}; & & \node (p0-6) [point] {}; & \node (p0-7) [lastPoint] {}; & \\
  };
  \draw[->] (P0start) -- (p0-2) ;
  \draw (p0-2) -- (p0-4) ;
  \draw[->] (p0-3) |- (p1-4) ;
  \draw[->] (p1-4) -- (p1-5) ;
  \draw[->] (p0-3) |- (p2-4) ;
  \draw[->] (p2-4) -- (p2-5) ;
  \draw (p0-4) -- (p0-6) ;
  \draw[->] (p1-5) -| (p0-6) ;
  \draw[->] (p2-5) -| (p0-6) ;
  \draw[->] (p0-6) -- (p0-7) ;
\end{tikzpicture}

\nonTerminalSection{expression\_7}{22}

\ruleSubsection{plm\_syntax}{expression-operator-priority}{116}

\begin{tikzpicture}
  \matrix[column sep=\ruleMatrixColumnSeparation, row sep=\ruleMatrixRowSeparation] {
    & & & & & & & & \node (p2-8) [point] {}; & \\
    & & & & & & \node (p1-6) [terminal] {\&}; & \node (p1-7) [nonterminal] {\nonTerminalSymbol{expression\_6}{23}}; & \\
    \node (P0start) [firstPoint] {}; & & \node (p0-2) [nonterminal] {\nonTerminalSymbol{expression\_6}{23}}; & \node (p0-3) [point] {}; & \node (p0-4) [point] {}; & \node (p0-5) [point] {}; & & & & \node (p0-9) [lastPoint] {}; & \\
  };
  \draw[->] (P0start) -- (p0-2) ;
  \draw (p0-2) -- (p0-4) ;
  \draw[->] (p0-5) |- (p1-6) ;
  \draw[->] (p1-6) -- (p1-7) ;
  \draw[->] (p2-8) -| (p0-3) ;
  \draw[->] (p1-7) -| (p2-8) ;
  \draw[->] (p0-4) -- (p0-9) ;
\end{tikzpicture}

\nonTerminalSection{expression\_8}{21}

\ruleSubsection{plm\_syntax}{expression-operator-priority}{98}

\begin{tikzpicture}
  \matrix[column sep=\ruleMatrixColumnSeparation, row sep=\ruleMatrixRowSeparation] {
    & & & & & & & & \node (p2-8) [point] {}; & \\
    & & & & & & \node (p1-6) [terminal] {\verb=^=}; & \node (p1-7) [nonterminal] {\nonTerminalSymbol{expression\_7}{22}}; & \\
    \node (P0start) [firstPoint] {}; & & \node (p0-2) [nonterminal] {\nonTerminalSymbol{expression\_7}{22}}; & \node (p0-3) [point] {}; & \node (p0-4) [point] {}; & \node (p0-5) [point] {}; & & & & \node (p0-9) [lastPoint] {}; & \\
  };
  \draw[->] (P0start) -- (p0-2) ;
  \draw (p0-2) -- (p0-4) ;
  \draw[->] (p0-5) |- (p1-6) ;
  \draw[->] (p1-6) -- (p1-7) ;
  \draw[->] (p2-8) -| (p0-3) ;
  \draw[->] (p1-7) -| (p2-8) ;
  \draw[->] (p0-4) -- (p0-9) ;
\end{tikzpicture}

\nonTerminalSection{expression\_9}{20}

\ruleSubsection{plm\_syntax}{expression-operator-priority}{80}

\begin{tikzpicture}
  \matrix[column sep=\ruleMatrixColumnSeparation, row sep=\ruleMatrixRowSeparation] {
    & & & & & & & & \node (p2-8) [point] {}; & \\
    & & & & & & \node (p1-6) [terminal] {|}; & \node (p1-7) [nonterminal] {\nonTerminalSymbol{expression\_8}{21}}; & \\
    \node (P0start) [firstPoint] {}; & & \node (p0-2) [nonterminal] {\nonTerminalSymbol{expression\_8}{21}}; & \node (p0-3) [point] {}; & \node (p0-4) [point] {}; & \node (p0-5) [point] {}; & & & & \node (p0-9) [lastPoint] {}; & \\
  };
  \draw[->] (P0start) -- (p0-2) ;
  \draw (p0-2) -- (p0-4) ;
  \draw[->] (p0-5) |- (p1-6) ;
  \draw[->] (p1-6) -- (p1-7) ;
  \draw[->] (p2-8) -| (p0-3) ;
  \draw[->] (p1-7) -| (p2-8) ;
  \draw[->] (p0-4) -- (p0-9) ;
\end{tikzpicture}

\nonTerminalSection{global\_variable\_declaration}{10}

\ruleSubsection{plm\_syntax}{declaration-global-variable}{77}

\begin{tikzpicture}
  \matrix[column sep=\ruleMatrixColumnSeparation, row sep=\ruleMatrixRowSeparation] {
    & & & & & & & & & & & & & & & & & & & & & & & & & \node (p11-25) [point] {}; & \\
    & & & & & & & & & & & & & & & & \node (p10-16) [terminal] {panic}; & \node (p10-17) [terminal] {func}; & \node (p10-18) [terminal] {identifier}; & \node (p10-19) [terminal] {integer}; & \\
    & & & & & & & & & & & & & & & & \node (p9-16) [terminal] {task}; & \node (p9-17) [terminal] {identifier}; & \\
    & & & & & & & & & & & & & & & & \node (p8-16) [terminal] {isr}; & \node (p8-17) [terminal] {identifier}; & \\
    & & & & & & & & & & & & & & & & \node (p7-16) [terminal] {init}; & \node (p7-17) [terminal] {integer}; & \\
    & & & & & & & & & & & & & & & & \node (p6-16) [terminal] {func}; & \node (p6-17) [terminal] {\$type}; & \node (p6-18) [terminal] {.}; & \node (p6-19) [terminal] {identifier}; & \\
    & & & & & & & & & & & & & & & & \node (p5-16) [terminal] {func}; & \node (p5-17) [terminal] {identifier}; & \\
    & & & & & & & & & & & & & & & & & & \node (p4-18) [terminal] {\$type}; & \node (p4-19) [terminal] {.}; & \\
    & & & & & & & & & & & & & & & & \node (p3-16) [terminal] {section}; & \node (p3-17) [point] {}; & \node (p3-18) [point] {}; & & \node (p3-20) [point] {}; & \node (p3-21) [terminal] {identifier}; & \\
    & & & & & & & & & & & & & \node (p2-13) [terminal] {@attribute}; & & & & & \node (p2-18) [terminal] {\$type}; & \node (p2-19) [terminal] {.}; & & & & & \node (p2-24) [point] {}; & \\
    & & & & & \node (p1-5) [terminal] {\$type}; & & & & & \node (p1-10) [terminal] {\{}; & \node (p1-11) [point] {}; & \node (p1-12) [point] {}; & \node (p1-13) [point] {}; & \node (p1-14) [point] {}; & \node (p1-15) [point] {}; & \node (p1-16) [terminal] {guard}; & \node (p1-17) [point] {}; & \node (p1-18) [point] {}; & & \node (p1-20) [point] {}; & \node (p1-21) [terminal] {identifier}; & \node (p1-22) [point] {}; & \node (p1-23) [point] {}; & & & \node (p1-26) [terminal] {\}}; & \\
    \node (P0start) [firstPoint] {}; & & \node (p0-2) [terminal] {var}; & \node (p0-3) [terminal] {identifier}; & \node (p0-4) [point] {}; & \node (p0-5) [point] {}; & \node (p0-6) [point] {}; & \node (p0-7) [terminal] {=}; & \node (p0-8) [nonterminal] {\nonTerminalSymbol{expression}{16}}; & \node (p0-9) [point] {}; & \node (p0-10) [point] {}; & & & & & & & & & & & & & & & & & \node (p0-27) [point] {}; & \node (p0-28) [lastPoint] {}; & \\
  };
  \draw[->] (P0start) -- (p0-2) ;
  \draw[->] (p0-2) -- (p0-3) ;
  \draw (p0-3) -- (p0-5) ;
  \draw[->] (p0-4) |- (p1-5) ;
  \draw (p0-5) -- (p0-6) ;
  \draw[->] (p1-5) -| (p0-6) ;
  \draw[->] (p0-6) -- (p0-7) ;
  \draw[->] (p0-7) -- (p0-8) ;
  \draw (p0-8) -- (p0-10) ;
  \draw[->] (p0-9) |- (p1-10) ;
  \draw (p1-10) -- (p1-13) ;
  \draw[->] (p1-12) |- (p2-13) ;
  \draw (p1-13) -- (p1-14) ;
  \draw[->] (p2-13) -| (p1-14) ;
  \draw[->] (p1-14) -- (p1-16) ;
  \draw (p1-16) -- (p1-18) ;
  \draw[->] (p1-17) |- (p2-18) ;
  \draw[->] (p2-18) -- (p2-19) ;
  \draw (p1-18) -- (p1-20) ;
  \draw[->] (p2-19) -| (p1-20) ;
  \draw[->] (p1-20) -- (p1-21) ;
  \draw[->] (p1-15) |- (p3-16) ;
  \draw (p3-16) -- (p3-18) ;
  \draw[->] (p3-17) |- (p4-18) ;
  \draw[->] (p4-18) -- (p4-19) ;
  \draw (p3-18) -- (p3-20) ;
  \draw[->] (p4-19) -| (p3-20) ;
  \draw[->] (p3-20) -- (p3-21) ;
  \draw[->] (p1-15) |- (p5-16) ;
  \draw[->] (p5-16) -- (p5-17) ;
  \draw[->] (p1-15) |- (p6-16) ;
  \draw[->] (p6-16) -- (p6-17) ;
  \draw[->] (p6-17) -- (p6-18) ;
  \draw[->] (p6-18) -- (p6-19) ;
  \draw[->] (p1-15) |- (p7-16) ;
  \draw[->] (p7-16) -- (p7-17) ;
  \draw[->] (p1-15) |- (p8-16) ;
  \draw[->] (p8-16) -- (p8-17) ;
  \draw[->] (p1-15) |- (p9-16) ;
  \draw[->] (p9-16) -- (p9-17) ;
  \draw[->] (p1-15) |- (p10-16) ;
  \draw[->] (p10-16) -- (p10-17) ;
  \draw[->] (p10-17) -- (p10-18) ;
  \draw[->] (p10-18) -- (p10-19) ;
  \draw (p1-21) -- (p1-22) ;
  \draw[->] (p3-21) -| (p1-22) ;
  \draw[->] (p5-17) -| (p1-22) ;
  \draw[->] (p6-19) -| (p1-22) ;
  \draw[->] (p7-17) -| (p1-22) ;
  \draw[->] (p8-17) -| (p1-22) ;
  \draw[->] (p9-17) -| (p1-22) ;
  \draw[->] (p10-19) -| (p1-22) ;
  \draw (p1-23) |- (p2-24) ;
  \draw[->] (p11-25) -| (p1-11) ;
  \draw[->] (p2-24) -| (p11-25) ;
  \draw[->] (p1-22) -- (p1-26) ;
  \draw (p0-10) -- (p0-27) ;
  \draw[->] (p1-26) -| (p0-27) ;
  \draw[->] (p0-27) -- (p0-28) ;
\end{tikzpicture}

\nonTerminalSection{guard}{15}

\ruleSubsection{plm\_syntax}{declaration-guard}{30}

\begin{tikzpicture}
  \matrix[column sep=\ruleMatrixColumnSeparation, row sep=\ruleMatrixRowSeparation] {
    & & & & & & & & & & & \node (p2-11) [point] {}; & \\
    & & & \node (p1-3) [terminal] {public}; & & & & & & & \node (p1-10) [terminal] {@attribute}; & & & & \node (p1-14) [terminal] {:}; & \node (p1-15) [nonterminal] {\nonTerminalSymbol{procedure\_call}{36}}; & \\
    \node (P0start) [firstPoint] {}; & & \node (p0-2) [point] {}; & \node (p0-3) [point] {}; & \node (p0-4) [point] {}; & \node (p0-5) [terminal] {guard}; & \node (p0-6) [terminal] {identifier}; & \node (p0-7) [point] {}; & \node (p0-8) [point] {}; & \node (p0-9) [point] {}; & & & \node (p0-12) [nonterminal] {\nonTerminalSymbol{procedure\_formal\_arguments}{14}}; & \node (p0-13) [point] {}; & \node (p0-14) [point] {}; & & \node (p0-16) [point] {}; & \node (p0-17) [terminal] {\{}; & \node (p0-18) [nonterminal] {\nonTerminalSymbol{instructionList}{31}}; & \node (p0-19) [terminal] {\}}; & \node (p0-20) [lastPoint] {}; & \\
  };
  \draw (P0start) -- (p0-3) ;
  \draw[->] (p0-2) |- (p1-3) ;
  \draw (p0-3) -- (p0-4) ;
  \draw[->] (p1-3) -| (p0-4) ;
  \draw[->] (p0-4) -- (p0-5) ;
  \draw[->] (p0-5) -- (p0-6) ;
  \draw (p0-6) -- (p0-8) ;
  \draw[->] (p0-9) |- (p1-10) ;
  \draw[->] (p2-11) -| (p0-7) ;
  \draw[->] (p1-10) -| (p2-11) ;
  \draw[->] (p0-8) -- (p0-12) ;
  \draw (p0-12) -- (p0-14) ;
  \draw[->] (p0-13) |- (p1-14) ;
  \draw[->] (p1-14) -- (p1-15) ;
  \draw (p0-14) -- (p0-16) ;
  \draw[->] (p1-15) -| (p0-16) ;
  \draw[->] (p0-16) -- (p0-17) ;
  \draw[->] (p0-17) -- (p0-18) ;
  \draw[->] (p0-18) -- (p0-19) ;
  \draw[->] (p0-19) -- (p0-20) ;
\end{tikzpicture}

\nonTerminalSection{guarded\_command}{35}

\ruleSubsection{plm\_syntax}{instruction-sync}{47}

\begin{tikzpicture}
  \matrix[column sep=\ruleMatrixColumnSeparation, row sep=\ruleMatrixRowSeparation] {
    & & & & \node (p3-4) [terminal] {while}; & & & & \node (p3-8) [terminal] {.}; & \node (p3-9) [terminal] {identifier}; & \node (p3-10) [nonterminal] {\nonTerminalSymbol{effective\_parameters}{30}}; & \\
    & & & \node (p2-3) [point] {}; & \node (p2-4) [terminal] {until}; & \node (p2-5) [point] {}; & \node (p2-6) [terminal] {identifier}; & \node (p2-7) [point] {}; & \node (p2-8) [nonterminal] {\nonTerminalSymbol{effective\_parameters}{30}}; & & & \node (p2-11) [point] {}; & \\
    & & & & & & \node (p1-6) [terminal] {while}; & & & & \node (p1-10) [terminal] {.}; & \node (p1-11) [terminal] {identifier}; & \node (p1-12) [nonterminal] {\nonTerminalSymbol{effective\_parameters}{30}}; & \\
    \node (P0start) [firstPoint] {}; & & \node (p0-2) [point] {}; & \node (p0-3) [terminal] {when}; & \node (p0-4) [nonterminal] {\nonTerminalSymbol{expression}{16}}; & \node (p0-5) [point] {}; & \node (p0-6) [terminal] {until}; & \node (p0-7) [point] {}; & \node (p0-8) [terminal] {identifier}; & \node (p0-9) [point] {}; & \node (p0-10) [nonterminal] {\nonTerminalSymbol{effective\_parameters}{30}}; & & & \node (p0-13) [point] {}; & \node (p0-14) [point] {}; & \node (p0-15) [lastPoint] {}; & \\
  };
  \draw[->] (P0start) -- (p0-3) ;
  \draw[->] (p0-3) -- (p0-4) ;
  \draw[->] (p0-4) -- (p0-6) ;
  \draw[->] (p0-5) |- (p1-6) ;
  \draw (p0-6) -- (p0-7) ;
  \draw[->] (p1-6) -| (p0-7) ;
  \draw[->] (p0-7) -- (p0-8) ;
  \draw[->] (p0-8) -- (p0-10) ;
  \draw[->] (p0-9) |- (p1-10) ;
  \draw[->] (p1-10) -- (p1-11) ;
  \draw[->] (p1-11) -- (p1-12) ;
  \draw (p0-10) -- (p0-13) ;
  \draw[->] (p1-12) -| (p0-13) ;
  \draw[->] (p0-2) |- (p2-4) ;
  \draw[->] (p2-3) |- (p3-4) ;
  \draw (p2-4) -- (p2-5) ;
  \draw[->] (p3-4) -| (p2-5) ;
  \draw[->] (p2-5) -- (p2-6) ;
  \draw[->] (p2-6) -- (p2-8) ;
  \draw[->] (p2-7) |- (p3-8) ;
  \draw[->] (p3-8) -- (p3-9) ;
  \draw[->] (p3-9) -- (p3-10) ;
  \draw (p2-8) -- (p2-11) ;
  \draw[->] (p3-10) -| (p2-11) ;
  \draw (p0-13) -- (p0-14) ;
  \draw[->] (p2-11) -| (p0-14) ;
  \draw[->] (p0-14) -- (p0-15) ;
\end{tikzpicture}

\nonTerminalSection{if\_instruction}{34}

\ruleSubsection{plm\_syntax}{instruction-if}{49}

\begin{tikzpicture}
  \matrix[column sep=\ruleMatrixColumnSeparation, row sep=\ruleMatrixRowSeparation] {
    & & & & & & & & & & \node (p4-10) [terminal] {@attribute}; & \\
    & & & & & & & \node (p3-7) [terminal] {else}; & \node (p3-8) [terminal] {if}; & \node (p3-9) [point] {}; & \node (p3-10) [point] {}; & \node (p3-11) [point] {}; & \node (p3-12) [nonterminal] {\nonTerminalSymbol{if\_instruction}{34}}; & \\
    & & & & & & & & & & \node (p2-10) [terminal] {@attribute}; & \\
    & & & & & & & \node (p1-7) [terminal] {else}; & \node (p1-8) [terminal] {\{}; & \node (p1-9) [point] {}; & \node (p1-10) [point] {}; & \node (p1-11) [point] {}; & \node (p1-12) [nonterminal] {\nonTerminalSymbol{instructionList}{31}}; & \node (p1-13) [terminal] {\}}; & \\
    \node (P0start) [firstPoint] {}; & & \node (p0-2) [nonterminal] {\nonTerminalSymbol{expression}{16}}; & \node (p0-3) [terminal] {\{}; & \node (p0-4) [nonterminal] {\nonTerminalSymbol{instructionList}{31}}; & \node (p0-5) [terminal] {\}}; & \node (p0-6) [point] {}; & \node (p0-7) [point] {}; & & & & & & & \node (p0-14) [point] {}; & \node (p0-15) [lastPoint] {}; & \\
  };
  \draw[->] (P0start) -- (p0-2) ;
  \draw[->] (p0-2) -- (p0-3) ;
  \draw[->] (p0-3) -- (p0-4) ;
  \draw[->] (p0-4) -- (p0-5) ;
  \draw (p0-5) -- (p0-7) ;
  \draw[->] (p0-6) |- (p1-7) ;
  \draw[->] (p1-7) -- (p1-8) ;
  \draw (p1-8) -- (p1-10) ;
  \draw[->] (p1-9) |- (p2-10) ;
  \draw (p1-10) -- (p1-11) ;
  \draw[->] (p2-10) -| (p1-11) ;
  \draw[->] (p1-11) -- (p1-12) ;
  \draw[->] (p1-12) -- (p1-13) ;
  \draw[->] (p0-6) |- (p3-7) ;
  \draw[->] (p3-7) -- (p3-8) ;
  \draw (p3-8) -- (p3-10) ;
  \draw[->] (p3-9) |- (p4-10) ;
  \draw (p3-10) -- (p3-11) ;
  \draw[->] (p4-10) -| (p3-11) ;
  \draw[->] (p3-11) -- (p3-12) ;
  \draw (p0-7) -- (p0-14) ;
  \draw[->] (p1-13) -| (p0-14) ;
  \draw[->] (p3-12) -| (p0-14) ;
  \draw[->] (p0-14) -- (p0-15) ;
\end{tikzpicture}

\nonTerminalSection{import\_file}{5}

\ruleSubsection{plm\_syntax}{syntax-grammar}{19}

\begin{tikzpicture}
  \matrix[column sep=\ruleMatrixColumnSeparation, row sep=\ruleMatrixRowSeparation] {
    \node (P0start) [firstPoint] {}; & & \node (p0-2) [terminal] {import}; & \node (p0-3) [terminal] {"string"}; & \node (p0-4) [lastPoint] {}; & \\
  };
  \draw[->] (P0start) -- (p0-2) ;
  \draw[->] (p0-2) -- (p0-3) ;
  \draw[->] (p0-3) -- (p0-4) ;
\end{tikzpicture}

\nonTerminalSection{instruction}{32}

\ruleSubsection{plm\_syntax}{directive-check}{18}

\begin{tikzpicture}
  \matrix[column sep=\ruleMatrixColumnSeparation, row sep=\ruleMatrixRowSeparation] {
    \node (P0start) [firstPoint] {}; & & \node (p0-2) [terminal] {check}; & \node (p0-3) [nonterminal] {\nonTerminalSymbol{expression}{16}}; & \node (p0-4) [lastPoint] {}; & \\
  };
  \draw[->] (P0start) -- (p0-2) ;
  \draw[->] (p0-2) -- (p0-3) ;
  \draw[->] (p0-3) -- (p0-4) ;
\end{tikzpicture}

\ruleSubsection{plm\_syntax}{instruction-assignment}{18}

\begin{tikzpicture}
  \matrix[column sep=\ruleMatrixColumnSeparation, row sep=\ruleMatrixRowSeparation] {
    \node (P0start) [firstPoint] {}; & & \node (p0-2) [nonterminal] {\nonTerminalSymbol{assignment\_target}{37}}; & \node (p0-3) [terminal] {=}; & \node (p0-4) [nonterminal] {\nonTerminalSymbol{expression}{16}}; & \node (p0-5) [lastPoint] {}; & \\
  };
  \draw[->] (P0start) -- (p0-2) ;
  \draw[->] (p0-2) -- (p0-3) ;
  \draw[->] (p0-3) -- (p0-4) ;
  \draw[->] (p0-4) -- (p0-5) ;
\end{tikzpicture}

\ruleSubsection{plm\_syntax}{instruction-assignment-operator}{70}

\begin{tikzpicture}
  \matrix[column sep=\ruleMatrixColumnSeparation, row sep=\ruleMatrixRowSeparation] {
    \node (P0start) [firstPoint] {}; & & \node (p0-2) [nonterminal] {\nonTerminalSymbol{assignment\_target}{37}}; & \node (p0-3) [nonterminal] {\nonTerminalSymbol{assignment\_operator}{33}}; & \node (p0-4) [nonterminal] {\nonTerminalSymbol{expression}{16}}; & \node (p0-5) [lastPoint] {}; & \\
  };
  \draw[->] (P0start) -- (p0-2) ;
  \draw[->] (p0-2) -- (p0-3) ;
  \draw[->] (p0-3) -- (p0-4) ;
  \draw[->] (p0-4) -- (p0-5) ;
\end{tikzpicture}

\ruleSubsection{plm\_syntax}{instruction-var}{26}

\begin{tikzpicture}
  \matrix[column sep=\ruleMatrixColumnSeparation, row sep=\ruleMatrixRowSeparation] {
    & & & & & \node (p1-5) [terminal] {\$type}; & \\
    \node (P0start) [firstPoint] {}; & & \node (p0-2) [terminal] {var}; & \node (p0-3) [terminal] {identifier}; & \node (p0-4) [point] {}; & \node (p0-5) [point] {}; & \node (p0-6) [point] {}; & \node (p0-7) [terminal] {=}; & \node (p0-8) [nonterminal] {\nonTerminalSymbol{expression}{16}}; & \node (p0-9) [lastPoint] {}; & \\
  };
  \draw[->] (P0start) -- (p0-2) ;
  \draw[->] (p0-2) -- (p0-3) ;
  \draw (p0-3) -- (p0-5) ;
  \draw[->] (p0-4) |- (p1-5) ;
  \draw (p0-5) -- (p0-6) ;
  \draw[->] (p1-5) -| (p0-6) ;
  \draw[->] (p0-6) -- (p0-7) ;
  \draw[->] (p0-7) -- (p0-8) ;
  \draw[->] (p0-8) -- (p0-9) ;
\end{tikzpicture}

\ruleSubsection{plm\_syntax}{instruction-var}{46}

\begin{tikzpicture}
  \matrix[column sep=\ruleMatrixColumnSeparation, row sep=\ruleMatrixRowSeparation] {
    \node (P0start) [firstPoint] {}; & & \node (p0-2) [terminal] {var}; & \node (p0-3) [terminal] {identifier}; & \node (p0-4) [terminal] {\$type}; & \node (p0-5) [lastPoint] {}; & \\
  };
  \draw[->] (P0start) -- (p0-2) ;
  \draw[->] (p0-2) -- (p0-3) ;
  \draw[->] (p0-3) -- (p0-4) ;
  \draw[->] (p0-4) -- (p0-5) ;
\end{tikzpicture}

\ruleSubsection{plm\_syntax}{instruction-let}{19}

\begin{tikzpicture}
  \matrix[column sep=\ruleMatrixColumnSeparation, row sep=\ruleMatrixRowSeparation] {
    & & & & & \node (p1-5) [terminal] {\$type}; & \\
    \node (P0start) [firstPoint] {}; & & \node (p0-2) [terminal] {let}; & \node (p0-3) [terminal] {identifier}; & \node (p0-4) [point] {}; & \node (p0-5) [point] {}; & \node (p0-6) [point] {}; & \node (p0-7) [terminal] {=}; & \node (p0-8) [nonterminal] {\nonTerminalSymbol{expression}{16}}; & \node (p0-9) [lastPoint] {}; & \\
  };
  \draw[->] (P0start) -- (p0-2) ;
  \draw[->] (p0-2) -- (p0-3) ;
  \draw (p0-3) -- (p0-5) ;
  \draw[->] (p0-4) |- (p1-5) ;
  \draw (p0-5) -- (p0-6) ;
  \draw[->] (p1-5) -| (p0-6) ;
  \draw[->] (p0-6) -- (p0-7) ;
  \draw[->] (p0-7) -- (p0-8) ;
  \draw[->] (p0-8) -- (p0-9) ;
\end{tikzpicture}

\ruleSubsection{plm\_syntax}{instruction-assert}{18}

\begin{tikzpicture}
  \matrix[column sep=\ruleMatrixColumnSeparation, row sep=\ruleMatrixRowSeparation] {
    \node (P0start) [firstPoint] {}; & & \node (p0-2) [terminal] {assert}; & \node (p0-3) [nonterminal] {\nonTerminalSymbol{expression}{16}}; & \node (p0-4) [lastPoint] {}; & \\
  };
  \draw[->] (P0start) -- (p0-2) ;
  \draw[->] (p0-2) -- (p0-3) ;
  \draw[->] (p0-3) -- (p0-4) ;
\end{tikzpicture}

\ruleSubsection{plm\_syntax}{instruction-panic}{18}

\begin{tikzpicture}
  \matrix[column sep=\ruleMatrixColumnSeparation, row sep=\ruleMatrixRowSeparation] {
    \node (P0start) [firstPoint] {}; & & \node (p0-2) [terminal] {panic}; & \node (p0-3) [nonterminal] {\nonTerminalSymbol{expression}{16}}; & \node (p0-4) [lastPoint] {}; & \\
  };
  \draw[->] (P0start) -- (p0-2) ;
  \draw[->] (p0-2) -- (p0-3) ;
  \draw[->] (p0-3) -- (p0-4) ;
\end{tikzpicture}

\ruleSubsection{plm\_syntax}{instruction-if}{23}

\begin{tikzpicture}
  \matrix[column sep=\ruleMatrixColumnSeparation, row sep=\ruleMatrixRowSeparation] {
    & & & & \node (p1-4) [terminal] {@attribute}; & & & & \node (p1-8) [terminal] {@attribute}; & \\
    \node (P0start) [firstPoint] {}; & & \node (p0-2) [terminal] {if}; & \node (p0-3) [point] {}; & \node (p0-4) [point] {}; & \node (p0-5) [point] {}; & \node (p0-6) [nonterminal] {\nonTerminalSymbol{if\_instruction}{34}}; & \node (p0-7) [point] {}; & \node (p0-8) [point] {}; & \node (p0-9) [point] {}; & \node (p0-10) [lastPoint] {}; & \\
  };
  \draw[->] (P0start) -- (p0-2) ;
  \draw (p0-2) -- (p0-4) ;
  \draw[->] (p0-3) |- (p1-4) ;
  \draw (p0-4) -- (p0-5) ;
  \draw[->] (p1-4) -| (p0-5) ;
  \draw[->] (p0-5) -- (p0-6) ;
  \draw (p0-6) -- (p0-8) ;
  \draw[->] (p0-7) |- (p1-8) ;
  \draw (p0-8) -- (p0-9) ;
  \draw[->] (p1-8) -| (p0-9) ;
  \draw[->] (p0-9) -- (p0-10) ;
\end{tikzpicture}

\ruleSubsection{plm\_syntax}{instruction-sync}{122}

\begin{tikzpicture}
  \matrix[column sep=\ruleMatrixColumnSeparation, row sep=\ruleMatrixRowSeparation] {
    & & & & & & & & & & & & & \node (p2-13) [point] {}; & \\
    & & & & & \node (p1-5) [terminal] {@attribute}; & & & & & & & \node (p1-12) [point] {}; & & & & \node (p1-16) [terminal] {@attribute}; & \\
    \node (P0start) [firstPoint] {}; & & \node (p0-2) [terminal] {sync}; & \node (p0-3) [terminal] {\{}; & \node (p0-4) [point] {}; & \node (p0-5) [point] {}; & \node (p0-6) [point] {}; & \node (p0-7) [point] {}; & \node (p0-8) [nonterminal] {\nonTerminalSymbol{guarded\_command}{35}}; & \node (p0-9) [terminal] {:}; & \node (p0-10) [nonterminal] {\nonTerminalSymbol{instructionList}{31}}; & \node (p0-11) [point] {}; & & & \node (p0-14) [terminal] {\}}; & \node (p0-15) [point] {}; & \node (p0-16) [point] {}; & \node (p0-17) [point] {}; & \node (p0-18) [lastPoint] {}; & \\
  };
  \draw[->] (P0start) -- (p0-2) ;
  \draw[->] (p0-2) -- (p0-3) ;
  \draw (p0-3) -- (p0-5) ;
  \draw[->] (p0-4) |- (p1-5) ;
  \draw (p0-5) -- (p0-6) ;
  \draw[->] (p1-5) -| (p0-6) ;
  \draw[->] (p0-6) -- (p0-8) ;
  \draw[->] (p0-8) -- (p0-9) ;
  \draw[->] (p0-9) -- (p0-10) ;
  \draw (p0-11) |- (p1-12) ;
  \draw[->] (p2-13) -| (p0-7) ;
  \draw[->] (p1-12) -| (p2-13) ;
  \draw[->] (p0-10) -- (p0-14) ;
  \draw (p0-14) -- (p0-16) ;
  \draw[->] (p0-15) |- (p1-16) ;
  \draw (p0-16) -- (p0-17) ;
  \draw[->] (p1-16) -| (p0-17) ;
  \draw[->] (p0-17) -- (p0-18) ;
\end{tikzpicture}

\ruleSubsection{plm\_syntax}{instruction-while}{20}

\begin{tikzpicture}
  \matrix[column sep=\ruleMatrixColumnSeparation, row sep=\ruleMatrixRowSeparation] {
    & & & & & \node (p1-5) [terminal] {@attribute}; & & & & & & & \node (p1-12) [terminal] {@attribute}; & \\
    \node (P0start) [firstPoint] {}; & & \node (p0-2) [terminal] {do}; & \node (p0-3) [terminal] {while}; & \node (p0-4) [point] {}; & \node (p0-5) [point] {}; & \node (p0-6) [point] {}; & \node (p0-7) [nonterminal] {\nonTerminalSymbol{expression}{16}}; & \node (p0-8) [terminal] {\{}; & \node (p0-9) [nonterminal] {\nonTerminalSymbol{instructionList}{31}}; & \node (p0-10) [terminal] {\}}; & \node (p0-11) [point] {}; & \node (p0-12) [point] {}; & \node (p0-13) [point] {}; & \node (p0-14) [lastPoint] {}; & \\
  };
  \draw[->] (P0start) -- (p0-2) ;
  \draw[->] (p0-2) -- (p0-3) ;
  \draw (p0-3) -- (p0-5) ;
  \draw[->] (p0-4) |- (p1-5) ;
  \draw (p0-5) -- (p0-6) ;
  \draw[->] (p1-5) -| (p0-6) ;
  \draw[->] (p0-6) -- (p0-7) ;
  \draw[->] (p0-7) -- (p0-8) ;
  \draw[->] (p0-8) -- (p0-9) ;
  \draw[->] (p0-9) -- (p0-10) ;
  \draw (p0-10) -- (p0-12) ;
  \draw[->] (p0-11) |- (p1-12) ;
  \draw (p0-12) -- (p0-13) ;
  \draw[->] (p1-12) -| (p0-13) ;
  \draw[->] (p0-13) -- (p0-14) ;
\end{tikzpicture}

\ruleSubsection{plm\_syntax}{instruction-for-in-do}{21}

\begin{tikzpicture}
  \matrix[column sep=\ruleMatrixColumnSeparation, row sep=\ruleMatrixRowSeparation] {
    \node (P0start) [firstPoint] {}; & & \node (p0-2) [terminal] {for}; & \node (p0-3) [terminal] {identifier}; & \node (p0-4) [terminal] {in}; & \node (p0-5) [nonterminal] {\nonTerminalSymbol{expression}{16}}; & \node (p0-6) [terminal] {\{}; & \node (p0-7) [nonterminal] {\nonTerminalSymbol{instructionList}{31}}; & \node (p0-8) [terminal] {\}}; & \node (p0-9) [lastPoint] {}; & \\
  };
  \draw[->] (P0start) -- (p0-2) ;
  \draw[->] (p0-2) -- (p0-3) ;
  \draw[->] (p0-3) -- (p0-4) ;
  \draw[->] (p0-4) -- (p0-5) ;
  \draw[->] (p0-5) -- (p0-6) ;
  \draw[->] (p0-6) -- (p0-7) ;
  \draw[->] (p0-7) -- (p0-8) ;
  \draw[->] (p0-8) -- (p0-9) ;
\end{tikzpicture}

\ruleSubsection{plm\_syntax}{instruction-for-in-lower-upper-bounds}{24}

\begin{tikzpicture}
  \matrix[column sep=\ruleMatrixColumnSeparation, row sep=\ruleMatrixRowSeparation] {
    \node (P0start) [firstPoint] {}; & & \node (p0-2) [terminal] {for}; & \node (p0-3) [terminal] {identifier}; & \node (p0-4) [terminal] {\$type}; & \node (p0-5) [terminal] {in}; & \node (p0-6) [nonterminal] {\nonTerminalSymbol{expression}{16}}; & \node (p0-7) [terminal] {..<}; & \node (p0-8) [nonterminal] {\nonTerminalSymbol{expression}{16}}; & \node (p0-9) [terminal] {\{}; & \node (p0-10) [nonterminal] {\nonTerminalSymbol{instructionList}{31}}; & \node (p0-11) [terminal] {\}}; & \node (p0-12) [lastPoint] {}; & \\
  };
  \draw[->] (P0start) -- (p0-2) ;
  \draw[->] (p0-2) -- (p0-3) ;
  \draw[->] (p0-3) -- (p0-4) ;
  \draw[->] (p0-4) -- (p0-5) ;
  \draw[->] (p0-5) -- (p0-6) ;
  \draw[->] (p0-6) -- (p0-7) ;
  \draw[->] (p0-7) -- (p0-8) ;
  \draw[->] (p0-8) -- (p0-9) ;
  \draw[->] (p0-9) -- (p0-10) ;
  \draw[->] (p0-10) -- (p0-11) ;
  \draw[->] (p0-11) -- (p0-12) ;
\end{tikzpicture}

\ruleSubsection{plm\_syntax}{instruction-procedure-call}{19}

\begin{tikzpicture}
  \matrix[column sep=\ruleMatrixColumnSeparation, row sep=\ruleMatrixRowSeparation] {
    \node (P0start) [firstPoint] {}; & & \node (p0-2) [nonterminal] {\nonTerminalSymbol{procedure\_call}{36}}; & \node (p0-3) [lastPoint] {}; & \\
  };
  \draw[->] (P0start) -- (p0-2) ;
  \draw[->] (p0-2) -- (p0-3) ;
\end{tikzpicture}

\nonTerminalSection{instructionList}{31}

\ruleSubsection{plm\_syntax}{instructionList}{23}

\begin{tikzpicture}
  \matrix[column sep=\ruleMatrixColumnSeparation, row sep=\ruleMatrixRowSeparation] {
    & & & & & & \node (p3-6) [point] {}; & \\
    & & & & & \node (p2-5) [terminal] {;}; & \\
    & & & & & \node (p1-5) [nonterminal] {\nonTerminalSymbol{instruction}{32}}; & \\
    \node (P0start) [firstPoint] {}; & & \node (p0-2) [point] {}; & \node (p0-3) [point] {}; & \node (p0-4) [point] {}; & & & \node (p0-7) [lastPoint] {}; & \\
  };
  \draw (P0start) -- (p0-3) ;
  \draw[->] (p0-4) |- (p1-5) ;
  \draw[->] (p0-4) |- (p2-5) ;
  \draw[->] (p3-6) -| (p0-2) ;
  \draw[->] (p1-5) -| (p3-6) ;
  \draw[->] (p2-5) -| (p3-6) ;
  \draw[->] (p0-3) -- (p0-7) ;
\end{tikzpicture}

\nonTerminalSection{isr}{4}

\ruleSubsection{plm\_syntax}{declaration-isr}{22}

\begin{tikzpicture}
  \matrix[column sep=\ruleMatrixColumnSeparation, row sep=\ruleMatrixRowSeparation] {
    & & & & & & & & \node (p2-8) [point] {}; & \\
    & & & & & & & \node (p1-7) [terminal] {@attribute}; & \\
    \node (P0start) [firstPoint] {}; & & \node (p0-2) [terminal] {isr}; & \node (p0-3) [terminal] {identifier}; & \node (p0-4) [point] {}; & \node (p0-5) [point] {}; & \node (p0-6) [point] {}; & & & \node (p0-9) [terminal] {\{}; & \node (p0-10) [nonterminal] {\nonTerminalSymbol{instructionList}{31}}; & \node (p0-11) [terminal] {\}}; & \node (p0-12) [lastPoint] {}; & \\
  };
  \draw[->] (P0start) -- (p0-2) ;
  \draw[->] (p0-2) -- (p0-3) ;
  \draw (p0-3) -- (p0-5) ;
  \draw[->] (p0-6) |- (p1-7) ;
  \draw[->] (p2-8) -| (p0-4) ;
  \draw[->] (p1-7) -| (p2-8) ;
  \draw[->] (p0-5) -- (p0-9) ;
  \draw[->] (p0-9) -- (p0-10) ;
  \draw[->] (p0-10) -- (p0-11) ;
  \draw[->] (p0-11) -- (p0-12) ;
\end{tikzpicture}

\nonTerminalSection{module\_variable}{11}

\ruleSubsection{plm\_syntax}{declaration-module}{17}

\begin{tikzpicture}
  \matrix[column sep=\ruleMatrixColumnSeparation, row sep=\ruleMatrixRowSeparation] {
    & & & & & \node (p1-5) [point] {}; & \\
    \node (P0start) [firstPoint] {}; & & \node (p0-2) [terminal] {var}; & \node (p0-3) [terminal] {identifier}; & \node (p0-4) [point] {}; & \node (p0-5) [terminal] {\$type}; & \node (p0-6) [point] {}; & \node (p0-7) [terminal] {=}; & \node (p0-8) [nonterminal] {\nonTerminalSymbol{expression}{16}}; & \node (p0-9) [lastPoint] {}; & \\
  };
  \draw[->] (P0start) -- (p0-2) ;
  \draw[->] (p0-2) -- (p0-3) ;
  \draw[->] (p0-3) -- (p0-5) ;
  \draw (p0-4) |- (p1-5) ;
  \draw (p0-5) -- (p0-6) ;
  \draw[->] (p1-5) -| (p0-6) ;
  \draw[->] (p0-6) -- (p0-7) ;
  \draw[->] (p0-7) -- (p0-8) ;
  \draw[->] (p0-8) -- (p0-9) ;
\end{tikzpicture}

\nonTerminalSection{primary}{29}

\ruleSubsection{plm\_syntax}{expression-operator-priority}{360}

\begin{tikzpicture}
  \matrix[column sep=\ruleMatrixColumnSeparation, row sep=\ruleMatrixRowSeparation] {
    \node (P0start) [firstPoint] {}; & & \node (p0-2) [terminal] {$\sim$}; & \node (p0-3) [nonterminal] {\nonTerminalSymbol{primary}{29}}; & \node (p0-4) [lastPoint] {}; & \\
  };
  \draw[->] (P0start) -- (p0-2) ;
  \draw[->] (p0-2) -- (p0-3) ;
  \draw[->] (p0-3) -- (p0-4) ;
\end{tikzpicture}

\ruleSubsection{plm\_syntax}{expression-operator-priority}{373}

\begin{tikzpicture}
  \matrix[column sep=\ruleMatrixColumnSeparation, row sep=\ruleMatrixRowSeparation] {
    \node (P0start) [firstPoint] {}; & & \node (p0-2) [terminal] {not}; & \node (p0-3) [nonterminal] {\nonTerminalSymbol{primary}{29}}; & \node (p0-4) [lastPoint] {}; & \\
  };
  \draw[->] (P0start) -- (p0-2) ;
  \draw[->] (p0-2) -- (p0-3) ;
  \draw[->] (p0-3) -- (p0-4) ;
\end{tikzpicture}

\ruleSubsection{plm\_syntax}{expression-operator-priority}{386}

\begin{tikzpicture}
  \matrix[column sep=\ruleMatrixColumnSeparation, row sep=\ruleMatrixRowSeparation] {
    \node (P0start) [firstPoint] {}; & & \node (p0-2) [terminal] {-}; & \node (p0-3) [nonterminal] {\nonTerminalSymbol{primary}{29}}; & \node (p0-4) [lastPoint] {}; & \\
  };
  \draw[->] (P0start) -- (p0-2) ;
  \draw[->] (p0-2) -- (p0-3) ;
  \draw[->] (p0-3) -- (p0-4) ;
\end{tikzpicture}

\ruleSubsection{plm\_syntax}{expression-operator-priority}{399}

\begin{tikzpicture}
  \matrix[column sep=\ruleMatrixColumnSeparation, row sep=\ruleMatrixRowSeparation] {
    \node (P0start) [firstPoint] {}; & & \node (p0-2) [terminal] {-\verb=%=}; & \node (p0-3) [nonterminal] {\nonTerminalSymbol{primary}{29}}; & \node (p0-4) [lastPoint] {}; & \\
  };
  \draw[->] (P0start) -- (p0-2) ;
  \draw[->] (p0-2) -- (p0-3) ;
  \draw[->] (p0-3) -- (p0-4) ;
\end{tikzpicture}

\ruleSubsection{plm\_syntax}{expression-operator-priority}{412}

\begin{tikzpicture}
  \matrix[column sep=\ruleMatrixColumnSeparation, row sep=\ruleMatrixRowSeparation] {
    \node (P0start) [firstPoint] {}; & & \node (p0-2) [terminal] {(}; & \node (p0-3) [nonterminal] {\nonTerminalSymbol{expression}{16}}; & \node (p0-4) [terminal] {)}; & \node (p0-5) [lastPoint] {}; & \\
  };
  \draw[->] (P0start) -- (p0-2) ;
  \draw[->] (p0-2) -- (p0-3) ;
  \draw[->] (p0-3) -- (p0-4) ;
  \draw[->] (p0-4) -- (p0-5) ;
\end{tikzpicture}

\ruleSubsection{plm\_syntax}{expression-convert}{19}

\begin{tikzpicture}
  \matrix[column sep=\ruleMatrixColumnSeparation, row sep=\ruleMatrixRowSeparation] {
    \node (P0start) [firstPoint] {}; & & \node (p0-2) [terminal] {convert}; & \node (p0-3) [nonterminal] {\nonTerminalSymbol{expression}{16}}; & \node (p0-4) [terminal] {:}; & \node (p0-5) [terminal] {\$type}; & \node (p0-6) [lastPoint] {}; & \\
  };
  \draw[->] (P0start) -- (p0-2) ;
  \draw[->] (p0-2) -- (p0-3) ;
  \draw[->] (p0-3) -- (p0-4) ;
  \draw[->] (p0-4) -- (p0-5) ;
  \draw[->] (p0-5) -- (p0-6) ;
\end{tikzpicture}

\ruleSubsection{plm\_syntax}{expression-extend}{19}

\begin{tikzpicture}
  \matrix[column sep=\ruleMatrixColumnSeparation, row sep=\ruleMatrixRowSeparation] {
    \node (P0start) [firstPoint] {}; & & \node (p0-2) [terminal] {extend}; & \node (p0-3) [nonterminal] {\nonTerminalSymbol{expression}{16}}; & \node (p0-4) [terminal] {:}; & \node (p0-5) [terminal] {\$type}; & \node (p0-6) [lastPoint] {}; & \\
  };
  \draw[->] (P0start) -- (p0-2) ;
  \draw[->] (p0-2) -- (p0-3) ;
  \draw[->] (p0-3) -- (p0-4) ;
  \draw[->] (p0-4) -- (p0-5) ;
  \draw[->] (p0-5) -- (p0-6) ;
\end{tikzpicture}

\ruleSubsection{plm\_syntax}{expression-truncate}{19}

\begin{tikzpicture}
  \matrix[column sep=\ruleMatrixColumnSeparation, row sep=\ruleMatrixRowSeparation] {
    \node (P0start) [firstPoint] {}; & & \node (p0-2) [terminal] {truncate}; & \node (p0-3) [nonterminal] {\nonTerminalSymbol{expression}{16}}; & \node (p0-4) [terminal] {:}; & \node (p0-5) [terminal] {\$type}; & \node (p0-6) [lastPoint] {}; & \\
  };
  \draw[->] (P0start) -- (p0-2) ;
  \draw[->] (p0-2) -- (p0-3) ;
  \draw[->] (p0-3) -- (p0-4) ;
  \draw[->] (p0-4) -- (p0-5) ;
  \draw[->] (p0-5) -- (p0-6) ;
\end{tikzpicture}

\ruleSubsection{plm\_syntax}{expression-constructor-call}{24}

\begin{tikzpicture}
  \matrix[column sep=\ruleMatrixColumnSeparation, row sep=\ruleMatrixRowSeparation] {
    & & & & & & & & & \node (p2-9) [point] {}; & \\
    & & & & & & & \node (p1-7) [terminal] {!}; & \node (p1-8) [nonterminal] {\nonTerminalSymbol{expression}{16}}; & \\
    \node (P0start) [firstPoint] {}; & & \node (p0-2) [terminal] {\$type}; & \node (p0-3) [terminal] {(}; & \node (p0-4) [point] {}; & \node (p0-5) [point] {}; & \node (p0-6) [point] {}; & & & & \node (p0-10) [terminal] {)}; & \node (p0-11) [lastPoint] {}; & \\
  };
  \draw[->] (P0start) -- (p0-2) ;
  \draw[->] (p0-2) -- (p0-3) ;
  \draw (p0-3) -- (p0-5) ;
  \draw[->] (p0-6) |- (p1-7) ;
  \draw[->] (p1-7) -- (p1-8) ;
  \draw[->] (p2-9) -| (p0-4) ;
  \draw[->] (p1-8) -| (p2-9) ;
  \draw[->] (p0-5) -- (p0-10) ;
  \draw[->] (p0-10) -- (p0-11) ;
\end{tikzpicture}

\ruleSubsection{plm\_syntax}{expression-typed-constant}{18}

\begin{tikzpicture}
  \matrix[column sep=\ruleMatrixColumnSeparation, row sep=\ruleMatrixRowSeparation] {
    & & & \node (p1-3) [terminal] {\$type}; & \\
    \node (P0start) [firstPoint] {}; & & \node (p0-2) [point] {}; & \node (p0-3) [point] {}; & \node (p0-4) [point] {}; & \node (p0-5) [terminal] {.}; & \node (p0-6) [terminal] {identifier}; & \node (p0-7) [lastPoint] {}; & \\
  };
  \draw (P0start) -- (p0-3) ;
  \draw[->] (p0-2) |- (p1-3) ;
  \draw (p0-3) -- (p0-4) ;
  \draw[->] (p1-3) -| (p0-4) ;
  \draw[->] (p0-4) -- (p0-5) ;
  \draw[->] (p0-5) -- (p0-6) ;
  \draw[->] (p0-6) -- (p0-7) ;
\end{tikzpicture}

\ruleSubsection{plm\_syntax}{expression-if}{22}

\begin{tikzpicture}
  \matrix[column sep=\ruleMatrixColumnSeparation, row sep=\ruleMatrixRowSeparation] {
    \node (P0start) [firstPoint] {}; & & \node (p0-2) [terminal] {if}; & \node (p0-3) [nonterminal] {\nonTerminalSymbol{expression}{16}}; & \node (p0-4) [terminal] {\{}; & \node (p0-5) [nonterminal] {\nonTerminalSymbol{expression}{16}}; & \node (p0-6) [terminal] {\}}; & \node (p0-7) [terminal] {else}; & \node (p0-8) [terminal] {\{}; & \node (p0-9) [nonterminal] {\nonTerminalSymbol{expression}{16}}; & \node (p0-10) [terminal] {\}}; & \node (p0-11) [lastPoint] {}; & \\
  };
  \draw[->] (P0start) -- (p0-2) ;
  \draw[->] (p0-2) -- (p0-3) ;
  \draw[->] (p0-3) -- (p0-4) ;
  \draw[->] (p0-4) -- (p0-5) ;
  \draw[->] (p0-5) -- (p0-6) ;
  \draw[->] (p0-6) -- (p0-7) ;
  \draw[->] (p0-7) -- (p0-8) ;
  \draw[->] (p0-8) -- (p0-9) ;
  \draw[->] (p0-9) -- (p0-10) ;
  \draw[->] (p0-10) -- (p0-11) ;
\end{tikzpicture}

\ruleSubsection{plm\_syntax}{expression-literal-integer}{17}

\begin{tikzpicture}
  \matrix[column sep=\ruleMatrixColumnSeparation, row sep=\ruleMatrixRowSeparation] {
    \node (P0start) [firstPoint] {}; & & \node (p0-2) [terminal] {integer}; & \node (p0-3) [lastPoint] {}; & \\
  };
  \draw[->] (P0start) -- (p0-2) ;
  \draw[->] (p0-2) -- (p0-3) ;
\end{tikzpicture}

\ruleSubsection{plm\_syntax}{expression-literal-string}{17}

\begin{tikzpicture}
  \matrix[column sep=\ruleMatrixColumnSeparation, row sep=\ruleMatrixRowSeparation] {
    \node (P0start) [firstPoint] {}; & & \node (p0-2) [terminal] {"string"}; & \node (p0-3) [lastPoint] {}; & \\
  };
  \draw[->] (P0start) -- (p0-2) ;
  \draw[->] (p0-2) -- (p0-3) ;
\end{tikzpicture}

\ruleSubsection{plm\_syntax}{expression-true-false}{17}

\begin{tikzpicture}
  \matrix[column sep=\ruleMatrixColumnSeparation, row sep=\ruleMatrixRowSeparation] {
    \node (P0start) [firstPoint] {}; & & \node (p0-2) [terminal] {true}; & \node (p0-3) [lastPoint] {}; & \\
  };
  \draw[->] (P0start) -- (p0-2) ;
  \draw[->] (p0-2) -- (p0-3) ;
\end{tikzpicture}

\ruleSubsection{plm\_syntax}{expression-true-false}{24}

\begin{tikzpicture}
  \matrix[column sep=\ruleMatrixColumnSeparation, row sep=\ruleMatrixRowSeparation] {
    \node (P0start) [firstPoint] {}; & & \node (p0-2) [terminal] {false}; & \node (p0-3) [lastPoint] {}; & \\
  };
  \draw[->] (P0start) -- (p0-2) ;
  \draw[->] (p0-2) -- (p0-3) ;
\end{tikzpicture}

\ruleSubsection{plm\_syntax}{expression-cst-registre}{27}

\begin{tikzpicture}
  \matrix[column sep=\ruleMatrixColumnSeparation, row sep=\ruleMatrixRowSeparation] {
    & & & & & & & & & & & & \node (p2-12) [point] {}; & \\
    & & & & & & & \node (p1-7) [terminal] {:}; & \node (p1-8) [nonterminal] {\nonTerminalSymbol{expression}{16}}; & & & \node (p1-11) [terminal] {,}; & \\
    \node (P0start) [firstPoint] {}; & & \node (p0-2) [terminal] {\$type}; & \node (p0-3) [terminal] {\{}; & \node (p0-4) [point] {}; & \node (p0-5) [terminal] {identifier}; & \node (p0-6) [point] {}; & \node (p0-7) [point] {}; & & \node (p0-9) [point] {}; & \node (p0-10) [point] {}; & & & \node (p0-13) [terminal] {\}}; & \node (p0-14) [lastPoint] {}; & \\
  };
  \draw[->] (P0start) -- (p0-2) ;
  \draw[->] (p0-2) -- (p0-3) ;
  \draw[->] (p0-3) -- (p0-5) ;
  \draw (p0-5) -- (p0-7) ;
  \draw[->] (p0-6) |- (p1-7) ;
  \draw[->] (p1-7) -- (p1-8) ;
  \draw (p0-7) -- (p0-9) ;
  \draw[->] (p1-8) -| (p0-9) ;
  \draw[->] (p0-10) |- (p1-11) ;
  \draw[->] (p2-12) -| (p0-4) ;
  \draw[->] (p1-11) -| (p2-12) ;
  \draw[->] (p0-9) -- (p0-13) ;
  \draw[->] (p0-13) -- (p0-14) ;
\end{tikzpicture}

\ruleSubsection{plm\_syntax}{expression-primary}{69}

\begin{tikzpicture}
  \matrix[column sep=\ruleMatrixColumnSeparation, row sep=\ruleMatrixRowSeparation] {
    & & & & & & & & & & & & & \node (p4-13) [point] {}; & \\
    & & & & & & & & & & \node (p3-10) [nonterminal] {\nonTerminalSymbol{effective\_parameters}{30}}; & \\
    & & & & & & & & & & \node (p2-10) [terminal] {[}; & \node (p2-11) [nonterminal] {\nonTerminalSymbol{expression}{16}}; & \node (p2-12) [terminal] {]}; & \\
    & & & \node (p1-3) [terminal] {self}; & \node (p1-4) [terminal] {.}; & & & & & & \node (p1-10) [terminal] {.}; & \node (p1-11) [terminal] {identifier}; & \\
    \node (P0start) [firstPoint] {}; & & \node (p0-2) [point] {}; & \node (p0-3) [point] {}; & & \node (p0-5) [point] {}; & \node (p0-6) [terminal] {identifier}; & \node (p0-7) [point] {}; & \node (p0-8) [point] {}; & \node (p0-9) [point] {}; & & & & & \node (p0-14) [lastPoint] {}; & \\
  };
  \draw (P0start) -- (p0-3) ;
  \draw[->] (p0-2) |- (p1-3) ;
  \draw[->] (p1-3) -- (p1-4) ;
  \draw (p0-3) -- (p0-5) ;
  \draw[->] (p1-4) -| (p0-5) ;
  \draw[->] (p0-5) -- (p0-6) ;
  \draw (p0-6) -- (p0-8) ;
  \draw[->] (p0-9) |- (p1-10) ;
  \draw[->] (p1-10) -- (p1-11) ;
  \draw[->] (p0-9) |- (p2-10) ;
  \draw[->] (p2-10) -- (p2-11) ;
  \draw[->] (p2-11) -- (p2-12) ;
  \draw[->] (p0-9) |- (p3-10) ;
  \draw[->] (p4-13) -| (p0-7) ;
  \draw[->] (p1-11) -| (p4-13) ;
  \draw[->] (p2-12) -| (p4-13) ;
  \draw[->] (p3-10) -| (p4-13) ;
  \draw[->] (p0-8) -- (p0-14) ;
\end{tikzpicture}

\nonTerminalSection{primitive}{3}

\ruleSubsection{plm\_syntax}{declaration-primitive}{23}

\begin{tikzpicture}
  \matrix[column sep=\ruleMatrixColumnSeparation, row sep=\ruleMatrixRowSeparation] {
    & & & & & & & & & & & \node (p2-11) [point] {}; & \\
    & & & \node (p1-3) [terminal] {public}; & & & & & & & \node (p1-10) [terminal] {@attribute}; & & & & \node (p1-14) [terminal] {->}; & \node (p1-15) [terminal] {\$type}; & \\
    \node (P0start) [firstPoint] {}; & & \node (p0-2) [point] {}; & \node (p0-3) [point] {}; & \node (p0-4) [point] {}; & \node (p0-5) [terminal] {primitive}; & \node (p0-6) [terminal] {identifier}; & \node (p0-7) [point] {}; & \node (p0-8) [point] {}; & \node (p0-9) [point] {}; & & & \node (p0-12) [nonterminal] {\nonTerminalSymbol{procedure\_formal\_arguments}{14}}; & \node (p0-13) [point] {}; & \node (p0-14) [point] {}; & & \node (p0-16) [point] {}; & \node (p0-17) [terminal] {\{}; & \node (p0-18) [nonterminal] {\nonTerminalSymbol{instructionList}{31}}; & \node (p0-19) [terminal] {\}}; & \node (p0-20) [lastPoint] {}; & \\
  };
  \draw (P0start) -- (p0-3) ;
  \draw[->] (p0-2) |- (p1-3) ;
  \draw (p0-3) -- (p0-4) ;
  \draw[->] (p1-3) -| (p0-4) ;
  \draw[->] (p0-4) -- (p0-5) ;
  \draw[->] (p0-5) -- (p0-6) ;
  \draw (p0-6) -- (p0-8) ;
  \draw[->] (p0-9) |- (p1-10) ;
  \draw[->] (p2-11) -| (p0-7) ;
  \draw[->] (p1-10) -| (p2-11) ;
  \draw[->] (p0-8) -- (p0-12) ;
  \draw (p0-12) -- (p0-14) ;
  \draw[->] (p0-13) |- (p1-14) ;
  \draw[->] (p1-14) -- (p1-15) ;
  \draw (p0-14) -- (p0-16) ;
  \draw[->] (p1-15) -| (p0-16) ;
  \draw[->] (p0-16) -- (p0-17) ;
  \draw[->] (p0-17) -- (p0-18) ;
  \draw[->] (p0-18) -- (p0-19) ;
  \draw[->] (p0-19) -- (p0-20) ;
\end{tikzpicture}

\nonTerminalSection{procedure}{0}

\ruleSubsection{plm\_syntax}{declaration-func}{70}

\begin{tikzpicture}
  \matrix[column sep=\ruleMatrixColumnSeparation, row sep=\ruleMatrixRowSeparation] {
    & & & & \node (p1-4) [terminal] {->}; & \node (p1-5) [terminal] {\$type}; & \\
    \node (P0start) [firstPoint] {}; & & \node (p0-2) [nonterminal] {\nonTerminalSymbol{procedure\_header}{13}}; & \node (p0-3) [point] {}; & \node (p0-4) [point] {}; & & \node (p0-6) [point] {}; & \node (p0-7) [terminal] {\{}; & \node (p0-8) [nonterminal] {\nonTerminalSymbol{instructionList}{31}}; & \node (p0-9) [terminal] {\}}; & \node (p0-10) [lastPoint] {}; & \\
  };
  \draw[->] (P0start) -- (p0-2) ;
  \draw (p0-2) -- (p0-4) ;
  \draw[->] (p0-3) |- (p1-4) ;
  \draw[->] (p1-4) -- (p1-5) ;
  \draw (p0-4) -- (p0-6) ;
  \draw[->] (p1-5) -| (p0-6) ;
  \draw[->] (p0-6) -- (p0-7) ;
  \draw[->] (p0-7) -- (p0-8) ;
  \draw[->] (p0-8) -- (p0-9) ;
  \draw[->] (p0-9) -- (p0-10) ;
\end{tikzpicture}

\nonTerminalSection{procedure\_call}{36}

\ruleSubsection{plm\_syntax}{instruction-procedure-call}{26}

\begin{tikzpicture}
  \matrix[column sep=\ruleMatrixColumnSeparation, row sep=\ruleMatrixRowSeparation] {
    \node (P0start) [firstPoint] {}; & & \node (p0-2) [nonterminal] {\nonTerminalSymbol{assignment\_target}{37}}; & \node (p0-3) [nonterminal] {\nonTerminalSymbol{effective\_parameters}{30}}; & \node (p0-4) [lastPoint] {}; & \\
  };
  \draw[->] (P0start) -- (p0-2) ;
  \draw[->] (p0-2) -- (p0-3) ;
  \draw[->] (p0-3) -- (p0-4) ;
\end{tikzpicture}

\nonTerminalSection{procedure\_formal\_arguments}{14}

\ruleSubsection{plm\_syntax}{declaration-func}{132}

\begin{tikzpicture}
  \matrix[column sep=\ruleMatrixColumnSeparation, row sep=\ruleMatrixRowSeparation] {
    & & & & & & & & & \node (p4-9) [point] {}; & \\
    & & & & & & \node (p3-6) [terminal] {?}; & \node (p3-7) [terminal] {identifier}; & \node (p3-8) [terminal] {\$type}; & \\
    & & & & & & \node (p2-6) [terminal] {?!}; & \node (p2-7) [terminal] {identifier}; & \node (p2-8) [terminal] {\$type}; & \\
    & & & & & & \node (p1-6) [terminal] {!}; & \node (p1-7) [terminal] {identifier}; & \node (p1-8) [terminal] {\$type}; & \\
    \node (P0start) [firstPoint] {}; & & \node (p0-2) [terminal] {(}; & \node (p0-3) [point] {}; & \node (p0-4) [point] {}; & \node (p0-5) [point] {}; & & & & & \node (p0-10) [terminal] {)}; & \node (p0-11) [lastPoint] {}; & \\
  };
  \draw[->] (P0start) -- (p0-2) ;
  \draw (p0-2) -- (p0-4) ;
  \draw[->] (p0-5) |- (p1-6) ;
  \draw[->] (p1-6) -- (p1-7) ;
  \draw[->] (p1-7) -- (p1-8) ;
  \draw[->] (p0-5) |- (p2-6) ;
  \draw[->] (p2-6) -- (p2-7) ;
  \draw[->] (p2-7) -- (p2-8) ;
  \draw[->] (p0-5) |- (p3-6) ;
  \draw[->] (p3-6) -- (p3-7) ;
  \draw[->] (p3-7) -- (p3-8) ;
  \draw[->] (p4-9) -| (p0-3) ;
  \draw[->] (p1-8) -| (p4-9) ;
  \draw[->] (p2-8) -| (p4-9) ;
  \draw[->] (p3-8) -| (p4-9) ;
  \draw[->] (p0-4) -- (p0-10) ;
  \draw[->] (p0-10) -- (p0-11) ;
\end{tikzpicture}

\nonTerminalSection{procedure\_header}{13}

\ruleSubsection{plm\_syntax}{declaration-func}{100}

\begin{tikzpicture}
  \matrix[column sep=\ruleMatrixColumnSeparation, row sep=\ruleMatrixRowSeparation] {
    & & & & & & & & & & \node (p2-10) [point] {}; & & & & & & \node (p2-16) [point] {}; & \\
    & & & \node (p1-3) [terminal] {public}; & & & & & & \node (p1-9) [point] {}; & & & & & & \node (p1-15) [terminal] {@attribute}; & \\
    \node (P0start) [firstPoint] {}; & & \node (p0-2) [point] {}; & \node (p0-3) [point] {}; & \node (p0-4) [point] {}; & \node (p0-5) [terminal] {func}; & \node (p0-6) [point] {}; & \node (p0-7) [terminal] {`mode}; & \node (p0-8) [point] {}; & & & \node (p0-11) [terminal] {identifier}; & \node (p0-12) [point] {}; & \node (p0-13) [point] {}; & \node (p0-14) [point] {}; & & & \node (p0-17) [nonterminal] {\nonTerminalSymbol{procedure\_formal\_arguments}{14}}; & \node (p0-18) [lastPoint] {}; & \\
  };
  \draw (P0start) -- (p0-3) ;
  \draw[->] (p0-2) |- (p1-3) ;
  \draw (p0-3) -- (p0-4) ;
  \draw[->] (p1-3) -| (p0-4) ;
  \draw[->] (p0-4) -- (p0-5) ;
  \draw[->] (p0-5) -- (p0-7) ;
  \draw (p0-8) |- (p1-9) ;
  \draw[->] (p2-10) -| (p0-6) ;
  \draw[->] (p1-9) -| (p2-10) ;
  \draw[->] (p0-7) -- (p0-11) ;
  \draw (p0-11) -- (p0-13) ;
  \draw[->] (p0-14) |- (p1-15) ;
  \draw[->] (p2-16) -| (p0-12) ;
  \draw[->] (p1-15) -| (p2-16) ;
  \draw[->] (p0-13) -- (p0-17) ;
  \draw[->] (p0-17) -- (p0-18) ;
\end{tikzpicture}

\nonTerminalSection{section}{1}

\ruleSubsection{plm\_syntax}{declaration-section}{23}

\begin{tikzpicture}
  \matrix[column sep=\ruleMatrixColumnSeparation, row sep=\ruleMatrixRowSeparation] {
    & & & & & & & & & & & \node (p2-11) [point] {}; & \\
    & & & \node (p1-3) [terminal] {public}; & & & & & & & \node (p1-10) [terminal] {@attribute}; & & & & \node (p1-14) [terminal] {->}; & \node (p1-15) [terminal] {\$type}; & \\
    \node (P0start) [firstPoint] {}; & & \node (p0-2) [point] {}; & \node (p0-3) [point] {}; & \node (p0-4) [point] {}; & \node (p0-5) [terminal] {section}; & \node (p0-6) [terminal] {identifier}; & \node (p0-7) [point] {}; & \node (p0-8) [point] {}; & \node (p0-9) [point] {}; & & & \node (p0-12) [nonterminal] {\nonTerminalSymbol{procedure\_formal\_arguments}{14}}; & \node (p0-13) [point] {}; & \node (p0-14) [point] {}; & & \node (p0-16) [point] {}; & \node (p0-17) [terminal] {\{}; & \node (p0-18) [nonterminal] {\nonTerminalSymbol{instructionList}{31}}; & \node (p0-19) [terminal] {\}}; & \node (p0-20) [lastPoint] {}; & \\
  };
  \draw (P0start) -- (p0-3) ;
  \draw[->] (p0-2) |- (p1-3) ;
  \draw (p0-3) -- (p0-4) ;
  \draw[->] (p1-3) -| (p0-4) ;
  \draw[->] (p0-4) -- (p0-5) ;
  \draw[->] (p0-5) -- (p0-6) ;
  \draw (p0-6) -- (p0-8) ;
  \draw[->] (p0-9) |- (p1-10) ;
  \draw[->] (p2-11) -| (p0-7) ;
  \draw[->] (p1-10) -| (p2-11) ;
  \draw[->] (p0-8) -- (p0-12) ;
  \draw (p0-12) -- (p0-14) ;
  \draw[->] (p0-13) |- (p1-14) ;
  \draw[->] (p1-14) -- (p1-15) ;
  \draw (p0-14) -- (p0-16) ;
  \draw[->] (p1-15) -| (p0-16) ;
  \draw[->] (p0-16) -- (p0-17) ;
  \draw[->] (p0-17) -- (p0-18) ;
  \draw[->] (p0-18) -- (p0-19) ;
  \draw[->] (p0-19) -- (p0-20) ;
\end{tikzpicture}

\nonTerminalSection{service}{2}

\ruleSubsection{plm\_syntax}{declaration-service}{23}

\begin{tikzpicture}
  \matrix[column sep=\ruleMatrixColumnSeparation, row sep=\ruleMatrixRowSeparation] {
    & & & & & & & & & & & \node (p2-11) [point] {}; & \\
    & & & \node (p1-3) [terminal] {public}; & & & & & & & \node (p1-10) [terminal] {@attribute}; & & & & \node (p1-14) [terminal] {->}; & \node (p1-15) [terminal] {\$type}; & \\
    \node (P0start) [firstPoint] {}; & & \node (p0-2) [point] {}; & \node (p0-3) [point] {}; & \node (p0-4) [point] {}; & \node (p0-5) [terminal] {service}; & \node (p0-6) [terminal] {identifier}; & \node (p0-7) [point] {}; & \node (p0-8) [point] {}; & \node (p0-9) [point] {}; & & & \node (p0-12) [nonterminal] {\nonTerminalSymbol{procedure\_formal\_arguments}{14}}; & \node (p0-13) [point] {}; & \node (p0-14) [point] {}; & & \node (p0-16) [point] {}; & \node (p0-17) [terminal] {\{}; & \node (p0-18) [nonterminal] {\nonTerminalSymbol{instructionList}{31}}; & \node (p0-19) [terminal] {\}}; & \node (p0-20) [lastPoint] {}; & \\
  };
  \draw (P0start) -- (p0-3) ;
  \draw[->] (p0-2) |- (p1-3) ;
  \draw (p0-3) -- (p0-4) ;
  \draw[->] (p1-3) -| (p0-4) ;
  \draw[->] (p0-4) -- (p0-5) ;
  \draw[->] (p0-5) -- (p0-6) ;
  \draw (p0-6) -- (p0-8) ;
  \draw[->] (p0-9) |- (p1-10) ;
  \draw[->] (p2-11) -| (p0-7) ;
  \draw[->] (p1-10) -| (p2-11) ;
  \draw[->] (p0-8) -- (p0-12) ;
  \draw (p0-12) -- (p0-14) ;
  \draw[->] (p0-13) |- (p1-14) ;
  \draw[->] (p1-14) -- (p1-15) ;
  \draw (p0-14) -- (p0-16) ;
  \draw[->] (p1-15) -| (p0-16) ;
  \draw[->] (p0-16) -- (p0-17) ;
  \draw[->] (p0-17) -- (p0-18) ;
  \draw[->] (p0-18) -- (p0-19) ;
  \draw[->] (p0-19) -- (p0-20) ;
\end{tikzpicture}

\nonTerminalSection{start\_symbol}{6}

\ruleSubsection{plm\_syntax}{syntax-grammar}{31}

\begin{tikzpicture}
  \matrix[column sep=\ruleMatrixColumnSeparation, row sep=\ruleMatrixRowSeparation] {
    & & & & & & \node (p8-6) [point] {}; & \\
    & & & & & \node (p7-5) [nonterminal] {\nonTerminalSymbol{import\_file}{5}}; & \\
    & & & & & \node (p6-5) [nonterminal] {\nonTerminalSymbol{isr}{4}}; & \\
    & & & & & \node (p5-5) [nonterminal] {\nonTerminalSymbol{primitive}{3}}; & \\
    & & & & & \node (p4-5) [nonterminal] {\nonTerminalSymbol{service}{2}}; & \\
    & & & & & \node (p3-5) [nonterminal] {\nonTerminalSymbol{section}{1}}; & \\
    & & & & & \node (p2-5) [nonterminal] {\nonTerminalSymbol{procedure}{0}}; & \\
    & & & & & \node (p1-5) [nonterminal] {\nonTerminalSymbol{declaration}{7}}; & \\
    \node (P0start) [firstPoint] {}; & & \node (p0-2) [point] {}; & \node (p0-3) [point] {}; & \node (p0-4) [point] {}; & & & \node (p0-7) [lastPoint] {}; & \\
  };
  \draw (P0start) -- (p0-3) ;
  \draw[->] (p0-4) |- (p1-5) ;
  \draw[->] (p0-4) |- (p2-5) ;
  \draw[->] (p0-4) |- (p3-5) ;
  \draw[->] (p0-4) |- (p4-5) ;
  \draw[->] (p0-4) |- (p5-5) ;
  \draw[->] (p0-4) |- (p6-5) ;
  \draw[->] (p0-4) |- (p7-5) ;
  \draw[->] (p8-6) -| (p0-2) ;
  \draw[->] (p1-5) -| (p8-6) ;
  \draw[->] (p2-5) -| (p8-6) ;
  \draw[->] (p3-5) -| (p8-6) ;
  \draw[->] (p4-5) -| (p8-6) ;
  \draw[->] (p5-5) -| (p8-6) ;
  \draw[->] (p6-5) -| (p8-6) ;
  \draw[->] (p7-5) -| (p8-6) ;
  \draw[->] (p0-3) -- (p0-7) ;
\end{tikzpicture}



}
