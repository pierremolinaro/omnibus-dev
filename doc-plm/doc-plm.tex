%!TEX TS-program = personallatex
%!TEX encoding = UTF-8 Unicode

% Le script utilisé pour compiler est :
%       xelatex --file-line-error --shell-escape --synctex=1

\documentclass[a4paper,10pt,obeyspaces,openany]{book}
\usepackage{verbatim}
% L'option 'openany' permet de démarrer un chapitre sur une page paire

% L'option 'obeyspaces' indique au paquetage hyperref de respecter les espaces dans les chemins \path ou \url.
%
% Par exemple : \path{ab cd} est écrit ab cd
% Sans cette option, \path{ab cd} est écrit abcd
% Voir : http://compgroups.net/comp.text.tex/space-in-path-hyperref-package

%-----------------------------------------------------------------------------------------------------------------------*
%                                                                                                                       *
%   « N O I R    E T    B L A N C »    O U    « C O U L E U R »                                                         *
%                                                                                                                       *
%-----------------------------------------------------------------------------------------------------------------------*

%--- Par défaut, l'impression se fait en couleur
\providecommand{\sortieEnCouleur}{true}

%-----------------------------------------------------------------------------------------------------------------------*
%                                                                                                                       *
%   E N C O D A G E    D E S    S O U R C E S     :     U T F 8                                                         *
%                                                                                                                       *
%-----------------------------------------------------------------------------------------------------------------------*

%--------------------------- Pour compilation XeLaTex
% http://www.tuteurs.ens.fr/logiciels/latex/xetex.html
\RequirePackage{etex}
\usepackage{fontspec}
\setmainfont[Ligatures=TeX, Scale=1.1]{Gill Sans} % {Gill Sans} % {Palatino} % {Optima} % {Hoefler Text}
\setsansfont[Ligatures={NoCommon}]{Courier}
\setmonofont[Ligatures=NoCommon, Scale=0.9]{Menlo}
\usepackage{polyglossia}
\setdefaultlanguage{french}

%-----------------------------------------------------------------------------------------------------------------------*

% http://forum.mathematex.net/latex-f6/supprimer-les-espaces-devant-la-ponctuation-t15951.html
\makeatletter
\newcommand{\nospace}[1]{\nofrench@punctuation\texttt{#1}}
\makeatother

%-----------------------------------------------------------------------------------------------------------------------*

%--- Latex demande ce paquetage pour mieux afficher le caractère "°" et \textquotesingle "'"
\usepackage{textcomp}

%--- Contrôle de l'indentation et de la séparation des paragraphes
\setlength{\parindent}{0pt} 
%\setlength{\parskip}{1.2ex} % Reporté avant les chapitres

%--- Ajouter une séparation à la fin des itemize
%\let\EndItemize\enditemize
%\def\enditemize{\EndItemize\vspace{1.2ex}}

%-----------------------------------------------------------------------------------------------------------------------*
%                                                                                                                       *
%   M I S E    E N    P A G E S    D E M A N D É E     P A R     E L L I P S E S                                        *
%                                                                                                                       *
%-----------------------------------------------------------------------------------------------------------------------*

%--- Interligne 1,5 ligne
\usepackage{setspace}
\onehalfspacing

%---------------------------------------------------- Mise en page
% Voir "Une courte introduction à Latex2e", § 6.4

%--- Marge gauche : 2 cm ; le paramètre \hoffset contient cette valeur, moins 1 pouce
%    \hoffset = 2 cm - 2,54 cm = -0,54 cm
\setlength{\hoffset}{-0.54 cm}
%
%%--- Hauteur de la page : 29,7 cm
%\setlength{\paperheight}{29.7 cm}
%
%--- Largeur de la page
%\setlength{\paperwidth}{18 cm}
%
%%--- Marges supplémentaires, différenciées pour les pages gauches et droites ; ici, aucune.
%\setlength{\oddsidemargin }{0 cm}
%\setlength{\evensidemargin}{0 cm}
%
%--- Largeur du texte
\setlength{\textwidth}{15cm}
%
%%--- Marge haute : 2,8 cm ; le paramètre \voffset contient cette valeur, moins 1 pouce
%%    \voffset = 2,8 cm - 2,54 cm = 0,26 cm
%\setlength{\voffset}{0.26 cm}
%
%%--- Distance entre la marge haute et l'en-tête : 0 cm
%\setlength{\topmargin}{0 cm}
%
%%--- Hauteur de l'en-tête de chaque page : 1 cm
%\setlength{\headheight}{1 cm}
%
%%--- Distance entre l'en-tête de chaque page et le corps : 0,5 cm
%\setlength{\headsep}{0.5 cm}
%
%%--- Hauteur du corps
%%    \textheight = 29,7 cm - 2,8 cm - 2,8 cm - 1,5 cm = 22,6 cm
%\setlength{\textheight}{22.6 cm}
%
%%--------------------- Changer la taille des notes en bas de page
%%  footnotesize ---> small
%
%\let\oldfootnotesize\footnotesize
%\renewcommand*{\footnotesize}{\oldfootnotesize\small}

%-----------------------------------------------------------------------------------------------------------------------*
%                                                                                                                       *
%   T R A C E R    U N E    L I G N E    H O R I Z O N T A L E                                                          *
%                                                                                                                       *
%-----------------------------------------------------------------------------------------------------------------------*

%\newcommand\ligne{\noindent\makebox[\linewidth]{\rule{\textwidth}{0.5pt}}}
\newcommand\ligne{\hrulefill}

%-----------------------------------------------------------------------------------------------------------------------*
%                                                                                                                       *
%   P A Q U E T A G E    « I F T H E N »                                                                                *
%                                                                                                                       *
%-----------------------------------------------------------------------------------------------------------------------*

%--- Ce paquetage permet d'effectuer des tests : \ifthenelse{test}{bloc then}{bloc else}
\usepackage{ifthen}

%-----------------------------------------------------------------------------------------------------------------------*
%                                                                                                                       *
%   G E S T I O N    D E    L ' I N D E X                                                                               *
%                                                                                                                       *
%-----------------------------------------------------------------------------------------------------------------------*

% http://www.cuk.ch/articles/4097
% http://www.tuteurs.ens.fr/logiciels/latex/makeindex.html
% http://linux.die.net/man/1/makeindex
%
% Attention ! Les deux commandes suivantes, ainsi que le \printindex placé plus bas ne
% sont pas suffisants pour construire l'index : il faut utiliser l'utilitaire "makeIndex"
% Voir le fichier de commande "build.command"
\usepackage{makeidx}
\makeindex

%-----------------------------------------------------------------------------------------------------------------------*
%                                                                                                                       *
%   N A M E R E F                                                                                                     *
%                                                                                                                       *
%-----------------------------------------------------------------------------------------------------------------------*
%  http://tex.stackexchange.com/questions/6238/get-the-title-instead-of-the-number-of-a-referenced-chapter-section

\usepackage{nameref}% Only if hyperref isn't loaded

%-----------------------------------------------------------------------------------------------------------------------*
%                                                                                                                       *
%   E N U M I T E M                                                                                                     *
%                                                                                                                       *
%-----------------------------------------------------------------------------------------------------------------------*
% Ce paquetage permet de personaliser les listes, par exemple : 
% http://www.xm1math.net/doculatex/listes.html
%\begin{enumerate}[label=(\arabic*)]
%  \item ... ; % (1) ...
%  \item ... ; % (2) ...
%\end{enumerate}

\usepackage{enumitem}

%-----------------------------------------------------------------------------------------------------------------------*
%                                                                                                                       *
%   H Y P E R R E F                                                                                                     *
%                                                                                                                       *
%-----------------------------------------------------------------------------------------------------------------------*

%--- Pour les hyperliens, et le contrôle de la génération PDF 
\usepackage[pagebackref]{hyperref}
\usepackage{bookmark} % see http://tex.stackexchange.com/questions/176113/problem-with-in-pdf-bookmark-under-xelatex

\hypersetup{pdftitle      = {PLM}}
\hypersetup{pdfauthor     = {Pierre Molinaro}}

\hypersetup{colorlinks    = true}
\hypersetup{anchorcolor   = black}
\hypersetup{filecolor     = black}
\hypersetup{menucolor     = black}
\hypersetup{plainpages    = false}
\hypersetup{pdfstartview  = FitH}
\hypersetup{pdfpagelayout = OneColumn}

%--- Contrôle des références de la bibliographie ; avec ces lignes, les pages où chaque item de la biblio
%    est appelée sont insérées à la fin de chaque entrée.
%\hypersetup{backref        = page}

\renewcommand*{\backrefalt}[4]{
  \ifcase #1 % case: not cited
    \ifthenelse{\equal{\sortieEnCouleur}{true}}{\textcolor{red}{\textbf{(non cité)}}}{\textbf{(non cité)}}. 
  \or % case: cited on exactly one page 
    (cité page #2).% 
  \else % case: cited on multiple pages 
    (cité pages #2). 
  \fi
} 
\renewcommand*{\backreftwosep}{ et }
\renewcommand*{\backreflastsep}{ et }


%--- Les hyperliens sont colorés en bleu dans la phase de mise au point, ils sont ainsi mieux visibles
\ifthenelse{\equal{\sortieEnCouleur}{true}}{
  \hypersetup{linkcolor  = blue}
  \hypersetup{urlcolor   = blue}
  \hypersetup{citecolor  = blue}
}{
  \hypersetup{linkcolor  = black}
  \hypersetup{urlcolor   = black}
  \hypersetup{citecolor  = black}
}

%-----------------------------------------------------------------------------------------------------------------------*
%------------------------------------------------------------------------------------------ RÉFÉRENCES À UN TABLEAU
% La référence au tableau "nom-du-tableau" est définie par \labelTableau{nom-du-tableau}
\newcommand\labelTableau[1]{\label{tab:#1}}
% Latex autorise deux types d'appel à une référence \ref{tab:nom-du-tableau} et \pageref{tab:nom-du-tableau}

% \refTableau{}{nom-du-tableau} ---> "tableau x.y"   où x.y est le n° du tableau
\newcommand\refTableau[1]{\hyperref[tab:#1]{tableau \ref*{tab:#1}}}

% \refTableauSansPrefixe{}{nom-du-tableau} ---> "x.y"   où x.y est le n° du tableau
\newcommand\refTableauSansPrefixe[1]{\hyperref[tab:#1]{\ref*{tab:#1}}}

% \refTableauPage{}{nom-du-tableau} ---> "tableau x.y page n"   où x.y est le n° du tableau
\newcommand\refTableauPage[1]{\hyperref[tab:#1]{tableau \ref*{tab:#1} page \pageref{tab:#1}}}

% \refTableauPageSansPrefixe{}{nom-du-tableau} ---> "x.y page n"   où x.y est le n° du tableau
\newcommand\refTableauPageSansPrefixe[1]{\hyperref[tab:#1]{\ref*{tab:#1} page \pageref{tab:#1}}}

%------------------------------------------------------------------------------------------ RÉFÉRENCES À UNE FIGURE
% La référence au tableau "nom-de-la-figure" est définie par \labelFigure{nom-de-la-figure}
\newcommand\labelFigure[1]{\label{fig:#1}}
% Latex autorise deux types d'appel à une référence \ref{fig:nom-de-la-figure} et \pageref{fig:nom-de-la-figure}

% \refFigure{}{nom-de-la-figure}   ---> "figure x.y"   où x.y est le n° de la figure
% \refFigure{z}{nom-de-la-figure}  ---> "figure x.y.z" où x.y est le n° de la figure
\newcommand\refFigure[2]{\hyperref[fig:#2]{figure \ref*{fig:#2}{\ifthenelse{\equal{#1}{}}{}{.#1}}}}

% \refFigureSansPrefixe{}{nom-de-la-figure}   ---> "x.y"   où x.y est le n° de la figure
% \refFigureSansPrefixe{z}{nom-de-la-figure}  ---> "x.y.z" où x.y est le n° de la figure
\newcommand\refFigureSansPrefixe[2]{\hyperref[fig:#2]{\ref*{fig:#2}{\ifthenelse{\equal{#1}{}}{}{.#1}}}}

% \refFigurePage{}{nom-de-la-figure}   ---> "figure x.y page n"   où x.y est le n° de la figure
% \refFigurePage{z}{nom-de-la-figure}  ---> "figure x.y.z page n" où x.y est le n° de la figure
\newcommand\refFigurePage[2]{\hyperref[fig:#2]{figure \ref*{fig:#2}{\ifthenelse{\equal{#1}{}}{}{.#1}} page \pageref{fig:#2}}}

% \refFigurePageSansPrefixe{}{nom-de-la-figure}   ---> "x.y page n"   où x.y est le n° de la figure
% \refFigurePageSansPrefixe{z}{nom-de-la-figure}  ---> "x.y.z page n" où x.y est le n° de la figure
\newcommand\refFigurePageSansPrefixe[2]{\hyperref[fig:#2]{\ref*{fig:#2}{\ifthenelse{\equal{#1}{}}{}{.#1}} page \pageref{fig:#2}}}

%------------------------------------------------------------------------------------------ RÉFÉRENCES À UN CHAPITRE
% Au lieu d'écrire \chapter{titre-chapitre}, on écrit \chapterLabel{titre-chapitre}{label-chapitre}
\newcommand\chapterLabel[2]{\chapter{#1}\label{chapter:#2}}


% \refChapter{label-chapter} ---> "chapitre n"
\newcommand\refChapter[1]{\hyperref[chapter:#1]{chapitre \ref*{chapter:#1}}}

\newcommand\refChapterPage[1]{\hyperref[chapter:#1]{chapitre \ref*{chapter:#1} page \pageref{chapter:#1}}}
\newcommand\refChapterTitlePage[1]{\hyperref[chapter:#1]{chapitre «~\emph{\nameref*{chapter:#1}}~» page \pageref{chapter:#1}}}
\newcommand\refChapterTitle[1]{\nameref{chapter:#1}}

%------------------------------------------------------------------------------------------ RÉFÉRENCES À UNE SECTION
% Au lieu d'écrire \section{titre-section}, on écrit \sectionLabel{titre-section}{label-section}
\newcommand\sectionLabel[2]{\section{#1}\label{sec:#2}}


% \refSectionPage{label-section} ---> "section x.y page n"   où x.y est le n° de la section
\newcommand\refSection[1]{\hyperref[sec:#1]{section \ref*{sec:#1}}}
\newcommand\refSectionPage[1]{\hyperref[sec:#1]{section \ref*{sec:#1} page \pageref{sec:#1}}}
\newcommand\refSectionTitlePage[1]{\hyperref[sec:#1]{section «~\emph{\nameref*{sec:#1}}~» page \pageref{sec:#1}}}
\newcommand\refSectionTitle[1]{\nameref{sec:#1}}

%------------------------------------------------------------------------------------------ RÉFÉRENCES À UNE SUB-SECTION
% Au lieu d'écrire \subsection{titre-section}, on écrit \subsectionLabel{titre-section}{label-section}
\newcommand\subsectionLabel[2]{\subsection{#1}\label{subsec:#2}}


% \refSubsectionPage{label-section} ---> "section x.y page n"   où x.y est le n° de la sub-section
\newcommand\refSubsection[1]{\hyperref[subsec:#1]{section \ref*{subsec:#1}}}
\newcommand\refSubsectionPage[1]{\hyperref[subsec:#1]{section \ref*{subsec:#1} page \pageref{subsec:#1}}}
\newcommand\refSubsectionTitlePage[1]{\hyperref[subsec:#1]{section «~\emph{\nameref*{subsec:#1}}~» page \pageref{subsec:#1}}}
\newcommand\refSubsectionTitle[1]{\nameref{subsec:#1}}

%------------------------------------------------------------------------------------------ RÉFÉRENCES À UNE SUB-SUB-SECTION
% Au lieu d'écrire \subsubsection{titre-section}, on écrit \subsubsectionLabel{titre-section}{label-section}
\newcommand\subsubsectionLabel[2]{\subsubsection{#1}\label{subsubsec:#2}}


% \refSubsectionPage{label-section} ---> "page n"   où n est la page de la sub-sub-section
\newcommand\refSubsubsection[1]{\hyperref[subsubsec:#1]{section \ref*{subsubsec:#1}}}
\newcommand\refSubsubsectionPage[1]{\hyperref[subsubsec:#1]{section \ref*{subsubsec:#1} page \pageref{subsubsec:#1}}}
\newcommand\refSubsubsectionTitlePage[1]{\hyperref[subsubsec:#1]{section «~\emph{\nameref*{subsubsec:#1}}~» page \pageref{subsubsec:#1}}}
\newcommand\refSubsubsectionTitle[1]{\nameref{subsubsec:#1}}

%------------------------------------------------------------------------------------------ RÉFÉRENCES À UNE SUB-SUB-SUB-SECTION
% Au lieu d'écrire \subsubsubsection{titre-section}, on écrit \subsubsubsectionLabel{titre-section}{label-section}
%\newcommand\subsubsubsectionLabel[2]{\subsubsubsection{#1}\label{subsubsubsec:#2}}


% \refSubsubsubsectionPage{label-section} ---> "page n"   où n est la page de la sub-sub-section
%\newcommand\refSubsubsubsectionPage[1]{\hyperref[subsubsubsec:#1]{page \pageref{subsubsubsec:#1}}}

%------------------------------------------------------------------------------------------ RÉFÉRENCES À UNE ÉQUATION
% Pour une équation, au lieu d'écrire \label{label-equation}, on écrit \labelEquation{label-equation}
\newcommand\labelEquation[1]{\label{equ:#1}}

% \refEquationPage{label-equation} ---> "équation x.y page n" où x.y est le n° de l'équation
\newcommand\refEquationPage[1]{\hyperref[equ:#1]{équation \ref*{equ:#1} page \pageref{equ:#1}}}

% \refEquation{label-equation} ---> "équation x.y" où x.y est le n° de l'équation
\newcommand\refEquation[1]{\hyperref[equ:#1]{équation \ref*{equ:#1}}}

%------------------------------------------------------------------------------------------ RÉFÉRENCES À UN SYSTÈME
% Pour latex, équation et systèmes (d'équation) sont numérotés grâce au même compteur ; on distingue les deux
% pour les référencements soient différents
% Pour un système, au lieu d'écrire \label{label-systeme}, on écrit \labelSystème{label-systeme}
\newcommand\labelSysteme[1]{\label{syst:#1}}

% \refSysteme{label-systeme} ---> "système x.y" où x.y est le n° du système
\newcommand\refSysteme[1]{\hyperref[syst:#1]{système \ref*{syst:#1}}}

% \refSystemeSansPrefixe{label-systeme} ---> "x.y" où x.y est le n° du système
\newcommand\refSystemeSansPrefixe[1]{\hyperref[syst:#1]{\ref*{syst:#1}}}

% \refSystemePage{label-systeme} ---> "système x.y page n" où x.y est le n° du système
\newcommand\refSystemePage[1]{\hyperref[syst:#1]{système \ref*{syst:#1} page \pageref{syst:#1}}}

%------------------------------------------------------------------------------------------ RÉFÉRENCES À UNE NOTE DE BAS DE PAGE
\newcommand\labelNote[1]{\label{footnote:#1}}

\newcommand\refNote[1]{\hyperref[footnote:#1]{$^{\ref*{footnote:#1}}$}}

\newcommand\refNotePage[1]{\hyperref[footnote:#1]{$^{\ref*{footnote:#1}~page~\pageref{footnote:#1}}$}}

%-----------------------------------------------------------------------------------------------------------------------*
%                                                                                                                       *
%   E X T E N S I O N S    P O U R    P R É S E N T E R    L E S    T A B L E A U X                                     *
%                                                                                                                       *
%-----------------------------------------------------------------------------------------------------------------------*

\usepackage{array}
\usepackage[usenames,dvipsnames,svgnames,table]{xcolor}

%\usepackage{arydshln} % Pour faire des séparateurs de ligne en pointillés (\hdashline)
%\renewcommand\dashlinedash{1pt}
%\renewcommand\dashlinegap{8pt}
%
%\renewcommand{\arraystretch}{1.2}

%--- Couleur de fond alternée des tableaux
%\ifthenelse{\equal{\sortieEnCouleur}{true}}{
%  \newcommand\fondTableau{yellow!25}
%}{
%  \newcommand\fondTableau{gray!25}
%}

%--- Ce paquetage permet d'utiliser les environnements wrapfigure et wraptable pour insérer les figures et
% les tableaux dans le texte.
%\usepackage{wrapfig}

%--- Ce paquetage permet de changer le style des légendes des tableaux et des figures (voir caption-eng.pdf) :
%  - l'étiquette est en italique gras
%  - le titre est en italique.
\usepackage[font=it, labelfont=bf]{caption}

%--- Changement des réglages par défaut 

%--- Par défaut, le paquetage nomme "Table" les tableaux. La commande
%   suivante impose le nom "Tableau"
% Voir http://fr.wikibooks.org/wiki/LaTeX/Éléments_flottants_et_figures
\addto\captionsfrench{\def\tablename{Tableau}}

%--- Par défaut, le paquetage écrit "Figure" en petites capitales. La commande
%   suivante l'écrit comme indiqué par les les paramètres du paquatage "caption"
% Voir http://fr.wikibooks.org/wiki/LaTeX/Éléments_flottants_et_figures
\addto\captionsfrench{\def\figurename{Figure}}

%--- De même, le sommaire est appelé "Table des matières"
% La commande suivante impose le nom "Sommaire"
%\addto\captionsfrench{\def\contentsname{Sommaire}}

%--- De même, la liste des figures sommaire est appelé "Table des figures"
% La commande suivante impose le nom "Liste des figures"
%\addto\captionsfrench{\def\listfigurename{Liste des figures}}

%-----------------------------------------------------------------------------------------------------------------------*
%                                                                                                                       *
%   E X T E N S I O N S    P O U R    L ' É C R I T U R E    D E S     F O R M U L E S    M A T H É M A T I Q U E S     *
%                                                                                                                       *
%-----------------------------------------------------------------------------------------------------------------------*

%--- Extensions pour l'écriture des formules mathématiques
\usepackage{amsmath}
\usepackage{amssymb}
\usepackage{amsfonts}

%--- Paquetage "IEEEtrantools"
% Pour créer des tableaux d'équations, bien alignées
% Voir courte-intro-latex.pdf, page §3.5.2 page 83
%\usepackage[retainorgcmds]{IEEEtrantools}

%-----------------------------------------------------------------------------------------------------------------------*
%                                                                                                                       *
%   P A Q U E T A G E    « S U B F I G »                                                                                *
%                                                                                                                       *
%-----------------------------------------------------------------------------------------------------------------------*

% Grâce à ce paquetage, il est possible de placer plusieurs figures, tables, côte à côte.
% Voir http://en.wikibooks.org/wiki/LaTeX/Floats,_Figures_and_Captions
%\usepackage{subfig}

%-----------------------------------------------------------------------------------------------------------------------*
%                                                                                                                       *
%   P A Q U E T A G E    « M U L T I C O L »                                                                            *
%                                                                                                                       *
%-----------------------------------------------------------------------------------------------------------------------*

%\usepackage{multicol}
%\setlength{\columnsep}{30pt}

%-----------------------------------------------------------------------------------------------------------------------*
%                                                                                                                       *
%   P A Q U E T A G E    « E M P T Y P A G E »                                                                          *
%                                                                                                                       *
%-----------------------------------------------------------------------------------------------------------------------*

% Supprime en-têtes et pieds de pages sur les pages vides
% Merci à Éric Le Carpentier !

\usepackage{emptypage}

%-----------------------------------------------------------------------------------------------------------------------*
%                                                                                                                       *
%   T I K Z    -    P G F                                                                                               *
%                                                                                                                       *
%-----------------------------------------------------------------------------------------------------------------------*

\usepackage{tikz}
\usetikzlibrary{calc}
\usepackage{pgfplots}
\usetikzlibrary{arrows}
\usetikzlibrary{decorations}
\usetikzlibrary{decorations.pathmorphing}
\usetikzlibrary{decorations.shapes}
\usetikzlibrary{shapes.callouts}
\usetikzlibrary{shapes.misc}
\usetikzlibrary{automata}
\usetikzlibrary{positioning}
\usepgflibrary{shapes.geometric}
\usepgfmodule{plot}

\usetikzlibrary{%
  arrows,%
  shapes.misc,% wg. rounded rectangle
  shapes.arrows,%
  chains,%
  matrix,%
  positioning,% wg. " of "
  scopes,%
  decorations.pathmorphing,% /pgf/decoration/random steps | erste Graphik
  shadows%
}

%-----------------------------------------------------------------------------------------------------------------------*
%   A F F I C H A G E    D U    C O D E    P L M                                                                        *
%-----------------------------------------------------------------------------------------------------------------------*

\usepackage{mdframed}
\usepackage{lineno}
\renewcommand\linenumberfont{\normalfont\bfseries\footnotesize}

%-----------------------------------------------------------------------------------------------------------------------*
% http://en.wikibooks.org/wiki/LaTeX/Colors

\newcommand\keywordsStyleplm[1]{\textcolor{blue}{\textbf{#1}}}
\newcommand\attributeStyleplm[1]{\textcolor{purple}{#1}}
\newcommand\delimitersStyleplm[1]{\textcolor{brown}{\textbf{#1}}}
\newcommand\integerStyleplm[1]{\textcolor{orange}{#1}}
\newcommand\commentStyleplm[1]{\textcolor{red}{#1}}
\newcommand\modeStyleplm[1]{\textcolor{violet}{#1}}
\newcommand\selectorStyleplm[1]{\textcolor{PineGreen}{#1}}
\newcommand\typeStyleplm[1]{\textcolor{PineGreen}{#1}}

\newcommand\characterStyleplm[1]{\textcolor{orange}{#1}}
\newcommand\stringStyleplm[1]{\textcolor{gray}{#1}}

\newcommand\lexicalErrorplm{\textcolor{red}{\textbullet ERRLEX\textbullet}}

\usepackage{tcolorbox}

\newmdenv[
  topline=false,
  bottomline=false,
  rightline=false,
  linecolor=blue!25,
  linewidth=2pt,
  backgroundcolor=yellow!10
]{siderules}

\newwrite\tempfile

\makeatletter
\newenvironment{PLM}[1][0]{%
  \providecommand{\ParametrePLM}{#1}
  \begingroup
  \@bsphack
  \immediate\openout\tempfile=temp.plm%
  \let\do\@makeother\dospecials
  \catcode`\^^M\active
  \verbatim@startline
  \verbatim@addtoline
  \verbatim@finish
  \def\verbatim@processline{\immediate\write\tempfile{\the\verbatim@line}}%
  \verbatim@start
}{
  \immediate\closeout\tempfile
  \@esphack
  \endgroup
  \immediate\write18{plm --mode=latex:plm temp.plm}
  \setstretch{1.2}
  \ifthenelse{\equal{\ParametrePLM}{0}}{
    {\begin{siderules}\ttfamilyc{}o{}n{}f{}i{}g{}u{}r{}a{}t{}i{}o{}n{}\end{siderules}}
  }{
    {\begin{linenumbers}\resetlinenumber[\ParametrePLM]\ttfamilyc{}o{}n{}f{}i{}g{}u{}r{}a{}t{}i{}o{}n{}\end{linenumbers}}
  }
}
\makeatother

%-----------------------------------------------------------------------------------------------------------------------*
% COMMANDE \plm : affichage de code en ligne PLM                                                                        *
%-----------------------------------------------------------------------------------------------------------------------*


\makeatletter
\newcommand*\plm{%
  \@bsphack%
  \begingroup%
  \let\do\@makeother\dospecials%
  \let\do\do@noligs\verbatim@nolig@list%
  \catcode`\^^M=15\relax%
  \@vobeyspaces%
  \@plm{\temporary}%
}%
\newcommand\@plm[2]{%
  \catcode`-=12\relax%
  \catcode`<=12\relax%
  \catcode`>=12\relax%
  \catcode`,=12\relax%
  \catcode`'=12\relax%
  \catcode``=12\relax%
  \catcode`#2\active%
  \catcode`~\active%
  \lccode`\~`#2\relax%
  \begingroup%
  \lowercase{%
    \def\@tempa##1~{%
      \expandafter\endgroup%
      \expandafter\DeclareRobustCommand%
      \expandafter*%
      \expandafter#1%
      \expandafter{\@tempa}%
      \@esphack%
      \immediate\openout\tempfile=temp.plm%
      \immediate\write\tempfile{##1}%
      \immediate\closeout\tempfile%
      \immediate\write18{plm --mode=latex:plm temp.plm}%
      \colorbox{gray!6}{\ttfamilyc{}o{}n{}f{}i{}g{}u{}r{}a{}t{}i{}o{}n{}\unskip}%
    }%
  }%
  \ifnum`#2=`\~\else\@makeother\~\fi%
  \expandafter\endgroup%
  \@tempa%
}%
\makeatother

%-----------------------------------------------------------------------------------------------------------------------*
%                                                                                                                       *
%   C OM M A N D E S    S H E L L                                                                                       *
%                                                                                                                       *
%-----------------------------------------------------------------------------------------------------------------------*

% http://forum.mathematex.net/latex-f6/supprimer-les-espaces-devant-la-ponctuation-t15951.html
% \nofrench@punctuation supprime l'insertion d'espace avant '!', ':'.
% \addfontfeatures{Ligatures={NoRequired, NoCommon, NoContextual}}
% \addfontfeatures{Mapping=}

\makeatletter
\newenvironment{SHELL}
{\begin{mdframed}[backgroundcolor=lightgray!50,linewidth=0]\tt\nofrench@punctuation}
{\end{mdframed}}
\makeatother


%-----------------------------------------------------------------------------------------------------------------------*
%                                                                                                                       *
%   F O O T N O T E    P A C K A G E                                                                                    *
%                                                                                                                       *
%-----------------------------------------------------------------------------------------------------------------------*

% Ce paquatage permet d'utiliser des \footnote dans un tableau
% http://texblog.org/2012/02/03/using-footnote-in-a-table/
\usepackage{footnote}

%-----------------------------------------------------------------------------------------------------------------------*
%                                                                                                                       *
%   E N - T Ê T E S    E T    P I E D S    D E    P A G E S                                                             *
%                                                                                                                       *
%-----------------------------------------------------------------------------------------------------------------------*

% Grâce au package "fancyhdr"
% voir http://www.exomatik.net/U-Latex/Personnaliser#toc2
%      http://www.trustonme.net/didactels/250.html
\usepackage{fancyhdr}
\pagestyle{fancy}
\fancyhead{} % clear all header fields
\fancyfoot{} % clear all footer fields
%--- Numéro de page : à gauche pages paires, à droite pages impaires
\fancyhead[EL,OR]{\thepage}
%--- Nom de chapitre : à droite page paires
\fancyhead[ER]{\leftmark}
%--- Nom de section : à gauche page impaires
\fancyhead[OL]{\rightmark}
%--- filet en haut de page
\renewcommand{\headrulewidth}{0.5 pt}
\renewcommand{\footrulewidth}{0 pt}

%-----------------------------------------------------------------------------------------------------------------------*
%                                                                                                                       *
%   T O C B I D I N D                                                                                                   *
%                                                                                                                       *
%-----------------------------------------------------------------------------------------------------------------------*

%-----------------------------------------------------------------------------------------------------------------------*
%                                                                                                                       *
%   C O N T R Ô L E    D E   L A   T A B L E   D E S   M A T I È R E S                                                  *
%                                                                                                                       *
%-----------------------------------------------------------------------------------------------------------------------*

% http://tex.stackexchange.com/questions/50471/question-about-indent-lengths-in-toc
\usepackage{tocloft}

%--- Gérer de l'indentation dans la table des matières
\cftsetindents{chapter}{0.0em}{1.5em}
\cftsetindents{section}{1.0em}{2.5em}
\cftsetindents{subsection}{2.0em}{3.5em}
\cftsetindents{subsubsection}{3.05em}{4.0em}
\cftsetindents{table}{0.0em}{2.5em}
\cftsetindents{figure}{0.0em}{2.5em}

%    Pour faire figurer la liste des tableaux (et la table des matières) dans la table des matières
\usepackage{tocbibind}

%--- Profondeur de la table des matières jusqu'au niveau 3 (subsubsection)
\setcounter{tocdepth}{3}

%--- Numéroter les entrées jusqu'au niveau 3 (subsubsection)
\setcounter{secnumdepth}{3}

%-----------------------------------------------------------------------------------------------------------------------*
%                                                                                                                       *
%   D P R O G R E S S                                                                                                   *
%                                                                                                                       *
%-----------------------------------------------------------------------------------------------------------------------*

%   Ce paquetage permet d'afficher dans le "log" les titres de section et de sous-section, facilitant ainsi la
%   localisation des warnings
\usepackage{dprogress}

%-----------------------------------------------------------------------------------------------------------------------*
%                                                                                                                       *
%   P A Q U E T A G E    « M I N I T O C »                                                                              *
%                                                                                                                       *
%-----------------------------------------------------------------------------------------------------------------------*

%\usepackage[french]{minitoc}
%\setcounter{minitocdepth}{2}
%\setlength{\mtcindent}{24pt}
%\renewcommand{\mtcfont}{\small\rm}
%\renewcommand{\mtcSfont}{\small\bf}


%-----------------------------------------------------------------------------------------------------------------------*
%                                                                                                                       *
%   Titre des tableaux et des figures                                                                                   *
%                                                                                                                       *
%-----------------------------------------------------------------------------------------------------------------------*

% Par défaut, le séparateur est « : », sans espace le précédent ; on obtient ainsi: « Figure 1.1: titre »
% La définition ci-dessous fixe le séparateur à un trait d'union long : « Figure 1.1 — titre »
% Ceci s'applique aux tableaux et aux figures

\DeclareCaptionLabelSeparator{bar}{ -- }
\captionsetup{
  labelsep=bar
}

%-----------------------------------------------------------------------------------------------------------------------*
%                                                                                                                       *
%   D É B U T    D U    D O C U M E N T                                                                                 *
%                                                                                                                       *
%-----------------------------------------------------------------------------------------------------------------------*

\begin{document} 

%-----------------------------------------------------------------------------------------------------------------------*
%                                                                                                                       *
%   P A G E    D E    T I T R E                                                                                         *
%                                                                                                                       *
%-----------------------------------------------------------------------------------------------------------------------*

\newcommand\titreOuvrage{\Huge\textbf{PLM}}

\ifthenelse{\equal{\sortieEnCouleur}{true}}{
  \title{\textcolor{blue}{\titreOuvrage}}
}{
  \title{\titreOuvrage}
}

\author{Pierre Molinaro}

\date{\today} 

\maketitle

%\dominitoc
%\dominilof
%\dominilot

%-----------------------------------------------------------------------------------------------------------------------*
%                                                                                                                       *
%   A V A N T    P R O P O S                                                                                            *
%                                                                                                                       *
%-----------------------------------------------------------------------------------------------------------------------*

%\input{chapitres/avant-propos.tex}

%-----------------------------------------------------------------------------------------------------------------------*
%                                                                                                                       *
%   T A B L E    D E S    M A T I È R E S                                                                               *
%                                                                                                                       *
%-----------------------------------------------------------------------------------------------------------------------*

%\clearpage
\tableofcontents
%--- La commande suivante supprime tout en-tête et tout pied de page sur la première page du sommaire
%    http://www.developpez.net/forums/d604749/autres-langages/autres-langages/latex/probleme-numerotation-bas-page-table-matieres/
\addtocontents{toc}{\protect\thispagestyle{empty}\protect\pagestyle{fancy}}

%-----------------------------------------------------------------------------------------------------------------------*
%                                                                                                                       *
%   L I S T E    D E S    T A B L E A U X                                                                               *
%                                                                                                                       *
%-----------------------------------------------------------------------------------------------------------------------*

\clearpage
\listoftables
\addtocontents{lot}{\protect\thispagestyle{empty}\protect\pagestyle{fancy}}

%-----------------------------------------------------------------------------------------------------------------------*
%                                                                                                                       *
%   L I S T E    D E S    F I G U R E S                                                                                 *
%                                                                                                                       *
%-----------------------------------------------------------------------------------------------------------------------*

\clearpage
\listoffigures
\addtocontents{lof}{\protect\thispagestyle{empty}\protect\pagestyle{fancy}}

%-----------------------------------------------------------------------------------------------------------------------*
%                                                                                                                       *
%   L E S    C H A P I T R E S                                                                                          *
%                                                                                                                       *
%-----------------------------------------------------------------------------------------------------------------------*

%--- Contrôle de la séparation des paragraphes
%    On met cette définition ici, sinon elle affecte la table des matières, la liste des tableaux, ...
\setlength{\parskip}{1.2ex}


%!TEX encoding = UTF-8 Unicode
%!TEX root = ../doc-plm.tex


\chapter{Tutorial}

%--- Pour supprimer tout en-tête et pied de page sur la 1re page d'un chapitre
\thispagestyle{empty}



\section{Cible}





%!TEX encoding = UTF-8 Unicode
%!TEX root = ../doc-plm.tex





\chapter{Cible \texttt{teensy-3-1-it}}\index{Cible!\texttt{teensy-3-1-it}}

Dans l'état actuel de PLM, une seule cible est définie : \texttt{teensy-3-1-it}.  Elle permet une programmation séquentielle avec routines d'interruption. L'interruption \texttt{systick} est programmée pour se déclencher chaque milliseconde. L'objet de ce chapitre est de décrire son utilisation.

Il est possible de définir sa propre cible (\refChapterPage{chapitreConfCible}).




















\sectionLabel{Organigramme d'exécution}{organigrammeExecutionTeensy31It}

La \refFigure{}{sequenceDemarrageTeensySequentialSystick} définit l'organigramme d'exécution d'un programme.

Le micro-contrôleur démarre sur une horloge interne, la mémoire vive n'étant pas initialisée. Il est dans le mode \emph{thread, priviliged access}, avec une seule pile. La configuration conservera cette pile unique, qui servira donc pour les routines de fond et les routines d'interruption.

La première étape est de configurer les horloges internes du micro-contrôleur : c'est le rôle des routines \plm=boot= (\refSectionPage{personalisationDemarrageTeensy31it}). À ce stade, la mémoire vive n'est toujours pas initialisée, aussi les routines \plm=boot= n'y accèdent pas (le compilateur l'assure).

La deuxième étape est d'initialiser les \emph{variables globales}, c'est-à-dire mettre à zéro la zone \texttt{bss}, et de recopier à partir de la flash les valeurs initiales des variables initialisées.

La troisième étape est l'exécution des routines \plm=init= (\refSectionPage{personalisationInitTeensy31it}). À partir de cette étape et pour les suivantes, les variables globales sont initialisées, et donc leur emploi est autorisé. Le rôle des routines \plm=init= est de configurer les entrées/sorties du micro-contrôleur.

Ensuite, le micro-contrôleur est passé en mode \emph{thread, unpriviliged access}, ce qui correspond au mode \plm+`user+ de PLM pour cette cible.

La routine \plm+setup+ est exécutée une fois, puis \plm+loop+ est exécutée indéfiniment.


\begin{figure}[t]
  \centering
  \small
  \begin{tikzpicture}[
      cloud/.style ={draw=red, thick, ellipse,fill=red!20, minimum height=2em},
      block/.style ={rectangle, draw=blue, thick, fill=green!20, align=center},
      decision/.style={chamfered rectangle, draw=blue, thick, fill=green!20},
      node distance=5mm
    ]
    \node [cloud] (start) {\textsc{Démarrage}} ;
    \node [block] (confDepart) [below=of start] {Mode \emph{thread}, accès priviliégié, une seule pile} ;
    \node [block] (boot) [below=of confDepart] {Routines \bf\texttt{boot}} ;
    \node [block] (raz) [below=of boot] {Initialisation des variables globales} ;
    \node [block] (init) [below=of raz] {Routines \bf\texttt{init}} ;
    \node [block] (user) [below=of init] {Passage en mode \emph{thread}, accès non priviliégié} ;
    \node [block] (setup) [below=of user] {Procédure \texttt{setup}} ;
    \node [block] (loop) [below=of setup] {Procédure \texttt{loop}} ;

    \draw [-stealth, thick] (start) -- (confDepart) ;
    \draw [-stealth, thick] (confDepart) -- (boot) ;
    \draw [-stealth, thick] (boot) -- (raz) ;
    \draw [-stealth, thick] (raz) -- (init) ;
    \draw [-stealth, thick] (init) -- (user) ;
    \draw [-stealth, thick] (user) -- (setup) ;
    \draw [-stealth, thick] (setup) -- (loop) ;
    \draw [-stealth, thick] (loop.south) -- +(0, -.25) -- +(1.7, -.25) -- +(1.7, 0.85)-- +(0, 0.85) ;
  \end{tikzpicture}
  \caption{Organigramme d'exécution de la cible \texttt{teensy-3-1-it}}
  \labelFigure{sequenceDemarrageTeensySequentialSystick}
  \ligne
\end{figure}










\sectionLabel{Personalisation du démarrage}{personalisationDemarrageTeensy31it}

La cible définit la routine \plm+boot 0+ qui configure le micro-contrôleur.

Vous pouvez ajouter vos propres routines \plm+boot+. À chaque routine \plm+boot+ est associée une priorité d'exécution, qui doit être unique. Les routines \plm+boot+ sont exécutées dans l'ordre croissant des priorités, c'est-à-dire que la routine \plm+boot 0+ est exécutée la première.





\sectionLabel{Personalisation de l'initialisation}{personalisationInitTeensy31it}

La cible définit la routine \plm+init 0+ qui configure le \emph{SysTick Timer} pour qu'il engendre une interruption toutes les millisecondes.

Vous pouvez ajouter vos propres routines \plm+init+. À chaque routine \plm+init+ est associée une priorité d'exécution, qui doit être unique. Les routines \plm+init+ sont exécutées dans l'ordre croissant des priorités, c'est-à-dire que la routine \plm+init 0+ est exécutée la première.












\section{API}

L'API de la cible \texttt{teensy-3-1-it} partagent les routines en trois groupes :
\begin{itemize}
  \item les routines appelables dans tous les modes ;
  \item les routines appelables uniquement dans le mode \plm+`user+ (\refSubsectionPage{RoutinesTousModeTeensy31}) ;
  \item les routines appelables uniquement dans le mode \plm+`exception+ ;
\end{itemize}

\subsectionLabel{Routines appelables dans tous les modes}{RoutinesTousModeTeensy31}

Ces routines sont appelables dans les modes \plm+`user+, \plm+`isr+ et \plm+`exception+ :
\begin{itemize}
\item \texttt{ledOff} : \refSubsubsectionTitlePage{routineLedOffTeensy31it} ;
\item \texttt{ledOn} : \refSubsubsectionTitlePage{routineLedOnTeensy31it}.
\end{itemize}


\subsubsectionLabel{Routine \texttt{ledOff}}{routineLedOffTeensy31it}

\begin{PLM}
proc ledOff `user `exception `isr (?inLeds $uint32)
\end{PLM}

\subsubsectionLabel{Routine \texttt{ledOn}}{routineLedOnTeensy31it}

\begin{PLM}
proc ledOn `user `exception `isr (?inLeds $uint32)
\end{PLM}


\section{Les routines d'interruption}

Les \refTableau{tableItTeensySequentialSystick1}, \refTableau{tableItTeensySequentialSystick2} et \refTableau{tableItTeensySequentialSystick3} listent les interruptions définies par le processeur qui équipe la carte \emph{Teensy 3.1}. L'utilisateur peut définir une routine d'interruption pour chacune d'entre elles, à l'exception de l'interruption n°1 (remise à zéro), et de la n°15 (\texttt{SysTick}), qui est prise en charge de façon particulière (voir la \refSubsectionTitlePage{SystickPourTeensy31It}). Celle des autres interruptions est décrite dans les sections suivantes :
\begin{itemize}
  \item \refSubsectionTitlePage{itsTeensy31AvecExceptions} ;
  \item \refSubsectionTitlePage{itsTeensy31SansExceptions} ;
  \item \refSubsectionTitlePage{itsTeensyRoutinesUtilisateur}.
\end{itemize}


\begin{table}[!t]
  \centering
  \begin{tabular}{llllll}
    \textbf{Numéro}& \textbf{Nom routine} \\
    1  & \emph{ResetHandler, réservé par PLM} \\
    2  & \texttt{NMIHandler}\\
    3  & \texttt{HardFaultHandler}\\
    4  & \texttt{MemManageHandler}\\
    5  & \texttt{BusFaultHandler}\\
    6  & \texttt{UsageFaultHandler}\\
    7 à 10 & \emph{réservées par ARM} \\
    11 & \texttt{svcHandler}\\
    12 & \texttt{DebugMonitorHandler}\\
    13 & \emph{réservée par ARM} \\
    14 & \texttt{PendSVHandler}\\
    15 & \texttt{userSystickHandler}, voir \refSubsectionPage{SystickPourTeensy31It} \\
  \end{tabular}
  \caption{Table des interruptions 1 à 15 de la cible \texttt{teensy-3-1-it}}
  \labelTableau{tableItTeensySequentialSystick1}
  \ligne
\end{table}

\begin{table}[!t]
  \centering
  \begin{tabular}{llllll}
    \textbf{Numéro} & \textbf{Nom routine} \\
    16  & \texttt{DMAChannel0TranfertCompleteHandler}\\
    17  & \texttt{DMAChannel1TranfertCompleteHandler}\\
    18  & \texttt{DMAChannel2TranfertCompleteHandler}\\
    19  & \texttt{DMAChannel3TranfertCompleteHandler}\\
    20  & \texttt{DMAChannel4TranfertCompleteHandler}\\
    21  & \texttt{DMAChannel5TranfertCompleteHandler}\\
    22  & \texttt{DMAChannel6TranfertCompleteHandler}\\
    23  & \texttt{DMAChannel7TranfertCompleteHandler}\\
    24  & \texttt{DMAChannel8TranfertCompleteHandler}\\
    25  & \texttt{DMAChannel9TranfertCompleteHandler}\\
    26  & \texttt{DMAChannel10TranfertCompleteHandler}\\
    27  & \texttt{DMAChannel11TranfertCompleteHandler}\\
    28  & \texttt{DMAChannel12TranfertCompleteHandler}\\
    29  & \texttt{DMAChannel13TranfertCompleteHandler}\\
    30  & \texttt{DMAChannel14TranfertCompleteHandler}\\
    31  & \texttt{DMAChannel15TranfertCompleteHandler}\\
    32  & \texttt{DMAErrorHandler}\\
    33  & \emph{inutilisée} \\
    34  & \texttt{flashMemoryCommandCompleteHandler}\\
    35  & \texttt{flashMemoryReadCollisionHandler}\\
    36  & \texttt{modeControllerHandler}\\
    37  & \texttt{LLWUHandler}\\
    38  & \texttt{WDOGEWMHandler}\\
    39  & \emph{inutilisée} \\
    40  & \texttt{I2C0Handler}\\
    41  & \texttt{I2C1Handler}\\
    42  & \texttt{SPI0Handler}\\
    43  & \texttt{SPI1Handler}\\
    44  & \emph{inutilisée} \\
    45  & \texttt{CAN0MessageBufferHandler}\\
    46  & \texttt{CAN0BusOffHandler}\\
    47  & \texttt{CAN0ErrorHandler}\\
    48  & \texttt{CAN0TransmitWarningHandler}\\
    49  & \texttt{CAN0ReceiveWarningHandler}\\
    50  & \texttt{CAN0WakeUpHandler}\\
    51  & \texttt{I2S0TransmitHandler}\\
    52  & \texttt{I2S0ReceiveHandler}\\
    53 à 59  & \emph{inutilisées} \\
  \end{tabular}
  \caption{Table des interruptions 16 à 59 de la cible \texttt{teensy-3-1-it}}
  \labelTableau{tableItTeensySequentialSystick2}
  \ligne
\end{table}

\begin{table}[!t]
  \centering
  \begin{tabular}{llllll}
    \textbf{Numéro}& \textbf{Nom routine} \\
    60  & \texttt{UART0LONHandler}\\
    61  & \texttt{UART0StatusHandler}\\
    62  & \texttt{UART0ErrorHandler}\\
    63  & \texttt{UART1StatusHandler}\\
    64  & \texttt{UART1ErrorHandler}\\
    65  & \texttt{UART2StatusHandler}\\
    66  & \texttt{UART2ErrorHandler}\\
    67 à 72  & \emph{inutilisées} \\
    73  & \texttt{ADC0Handler}\\
    74  & \texttt{ADC1Handler}\\
    75  & \texttt{CMP0Handler}\\
    76  & \texttt{CMP1Handler}\\
    77  & \texttt{CMP2Handler}\\
    78  & \texttt{FMT0Handler}\\
    79  & \texttt{FMT1Handler}\\
    80  & \texttt{FMT2Handler}\\
    81  & \texttt{CMTHandler}\\
    82  & \texttt{RTCAlarmHandler}\\
    83  & \texttt{RTCSecondHandler}\\
    84  & \texttt{PITChannel0Handler}\\
    85  & \texttt{PITChannel1Handler}\\
    86  & \texttt{PITChannel2Handler}\\
    87  & \texttt{PITChannel3Handler}\\
    88  & \texttt{PDBHandler}\\
    89  & \texttt{USBOTGHandler}\\
    90  & \texttt{USBChargerDetectHandler}\\
    91 à 96  & \emph{inutilisées} \\
    97  & \texttt{DAC0Handler}\\
    98  & \emph{inutilisée} \\
    99  & \texttt{TSIHandler}\\
    100  & \texttt{MCGHandler}\\
    101  & \texttt{lowPowerTimerHandler}\\
    103  & \texttt{pinDetectPortAHandler}\\
    104  & \texttt{pinDetectPortBHandler}\\
    105  & \texttt{pinDetectPortCHandler}\\
    106  & \texttt{pinDetectPortDHandler}\\
    107  & \texttt{pinDetectPortEHandler}\\
    108 et 109  & \emph{inutilisées} \\
    110  & \texttt{softwareInterruptHandler}
  \end{tabular}
  \caption{Table des interruptions 60 à 110 de la cible \texttt{teensy-3-1-it}}
  \labelTableau{tableItTeensySequentialSystick3}
  \ligne
\end{table}


\subsectionLabel{Routines d'interruption par défaut, exceptions activées}{itsTeensy31AvecExceptions}

Quand les exceptions sont activées, une routine par défaut est prédéfinie pour chaque interruption (sauf la n°1 et la n°15). Celle-ci exécute l'instruction \plm+panic+, dont l'argument est le numéro de l'interruption. Par exemple, pour l'interruption n°2 (\texttt{NMI}), la routine prédéfinie est :
\begin{PLM}[1]
proc NMIHandler `isr @nullWhenPanicDisabled @weak () {
  panic 2
}
\end{PLM}


\subsectionLabel{Routines d'interruption par défaut, exceptions inactivées}{itsTeensy31SansExceptions}

Quand les exceptions sont inactivées, l'attribut \plm+@nullWhenPanicDisabled+ associé à chaque routine d'interruption provoque la suppression de cette routine d'interruption, et son remplacement de son adresse dans la table des vecteurs d'interruption par la valeur $0$.



\subsectionLabel{Routines d'interruption définies par l'utilisateur}{itsTeensyRoutinesUtilisateur}

L'attribut \plm+@weak+ associé à chaque routine d'interruption permet sa redéfinition par l'utilisateur. La routine par défaut est alors ignorée, que les exceptions soient activées ou non.

Par exemple, l'utilisateur peut définir la routine associée à \texttt{NMI} par :
\begin{PLM}[1]
proc NMIHandler `isr {
  ...
}
\end{PLM}

Il est important de conserver le nom. Si celui-ci est mal orthographié et ne correspond à aucune interruption :
\begin{itemize}
  \item la routine d'interruption par défaut est conservée (execeptions activées) ou supprimée (exception inactivées) ;
  \item un \emph{warning} est déclenché, signalant que la routine utilisateur est inutilisée.
\end{itemize}



\subsectionLabel{Routine associée à l'interruption \texttt{SysTick}}{SystickPourTeensy31It}

L'interruption \texttt{SysTick} est particulière. Elle est programmé par la routine \plm+init 0+ de façon à engendrer une interruption chaque milliseconde. Cette interruption se déclenche après la fin de l'exécution la routine \plm+init 0+, y compris éventuellement dans les routines \plm+init+ qui suivent. La routine d'interruption associée, inaccessible à l'utilisateur :
\begin{itemize}
  \item incrémente une variable globale de comptage du temps ;
  \item appelle la routine \plm+userSystickHandler+.
\end{itemize}

La routine \plm+userSystickHandler+ est définie par :
\begin{PLM}[1]
proc userSystickHandler `isr @weak () {
}
\end{PLM}

L'attribut \plm+@weak+ permet sa redéfinition par l'utilisateur.




%!TEX encoding = UTF-8 Unicode
%!TEX root = ../doc-plm.tex


\newcommand\OPTION[1]{\colorbox{gray!15}{\ttfamily\bfseries #1}}


\chapter{Options de la ligne de commande}

Les options de la ligne de commande commencent toutes par un caractère « \texttt{-} ». La forme courte débute par un simple « \texttt{-} », la forme longue par un double « \texttt{-} ». Certaines options acceptent au choix les deux formes (par exemple \texttt{-{}-verbose} ou \texttt{-v}). Une option ne commençant par un « \texttt{-} » est considérée comme un nom de fichier source. Les deux seules extensions autorisées pour les fichiers sources sont « \texttt{.plm} » (fichier source PLM) et  « \texttt{.plm-target} » (fichier de description d'une cible). Par exemple, pour compiler un fichier source \texttt{source.plm} avec l'option \emph{verbose} :

\begin{SHELL}
\bfseries plm -v source.plm
\end{SHELL}

Les options peuvent apparaître avant ou après le nom du fichier source, elles sont toujours examinées avant que la compilation ne commence.


\sectionLabel{Options générales}{optionsGenerales}

\OPTION{-{}-help} Affiche l'aide.


\OPTION{-{}-version} Affiche la version du compilateur PLM.



\OPTION{-{}-log-file-read} Affiche l'accès en lecture à tout fichier.

\OPTION{-{}-max-errors=$n$} Arrête la compilation si le nombre de $n$ erreurs est atteint. Par défaut, la borne est égale à $100$.

\OPTION{-{}-max-warnings=$n$} Arrête la compilation si le nombre de $n$ alertes est atteint. Par défaut, la borne est égale à $100$.

\OPTION{-{}-Werror} Transforme toute alerte en erreur.

\OPTION{-{}-verbose}, \OPTION{-v} Affiche des messages indiquant la progression de la compilation.

\OPTION{-{}-Werror} Considère tout \emph{warning} comme une erreur.

\OPTION{-{}-no-color} Les textes affichés sur le terminal sont sans l'enrichissement en couleur.








\sectionLabel{Options affectant le code engendré}{optionCodeEngendre}

\OPTION{-{}-no-panic-generation} Inhibe la génération du code de la panique organisée.


\OPTION{-{}-no-file-generation} Inhibe l'écriture de fichiers par le compilateur.

\OPTION{-{}-O1} Premier niveau d'optimisation (correspond à l'option \texttt{-O1} de LLVM).

\OPTION{-{}-O2} Premier niveau d'optimisation (correspond à l'option \texttt{-O2} de LLVM).

\OPTION{-{}-O3} Premier niveau d'optimisation (correspond à l'option \texttt{-O3} de LLVM).

\OPTION{-{}-Os} Comme l'option \texttt{-{}-O2}, mais avec des optimisations supplémentaires pour réduire la taille (correspond à l'option \texttt{-Os} de LLVM).

\OPTION{-{}-Oz} Comme l'option \texttt{-{}-Os}, mais avec encore d'autres optimisations pour réduire la taille (correspond à l'option \texttt{-Oz} de LLVM).



\sectionLabel{Options de débogage}{optionsDebogage}

\OPTION{-{}-output-concrete-syntax-tree} Engendre un fichier au format \texttt{dot} contenant l'arbre syntaxique concret du texte source.

\OPTION{-{}-routine-invocation-graph}, \OPTION{-i} Engendre un fichier au format \texttt{dot} pouvant être ouvert par \texttt{graphviz}\index{graphviz} contenant le graphe d'invocation des routines.

\OPTION{-{}-do-not-detect-recursive-calls}, \OPTION{-r} N'effectue pas la détection des routines récursives. Par défaut le compilateur affiche un message d'erreur si il trouve des routines récursives.\index{Routines!recursives@récursives}

\OPTION{-{}-display-deadcode-elimination}, \OPTION{-z} Affiche sur le terminal les détails d'élimination du code mort.

\OPTION{-{}-control-register-map} Écrit dans un fichier HTML la tables des registres de contrôle.

\OPTION{-{}-global-constant-dependency-graph}, \OPTION{-c} Écrit dans un fichier au format \texttt{dot}  pouvant être ouvert par \texttt{graphviz}\index{graphviz} le graphe de dépendance des constantes statiques.

\OPTION{-{}-routine-invocation-graph}, \OPTION{-i} Écrit dans un fichier au format \texttt{dot}  pouvant être ouvert par \texttt{graphviz}\index{graphviz} le graphe d'invocation des routines.

\OPTION{-{}-type-dependency-graph}, \OPTION{-t} Écrit dans un fichier au format \texttt{dot}  pouvant être ouvert par \texttt{graphviz}\index{graphviz} le graphe de dépendance des types.

\OPTION{-{}-mode=$string$}, où $string$ est l'une des chaînes suivantes : \texttt{lexical-only},\texttt{syntax-only},\texttt{latex}.

\OPTION{-{}-output-keyword-list-file=$string$}, où $string$ est une chaîne décrivant le formatage de la sortie.









\sectionLabel{Option de flashage du code engendré}{optionFlashageCodeEngendre}


\OPTION{-{}-flash-target}, \OPTION{-f} Après une compilation sans erreur, effectue le flashage du micro-contrôleur cible.



\section{Options de débogage du compilateur}


\OPTION{-{}-output-concrete-syntax-tree} Engendre un fichier au format \texttt{dot} pouvant être ouvert par \texttt{graphviz}\index{graphviz} contenant l'arbre syntaxique concret du texte source.


\OPTION{-{}-mode=$nom$}, où \emph{nom} peut prendre pour valeur :
\begin{itemize}
  \item « \emph{vide} » : fonctionnement nominal, le compilateur effectue toutes les phases : analyse lexicale, analyse syntaxique, analyse sémantique, et génération de code ;
  \item \texttt{lexical-only} : le compilateur s'arrête après l'analyse lexicale, et affiche la séquence des symboles terminaux obtenue ;
  \item \texttt{syntax-only} : le compilateur s'arrête après l'analyse syntaxique, et affiche l'arbre de dérivation.
  \item \texttt{latex} : le compilateur s'arrête après l'analyse lexicale, et engendre un fichier latex contenant le texte source.
\end{itemize}

Écrire l'option \texttt{-{}-mode=} est équivalent à l'absence de cette option.






\sectionLabel{Options d'accès aux fichiers d'exemple embarqués}{optionsExemplesEmbarques}

\OPTION{-{}-list-embedded-samples}, \OPTION{-l} Affiche la liste des fichiers d'exemple embarqués dans le compilateur.

\OPTION{-{}-extract-embedded-sample-code=$nom$}, \OPTION{-x=$nom$} Extrait le fichier d'exemple $nom$ et l'écrit dans le répertoire courant.

L'utilisation de ces deux options est illustrée à la \refSectionPage{exempleBlinkled}.






\sectionLabel{Options d'accès aux cibles embarquées}{optionsCiblesEmbarquees}


\OPTION{-{}-use-target-dir=$repertoire$}, \OPTION{-T=$repertoire$} N'utilise pas les cibles embarquées, au profit des cibles définies dans le répertoire $repertoire$.

\OPTION{-{}-list-targets}, \OPTION{-L} Affiche la liste des cibles disponibles, soit celles embarquées dans le compilateur, soit, si l'option précédente est présente, celles définies dans le répertoire $repertoire$.

\OPTION{-{}-extract-embedded-targets=$repertoire$}, \OPTION{-X=$repertoire$} Extrait les fichiers de définition des cibles embarquées dans le compilateur et les écrit dans le répertoire $repertoire$.

L'utilisation de ces options est illustrée à la \refSectionPage{exempleDefinitionCible}.

%!TEX encoding = UTF-8 Unicode
%!TEX root = ../doc-plm.tex



\chapter{Élements lexicaux}

%--- Pour supprimer tout en-tête et pied de page sur la 1re page d'un chapitre
\thispagestyle{empty}


\section{Commentaires}

Un commentaire commence par deux barres obliques consécutives \plm+//+, et s’étend jusqu’à la fin de la ligne courante.

\section{Délimiteurs}

PLM définit les délimiteurs listés dans le \refTableau{delimiteursLangage}.

\begin{table}[!ht]
  \centering
  \begin{tabular}{ccccccccccccccccc}
    \hline
    \plm!:!  & \plm!.! & \plm!,!  & \plm+!=+ & \plm!<=! & \plm!>=! & \plm!;! & \plm!==! & \plm!<! & \plm!>! & \plm![! & \plm!]! \\
    \plm!&! & \plm!&=! & \plm!|! & \plm!|=!  & \plm!}! & \plm!}! & \plm!(!  & \plm!)!  & \plm!/!  & \plm!&/! \\
    \plm!-! & \plm!&-! & \plm!+!  & \plm!&+! & \plm!^!  & \plm!^=! & \plm!<<! & \plm!>>! & \plm!~!\\
    \plm!%! & \plm!&%! & \plm!=! & \plm!\\! & \plm!&\\!\\
    \plm!->! & \plm!::! & \plm!++! & \plm!&++! & \plm!--! & \plm!&--! & \plm!*! & \plm!&*!
  \end{tabular}
  \caption{Délimiteurs du langage PLM}
  \labelTableau{delimiteursLangage}
  \ligne
\end{table}

\section{Séparateurs}

Tout caractère dont le code ASCII est compris entre \texttt{0x01} et \texttt{0x20} est considéré comme un séparateur (ceci inclut donc la tabulation horizontale \texttt{HT} (\texttt{0x09}), le passage à la ligne \texttt{LF} (\texttt{0x0A}), le retour-chariot \texttt{CR} (\texttt{0x0D}), et l’espace (\texttt{0x20}).

\section{Identificateurs}
Un identificateur commence par une lettre (minuscule ou majuscule), qui est suivie par zéro, un ou plusieurs lettres (minuscules ou majuscules), chiffres décimaux, caractères \texttt{\_}.

La casse est significative pour les identificateurs.

\section{Mots réservés}
\index{Mots réservés}

Certains identificateurs sont réservés (\refSubsectionPage{motsReservesLangage}) .

\subsectionLabel{Mots réservés correspondants aux éléments du langage}{motsReservesLangage}

Les mots réservés correspondant aux éléments du langage sont listés dans le \refTableau{motReservesLangage}.

\begin{table}[!t]
  \centering
  \begin{tabular}{llllll}
   \plm!and! & \plm!assert! & \plm!at! & \plm!boolset! \\
   \plm!booleanType! & \plm!boot! & \plm!case! & \plm!check! \\
   \plm!do! & \plm!else! & \plm!elsif! & \plm!end! \\
   \plm!enum! & \plm!exception! & \plm!false! & \plm!forever! \\
   \plm!func! & \plm!if! & \plm!import! & \plm!init!  \\
   \plm!let! & \plm!mutating! & \plm!mode! & \plm!newIntegerType! \\
   \plm!newSignedRepresentation! & \plm!newUnsignedRepresentation! & \plm!not! & \plm!or!\\
   \plm!proc! & \plm!register! & \plm!required! & \plm!self! \\
   \plm!struct! & \plm!target! & \plm!then! & \plm!throw! \\
   \plm!true! & \plm!var! & \plm!while! & \plm!xor! \\
  \end{tabular}
  \caption{Mots réservés correspondant aux éléments du langage PLM}
  \labelTableau{motReservesLangage}
  \ligne
\end{table}







%\section{Constante chaîne de caractères}
%
%Comme en C, les chaînes de caractères sont délimitées par des caractères \texttt{"}. Les séquences d’échappement suivantes sont acceptées : \texttt{\textbackslash f}, \texttt{\textbackslash n}, \texttt{\textbackslash r}, \texttt{\textbackslash v}, \texttt{\textbackslash\textbackslash}, \texttt{\textbackslash\textquotedbl}, \texttt{\textbackslash\textquotesingle}, \texttt{\textbackslash0}.
%
%\section{Constante caractère}
%
%Comme en C, les caractères sont délimités par des caractères « \texttt{\textquotesingle} ». Par exemple :\plm!'A'!, \plm!'+'!.


%Les séquences d’échappement suivantes sont acceptées : \plm!'\f'!, \plm!'\n', !\plm!'\r'!, \plm!'\v'!, \plm!'\\'!, \plm!'\''!, \plm!'\0'!.

\section{Constante entière}

Vous pouvez écrire les constantes entières en décimal, en hexadécimal ou en binaire. 

\textbf{Décimal.} Une constante entière décimale commence par un chiffre décimal, et est suivie par zéro, un ou plusieurs chiffres décimaux, ou caractères \texttt{\_}.

\textbf{Hexadécimal.} Un chiffre hexadécimal est soit un chiffre décimal, soit une lettre entre \texttt{a} et \texttt{f}, écrite indifféremment en minuscule ou en majuscule. Une constante hexadécimale commence par la séquence \texttt{0x} », et est suivie par un ou plusieurs chiffres hexadécimal, ou caractères \texttt{\_}.

\textbf{Binaire.} Une constante entière binaire commence par la séquence \texttt{0b} suivie par un ou plusieurs chiffres binaires, \texttt{0} ou \texttt{1}, ou le caractère \texttt{\_}.

Dans une constante entière, écrite en binaire, décimal ou en hexadécimal, le caractère \texttt{\_} peut servir de séparateur ; on peut ainsi écrire indifféremment : \plm!123!, \plm!1_23!, \plm!1_2_3!, \plm!1___23!, \dots

Attention :
\begin{itemize}
  \item contrairement au C, un nombre qui commence par un zéro est un nombre écrit en décimal ;
  \item contrairement à l’assembleur PIC, le préfixe \texttt{0x} est indispensable pour écrire un nombre en hexadécimal.
\end{itemize}

\sectionLabel{Attributs}{attribut}


Un attribut est une séquence commençant par un \texttt{@} et suivi d'un ou plusieurs chiffres, lettres. Voici quelques étiquettes valides : \plm!@toto!, \plm!@truc1!, \plm!@123!. Un attribut sert à différents usages comme par exemple (liste non exhautive) :
\begin{itemize}
  \item \plm!@ro! indique qu'un registre de contrôle est accessible en lecture seule (\refSectionPage{attributRo}) ;
  \item \plm+@weak+ signifie qu'une routine peut être redéfinie par une routine de même nom (\refSubsectionPage{attributWeak}) ;
  \item ajouter une étiquette à une instruction \plm!if! (\refSubsectionPage{etiquette-if}), pour en augmenter la lisibilité.
\end{itemize}








\sectionLabel{Modes}{mode}

Un mode est une séquence commençant par un accent grave « \texttt{\`} » et suivi d'un ou plusieurs chiffres, lettres ; par exemple : \plm!`user!, \plm!`boot!, \plm!`init!. Un mode permet de caractériser le mode d'exécution logique d'une procédure, et fait l'objet du \refChapterPage{chapitreModesExecution}.









\sectionLabel{Sélecteurs}{selecteur}

Un sélecteur spécifie le mode de passage d'un argument formel et d'un paramètre effectif. Ils se présentent sous plusieurs formes :
\begin{itemize}
  \item une forme anonyme : \plm+?+, \plm+!+, \plm+?!+, \plm+!?+ ;
  \item \plm+?selecteur:+, \plm+!selecteur:+, \plm+?!selecteur:+, \plm+!?selecteur:+, où $selecteur$ est une séquence de lettres ou de chiffres.
\end{itemize}



Les sélecteurs sont l'objet de la \refSectionPage{argumentFormel}.






\section{Format}

Le format est libre, la fin de ligne n’est pas significative (sauf pour les commentaires, qui commencent par deux barres obliques consécutives \plm!//!, et s’étendent jusqu’à la fin de la ligne courante). Le compilateur accepte de manière indifférente que les fins de ligne soient codés par un caractère LF (\texttt{0x0A}), un caractère CR (\texttt{0x0D}), ou par la séquence CRLF (\texttt{0x0D}, \texttt{0x0A}).


%!TEX encoding = UTF-8 Unicode
%!TEX root = ../doc-plm.tex





\chapter{Le type booléen}





\section{Le type \texttt{Bool}}

L'identificateur \plm=Bool= dénote le type booléen. Sa taille est fixée par la définition de la cible.

\section{Les mots réservés \texttt{true} et \texttt{false}}

Les mots réservés \plm+true+ et \plm+false+ dénotent respectivement la valeur logique \emph{vraie} et la valeur logique \emph{fausse}.

\section{Les opérateurs infix de comparaison}

Les valeurs booléennes sont comparables, les six opérateurs \plm+==+, \plm+!=+, \plm+>=+, \plm+>+, \plm+<=+ et \plm+<+ sont acceptés, avec \plm=false= < \plm=true=.
 
\section{Les opérateurs infixes \texttt{and}, \texttt{or} et \texttt{xor}}

Les opérateurs infixes \plm=and=, \plm=or= et \plm=xor= implémentent respectivement le \emph{et} logique, \emph{ou} logique, \emph{ou exclusif} logique. Les deux premiers évaluent les opérandes en \emph{court-circuit}, c'est-à-dire que si la valeur de l'opérande de gauche détermine la valeur de l'expression, alors l'opérande de droite n'est pas évalué.

Noter que les opérateurs infixes \plm=&=, \plm=|= et \plm=^= sont des opérateurs bit-à-bit sur les entiers non signés.


\section{L'opérateur préfixé \texttt{not}}

L'opérateur préfixé \plm=not= est la complémentation booléenne. Noter que l'opérateur préfixé \plm=~= effectue la complémentation bit-à-bit d'un entier non signé.

\section{Conversion en une valeur entière}

\section{Conversion d'une valeur entière en booléen}

Il n'y a pas d'opérateur dédié à la conversion d'une valeur entière vers un booléen. Il suffit d'utiliser des opérateurs entre entiers comme \plm+==+ ou \plm+!=+ pour réaliser une conversion :

\begin{PLM}
let result : Bool = x != 0 // x est une expression entière
\end{PLM}


%!TEX encoding = UTF-8 Unicode
%!TEX root = ../doc-plm.tex





\chapterLabel{Le type \emph{entier statique}}{chapitreTypeEntierStatique}

Ce chapitre est consacré au type \emph{entier statique} \plm!$staticInt!, qui est le type de toute constante entière littérale. Mais ce type est aussi applicable à des constantes \emph{entières statiques}, c'est-à-dire dont la valeur est calculée à la compilation.

Le type \plm!$staticInt! n'est pas applicable à une variable : une variable entière doit être déclarée du type \plm!$uintN! (entier non signé de $N$ bits) ou \plm!$intN! (entier signé de $N$ bits), décrit au \refChapterPage{chapitreTypesEntiers}.


Le langage accepte des constantes littérales entières d'une valeur quelconque. Une constante littérale entière a pour type \plm!$staticInt!.

Il est valide de déclarer des constantes de type \plm!$staticInt! :
\begin{PLM}
let N $staticInt = 1_000_000 // Ok, N a pour type $staticInt
\end{PLM}

L'annotation de type peut être omise :
\begin{PLM}
let N = 1_000_000 // Ok, N a pour type $staticInt
\end{PLM}



Il est possible d'utiliser une constante entière statique pour définir une autre constante :
\begin{PLM}
let P = N + 1 // Ok, P a pour type $staticInt, et vaut 1_000_001
\end{PLM}

Le type \plm!$staticInt! n'est pas acceptable pour une variable, aussi la déclaration suivant provoque une erreur de compilation :
\begin{PLM}
var N = 1_000_000 // Erreur, $staticInt invalide pour une variable
\end{PLM}

Il faut une annotation de type qui nomme un type \plm!$uintN! (entier non signé de $N$ bits) ou \plm!$intN! (entier signé de $N$ bits) :
\begin{PLM}
var v $uint32 = 1_000_000 // Ok
\end{PLM}

Une constante entière statique de type \plm!$staticInt! est silencieusement convertie en un type entier \plm!$uintN! ou \plm!$intN!, en vérifiant si la conversion est possible ; par exemple, l'écriture suivante déclenche une erreur de compilation :
\begin{PLM}
var w $uint8 = 1_000 // Erreur, 1_000 ne peut pas être représenté
                     // par un entier non signé de 8 bits
\end{PLM}

Comme les constantes entières statiques sont calculées à la compilation, l'écriture suivante est correcte :
\begin{PLM}
let A = 1_000_000_000_000_000_000_000_000_000_000_000_000_000
let B = 1_000_000_000_000_000_000_000_000_000_000_000_000_001
var z $uint1 = B - A
\end{PLM}

En effet, la valeur initiale de \plm!z! est $1$, représentable par un entier non signé de $1$ bit.



%!TEX encoding = UTF-8 Unicode
%!TEX root = ../doc-omnibus.tex





\chapterLabel{Les types entiers}{chapitreTypesEntiers}

Sont définis implicitement les types entiers signés et non signés d'une taille variant entre $1$ bit et $32768$ bits, et sont notés :
\begin{itemize}
  \item \omnibus!UInt1! à \omnibus!UInt32768! pour les types entiers non signés de $1$ à $32768$ bits~;
  \item \omnibus!Int1! à \omnibus!Int32768! pour les types entiers signés de $1$ à $32768$ bits.
\end{itemize}

OMNIBUS définit aussi le type \omnibus!LiteralInt!, qui est le type de toute constante entière littérale. Mais ce type est aussi applicable à des constantes entières \emph{statiques}, c'est-à-dire dont la valeur est calculée à la compilation. Ce type est décrit au \refChapterPage{chapitreTypeEntierStatique}.

\section{Constante littérale entière}

Le langage accepte des constantes littérales entières d'une taille quelconque. Une constante est convertie dans le type entier requis par le contexte sémantique, et une erreur est déclenchée à la compilation en cas d'impossibilité. Par exemple :

\begin{OMNIBUS}
var v Int8 = 128  // Erreur de compilation : 128 non
                   // représentable par un entier signé 8 bits
var v Int8 = -128 // Ok
\end{OMNIBUS}

Une constante littérale entière a pour type \omnibus!LiteralInt!, or ce type n'est pas acceptable pour une variable. Par exemple, si on écrit :
\begin{OMNIBUS}
var v = 28 // Erreur, le type LiteralInt n'est pas valide pour une variable
\end{OMNIBUS}

Dans ce cas, il faut que la déclaration contienne l'annotation de type :
\begin{OMNIBUS}
var v Int32 = 28 // Ok
\end{OMNIBUS}




\section{Conversion entre objets de type entier}

Il y a trois types de conversion entre objets de type entier :
\begin{itemize}
  \item les conversions toujours possibles \omnibus+extend Type (exp)+ (\refSubsectionPage{conversionsToujoursPossibles})~;
  \item les conversions pouvant échouer \omnibus+convert Type (exp)+ (\refSubsectionPage{conversionsPouvantEchouer})~;
  \item les troncatures \omnibus+truncate Type (exp)+  (\refSubsectionPage{conversionsTroncature}).
\end{itemize}


\subsectionLabel{Conversions toujours possibles : \texttt{extend}}{conversionsToujoursPossibles}

Les conversions qui sont toujours possibles sont exprimées par le mot réservé \omnibus=extend=. Par exemple :
\begin{OMNIBUS}
let v UInt8 = ...
let x UInt9 = extend UInt9 (v)
let y Int9 =  extend Int9 (v)
let z Int10 = extend Int10 (y)
\end{OMNIBUS}

D'une manière générale :
\begin{itemize}
\item un entier non signé peut être étendu en un entier non signé de taille strictement supérieure~;
\item un entier non signé peut être étendu en un entier signé de taille strictement supérieure~;
\item un entier signé peut être étendu en un entier signé de taille strictement supérieure.
\end{itemize}

Par contre, une conversion pouvant provoquer un débordement est rejetée à la compilation :
\begin{OMNIBUS}
let s Int8 = ...
let x UInt16 = x // Erreur de compilation
\end{OMNIBUS}

L'annotation de type après le mot-clé \omnibus=extend= est optionnel~: par défaut, le type est inféré par le contexte. Par exemple, on peut écrire~:
\begin{OMNIBUS}
let x Int9 =  ...
let y Int10 = extend (x) // Le type Int10 est inféré du contexte
let z = extend (x) // Invalide : aucun type ne peut être inféré.
\end{OMNIBUS}


\subsectionLabel{Conversions pouvant échouer : \texttt{convert}}{conversionsPouvantEchouer}

Les conversions pouvant échouer sont exprimées par le mot réservé \omnibus=convert=. Par exemple :

\begin{OMNIBUS}
let s Int8 = ...
let x UInt16 = convert UInt16 (s)
\end{OMNIBUS}

L'opérateur \omnibus+convert+ engendre un code qui vérifie à l'exécution que l'expression source (ici \omnibus+x+) peut être convertie dans le type cible (ici \omnibus+UInt16+) sans débordement. En cas de débordement détecté à l'exécution, la panique dont le code est donné dans le \refTableauPage{tableauCodePanique} est déclenchée. L'opérateur \omnibus+convert+ est donc interdit dans les constructions où la panique ne peut être déclenchée : il faut alors utiliser l'opérateur \omnibus=truncate=.

L'annotation de type après le mot-clé \omnibus=convert= est optionnel~: par défaut, le type est inféré par le contexte. Par exemple, on peut écrire~:
\begin{OMNIBUS}
let s Int8 = ...
let x UInt16 = convert (s) // Le type UInt16 est inféré du contexte
let y = convert (s) // Invalide : aucun type ne peut être inféré
\end{OMNIBUS}

L'opérateur \omnibus+convert+ ne peut pas apparaître dans une expression statique.

De plus, une erreur de compilation est déclenchée si l'opérateur \omnibus+convert+ est utilisé alors que la conversion est toujours possible :
\begin{OMNIBUS}
let v UInt8 = ...
let y = convert Int16 (v) // Erreur, conversion toujours possible
\end{OMNIBUS}

\subsectionLabel{Troncatures : \texttt{truncate}}{conversionsTroncature}

L'opérateur \omnibus+truncate+ permet de spécifier une conversion explicite silencieuse, qui ne déclenche aucune panique. La valeur de l'expression source est tronquée en cas de débordement\footnote{L'opérateur \texttt{truncate} est équivalent au \emph{type cast} entre entiers du langage C.}. Par exemple :

\begin{OMNIBUS}
let s Int8 = -10
let x UInt16 = truncate UInt16 (x)
\end{OMNIBUS}

L'annotation de type après le mot-clé \omnibus=truncate= est optionnel~: par défaut, le type est inféré par le contexte. Par exemple, on peut écrire~:
\begin{OMNIBUS}
let s Int8 = -10
let x UInt16 = truncate (s) // Le type UInt16 est inféré du contexte
let y = truncate (s) // Invalide : aucun type ne peut être inféré
\end{OMNIBUS}


L'opérateur \omnibus+truncate+ ne peut pas apparaître dans une expression statique.

De plus, une erreur de compilation est déclenchée si l'opérateur \omnibus+truncate+ est utilisé alors qu'une conversion implicite est possible :
\begin{OMNIBUS}
let v UInt8 = ...
let y = truncate Int16 (v) // Erreur, conversion toujours possible
\end{OMNIBUS}

\section{Opérateurs infixes de comparaison}

Les valeurs entières sont comparables, les six opérateurs \omnibus+==+, \omnibus+≠+, \omnibus+≥+, \omnibus+>+, \omnibus+≤+ et \omnibus+<+ sont acceptés.

La comparaison ne peut s'effectuer qu'entre objets du même type entier, ou entre un objet de type entier et une constante littérale entière.









\sectionLabel{Opérateurs infixes arithmétiques}{operateursInfixArithmétiques}


Les opérateurs infixes arithmétiques sont listés dans le \refTableau{operateursInfixesArithmetiques} avec leur signification. Ils ne peuvent opérer qu'entre objets du même type entier, ou entre un objet de type entier et une constante littérale entière.


\begin{table}[!ht]
\centering
\begin{tabular}{lllll}
  \textbf{Opérateur} & \textbf{Signification} \\
  \omnibus=+= & Addition avec détection de débordement\\
  \omnibus=-= & Soustraction avec détection de débordement\\
  \omnibus=*= & Multiplication avec détection de débordement\\
  \omnibus=/= & Division avec détection de débordement\\
  \omnibus=%= & Modulo avec détection de division par zéro\\
  \omnibus=+%= & Addition sans détection de débordement\\
  \omnibus=-%= & Soustraction sans détection de débordement\\
  \omnibus=*%= & Multiplication sans détection de débordement\\
  \omnibus=!/= & Division sans détection de débordement\\
  \omnibus=!%= & Modulo sans détection de division par zéro\\
\end{tabular}
\caption{Opérateurs infixes arithmétiques}
\labelTableau{operateursInfixesArithmetiques}
\ligne
\end{table}




\section{Opérateurs préfixés de négation arithmétique}

\subsectionLabel{Opérateur \texttt{-}}{negationOvf}

L'opérateur préfixé \omnibus=-= est la négation arithmétique avec détection de débordement. Il n'est accepté que sur les types signés. La négation de la borne inférieure d'un type signé (\omnibus+-128+ pour \omnibus+Int8+, \omnibus+-32768+ pour \omnibus+Int16+, ...) entraîne un débordement arithmétique qui déclenche une panique dont le code est donné dans le \refTableau{tableauCodePanique}.


\subsectionLabel{Opérateur \texttt{-\%}}{negationNoOvf}

L'opérateur préfixé \omnibus=-%= est la négation arithmétique sans détection de débordement. Il n'est accepté que sur les types signés. La négation de la borne inférieure d'un type signé (\omnibus+-128+ pour \omnibus+Int8+, \omnibus+-32768+ pour \omnibus+Int16+, ...) retourne cette même valeur. Cet opérateur ne déclenche jamais de panique.




\sectionLabel{Opérateurs infixes bit-à-bit}{operateurBitABitEntier}
\index{"\&!Entier}
\index{\textbar!Entier}
\index{\^!Entier}

Les opérateurs infixes bit-à-bit acceptent les types entiers non signés (\refTableau{operateursInfixesBitABit}).

\begin{table}[h]
\centering
\begin{tabular}{lllll}
  \textbf{Opérateur} & \textbf{Signification} \\
  \omnibus=|= & \emph{ou} bit-à-bit\\
  \omnibus=&= & \emph{et} bit-à-bit\\
  \omnibus=^= & \emph{ou exclusif} bit-à-bit\\
\end{tabular}
\caption{Opérateurs infixes bit-à-bit sur les entiers non signés}
\labelTableau{operateursInfixesBitABit}
\ligne
\end{table}





\section{Opérateur préfixé bit-à-bit}

L'opérateur préfixé \omnibus=~= retourne la complémentation bit-à-bit d'une valeur entière non signée.




\section{Opérateurs infixes de décalage}

Les opérateurs infixes \omnibus=<<= et \omnibus=>>= réalisent respectivement le décalage à gauche et à droite de l'opérande de gauche. L'amplitude du décalage est spécifiée par la valeur de l'opérande droite (\refTableau{operateursInfixesDecalage}). \omnibus=a= est une expression entière signée ou non signée, et l'expression renvoie une valeur de même type que \omnibus=a=. L'expression \omnibus=b= est une expression entière non signée.

\begin{table}[h]
\centering
\begin{tabular}{lllll}
  \textbf{Expression} & \textbf{Signification} \\
  \omnibus=a << b= & Décalage à gauche de \omnibus=a= d'une amplitude de \omnibus=b= bits\\
  \omnibus=a >> b= & Décalage à droite de \omnibus=a= d'une amplitude de \omnibus=b= bits\\
\end{tabular}
\caption{Opérateurs infixes de décalage sur les entiers}
\labelTableau{operateursInfixesDecalage}
\ligne
\end{table}








\sectionLabel{Opérateurs combinées avec une affectation}{operateursCombinesAffectationEntier}
\index{\&=!Entier}
\index{\textbar=!Entier}
\index{\^{}=!Entier}

Les opérateurs suivants sont définis pour les entiers.

\omnibus!a &= b! est équivalent à \omnibus!a = a & b!.

\omnibus!a |= b! est équivalent à \omnibus!a = a | b!.

\omnibus!a ^= b! est équivalent à \omnibus!a = a ^ b!.

\omnibus!a += b! est équivalent à \omnibus!a = a + b!.

\omnibus!a +%= b! est équivalent à \omnibus!a = a +% b!.

\omnibus!a -= b! est équivalent à \omnibus!a = a - b!.

\omnibus!a -%= b! est équivalent à \omnibus!a = a -% b!.

\omnibus!a *= b! est équivalent à \omnibus!a = a * b!.

\omnibus!a *%= b! est équivalent à \omnibus!a = a *% b!.

Les opérateurs infixes \omnibus!&=!, \omnibus!|=! et \omnibus!^=! sont décrits à la \refSectionPage{operateurBitABitEntier}.

Les opérateurs infixes \omnibus!+!, \omnibus!+%!, \omnibus!-!, \omnibus!-%!, \omnibus!*! et \omnibus!*%! sont décrits à la \refSectionPage{operateursInfixArithmétiques}.








\sectionLabel{Accesseurs}{accesseursEntiers}


\subsection{Accesseur \texttt{bitReversed}}

L'accesseur \omnibus=bitReversed= est implémenté pour tout type entier. Pour un entier sur $n$ bits, il retourne une valeur dont le bit d'indice $i$ (avec $0 \leqslant i < n$) a la valeur du bit $n-i-1$ du récepteur. Par exemple~:

\begin{OMNIBUS}
let x UInt3 = 0x_3
let y = x.bitReversed () // 0x_6
let z UInt16 = 0x_1234
let t = z.bitReversed () // 0x_2C48
\end{OMNIBUS}



\subsection{Accesseur \texttt{byteSwapped}}

L'accesseur \omnibus=byteSwapped= est implémenté pour tout type entier dont la taille est un multiple de 16 bits, qu'il soit signé ou non (\omnibus=Int16=, \omnibus=UInt16=, \omnibus=Int32=, \omnibus=UInt32=, \omnibus=Int48=, \omnibus=UInt48=, \omnibus=Int64=, \omnibus=UInt64=, …). Il réalise la conversion \emph{big endian} $\leftrightarrow$ \emph{little endian}. Par exemple~:

\begin{OMNIBUS}
let x : UInt32 = 0x_1234_5678
let y = x.byteSwapped () // 0x_8756_3412
let z UInt48 = 0x_1234_5678_ABCD
let t = z.byteSwapped () // 0x_CDAB_8756_3412
\end{OMNIBUS}



\subsection{Accesseur \texttt{leadingZeroCount}}

L'accesseur \omnibus=leadingZeroCount= est implémenté pour tout type entier. Il retourne le nombre de bits à zéro à partir du bit le plus significatif. Par exemple~:

\begin{OMNIBUS}
let x : UInt32 = 0x_1234_5678
let y = x.leadingZeroCount () // 3
\end{OMNIBUS}



\subsection{Accesseur \texttt{setBitCount}}

L'accesseur \omnibus=setBitCount= est implémenté pour tout type entier. Il retourne le nombre de bits à un dans la valeur du récepteur. Par exemple~:

\begin{OMNIBUS}
let x UInt16 = 0x_1234
let y = x.setBitCount () // 5
\end{OMNIBUS}



\subsection{Accesseur \texttt{trainingZeroCount}}

L'accesseur \omnibus=trainingZeroCount= est implémenté pour tout type entier. Il retourne le nombre de bits à zéro à partir du bit le moins significatif. Par exemple~:

\begin{OMNIBUS}
let x UInt16 = 0x_1234
let y = x.trainingZeroCount () // 2
\end{OMNIBUS}




\section{Construction d'un entier non signé par tranches}

Cette expression permet de construire un entier non signé à partir d'entiers non signés ou de booléens. Par exemple~:
\begin{OMNIBUS}
let x = {UInt8 !1:1 !1:0 !6:12} // 0x8C
\end{OMNIBUS}

Dans l'exemple ci-dessus, le bit n°7 est à 1, le bit n°6 à 0, et les 6 bits de poids faibles à 12.

Les tranches sont désignés par des sélecteurs (\omnibus=!1:=, \omnibus=!6:=) qui indiquent le nombre de bits de la tranche. La description commence à partir du bit de poids fort. L'expression qui suit le sélecteur doit être du type entier non signé correspondant~: par exemple, \omnibus=!1:= $\rightarrow$ \omnibus=UInt1=, \omnibus=!6:= $\rightarrow$ \omnibus=UInt6=.

Attention, une tranche \omnibus=!1:= signifie que l'expression doit être un entier non signé sur un bit, et non pas un booléen. Si l'on veut initialiser une tranche de 1 bit à partir d'un booléen, il faut utiliser le sélecteur particulier \omnibus=!b:=. Par exemple~:
\begin{OMNIBUS}
let x = {UInt8 !b:yes !1:0 !6:12} // 0x8C
\end{OMNIBUS}

Si toutes les expressions associées aux sélecteurs sont statiques, alors l'expression de construction d'un entier non signé par tranches est aussi une expression statique~: on peut donc utiliser cette construction pour définir des constantes globales.




%!TEX encoding = UTF-8 Unicode
%!TEX root = ../doc-plm.tex





\chapter{Les types flottants}

Les types flottants ne sont pas pris en charge dans la version actuelle.
%!TEX encoding = UTF-8 Unicode
%!TEX root = ../doc-plm.tex





\chapter{Le type caractère}

\colorbox{red}{Non pris en charge actuellement.}

%!TEX encoding = UTF-8 Unicode
%!TEX root = ../doc-plm.tex





\chapter{Les types chaîne de caractères}

\colorbox{red}{Non pris en charge actuellement.}
%!TEX encoding = UTF-8 Unicode
%!TEX root = ../doc-plm.tex





\chapter{Les types énumérés}


\section{Déclaration d'un type énuméré}

La déclaration d'un type énuméré est introduite par le mot réservé \plm!enum! :

\begin{PLM}
enum $feu {
  case vert
  case orange
  case rouge
}
\end{PLM}

\section{Utilisation d'un type énuméré}

\subsection{Constructeurs}
La déclaration d'un type énuméré définit les constructeurs associés au type énuméré, ici \plm!$feu.vert!, \plm!$feu.orange! et \plm!$feu.rouge!.


\subsection{Constante globale et variable globale}
Les constructeurs d'un type énuméré sont \emph{statiques}, c'est-à-dire qu'ils permettent d'initialiser des variables globales et des constantes globales :

\begin{PLM}
let ROUGE $feu = $feu.rouge
\end{PLM}

L'annotation de type peut être omis une fois, c'est-à-dire que l'on peut aussi écrire :
\begin{PLM}
let ROUGE $feu = .rouge
\end{PLM}

Ou encore :
\begin{PLM}
let ROUGE = $feu.rouge
\end{PLM}


De même, on peut déclarer une variable globale appartenant à un type énuméré :
\begin{PLM}
var unFeu $feu = $feu.vert { ... }
\end{PLM}

L'annotation de type peut être omis une fois, c'est-à-dire que l'on peut aussi écrire :
\begin{PLM}
var unFeu $feu = .vert { ... }
\end{PLM}

Ou encore :
\begin{PLM}
var unFeu = $feu.vert { ... }
\end{PLM}

\subsection{Comparaison}

Les opérateurs \plm!==!, \plm+!=+, \plm!<!, \plm!<=!, \plm!>! et \plm!>=! permettent de comparer les valeurs d'un type énuméré ; la relation d'ordre est donnée par l'ordre de déclaration des constantes, c'est-à-dire que \plm!$feu.vert < $feu.orange! et \plm!$feu.orange < $feu.rouge!.


\sectionLabel{Représentation d'un type énuméré}{representation-type-enumere}

Un type énuméré est représenté dans le code engendré par une valeur codée sur le plus petit nombre de bits nécessaire. Par exemple, le type énuméré suivant code trois valeurs.
\begin{PLM}
enum $feu {
  case vert
  case orange
  case rouge
}
\end{PLM}

Un objet de ce type est donc codé sur deux bits, et \plm+vert+ est représenté par $0$, \plm+orange+ par $1$ et \plm=rouge= par $2$.


%!TEX encoding = UTF-8 Unicode
%!TEX root = ../doc-plm.tex





\chapter{Les types structure}


\section{Déclaration d'un type structure}

La déclaration d'un type structure est introduite par le mot réservé \plm!struct! :

\begin{PLM}
struct $point {
  var x $int32 = 0
  var y $int32 = 0
}
\end{PLM}

Dans la version actuelle de PLM, tous les champs doivent être initialisés.

Déclarer un type structure définit implicitement un initialiseur sans argument, qu'il faut appeler lorsque l'on initialise une variable ou une constante. Par exemple :
\begin{PLM}
var pt $point = $point ()
\end{PLM}

Et l'annotation de type peut être omise :
\begin{PLM}
var pt = $point ()
\end{PLM}

Tout type acceptable pour un champ. Par exemple : 
\begin{PLM}
struct $point3D {
  var p $point = $point ()
  var z $int32 = 0
}
\end{PLM}

\section{Procédure de structure}

Il existe deux sortes de procédure de structure :
\begin{itemize}
  \item les procédures constantes (\plm!proc!), \refSubsectionTitlePage{procConstStructure} ;
  \item les procédures mutables (\plm!mutating proc!), \refSubsectionTitlePage{procMutableStructure}.
\end{itemize}


\subsectionLabel{Procédure constante}{procConstStructure}

Un type structure peut définir des procédures grâce à la construction \plm!proc! :

\begin{PLM}
struct $point {
  var x $int32 = 0
  var y $int32 = 0
  
  proc `user getX (!outX $uint32) {
    outX = self.x
  }
}
\end{PLM}

Trois remarques :
\begin{itemize}
  \item une telle procédure ne peut pas modifier l'objet auquel il s'applique (pour le faire, il faut définir une \plm!mutating proc!) ;
  \item l'accès aux variables globales n'est pas possible : il faut appeler une procédure qui effectue l'accès ;
  \item préfixer l'accès à un champ par \plm!self.! est obligatoire. 
\end{itemize}




\subsectionLabel{Procédure mutable}{procMutableStructure}

Si on veut modifier un champ, il faut déclarer la procédure de structure avec le qualificatif \plm!mutating! :

\begin{PLM}
struct $point {
  var x $int32 = 0
  var y $int32 = 0
  
  mutating proc `user setX (?inX $uint32) {
    self.x = inX
  }
}
\end{PLM}

L'accès aux variables globales n'est pas possible, et préfixer l'accès à un champ par \plm!self.! est obligatoire. 

Il est possible de modifier l'objet courant en affectant \plm!self! :

\begin{PLM}
struct $point {
  var x $int32 = 0
  var y $int32 = 0
  
  mutating proc reset `user () {
    self = $point ()
  }
}
\end{PLM}

Ou encore :

\begin{PLM}
struct $point {
  var x $int32 = 0
  var y $int32 = 0
  
  mutating proc reset `user () {
    newPoint (?self)
  }
}

proc `user newPoint (!p $point) {
  p = $point ()
}
\end{PLM}














\section{Visibilité des champs et des méthodes}

Dans la version actuelle, les champs et les méthodes sont publiques.



%!TEX encoding = UTF-8 Unicode
%!TEX root = ../doc-plm.tex





\chapter{Les types opaque}

Un \emph{type opaque} est un type dont la définition est externe à PLM, dans le code C associé. Par défaut, les objets de ce type ne peuvent pas être copiés, ni comparés, ni instanciés et leur contenu est inaccessible. La déclaration d'attributs associés modifie ces propriétés par défaut (\refSectionPage{attributTypeOpaque}).

\section{Déclaration d'un type opaque}

La déclaration d'un type opaque est introduite par le mot réservé \plm!type! :

\begin{PLM}
type $monTypeOpaque : ((32))
\end{PLM}

La séquence \plm!((nombre))! caractérise la déclaration d'un type opaque. Les deux niveaux de parenthèses sont obligatoires. Le nombre associé (ici \plm!32!) est le nombre de bits pour représenter un objet de ce type.

{\bf Note.} Actuellement, une expression statique n'est pas acceptée pour spécifier ce nombre de bits. 

\sectionLabel{Attributs d'un type opaque}{attributTypeOpaque}

La déclaration d'un type opaque accepte deux attributs :
\begin{itemize}
\item \plm=@instantiable=, qui rend instanciable un objet d'un type opaque ;
\item \plm=@copyable=, qui rend copiable un objet d'un type opaque.
\end{itemize}

Ces deux attributs sont cumulables.

\subsection{Attribut \texttt{@instantiable}}

L'attribut \plm=@instantiable= est spécifié après l'indication de taille du type :

\begin{PLM}
type $monTypeOpaque : ((32)) @instantiable
\end{PLM}


Par défaut, un type opaque n'est pas instanciable ; l'attribut \plm=@instantiable= le rend instanciable, c'est-à-dire que l'on peut écrire :

\begin{PLM}
var t $monTypeOpaque = $monTypeOpaque ()
\end{PLM}

Ou encore, en supprimant l'annotation de type :

\begin{PLM}
var t = $monTypeOpaque ()
\end{PLM}

L'expression \plm=$monTypeOpaque ()= est à une instance de \plm=$monTypeOpaque= dont la valeur correspond à des zéros binaires.






\subsection{Attribut \texttt{@copyable}}

L'attribut \plm=@copyable= est spécifié après l'indication de taille du type :

\begin{PLM}
type $monTypeOpaque : ((32)) @copyable
\end{PLM}


Par défaut, un type opaque n'est pas copiable ; l'attribut \plm=@copyable= le rend copiable, c'est-à-dire que l'on peut écrire :

\begin{PLM}
var t = $monTypeOpaque ()
var u = t // Copie d'une instance de $monTypeOpaque
\end{PLM}



%!TEX encoding = UTF-8 Unicode
%!TEX root = ../doc-plm.tex





\chapter{Les types tableau}

PLM implémente les tableaux de taille fixe. La déclaration d'un type tableau utilise obligatoirement la construction \plm=type= (\refSectionPage{DecTypeTableau}), il n'est pas possible de déclarer un type tableau anonyme.








\sectionLabel{Déclaration d'un type tableau}{DecTypeTableau}

La déclaration d'un type tableau est réalisée par la construction \plm=newtype=~:

\begin{PLM}
newtype MonTypeTableau : UInt32 [20]
\end{PLM}

Cette construction déclare le type \plm=MonTypeTableau= comme tableau de $20$ instances de \plm=UInt32=.

Tout type est acceptable comme élément de tableau (ici \plm=UInt32=), du moment qu'il est instanciable et copiable. Essayer de définir un type tableau avec un type non instanciable et/ou non copiable entraîne une erreur de compilation.

La taille du tableau est une expression statique, de type entier. Il est donc possible de faire référence à des constantes globales, comme par exemple :

\begin{PLM}
let SIZE = 20
typealias MonTypeTableau = [SIZE @x UInt32]
\end{PLM}


\sectionLabel{Construction d'un tableau}{ConstructionTableau}

Un tableau statique implémente deux constructeurs~:
\begin{itemize}
  \item un constructeur qui initialise tous les éléments du tableau à la même valeur (\refSubsectionPage{constructeurTableauRepeated})~;
  \item un constructeur qui initialise chaque élément du tableau à une valeur particulière (\refSubsectionPage{constructeurTableauValeurParticuliere}).
\end{itemize}




\subsectionLabel{Constructeur \texttt{(!repeated)}}{constructeurTableauRepeated}
L'expression \plm=MonTypeTableau (!repeated:exp)= est une instance du type \plm=MonTypeTableau=, dont tous les éléments sont initialisés avec la valeur de \plm=exp=. Par exemple~:
\begin{PLM}
let v : UInt32 = 0
var t : UInt32 [10] = UInt32 [10] (!repeated:v)
\end{PLM}


\subsectionLabel{Constructeur \texttt{(!!...)}}{constructeurTableauValeurParticuliere}
Ce constructeur possède autant d'arguments que d'éléments dans le tableau. Le premier est affecté à l'élément d'indice 0, le deuxième à celui d'indice 1, ... Par exemple~:
\begin{PLM}
var t UInt32 [3] = UInt32 [3] (!0 !1 !2)
\end{PLM}



\sectionLabel{Déclaration d'une instance de tableau}{DecInstanceTableau}

La déclaration d'une instance de tableau s'effectue en nommant le type tableau et l'expression de construction de ce tableau, où \plm=v= est une valeur du type de l'élément de tableau (ici, \plm=UInt32=)~:

\begin{PLM}
let v UInt32 = 0
var t : MonTypeTableau = MonTypeTableau (!repeated:v)
\end{PLM}

Et on peut omettre l'annotation de type~:

\begin{PLM}
var t = MonTypeTableau (!repeated:v)
\end{PLM}




\sectionLabel{Obtention de la taille d'un tableau}{ObtentionTailleTableau}

L'expression \plm=MonTypeTableau.count= renvoie la taille du tableau, sous la forme d'un entier statique (c'est-à-dire de type \plm=LiteralInt=, voir \refChapterPage{chapitreTypeEntierStatique}). Ou peut aussi appliquer \plm=count= à une instance de tableau.

On peut donc utiliser les expressions \plm=$t.count= et \plm=$monTypeTableau.count= pour itérer sur tous les élements d'un tableau~:
\begin{PLM}
var t = MonTypeTableau (!repeated:0)
for i UInt32 in 0 ..< t.count {
  t [i] = i
}
for i UInt32 in 0 ..< MonTypeTableau.count {
  t [i] = i
}
\end{PLM}


Une erreur de compilation est déclenchée si on écrit~:
\begin{PLM}
var s = MonTypeTableau.count // ERREUR
\end{PLM}

En effet, \plm=MonTypeTableau.count= est un entier statique et le type \emph{entier statique} ne peut pas être attribué à une variable. Il faut préciser obligatoirement un type d'entier~:

\begin{PLM}
var s UInt32 = MonTypeTableau.count // Ok
\end{PLM}

Ce type peut être signé ou non signé, du moment que sa plage de valeur permet de représenter la valeur statique.






\sectionLabel{Accès à un élément d'un tableau}{AccesElementTableau}

L'accès à un élément d'un tableau s'effectue par la notation classique \plm=[expression_indice]=. Les indices valides commencent à $0$ jusqu'à la taille du tableau moins $1$.

L'accès à un élément de tableau est valide dans les constructions suivantes~:
\begin{itemize}
  \item expression~: \plm!var x = t [1]! ;
  \item cible d'une affectation~: \plm!t [1] = x! ;
  \item cible d'un opérateur combiné à une affectation~: \plm!t [1] += x! ;
\end{itemize}

L'\plm=expression_indice= doit être une expression entière, signée ou non signée.

\subsection{Expression indice statique}

Si l'\plm=expression_indice= est statique, c'est-à-dire dont la valeur est calculée à la compilation, la vérification de sa validité est effectuée à la compilation. Par exemple, pour un tableau \plm=t= de $20$ éléments~:
\begin{PLM}
t [0] // Ok
t [19] // Ok
t [-1] // Erreur de compilation, indice négatif
t [20] // Erreur de compilation, indice trop grand
\end{PLM}

Comme la validité est effectuée à la compilation, aucune vérification de validité n'est effectuée à l'exécution.


\subsectionLabel{Expression indice non signée}{indiceNonSigne}

Si l'\plm=expression_indice= est une instance d'un type entier non signé, il y a deux possibilités.

Soit ce type entier non signé présente une valeur maximum supérieure ou égale à la taille du tableau~: alors le code engendré vérifie à l'exécution que l'\plm=expression_indice= a une valeur valide. Dans l'exemple ci-dessous, le tableau \plm=t= contient $20$ éléments, l'\plm=expression_indice= doit donc être inférieure ou égale à $19$ ; or la plage de valeurs de \plm=UInt32= dépasse cette borne, aussi la validité est vérifiée à l'exécution. En cas d'échec, la panique est déclenchée (\refTableauPage{tableauCodePanique}).
\begin{PLM}
var i UInt32 = ...
var x = t [i] // Vérification à l'exécution que i < 20
\end{PLM}



La seconde possibilité est que ce type entier non signé présente une valeur maximum strictement inférieure à la taille du tableau~: la compilation garantit que l'indice sera toujours valide, aucune vérification n'est effectuée à l'exécution. Par exemple~:
\begin{PLM}
var i UInt4 = ... // Donc 0 ≤ i ≤ 15
var x = t [i] // Aucune vérification à l'exécution car toujours i < 20
\end{PLM}



\subsectionLabel{Expression indice signée}{indiceSigne}

Quand l'\plm=expression_indice= est une expression entière signée, il faut vérifier~:
\begin{itemize}
  \item qu'elle est positive ou nulle ;
  \item qu'elle est strictement inférieure à la taille du tableau.
\end{itemize}

La première vérification est toujours réalisée à l'exécution ; la seconde dépend de la valeur maximum du type entier, de manière analogue à ce qui est fait pour un indice entier non signé. Donc, pour un tableau \plm=t= de $20$ éléments~:

\begin{PLM}
var i Int32 = ...
var x = t [i] // Vérification à l'exécution que 0 ≤ i < 20
\end{PLM}


\begin{PLM}
var i Int4 = ... // Donc -16 ≤ i < 15
var x = t [i] // Vérification à l'exécution que 0 ≤ i
\end{PLM}

En cas d'échec de la vérification à l'exécution, la panique est déclenchée (\refTableauPage{tableauCodePanique}).

\subsection{Accès à un élément en mode panique}

Dans une liste d'instructions devant être exécutée en mode panique, les instructions pouvant engendrer une panique sont interdites et leur présence entraîne une erreur de compilation.

Ainsi l'accès à un élément de tableau est donc accepté par le compilateur si l'\plm=expression_indice= est~:
\begin{itemize}
\item une expression statique ;
\item ou une expression entière non signée, dont la valeur maximum est strictement inférieure à la taille du tableau.
\end{itemize}


Donc l'accès à un élément de tableau est donc rejeté par le compilateur si l'\plm=expression_indice= est~:
\begin{itemize}
\item une expression entière signée ;
\item ou une expression entière non signée, dont la valeur maximum est supérieure ou égale à la taille du tableau.
\end{itemize}



%!TEX encoding = UTF-8 Unicode
%!TEX root = ../doc-plm.tex





\chapter{Tableaux statiques constants}

PLM permet de construire des tableaux statiques constants, en séparant déclaration du tableau et constitution. Cette caractéristique s'appuie sur les trois constructions suivantes :
\begin{itemize}
  \item déclaration du tableau statique (\refSection{DecTypeTableauStatique}) ;
  \item ajout d'un élément au tableau statique (\refSection{ajoutElementTableauStatique}) ;
  \item parcours de'un tableau statique (\refSection{parcoursTableauStatique}).
\end{itemize}











\sectionLabel{Déclaration}{DecTypeTableauStatique}

La déclaration d'un tableau statique est réalisée par la construction \plm=staticArray= :

\begin{PLM}
staticArray maListeStatique {
  let a $uint32
  let b $uint32
}
\end{PLM}

Cette construction déclare la constante \plm=maListeStatique= comme tableau constant vide. Pour le remplir, utiliser la construction \plm=extend staticArray= (\refSection{ajoutElementTableauStatique}).

La composition de chaque élément est spécifiée par la liste des propriétés, chacune d'elles étant définie par son nom et son type.









\sectionLabel{Ajout d'un élément au tableau}{ajoutElementTableauStatique}

Un tableau statique est construit élément par élément : 

\begin{PLM}
extend staticArray maListeStatique (5, 9)
\end{PLM}

Cette déclaration ajoute un élément au tableau statique, élément dont toutes les propriétés doivent être initialisées par une expression statique. 

\fbox{\begin{minipage}{1.0\textwidth}
   {\bf Attention !} L'ordre des éléments ne peut pas être spécifié. Il peut varier d'une façon imprévisible d'une compilation à une autre. Aussi, il faut veiller que les opérations réalisées soient indépendantes de l'ordre dans lequel les éléments sont placés dans le tableau statique.
\end{minipage}}





\sectionLabel{Parcours d'un tableau statique}{parcoursTableauStatique}

L'instruction \plm=for= (\refSectionPage{instructionFor}) est la seule qui accède à un tableau statique. Elle permet de parcourir tous les éléments du tableau :

\begin{PLM}
var total $uint32 = 0
for élément in maListeStatique {
  total += élément.a
  total += élément.b
}
\end{PLM}

L'élément courant est désigné par la constante \plm=élément=, et on accède aux propriétés par la notation pointée habituelle. 






%!TEX encoding = UTF-8 Unicode
%!TEX root = ../doc-plm.tex


\chapter{Les registres de contrôle}

%La déclaration d'un registre de contrôle obéit à une syntaxe particulière, ne serait-ce que parce que son adresse absolue doit y être spécifiée. Pour de nombreux registres, un bit ou un groupe de bits ont une signification particulière, et obtenir la valeur d'un champ ou modifier sa valeur est une opération courante.

%À titre d'exemple, nous allons nous intéresser au registre \texttt{ICSR} du processeur ARMv7-M. Le \emph{manuel de référence de l'architecture ARMv7-M}\footnote{\url{http://infocenter.arm.com/help/index.jsp?topic=/com.arm.doc.ddi0403e.b/index.html}} décrit ce registre comme indiqué à la \refFigure{}{definitionPORTxPCRn}, et indique que son adresse est \texttt{0xE000ED04}.

Dans ce chapitre, nous allons décrire~:
\begin{itemize}
  \item comment déclarer un registre de contrôle, comment lui affecter une valeur, et le lire (\refSectionPage{simpleDeclarationRegistre})~;
  \item comment déclarer les champs d'un registre de contrôle, et comment les utiliser (\refSectionPage{declarationRegistreEtChamps})~;
  \item comment déclarer plusieurs registres de contrôle ayant la même composition de champs (\refSectionPage{declarationPlusieursRegistres})~;
  \item comment déclarer et utiliser un tableau de registres de contrôle (\refSectionPage{declarationTableauRegistres})~;
  \item les attributs applicables aux registres de contrôle (\refSectionPage{attributsRegistreControle})~;
  \item les restrictions d'usage des registres de contrôle (\refSectionPage{restrictionsUsageRegistres}).
\end{itemize}

À titre d'exemple, nous allons nous intéresser aux registres \texttt{PORTx\_PCRn} du micro-contrôleur \texttt{MK20DX256VLH7} qui équipe les cartes \emph{Teensy 3.x} (\refFigure{}{definitionPORTxPCRn}). La documentation de ce micro-contrôleur indique que l'un des registres de cette famille, \texttt{PORTA\_PCR0} est à l'adresse \texttt{0x4004\_9000}.


\begin{figure}[htbp]
\centering
\includegraphics[width=14cm]{chapitres/PORTx_PCRn.pdf}
\caption{Registres de contrôle \texttt{PORTx\_PCRn} intégré dans le \texttt{MK20DX256VLH7}}
\labelFigure{definitionPORTxPCRn}
%\ligne
\end{figure}




\sectionLabel{Groupe de registres}{GroupeRegistre}




\sectionLabel{Simple déclaration d'un registre}{simpleDeclarationRegistre}

Pour déclarer le registre \texttt{PORTA\_PCR0} (\refFigure{}{definitionPORTxPCRn}), situé à l'adresse \texttt{0x4004\_9000}, on écrit~:

\begin{PLM}
registers PORTA {
  PCR0 at 0x4004_9000 $uint32
}
\end{PLM}

Le type \plm+$uint32+ qui est mentionné signifie que les valeurs écrites et lues de ce registre sont des entiers non signés de 32 bits. Tout type entier, signé ou non signé est autorisé.

Pour lire ou écrire ce registre, on le nomme comme s'il s'agissait d'une simple variable. Par exemple, pour configurer le bit $0$ du port \texttt{A} en entrée ou en sortie logique, il faut écrire $1$ dans le champ \texttt{MUX} et zéro dans les autres champs. Comme le champ \texttt{MUX} commence au $8^e$ bit, on écrit~:

\begin{PLM}
PORTA.PCR0 = 1 << 8
\end{PLM}

Si l'on veut que ce port soit un \emph{collecteur ouvert} si il est programmé en sortie, il faut mettre le champ \texttt{ODE} à $1$. On écrit donc~:
\begin{PLM}
PORTA.PCR0 = (1 << 8) | (1 << 5)
\end{PLM}

Lire le contenu du registre est réalisé par une instruction d'affectation~:
\begin{PLM}
let x = PORTA.PCR0
\end{PLM}
Le type de la constante \plm+x+ est \plm+$uint32+, déduit du type du registre de contrôle \texttt{PORTA\_PCR0}.

Pour savoir si le bit \texttt{ODE} est activé, on réalise un masquage (l'annotation du type \plm+$bool+ est facultative)~:
\begin{PLM}
let ODEactivé $bool = (PORTA.PCR0 & (1 << 5)) ≠ 0
\end{PLM}

Pour obtenir la valeur du champ \texttt{MUX}, on effectue un décalage, suivi d'un masquage~:
\begin{PLM}
let champMUX $uint32 = (PORTA.PCR0 >> 8) & 7
\end{PLM}
 
 
Toutes ces formulations peuvent être rendues plus intelligibles en précisant la composition du registre \texttt{PORTA\_PCR0} dans sa déclaration. C'est ce qui va être réalisé dans la section suivante.








\sectionLabel{Déclaration d'un registre et de ses champs}{declarationRegistreEtChamps}

Lors de la déclaration d'un registre, il est possible de préciser la composition de ses champs entiers et booléens. Par exemple, pour le registre \texttt{PORTA\_PCR0} et en s'appuyant par sa description dans la \refFigurePage{}{definitionPORTxPCRn}~:

\begin{PLM}
registers PORTA {
  PCR0 at 0x4004_9000 $uint32 {
    7, ISF, 4, IRQC[4], LK, 4, MUX[3], 1, DSE, ODE, PFE, 1, SRE, PE, PS
  }
}
\end{PLM}

Entre accolades, trois définitions différentes peuvent apparaître~:
\begin{itemize}
\item un nombre indique le nombre de bits consécutifs inutilisés~;
\item un identificateur (par exemple \plm=ISF=) nomme un champ booléen~;
\item un identificateur suivi d'un nombre entre crochets (par exemple \plm=IRQC[4]=) nomme un champ entier constitué du nombre indiqué de bits consécutifs.
\end{itemize}

La description commence par le bit le plus significatif~: comme le type du registre est \plm+$uint32+ (entier non signé sur 32 bits), le premier bit nommé \texttt{ISF} porte le n°24, \texttt{IRQC} s'étend sur 4 bits à partir du n°16,~...

Cette écriture n'est autorisée que si le type nommé (ici \plm+$uint32+) est une type entier non signé. Les types signés (\plm+$int32+, ...) sont interdits. Le compilateur vérifie que la description des champs définit exactement le nombre de bits du type nommé, ici les 32 bits du type \plm+$uint32+.

Définir la composition des champs d'un registre permet d'utiliser des constructions qui simplifient~:
\begin{itemize}
  \item l'obtention de leur valeur (\refSubsectionPage{accesValeurChamps})~;
  \item la construction d'une valeur à affecter à un registre de contrôle (\refSubsectionPage{constructionChampEntierRegistre}).
\end{itemize}










\subsectionLabel{Accès en lecture aux champs}{accesValeurChamps}

À la \refSection{simpleDeclarationRegistre}, pour obtenir la valeur du champ \texttt{MUX} est activé, on réalisait un décalage suivi d'un masquage~:
\begin{PLM}
let champMUX $uint32 = (PORTA.PCR0 >> 8) & 7 // 0, 1, 2, ..., 7
\end{PLM}

Plusieurs formulations nommant le champ \texttt{MUX} sont possibles.

La première renvoie la valeur du champ non décalée~:

\begin{PLM}
let résultatNonDécalé $uint32 = PORTA.PCR0.MUX
  // 0, 0x100, 0x200, ..., 0x700
\end{PLM}

Pour obtenir la valeur d'un champ justifiée à droite, on utilise l'accesseur \plm+shifted+~:
\begin{PLM}
let champMUX $uint32 = PORTA.PCR0.MUX.shifted // 0, 1, 2, ..., 7
\end{PLM}

L'expression \plm+PORTA.PCR0.MUX.shifted+ est équivalente à \plm+(PORTA.PCR0 >> 8) & 7+.


Pour le champ booléen \texttt{ODE}, on écrivait à la \refSection{simpleDeclarationRegistre}~:

\begin{PLM}
let ODEactivé $bool = (PORTA.PCR0 & (1 << 5)) ≠ 0
\end{PLM}

On peut maintenant écrire (noter que le type du résultat est \plm=$uint32=)~:
\begin{PLM}
let champODEnonDécalé $uint32 = PORTA.PCR0.ODE // 0 ou 2**5
\end{PLM}

De même, on peut obtenir la valeur justifiée à droite (noter que le type du résultat est toujours \plm=$uint32=)~:
\begin{PLM}
let champODEdécalé $uint32 = PORTA.PCR0.ODE.shifted // 0 ou 1
\end{PLM}

Pour obtenir la valeur valeur booléenne, on utilise l'accesseur \plm=bool= (noter que le type du résultat est maintenant \plm=$bool=)~:
\begin{PLM}
let ODEactivé $bool = PORTA.PCR0.ODE.bool // no ou yes
\end{PLM}

L'expression \plm+PORTA.PCR0.ODE.bool+ est équivalente à \plm+(PORTA.PCR0 & (1 << 5)) ≠ 0+.


















\subsectionLabel{Construction à partir des valeurs de champs d'un registre de contrôle}{constructionChampEntierRegistre}

La construction particulière \plm+$registre {champ:expression, ...}+ permet de définir facilement la valeur à affecter à un registre de contrôle. Prenons toujours l'exemple du registre \texttt{PORTA\_PCR0} dont la composition est décrite à la \refFigurePage{}{definitionPORTxPCRn}.


À la \refSection{simpleDeclarationRegistre}, pour écrire $1$ dans le champ \texttt{MUX} et zéro dans les autres champs on écrivait~:

\begin{PLM}
PORTA.PCR0 = 1 << 8
\end{PLM}

En utilisant la notation dédiée, on écrit maintenant~: 

\begin{PLM}
PORTA.PCR0 = {PORTA.PCR0 !MUX:1}
\end{PLM}

L'expression \plm+{PORTA.PCR0 !MUX:1}+ est équivalente à \plm+1 << 8+.

Si l'on veut que ce port soit un \emph{collecteur ouvert} si il est programmé en sortie, il faut mettre le champ \texttt{ODE} à $1$, et on écrivait~:
\begin{PLM}
PORTA.PCR0 = (1 << 8) | (1 << 5)
\end{PLM}

On peut maintenant écrire~:

\begin{PLM}
PORTA.PCR0 = {PORTA.PCR0 !MUX:1 !ODE:1}
\end{PLM}


\subsection{Vérifications sémantiques}
 
Examinons maintenant les conditions de validité de l'\plm=expression= dans la construction (décrite à la \refSubsectionPage{constructionChampEntierRegistre}) \plm+$registre {champ:expression, ...}+.

{\bf Expression entière statique.} Le compilateur vérifie qu'elle est comprise entre $0$ et $2^n-1$, $n$ étant le nombre de bits du champ~: par exemple, pour le champ \texttt{MUX} de $3$ bits, une valeur entre $0$ et $7$. Ainsi~:

\begin{PLM}
PORTA.PCR0 = {PORTA.PCR0 !MUX:1-2} // Erreur de compilation, exp. < 0
PORTA.PCR0 = {PORTA.PCR0 !MUX:8} // Erreur de compilation, exp. > 7
\end{PLM}

{\bf Expression entière non statique signée.} Le compilateur considère que c'est une erreur~: uniquement une expression entière non statique non signée est acceptable.


{\bf Expression entière non statique non signée.} Il y a plusieurs sous cas à examiner.

Si l'\plm=expression= est d'un type entier non signé dont le nombre de bits est inférieur ou égal au nombre de bits du champ, alors toute valeur de l'\plm+expression+ est acceptable~: le code engendré se borne à faire le décalage à gauche de la valeur de l'\plm+expression+.

Par exemple, le champ \texttt{MUX} s'étendant sur $3$ bits, une expression de type \plm=$uint3=, \plm=$uint2= ou \plm=$uint1= est toujours acceptée~:

\begin{PLM}
let x $uint2 = 1
PORTA.PCR0 = {PORTA.PCR0 !MUX:x} // Ok
\end{PLM}

Dans le cas contraire, c'est-à-dire si l'\plm=expression= est d'un type entier non signé dont le nombre de bits est strictement supérieur au nombre de bits du champ, une vérification de la valeur à l'exécution est effectuée~: si la valeur de l'\plm+expression+ est trop grande, la panique (code~: voir \refTableauPage{tableauCodePanique}) est déclenchée. Si la génération de code panique n'est pas activée, le débordement est silencieusement ignoré. Par exemple~:

\begin{PLM}
let x $uint8 = ... 
PORTA.PCR0 = {PORTA.PCR0 !MUX:x} // Vérification à l'exécution
\end{PLM}

Si le code ci-dessus apparaît dans une routine où la génération de panique est interdite (par exemple, dans une routine \plm=boot=), alors il déclenche une erreur de compilation. Il faut ajouter une troncature explicite pour le code soit accepté~:

\begin{PLM}
let x $uint8 = ... 
PORTA.PCR0 = {PORTA.PCR0 !MUX:truncate $uint3 (x)}
\end{PLM}

Ce code n'effectue aucune vérification à l'exécution.
















\sectionLabel{Déclaration de plusieurs registres}{declarationPlusieursRegistres}

Il est possible de regrouper les déclarations de registres partageant la même décomposition de leur champs. Par exemple, pour les registres \texttt{PORT$n$\_PCR$m$} du \texttt{mk20dx256} (seule la déclaration de deux premiers registres est montrée)~:

\begin{PLM}
registers PORTA {
  PCR0 at 0x4004_9000
  PCR1 at 0x4004_9004
  $uint32 {
    7, ISF, 4, IRQC[4], LK, 4, MUX[3], 1, DSE, ODE, PFE, 1, SRE, PE, PS
  }
}
\end{PLM}











\sectionLabel{Déclaration d'un tableau de registres}{declarationTableauRegistres}

Le micro-contrôleur \texttt{LPC2294} de NXP possède 4 modules CAN.

\fbox{\begin{minipage}{1.0\textwidth}
  La documentation du \texttt{LPC2294} numérote ces modules de $1$ à $4$. Dans ce document, ils sont numérotés de $0$ à $3$, ce qui s'avère beaucoup plus pratique à l'usage.
\end{minipage}}

Les registres de ces modules sont aux adresses~:

\texttt{0xE004\_4000 + (canal << 14) + register\_offset}

où \texttt{canal} vaut $0$ pour le module $0$, …, $3$ pour le module 3~; \texttt{register\_offset} est une valeur propre à chaque type de registre. Par exemple, pour les registres \texttt{CANCMR}, l'offset est égal à 4. Les quatre registres \texttt{CANCMR} sont donc aux adresses~:

\texttt{CANCMR0} : \texttt{0xE004\_4000 + (0 << 14) + 4 = 0xE004\_4004}

\texttt{CANCMR1} : \texttt{0xE004\_4000 + (1 << 14) + 4 = 0xE004\_8004}

\texttt{CANCMR2} : \texttt{0xE004\_4000 + (2 << 14) + 4 = 0xE004\_C004}

\texttt{CANCMR3} : \texttt{0xE004\_4000 + (3 << 14) + 4 = 0xE005\_0004}

Il est possible de déclarer ces registres individuellement~:

\begin{PLM}
register
  CANCMR0 at 0xE004_4004
  CANCMR1 at 0xE004_8004
  CANCMR2 at 0xE004_C004
  CANCMR3 at 0xE005_0004
$uint32 {
  STB3, STB2, STB1, SRR, CDO, RRB, AT, TR
}
\end{PLM}

Mais on n'a pas de solution simple pour sélectionner un de ces registres en fonction du numéro de module. Si on veut écrire une valeur \plm=v= dans le registre désigné par la variable \plm=n= (dont la valeur est comprise entre $0$ et $3$), il faut écrire~:

\begin{PLM}
if n == 0 {
  CANCMR0 = v
}else if n == 1 {
  CANCMR1 = v
}else if n == 2 {
  CANCMR2 = v
}else{
  CANCMR3 = v
}
\end{PLM}

On peut simplifier l'accès en déclarant les registres \texttt{CANCMR$n$} comme un tableau de registres de contrôle~:

\begin{PLM}
register
  CANCMR[4] at 0xE004_4004 : 1 << 14
$uint8 {
  STB3, STB2, STB1, SRR, CDO, RRB, AT, TR
}
\end{PLM}

D'une manière générale, la déclaration d'un tableau de registres de contrôle est de la forme~:
\begin{PLM}
nom_registre [taille] at adresse_base~: multiplicateur
$type { ... }
\end{PLM}

La \plm=taille= doit toujours être égale à une puissance de $2$. L'adresse du registre d'indice $i$ est égale à \texttt{adresse\_base + $i$ * multiplicateur}. Ici~:

Adresse de \texttt{CANCMR[0]} : \texttt{0xE004\_4004 + 0 * (1 << 14) = 0xE004\_4004}

Adresse de \texttt{CANCMR[1]} : \texttt{0xE004\_4004 + 1 * (1 << 14) = 0xE004\_8004}

Adresse de \texttt{CANCMR[2]} : \texttt{0xE004\_4004 + 2 * (1 << 14) = 0xE004\_C004}

Adresse de \texttt{CANCMR[3]} : \texttt{0xE004\_4004 + 3 * (1 << 14) = 0xE005\_0004}

Si vous voulez confirmer le calcul des adresses des registres de contrôle, utilisez l'option de la ligne de commande \OPTION{-{}-control-register-map} (\refSectionPage{optionsDebogage}) qui affiche le détail de la définition des registres de contrôle dans un fichier HTML.
 
L'accès aux registres de contrôle s'effectue alors en utilisant la notation \plm=[...]= habituelle de l'accès à un élément de tableau. Par exemple, en reprenant l'exemple précédent, écrire une valeur \plm=v= dans le registre désigné par la variable \plm=n= (dont la valeur est comprise entre $0$ et $3$) s'exprime simplement par~:

\begin{PLM}
CANCMR[n] = v
\end{PLM}

L'indice d'un tableau de registre peut-être une expression entière statique, ou une expression dynamique signée ou non signée. Les vérifications à la compilation et à l'exécution sont les même que pour l'accès à un élément de tableau (voir la \refSectionPage{AccesElementTableau}).

En particulier, si l'indice est une expression de type non signée dont la valeur maximum est strictement inférieure à la taille du tableau, aucune vérification n'est faite à l'exécution, puisque l'indice sera toujours valide~: 

\begin{PLM}
let n $uint2 = ...
CANCMR[n] = v // Aucune vérification, indice toujours valide
\end{PLM}







\sectionLabel{Attributs d'un registre de contrôle}{attributsRegistreControle}

\subsectionLabel{Attribut \texttt{@ro}}{attributRo}\index{"@ro}
La déclaration d'un registre accepte l'attribut \plm+@ro+, qui signifie qu'il est en lecture seule. Par exemple~:
\begin{PLM}
registers {
   CALIB @ro at 0xE000_E01C $uint32
}
\end{PLM}

Toute tentative de faire figurer ce registre dans une construction qui provoque une écriture de celui-ci entraîne l'apparition d'une erreur de compilation.






\subsectionLabel{Attribut \texttt{@user}}{attributUser}\index{"@user}
La déclaration d'un registre accepte l'attribut \plm+@user+, qui signifie qu'il est accessible en mode \plm!user!. Par défaut, un registre de contrôle n'est pas accessible en mode  \plm!user!. Par exemple~:
\begin{PLM}
registers GPIOE {
  PSOR @user at 0x400F_F104 $uint32
}
\end{PLM}

Évidemment, il faut que le matériel accepte effectivement que le registre soit accessible quand le processeur en mode \emph{utilisateur}.









\sectionLabel{Restrictions d'usage des registres}{restrictionsUsageRegistres}\index{Parametre effectif@Paramètre effectif!Registre}

Un registre ne peut pas~:
\begin{itemize}
  \item apparaître comme paramètre effectif en entrée d'une procédure~;
  \item apparaître comme paramètre effectif en sortie/entrée d'une procédure.
\end{itemize}

Prenons un exemple~; la procédure \plm=uneProcedure= présente un argument formel en sortie, et on suppose que \texttt{REGISTRE} est un registre de type \plm!$uint32! :
\begin{PLM}
func user uneProcedure (!outValue $uint32) {
  outValue = 5
}
\end{PLM}



L'écriture suivante est rejetée par le compilateur (passage d'un registre comme paramètre effectif en entrée)~:
\begin{PLM}
func user autreProcedure () {
  uneProcedure (?REGISTRE) // Erreur
}
\end{PLM}

Par contre, l'écriture suivante est correcte (écriture du registre)~:
\begin{PLM}
func user autreProcedure () {
  REGISTRE = 5 // Ok
}
\end{PLM}


%!TEX encoding = UTF-8 Unicode
%!TEX root = ../doc-omnibus.tex





\chapter{Déclaration des constantes globales}

Les constantes peuvent être déclarées en deux endroits :
\begin{itemize}
  \item en dehors de toute routine : c'est une constante globale (voir ci-après) ;
  \item parmi les instructions d'une routine : c'est une constante locale à la routine (voir \refSectionPage{declarationConstanteLocale}).
\end{itemize}




\sectionLabel{Déclaration d'une constante globale}{declarationConstanteGlobale}\index{Constante!globale}

La déclaration d'une constante globale est la suivante :

\begin{OMNIBUS}
let nom : Type = expression_statique
\end{OMNIBUS}

Où :
\begin{itemize}
  \item \omnibus=nom= est le nom de la constante globale ;
  \item \omnibus=Type= est le nom du type de la constante globale ;
  \item \omnibus=expression_statique= est l'expression qui fournit la valeur de cette constante~; cette expression est calculée lors de la compilation.
\end{itemize}

La portée d'une constante globale est le programme dans son intégralité : peut importe le fichier et la ligne où elle est déclarée.

L'\omnibus=expression_statique= ne peut pas nommer une autre constante globale, ni une variable globale.


%!TEX encoding = UTF-8 Unicode
%!TEX root = ../doc-plm.tex





\chapter{Variables globales}

Les variables peuvent être déclarées en deux endroits :
\begin{itemize}
  \item en dehors de toute routine : c'est une variable globale (voir ci-après) ;
  \item parmi les instructions d'une routine : c'est une variable locale à la routine (voir \refSectionPage{declarationVariableLocale}).
\end{itemize}





\sectionLabel{Déclaration d'une variable globale}{declarationVariableGlobale}\index{Variable!globale}

La déclaration d'une variable globale est la suivante :

\begin{PLM}
var nom : Type = expression_statique {
  en_tetes_routines_autorisees
}
\end{PLM}

Où :
\begin{itemize}
  \item \plm=nom= est le nom de la variable globale ;
  \item \plm=Type= est le nom du type de la variable globale ;
  \item \plm=expression_statique= est l'expression qui fournit la valeur initiale de cette variable ; cette expression est calculée lors de la compilation ;
  \item \plm=en_tetes_routines_autorisees= est la liste des routines qui sont autorisées à accéder à la variable globale (\refSectionPage{DecRoutinesAutoriseesVarGlobale}) ; par défaut, l'accès est autorisé en lecture seule ; pour avoir l'autorisation d'écrire la variable, l'attribut \plm=@rw=\index{"@rw} doit être placé avant la routine.
\end{itemize}

La portée d'une variable globale est le programme dans son intégralité : peu importe où est déclarée la variable. Toutefois, seules les routines explicitement autorisées dans peuvent y accéder.

L'\plm=expression_statique= peut utiliser la valeur des constantes globales : le compilateur évalue les constantes globales avant d'évaluer les valeurs initiales des variables globales. Par contre, \plm=expression_statique= ne peut pas utiliser les valeurs d'une autre variable globale.

Par exemple, voici comment on peut implémenter une variable globale \texttt{gCompteur}, incrémentée par la routine d'interruption \texttt{timerHandler}, et écrire un service d'attente de délai \texttt{wait} avec une boucle d'attente active :

\begin{PLM}
var gCompteur $uint32 = 0 {
  @rw func timerHandler // Ne pas faire figurer
  func waitMS           // la liste des arguments
}

func timerHandler `isr () {
  gCompteur +%= 1
}

func wait `user (?inDuration $uint32) {
  let deadline = gCompteur + inDuration
  while gCompteur < deadline do
  end
}
\end{PLM}

Note : si une variable globale est accédée par des routines appartenant à plusieurs modes, le compilateur ajoute le qualificatif \texttt{volatile}\index{volatile@\texttt{volatile}} dans le code engendré pour déclarer cette variable. C'est le cas de l'exemple ci-dessus, la variable \texttt{gCompteur} pouvant être accédée dans les modes \plm=`isr= et \plm=`user=.


Les routines d'initialisation (\refSectionPage{initRoutine})\index{init@\plm=init=!Routine} et les routines de panique (\refSubsectionPage{routinesPanique})\index{Routine!panique \plm=panic=}
 doivent de même être déclarées pour pouvoir accéder à une variable globale (et avec l'attribut \plm=@rw= si on veut l'accès en écriture). Par exemple :
\begin{PLM}
var gCompteur $uint32 = 0 {
  @rw init 112
  @rw func panic loop 179
}

func init 112 {
  gCompteur = 10
}

func panic loop 179 {
  gCompteur = 100
}
\end{PLM}


\sectionLabel{Déclarations des routines autorisées}{DecRoutinesAutoriseesVarGlobale}

Dans la déclaration d'une variable globale, \plm=en_tetes_routines_autorisees= liste l'ensemble des routines pouvant accédéer à la variable globale. Cette liste doit être non vide. On peut ainsi autoriser l'accès à :
\begin{itemize}
  \item une routine d'initialisation (\plm!init!) ;
  \item une fonction (\plm!func!) ;
  \item une section (\plm!section!) ;
  \item une routine de panique (\plm!func panic!) ;
  \item une fonction de type (\plm!func!).
\end{itemize}

Le \refTableauPage{DecRoutinesAutoriseesAccesVarGlobale} résume les déclarations des routines autorisées. Par défaut, seul l'accès en lecture est permis. L'attribut \plm!@rw! permet d'autoriser pour une routine l'accès en écriture, sauf pour les fonctions ou cet attribut est interdit.

Le compilateur interdit l'accès des routines \plm+boot+ aux variables globales\footnote{En effet, comme les routines \texttt{boot} sont exécutées avant l'initialisation des variables globales, le comportement à l'exécution serait indéfini.}.

\begin{table}[t]
\centering
\begin{tabular}{lp{5cm}l}
  \textbf{Déclaration} & \textbf{Routine autorisée} & \textbf{Accés en écriture} \\
  \plm=init priorite= & Routine d'initialisation de priorité \plm=priorite= & Nécessite \plm!@rw! \\
  \plm=func nom= & Fonction \plm!nom! & Jamais \\
  \plm=section nom= & Section \plm!nom! & Nécessite \plm!@rw! \\
  \plm=func panic nom prorite= & Routine de panique \plm!nom! de priorité \plm=priorite= & Nécessite \plm!@rw! \\
  \plm=func $type nom= & Procédure \plm!nom! du type \plm!$type! & Nécessite \plm!@rw! \\
  \plm=func $type nom= & Fonction \plm!nom! du type \plm!$type! & Jamais \\
\end{tabular}
\caption{Déclaration des routines autorisées}
\labelTableau{DecRoutinesAutoriseesAccesVarGlobale}
\ligne
\end{table}













\section{Restrictions d'usage des variables globales}
\index{Parametre effectif@Paramètre effectif!Variable globale}

Une variable globale ne peut pas :
\begin{itemize}
  \item apparaître comme paramètre effectif en entrée d'une procédure ;
  \item apparaître comme paramètre effectif en sortie/entrée d'une procédure.
\end{itemize}

Prenons un exemple ; la procédure \texttt{uneProcedure} présente un argument formel en sortie, et la variable globale \texttt{gGlobale} peut être accédée en lecture et écriture par la procédure \texttt{autreProcedure} :
\begin{PLM}
func uneProcedure `user (!outValue $uint32) {
  outValue = 5
}

var gGlobale $uint32 {
  @rw func autreProcedure
}
\end{PLM}

L'écriture suivante est correcte (écriture de la variable globale) :
\begin{PLM}
func autreProcedure `user () {
  gGlobale = 10 // Ok
}
\end{PLM}


Par contre, l'écriture suivante est rejetée par le compilateur (passage d'une variable globale comme paramètre effectif en entrée) :
\begin{PLM}
func autreProcedure `user () {
  uneProcedure (?gGlobale) // Erreur, variable globale comme
                           // paramètre effectif en entrée
}
\end{PLM}


%!TEX encoding = UTF-8 Unicode
%!TEX root = ../doc-plm.tex





\chapter{Procédures}\index{Procedure@Procédure}

Le langage définit plusieurs natures de sous-programmes :
\begin{itemize}
  \item les \emph{procédures}, dont l'appel est une instruction ;
  \item les \emph{fonctions}, dont l'appel apparaît dans une expression ;
  \item les \emph{boot routines}, exécutées une fois avant l'initialisation des variables globales (\refSectionPage{bootRoutine}) ;
  \item les \emph{init routines}, exécutées une fois après l'initialisation des variables globales (\refSectionPage{initRoutine}) ;
  \item les \emph{routines d'exception}, exécutées lors d'une exception (\refSectionPage{routineException}).
\end{itemize}

Ce chapitre décrit les \emph{procédures} et les \emph{fonctions}. Les \emph{routines}\index{Routine} se distinguent des \emph{procédures} par le fait qu'elles n'ont pas d'arguments formels explicites, et qu'elles ne peuvent pas être appelées explicitement. Elles sont présentées dans des sections particulières, citées ci-dessus.






\sectionLabel{Déclaration d'une procédure}{declarationProcedure}

La déclaration d'une procédure est la suivante :
\begin{PLM}
proc nom $mode1 $mode2 @attribut1 @attribut2 (arguments_formels) {
  liste_instructions
}
\end{PLM}
Où :
\begin{itemize}
  \item \plm=nom= est le nom de la procédure ;
  \item \plm=$mode1= \plm=$mode2= est la liste non vide de l'ensemble des modes associés à la procédure ;
  \item \plm=@attribut1= \plm=@attribut2= est une liste éventuellement vide d'attributs associés à la procédure.
\end{itemize}

Par exemple :

\begin{PLM}
proc setup $user () {
}
\end{PLM}

Ceci définit la procédure \plm=setup=, sans argument, sans attribut, appelable uniquement en mode \plm=$user=.

\begin{PLM}
proc goto $user @noWarningIfUnused (
  ?line:inLine : UInt32
  ?column:inColumn : UInt8) {
}
\end{PLM}

Ceci définit la procédure \plm=goto=, appelable uniquement en mode \plm=$user=, avec deux arguments formels en entrée. L'attribut \plm+@noWarningIfUnused+ signifie qu'aucune alerte n'est émise si la procédure n'est pas utilisée.






\sectionLabel{Arguments formels, paramètres effectifs, sélecteurs}{argumentFormel}
\index{Argument formel}
\index{Parametre effectif@Paramètre effectif}
\index{Selecteur@Sélecteur}

Il existe trois natures d'arguments formels : \emph{entrée}, \emph{sortie} et \emph{entrée / sortie}, décrits dans le \refTableau{argumentsFormels}. Un sélecteur peut être \emph{anonyme} (par exemple \plm+?+ pour un paramètre formel en entrée), ou comporter un nom (le nom « \texttt{nom} » pour \plm+?nom:+).

Une procédure déclare zéro, un ou plusieurs arguments formels qui peuvent être en \emph{entrée}, en \emph{sortie} ou en \emph{entrée/sortie}. Une fonction déclare zéro, un ou plusieurs arguments formels en \emph{entrée}.

La syntaxe des différents arguments formels et de leur paramètre effectif est résumée dans le \refTableau{argumentsFormelsParametresEffectifs}. 

\begin{table}[t]
  \centering
  \begin{tabular}{lp{6.5cm}l}
    \textbf{Argument formel} & \textbf{Transfert d'information} & \textbf{Sélecteur} \\
    Entrée & Lors de l'appel, de l'appelant vers l'appelé & \plm+?+ ou \plm+?selecteur:+\\
    Sortie & Lors du retour, de l'appelé vers l'appelant & \plm+!+ ou \plm+!selecteur:+\\
    Entrée / sortie & Lors de l'appel, de l'appelant vers l'appelé, et lors du retour, de l'appelé vers l'appelant & \plm+?!+ ou \plm+?!selecteur:+\\
  \end{tabular}
  \caption{Arguments formels}
  \labelTableau{argumentsFormels}
  \ligne
\end{table}



\begin{table}[t]
  \centering
  \begin{tabular}{llll}
    \textbf{Argument formel} & \textbf{Sélecteur} & \textbf{Paramètre effectif} & \textbf{Sélecteur} \\
    Entrée & \plm+?+         & Sortie & \plm+!expression+ \\
           & \plm+?selecteur:+ & & \plm+!selecteur:expression+ \\
    Sortie & \plm+!+         & Entrée & \plm+?variable+ \\
           & \plm+!selecteur:+ & & \plm+?selecteur:variable+ \\
    Entrée/sortie & \plm+?!+         & Sortie/entrée & \plm+!?variable+ \\
           & \plm+?!selecteur:+ & & \plm+!?selecteur:variable+ \\
  \end{tabular}
  \caption{Argument formel et paramètre effectif}
  \labelTableau{argumentsFormelsParametresEffectifs}
  \ligne
\end{table}






\sectionLabel{Attribut \texttt{@noWarningIfUnused}}{attributNoWarningIfUnused}\index{"@noWarningIfUnused}

L'attribut \plm+@noWarningIfUnused+ signifie qu'aucune alerte n'est émise si la procédure n'est pas utilisée.







\sectionLabel{Attribut \texttt{@weak}}{attributWeak}\index{"@weak}

L'attribut \plm+@weak+ signifie que la routine peut être redéfinie par une routine de même nom, avec les mêmes arguments formels, les mêmes modes. Les fichiers de définition de cible utilisent cette possibilité pour définir des routines par défaut.

Par exemple, la cible \texttt{teensy-3-1-sequential-systick} définit cette routine d'interruption (voir le deuxième exemple du tutorial, \refSectionPage{deuxiemeExemple}) :

\begin{PLM}
proc systickHandler $isr @weak () {
}
\end{PLM}

Par défaut, la routine d'interruption \plm+systickHandler+ est donc définie, vide. Le programme d'exemple définit cette routine :

\begin{PLM}
proc systickHandler $isr () {
  gUpTimeMS ++
}
\end{PLM}

Comme cette nouvelle routine a le même nom, les mêmes arguments formels et le même mode que la routine marquée \plm+@weak+, elle est utilisée à la place de celle-ci.






\sectionLabel{Procédures requises}{procedureRequise}

La déclaration \plm+required proc+ permet de signifier au compilateur qu'une procédure doit être définie, soit par la cible, soit par le programme utilisateur.

Cette déclaration est la suivante :
\begin{PLM}
required proc nom $mode1 $mode2 @attribut1 @attribut2 (arguments_formels)
\end{PLM}

Elle consiste à la déclaration de l'en-tête d'une procédure (voir \refSectionPage{declarationProcedure}), précédée par le mot réservée \plm+required+.

Par exemple, la cible \texttt{teensy-3-1-sequential-systick} comporte ces deux déclarations :

\begin{PLM}
required proc setup $user ()
required proc loop $user ()
\end{PLM}

Ceci impose au programme utilisateur de définir ces deux procédures.









\section{Procédures utiles}

Le compilateur élimine les procédures qui ne sont jamais appelées, en calculant le graphe des appels. Les racines de ce graphe sont les procédures requises (\refSectionPage{procedureRequise}). Les procédures inatteignables sont éliminées, avec ou sans message d'alerte (\refSectionPage{attributNoWarningIfUnused}).











\sectionLabel{Récursivité}{routinesRecursives}\index{Routines!recursives@récursives}

Par défaut, le compilateur émet un message d'erreur si une ou plusieurs routines récursives sont détectées. L'option \texttt{-{}-do-not-detect-recursive-calls} (\refSectionPage{optionCodeEngendre}) permet d'inhiber cette recherche.

L'option \texttt{-{}-routine-invocation-graph} permet d'obtenir un fichier contenant le graphe d'invocation, qui peut être affiché par le logiciel \texttt{graphviz}\index{graphviz}. Si le fichier source est \texttt{source.plm}, le fichier engendré s'appelle \texttt{source.subprogramInvocation.dot}.


%!TEX encoding = UTF-8 Unicode
%!TEX root = ../doc-plm.tex

\chapterLabel{Démarrage du micro-contrôleur}{chapitreDemarrageMicro}


\sectionLabel{Séquence de démarrage}{sequenceDemarrage}

La séquence de démarrage du micro-contrôleur est illustrée par la \refFigure{}{sequenceDemarrage}.

La première étape est de configurer les horloges internes du micro-contrôleur : c'est le rôle des routines \plm=boot=. À ce stade, la mémoire vive n'est toujours pas initialisée, aussi les routines \plm=boot= n'y accèdent pas (le compilateur l'assure).

La deuxième étape est d'initialiser les \emph{variables globales}, c'est-à-dire mettre à zéro la zone «~\texttt{.bss.*}~», et de recopier à partir de la flash les valeurs initiales des variables initialisées.

La troisième étape est l'exécution des routines \plm=init=. À partir de cette étape et pour les suivantes, les variables globales sont initialisées, et donc leur emploi est autorisé. Le rôle des routines \plm=init= est de configurer les entrées/sorties du micro-contrôleur.

Ensuite, les tâches sont lancées, et exécutées en fonction de leurs priorités et synchronisations.

\begin{figure}[t]
  \centering
  \small
  \begin{tikzpicture}[
      cloud/.style ={draw=red, thick, ellipse,fill=red!20, minimum height=2em},
      block/.style ={rectangle, draw=blue, thick, fill=green!20, align=center},
      decision/.style={chamfered rectangle, draw=blue, thick, fill=green!20},
      node distance=5mm
    ]
    \node [cloud] (start) {\textsc{Démarrage du micro-contrôleur}} ;
    \node [block] (boot) [below=of start] {Routines \bf\texttt{boot}} ;
    \node [block] (raz) [below=of boot] {Initialisation des variables globales} ;
    \node [block] (init) [below=of raz] {Routines \bf\texttt{init}} ;
    \node [block] (setup) [below=of init] {Démarrage des tâches} ;

    \draw [-stealth, thick] (start) -- (boot) ;
    \draw [-stealth, thick] (boot) -- (raz) ;
    \draw [-stealth, thick] (raz) -- (init) ;
    \draw [-stealth, thick] (init) -- (setup) ;
  \end{tikzpicture}
  \caption{Organigramme d'exécution de la séquence de démarrage}
  \labelFigure{sequenceDemarrage}
  \ligne
\end{figure}


\sectionLabel{\texttt{boot} routines}{bootRoutine}
\index{boot@\plm=boot=!Routine}
\index{Routine!boot@\plm=boot=}

Une routine \plm=boot= est exécutée une et une seule fois, lors du démarrage du micro-contrôleur, avant que les variables globales ne soient initialisées. Elle a la syntaxe suivante :
\begin{PLM}
boot priorité {
  liste_instructions
}
\end{PLM}
Où \plm=priorité= est la priorité de la routine. C'est une constante entière statique. Les routines \plm=boot= sont exécutées dans l'ordre des priorités croissantes. Le compilateur vérifie que deux routines \plm=boot= n'ont pas la même priorité.

Les routines \plm=boot= s'exécutent dans le mode \plm=boot=.\index{boot}

Comme les routines \plm=boot= s'exécutent avant que les variables globales soient initialisées, l'accès aux variables globales y est interdit. L'interdiction est mise en place de la façon suivante : il n'est pas possible d'associer une routine \plm=boot= à une variable globale lors de sa déclaration (\refSectionPage{declarationVariableGlobale}).

Par contre, l'accès aux registres de contrôle est autorisé dans une routine \plm=boot= (d'ailleurs, elle sert à cela : configurer le micro-contrôleur au démarrage).







\sectionLabel{\texttt{init} routines}{initRoutine}
\index{init@\plm=init=!Routine}
\index{Routine!init@\plm=init=}

Une routine \plm=init= est exécutée une et une seule fois, lors du démarrage du micro-contrôleur, après l'initialisation des variables globales. Elle a la syntaxe suivante :
\begin{PLM}
init priorité {
  liste_instructions
}
\end{PLM}
Où \plm=priorité= est la priorité de la routine. C'est une constante entière statique. Les routines \plm=init= sont exécutées dans l'ordre des priorités croissantes. Le compilateur vérifie que deux routines \plm=init= n'ont pas la même priorité.

Les routines \plm=init= s'exécutent dans le mode \plm=init=.\index{init}

Comme les routines \plm=init= s'exécutent après l'initialisation des variables globales, l'accès aux variables globales y est autorisé, du moment que la variable globale cite le mode \plm=init= dans sa déclaration (\refSectionPage{declarationVariableGlobale}).



















%!TEX encoding = UTF-8 Unicode
%!TEX root = ../doc-omnibus.tex





\chapter{Expressions}


\begin{table}[htbp]
\centering
\begin{tabular}{llll}
  \textbf{Priorité} & \textbf{Opérateur} & \textbf{Commentaire}\\
   0 & \omnibus+-+, \omnibus+-%+ & \emph{moins} unaire \\
   0 & \omnibus+~+, \omnibus+not+ & \emph{complémentation} binaire et \emph{non} logique \\
   1 & \omnibus=convert= & Conversion \\
   2 & \omnibus+*+, \omnibus+*%+, \omnibus+/+, \omnibus+!/+, \omnibus-%-, \omnibus-!%- & Multiplication, division, modulo \\
   3 & \omnibus-+-, \omnibus-+%-, \omnibus+-+, \omnibus+-%+ & Addition, soustraction \\
   4 & \omnibus+<<+, \omnibus+>>+ & Décalage à gauche et à droite \\
   5 & \omnibus+≤+, \omnibus+<+, \omnibus+>=+, \omnibus+>+ & Comparaison \\
   6 & \omnibus+==+, \omnibus+≠+ & Test d'égalité, d'inégalité \\
   7 & \omnibus+&+ & \emph{et} binaire \\
   8 & \omnibus+^+ & \emph{ou exclusif} binaire \\
   9 & \omnibus+|+ & \emph{ou} binaire \\
   10 & \omnibus+and+ & \emph{et} logique \\
   11 & \omnibus+xor+ & \emph{ou exclusif} logique \\
   12 & \omnibus+or+ & \emph{ou} logique \\
\end{tabular}
\caption{Priorité des opérateurs}\index{Operateur@Opérateur!Priorite@Priorité}
\labelTableau{tableauPrioriteOperateurs}
\end{table}



\section{Opérateur $\sim$}

L'opérateur $\sim$ renvoie la complémentation bit-à-bit d'un entier non signé. Une erreur de compilation est déclenchée si l'opérateur est appliqué à un entier signé~:
\begin{OMNIBUS}
let x Int8 = 3
let y = ~ x // Erreur, x est signé
\end{OMNIBUS}

Le nombre de bits complémentés dépend du nombre de bits du type entier non signé~:
\begin{OMNIBUS}
let x UInt8 = 1
let y = ~ x // y est égal à 0xFE
let z UInt16 = 1
let t = ~ z // t est égal à 0xFFFE
\end{OMNIBUS}

L'opérateur $\sim$ ne peut s'appliquer à une constante entière statique uniquement si le type du résultat peut être inféré, et que ce type est un entier non signé~:
\begin{OMNIBUS}
let x = ~ 1 // Erreur, le type du résultat ne peut pas être inféré
let y Int8 = ~ 1 // Erreur, le type inféré est signé
let z UInt8 = ~ 1 // Ok, z = 0xFE
let t UInt16 = ~ 1 // Ok, z = 0xFFFE
\end{OMNIBUS}





\section{Expression \texttt{if}}

\begin{OMNIBUS}
let x = if expression_1 { expression_2 } else { expression_3 }
\end{OMNIBUS}

L'expression \omnibus=if= fonctionne comme suit~:
\begin{itemize}
  \item l'\omnibus=expression_1= est une expression booléenne~;
  \item si l'\omnibus=expression_1= est vraie, l'\omnibus=expression_2= est calculée et sa valeur est celle renvoyée par l'expression \omnibus=if=~;
  \item si l'\omnibus=expression_1= est fausse, l'\omnibus=expression_3= est calculée et sa valeur est celle renvoyée par l'expression \omnibus=if=.
\end{itemize}

Les expressions \omnibus=if= peuvent se succéder, dans tous les cas il faut terminer par une clause \omnibus=else=~:

\begin{OMNIBUS}
let x =
  if expression_1 {
    expression_2
  }else if expression_3 {
    expression_4
  }else{
    expression_5
  }
\end{OMNIBUS}



\section{Expression \texttt{addressof}}


L'expression \omnibus=addressof= permet d'obtenir l'adresse de toute \emph{lvalue}. La valeur retournée a pour type l'entier non signé de la taille d'un pointeur (sur Cortex, c'est donc \omnibus=UInt32=).

Par exemple~:
\begin{OMNIBUS}
var x = yes // x est booléen
let adresse UInt32 = addressof (x)
let adresse_registre_controle UInt32 = addressof (PORTA_PCR0)
\end{OMNIBUS}




\section{Expression \texttt{sizeof}}


L'expression \omnibus=sizeof= permet d'obtenir la taille (en nombre d'octets) de toute \emph{lvalue}, ou de tout type. La valeur retournée a pour type l'entier non signé de la taille d'un pointeur (sur Cortex, c'est donc \omnibus=UInt32=).

Par exemple~:
\begin{OMNIBUS}
var x = yes // x est booléen
let s1 UInt32 = sizeof (x) // 1
let s2 UInt32 = sizeof (Bool) // 1
\end{OMNIBUS}


%!TEX encoding = UTF-8 Unicode
%!TEX root = ../doc-omnibus.tex





\chapter{Instructions}





\sectionLabel{Déclaration d'une variable locale}{declarationVariableLocale}\index{Variable!locale}

La déclaration d'une variable locale peut prendre plusieurs formes, suivant que la variable est initialisée ou non.

\subsection{Déclaration d'une variable locale initialisée}

\begin{OMNIBUS}
var nom : Type = expression
\end{OMNIBUS}

Où :
\begin{itemize}
  \item \omnibus=nom= est le nom de la variable locale ;
  \item \omnibus=Type= est le nom du type de la variable globale ;
  \item \omnibus=expression= est l'expression qui fournit la valeur initiale de cette variable ; cette expression peut être statique ou non.
\end{itemize}

L'annotation de type peut être omise ; la variable a alors le type de l'expression qui l'initialise~:
\begin{OMNIBUS}
var nom = expression
\end{OMNIBUS}


\subsection{Déclaration d'une variable locale non initialisée}
\begin{OMNIBUS}
var nom : Type
\end{OMNIBUS}
Où~:
\begin{itemize}
  \item \omnibus=nom= est le nom de la variable locale ;
  \item \omnibus=Type= est le nom du type de la variable globale.
\end{itemize}

Le compilateur garantit qu'aucune lecture n'est faite avant que la variable reçoive une valeur.










\sectionLabel{Déclaration d'une constante locale}{declarationConstanteLocale}\index{Constante!locale}

La déclaration d'une constante locale apparaît dans une liste d'instructions et sa syntaxe est la suivante~:

\begin{OMNIBUS}
let nom : Type = expression
\end{OMNIBUS}

Où~:
\begin{itemize}
  \item \omnibus=nom= est le nom de la constante globale ;
  \item \omnibus=Type= est le nom du type de la constante globale ;
  \item \omnibus=expression= est l'expression qui fournit la valeur de cette constante ; cette expression est soit calculable statiquement, soit à l'exécution.
\end{itemize}

Il n'y a aucune restriction pour l'\omnibus=expression=~: elle peut nommer constantes locales et globales, variables locales et globales.






\section {Instruction d'affectation}

L'instruction d'affectation se présente sous plusieurs formes ; la plus simple est la suivante~:

\begin{OMNIBUS}
nom = expression
\end{OMNIBUS}

Où~:
\begin{itemize}
  \item \omnibus=nom= est le nom d'une variable globale, d'une variable locale, ou d'un registre de contrôle ;
  \item \omnibus=expression= est l'expression qui fournit la valeur.
\end{itemize}

On peut aussi accéder à une propriété d'une variable instance de structure~:
\begin{OMNIBUS}
nom.propriété = expression
\end{OMNIBUS}

Ainsi qu'à un élément de tableau~:
\begin{OMNIBUS}
nom [expression_indice] = expression
\end{OMNIBUS}

Ou encore toute combinaison de propriétés et d'éléments de tableau.





\sectionLabel {Opérateurs combinés avec l'affectation}{operateursCombinesAffectation}
\index{\&=!Entier}
\index{\textbar=!Entier}
\index{\^{}=!Entier}

Le \refTableau{tableauOperateursCombinesAffectation} liste les opérateurs combinés avec une affectation.
\begin{table}[ht]
\centering
\begin{tabular}{rcl}
  \textbf{Opérateur combiné} & \textbf{Écriture équivalente} & \textbf{Lien}\\
  \omnibus!a &= b! & \omnibus!a = a & b! & \refSectionPage{operateursCombinesAffectationEntier}\\
  \omnibus!a |= b! & \omnibus!a = a | b! & \refSectionPage{operateursCombinesAffectationEntier}\\
  \omnibus!a ^= b! & \omnibus!a = a ^ b! & \refSectionPage{operateursCombinesAffectationEntier}\\
  \omnibus!a += b! & \omnibus!a = a + b! \\
  \omnibus!a +%= b! & \omnibus!a = a +% b! \\
  \omnibus!a -= b! & \omnibus!a = a - b! \\
  \omnibus!a -%= b! & \omnibus!a = a -% b! \\
  \omnibus!a *= b! & \omnibus!a = a * b! \\
  \omnibus!a *%= b! & \omnibus!a = a *% b! \\
  \omnibus-a /= b- & \omnibus-a = a / b- \\
  \omnibus-a !/= b- & \omnibus-a = a !/ b- \\
  \omnibus-a %= b- & \omnibus-a = a % b- \\
  \omnibus-a !%= b- & \omnibus-a = a !% b- \\
  \omnibus-a <<= b- & \omnibus-a = a << b- \\
  \omnibus-a >>= b- & \omnibus-a = a >> b- \\
\end{tabular}
\caption{Opérateurs combinés avec l'affectation}
\labelTableau{tableauOperateursCombinesAffectation}
\end{table}









\sectionLabel{Instruction d'affectation «~Bit banding~»}{bitbandRegistreControle}

La technique de «~Bit banding~» est implémentée sur certains processeurs pour pouvoir mettre à 0 ou à 1 un bit d'un registre, et ce de manière atomique. C'est par exemple le cas des Cortex-M4.

Considérons l'instruction suivante~:
\begin{OMNIBUS}
GPIOB.PDDR |= 1 << 16
\end{OMNIBUS}
Son exécution met à 1 le bit n°16 du registre \omnibus+GPIOB.PDDR+. Mais cette opération n'est pas atomique sur un Cortex-M4. Si elle apparaît dans une fonction s'exécutant en mode utilisateur, elle peut être interrompue.

Pour la rendre atomique, on pourrait l'enfermer dans une \omnibus=section=. En OMNIBUS, une autre possibilité est d'utiliser l'instruction de «~Bit banding~»~:
\begin{OMNIBUS}
\~{GPIOB.PDDR : 16} = 1
\end{OMNIBUS}

La forme générale de cette instruction est~:
\begin{OMNIBUS}
\~{registre_contrôle : numéro_bit} = expression_source
\end{OMNIBUS}

OMNIBUS limite le bit-banding à une zone de registres, aussi le premier argument \omnibus=registre_contrôle= doit nommer un registre de contrôle. L'argument \omnibus=numéro_bit= désigne le bit du registre par son numéro, c'est une expression du type \omnibus=UInt5= pour un registre 32-bits, \omnibus=UInt4= pour un registre 16-bits et \omnibus=UInt3= pour un registre 8-bits.
L'\omnibus=expression_source= est de type \omnibus=UInt1=, et donne la valeur à affecter au bit du registre.






\section{Instruction de décomposition d'un entier non signé en tranches}

Cette forme particulière d'affectation permet de décomposer une valeur entière non signée en tranches. Par exemple~:
\begin{OMNIBUS}
\~{UInt8 ?1:var b1 ?2:let b2 ?5:let b3} = 0xCC // b1 <- 1, b3 <- 2, b3 <- 12
\end{OMNIBUS}

\sectionLabel{Instruction \texttt{check}}{directiveCheck}\index{check@\omnibus=check=}

La directive \omnibus=check= apparaît dans une liste d'instructions et a la syntaxe suivante :
\begin{OMNIBUS}
check expression
\end{OMNIBUS}

L'\omnibus=expression= est une expression booléenne calculée statiquement.

Contrairement à l'instruction \omnibus=assert= (\refSectionPage{instructionAssert}) qui évalue l'expression booléenne à l'exécution, la directive \omnibus=check= est toujours évaluée à la compilation. Elle permet d'exprimer des assertions qui sont évaluées lors de la compilation.

Aucun code n'est engendré. La directive \omnibus=check= peut donc apparaître dans des listes d'instructions où la panique est interdite.

Exemples :
\begin{OMNIBUS}
check yes  // Ok
check no // Erreur, expression fausse
\end{OMNIBUS}



\sectionLabel{L'instruction \texttt{assert}}{instructionAssert}\index{assert@\omnibus=assert=}

L'instruction \omnibus=assert= a la syntaxe suivante :
\begin{OMNIBUS}
assert expression
\end{OMNIBUS}

L'\omnibus=expression= est une expression booléenne non calculable statiquement.

Si le programme est compilé avec la panique activée, alors le compilateur engendre le code de calcul de l'expression booléenne. Celle-ci sera calculée à l'exécution. Si le résultat est faux, une panique (dont le code est donné par le \refTableau{tableauCodePanique}) est déclenchée.

Si le programme est compilé avec l'option \texttt{-{}-no-panic-generation}, alors aucun code n'est engendré.

Noter que \omnibus=expression= ne doit pas être calculable statiquement. Si elle est calculable statiquement, il faut utiliser la directive \omnibus=check=, \refSectionPage{directiveCheck}. Par exemple, le code suivant provoque une erreur de compilation :
\begin{OMNIBUS}
assert yes // Erreur de compilation, l'expression est calculable statiquement
\end{OMNIBUS}







\section{L'instruction \texttt{panic}}\index{panic@\omnibus=panic=}

L'instruction \omnibus=panic= a la syntaxe suivante :
\begin{OMNIBUS}
panic expression
\end{OMNIBUS}

L'\omnibus=expression= est une expression entière, calculée statiquement. Son type est défini pour chaque cible (\refSubsectionPage{configurationPanic}), c'est un type entier non signé.

Si le programme est compilé avec la panique activée, alors l'exécution de l'instruction \omnibus=panic= déclenche la panique avec le code est la valeur de l'\omnibus=expression=. Pour éviter un conflit de code avec les codes prédéfinis dans OMNIBUS, consulter le \refTableauPage{tableauCodePanique}.

Si le programme est compilé avec l'option \texttt{-{}-no-panic-generation}, alors aucun code n'est engendré, l'instruction est ignorée.




\sectionLabel{Instruction d'appel de procédure}{instructionAppelProc}

\colorbox{red}{À définir}


\sectionLabel{Instruction \texttt{if}}{instructionIF}
\index{if@\omnibus=if=}

L'instruction \omnibus=if= a une structure classique, où \omnibus=condition= est une expression booléenne :
\begin{OMNIBUS}
if condition {
  instructions_then
}else{
  instructions_else
}
\end{OMNIBUS}

Le compilateur vérifie que la \omnibus=condition= n'est pas une expression statique : une erreur de compilation est émise si elle l'est.

La branche \omnibus=else= est optionnelle :
\begin{OMNIBUS}
if condition {
  instructions_then
}
\end{OMNIBUS}


Une ou plusieurs branches \omnibus=else if= peuvent être ajoutées, avec ou sans branche \omnibus=else= :
\begin{OMNIBUS}
if condition {
  instructions_then
}else if condition2 {
  instructions_elsif
}else if condition3 {
  instructions_elsif_3
}else{
  instructions_else
}
\end{OMNIBUS}


\section{Instruction \texttt{while}}
\index{if@\omnibus=while=}

L'instruction \omnibus=while= permet d'exprimer une répétition, où la \omnibus=condition= est testée avant l'exécution des instructions de la boucle :
\begin{OMNIBUS}
while condition {
  instructions_while
}
\end{OMNIBUS}

\omnibus=condition= est une expression booléenne, qui ne doit pas être statique : une erreur de compilation est émise si elle l'est.












\sectionLabel{Instruction \texttt{for}}{instructionFor}


\subsection{Énumération entière}
\begin{OMNIBUS}
for nom : Type in lower_bound ..< upper_bound {
  instructions_for
}
\end{OMNIBUS}

Si il n'est pas lu par les \omnibus=instructions_for=, \omnibus=nom= peut être remplacé par le joker « \omnibus=_= ».




\subsection{Énumération d'un tableau ou d'une chaîne de caractères}

\begin{OMNIBUS}
for nom in objet while expression {
  instructions_for
}
\end{OMNIBUS}













\sectionLabel{Instruction d'appel de routine}{instructionAppelRoutine}













\sectionLabel{Instruction \texttt{switch}}{instructionSwitch}













\section{Instruction \texttt{sync}}

L'instruction \omnibus=sync= est décrite à la \refSectionPage{instructionSync}.





\sectionLabel{Instruction \texttt{nop}}{instructionNop}

\begin{OMNIBUS}
nop
\end{OMNIBUS}

Le mot réservé \omnibus=nop= engendre une instruction assembleur \emph{nop}, telle que défini par le paramètre \texttt{NOP} dans la définition de la cible (\refSubsectionPage{configurationNop}).


%!TEX encoding = UTF-8 Unicode
%!TEX root = ../doc-plm.tex





\chapter{Directives}



\section{Directive \texttt{target}}\index{target@\plm=target=}





\section{Directive \texttt{import}}\index{import@\plm=import=}


\sectionLabel{Directive \texttt{check}}{directiveCheck}\index{check@\plm=check=}

La directive \plm=check= apparaît dans une liste d'instructions et a la syntaxe suivante :
\begin{PLM}
check expression
\end{PLM}

L'\plm=expression= est une expression booléenne calculée statiquement.

Contrairement à l'instruction \plm=assert= (\refSectionPage{instructionAssert}) qui évalue l'expression booléenne à l'exécution, la directive \plm=check= est toujours évaluée à la compilation. Elle permet d'exprimer des assertions qui sont évaluées lors de la compilation.

Aucun code n'est engendré. La directive \plm=check= peut donc apparaître dans des listes d'instructions où les exceptions sont interdites.

Exemples :
\begin{PLM}
check true  // Ok
check false // Erreur, expression fausse
\end{PLM}



%!TEX encoding = UTF-8 Unicode
%!TEX root = ../doc-plm.tex





\chapter{Exceptions}

\begin{table}[ht]
\centering
\small
\begin{tabular}{lp{10cm}l}
  \textbf{Numéros} & \textbf{Signification} & \textbf{Lien} \\
  \hline
   0 à 999 & Exceptions liées aux vecteurs d'interruption & \\
   1000 & Dépassement de capacité de l'incrémentation (\plm-++-) & \\
   1001 & Dépassement de capacité de l'incrémentation (\plm+--+) & \\
   1002 & Dépassement de capacité de la négation (\plm+-+) & \\
   1003 & Dépassement de la construction d'un champ entier d'un registre (\plm+registre::champ (...)+) & \refSubsectionPage{constructionChampEntierRegistre}\\
   1010 & Dépassement de capacité de l'addition (\plm-+-) & \\
   1011 & Dépassement de capacité de la soustraction (\plm+-+) & \\
   1012 & Dépassement de capacité de la multiplication (\plm+*+) & \\
   1013 & Dépassement de capacité de la division (\plm+/+) & \\
   1020 & Échec de l'instruction \plm+assert+ & \\
   1021 & Instruction \plm+throw+ & \\
\end{tabular}
\caption{Code des exceptions}
\labelTableau{tableauCodeExceptions}
\end{table}


%!TEX encoding = UTF-8 Unicode
%!TEX root = ../doc-plm.tex





\chapterLabel{Modules}{chapitreModule}

Un \emph{module} PLM est un singleton, c'est-à-dire une type structure qui n'est être instancié qu'une fois et qui est visible globalement. C'est l'outil qui permet d'implémenter des pilotes matériels.






%!TEX encoding = UTF-8 Unicode
%!TEX root = ../doc-plm.tex





\chapterLabel{Tâches}{chapitreTache}

En PLM, la tâche est l'unité d'exécution. Une tâche est déclarée statiquement, de priorité fixe. Une tâche peut déclarer des variables privées et du code à exécuter.


\section{Un exemple de tâche}


\begin{PLM}
task T1 priority 1 stackSize 512 {
  var compteur $uint32 = 0

  setup 0 {
    lcd.print (!string:"Hello")
  }
  
  on time.waitUntilMS (!deadline:self.compteur) continue {
    digitalWrite (!port:LED_L0 !yes)
    self.compteur +%= 500
    time.waitUntilMS (!deadline:self.compteur)
    digitalWrite (!port:LED_L0 !no)
    self.compteur +%= 500
  }
}
\end{PLM}

\section{Déclaration d'une tâche}

L'en-tête de la déclaration d'une tâche définit~:
\begin{itemize}
  \item son nom~;
  \item sa priorité~;
  \item la taille de sa pile.
\end{itemize}

Toutes les priorités doivent être différentes~; $0$ est la plus forte priorité.

Par exemple~:
\begin{PLM}
task T priority 12 stackSize 512 {
  ...
}
\end{PLM}

Peuvent être déclarés dans le corps d'une tâche~:
\begin{itemize}
  \item des variables privées~;
  \item des routines d'initialisation (\plm=setup=)~;
  \item des fonctions~;
  \item des commandes gardées.
\end{itemize}

La déclaration d'une variable privée doit comporter une expression qui fixe sa valeur initiale (ainsi toutes les variables privées d'une tâche sont initialisées lorsqu'elle démarre)~; cette expression doit être calculable statiquement. Ces variables sont privées, c'est-à-dire qu'aucune autre entité extérieure (comme par exemple une autre tâche) ne peut y accéder. L'accès aux variables privées doit être obligatoirement préfixé par \plm=self.=.

Une tâche peut déclarer zéro, une ou plusieurs routines d'initialisation. Chacune présente une priorité (exprimée sous la forme d'un nombre positif ou nul). Deux routines d'initialisation d'une même tâche ne peuvent pas avoir la même priorité. Les routines d'initialisation sont exécutées dans l'ordre croissant de leur priorité, et en mode utilisateur (\refSectionPage{modesLogiques}). Une routine d'initialisation peut servir par exemple à donner une valeur initiale calculée dynamiquement à une variable privée, ou encore à réaliser des initialisations matérielles.

Une tâche peut déclarer zéro, une ou plusieurs gardes. Une garde exprime l'attente qu'une condition de synchronisation se réalise. Si une tâche ne définit aucune garde, alors son exécution se termine à la fin de l'exécution des routines d'initialisation \plm=setup=.

Enfin, une tâche peut déclarer des fonctions privées.






\section{Extensions}






\section{Exécution des tâches}

Les tâches sont toutes démarrées à la fin de la phase d'initialisation (\refFigurePage{}{sequenceDemarrage})\footnote{Si aucune tâche n'est déclarée, l'exécution s'arrête à la fin de l'exécution des routines \texttt{\bf init}.}. Après avoir exécuté ses routines \plm=setup= (dans l'ordre croissant des valeurs associées), la tâche se met en attente des gardes tant qu'aucune n'est vraie. Dès que l'une d'elles devient vraie, sa liste d'instructions est exécutée~; par défaut, la tâche se remet en attente de l'évolution des gardes. Si la garde nomme le qualificatif \plm=exit=, l'exécution est terminée.

La \refFigure{}{executionTache} illustre l'exécution de la tâche \plm=T= suivante~: 
\begin{PLM}
task T priority 1 stackSize 512 {
  setup 1 { ... }
  on G1 exit { L1 }
  on G2 { L2 }
}
\end{PLM}

\begin{figure}[htbp]
  \centering
  \small
  \begin{tikzpicture}[
      cloud/.style ={draw=red, thick, chamfered rectangle,fill=red!20, minimum height=2em},
      block/.style ={rectangle, draw=blue, thick, fill=green!20, align=center},
      decision/.style={chamfered rectangle, draw=blue, thick, fill=green!20},
      node distance=5mm
    ]
    \node [cloud] (start) {\textsc{Démarrage de la tâche}};
    \node [block] (setup) [below=of start] {Routines \bf\texttt{setup $n$}};
    \node [block] (loop) [below=of setup] {Attente évolution gardes};
    \node [block] (G1) [below left=of loop] {Garde \texttt{G1 \bf exit}};
    \node [block] (L1) [below=of G1] {Liste d'instructions \texttt{L1}};
    \node [cloud] (fin) [below=of L1] {Fin de la tâche};
    \node [block] (G2) [below right=of loop] {Garde \texttt{G2}};
    \node [block] (L2) [below=of G2] {Liste d'instructions \texttt{L2}};

    \draw [-stealth, thick] (start) -- (setup);
    \draw [-stealth, thick] (setup) -- (loop);
    \draw [-stealth, thick] (loop) -- (G1);
    \draw [-stealth, thick] (G1) -- (L1);
    \draw [-stealth, thick] (L1) -- (fin);
    \draw [-stealth, thick] (loop) -- (G2);
    \draw [-stealth, thick] (G2) -- (L2);
    \draw [-stealth, thick] (G2) -- (L2);
    \draw [-stealth, thick] (L2) -- + (0, -.5) -- + (2, -.5) |- (loop);
  \end{tikzpicture}
  \caption{Organigramme d'exécution d'une tâche}
  \labelFigure{executionTache}
%  \ligne
\end{figure}

 %!TEX encoding = UTF-8 Unicode
%!TEX root = ../doc-plm.tex





\chapterLabel{Synchronisation et communication}{chapitreSynchros}



%\sectionLabel{Instruction \texttt{sync}}{instructionSync}


\sectionLabel{Sémaphore de Dijkstra}{semaphore}

\begin{PLM}
struct $semaphore {
  var value $uint32
  var list = $taskList ()
  var guardList = $guardList ()

  public service V @noUnusedWarning () {
    makeTaskReady (!?list:self.list ?found:let found)
    if not found {
      self.value += 1
      guardDidChange (!?guard:self.guardList)
    }
  }

  public primitive P @noUnusedWarning () {
    if self.value > 0 {
      self.value -= 1
    }else{
      blockInList (!?list:self.list)
    }
  }

  public primitive P_until @noUnusedWarning (?deadline:inDeadline $uint32) -> $bool {
    result = self.value > 0
    if result {
      self.value -= 1
    }else if inDeadline > time.millis () { 
      blockInListAndOnDeadline (!?list:self.list !deadline:inDeadline)
    }
  }

  public guard P @noUnusedWarning () {
    accept = self.value > 0
    if accept {
      self.value -= 1
    }else{
      handleGuardedCommand (!?guard:self.guardList)
    }
  }

}
\end{PLM}




\section{Implémentation des commandes gardées}

\begin{figure}[ht]
  \centering
  \small
  \begin{tikzpicture}[
      block/.style ={chamfered rectangle, draw=black, fill=green!20, align=center, minimum height=0.75cm, minimum width=4cm},
      node distance=2cm and 4cm
    ]
    \node [block] (horsGarde) at (0, 0) {Hors garde \textbar~$n == 0$} ;
    \node [block] (gardeModifie) [right=of horsGarde] {Garde modifiée \textbar~$n == 0$} ;
    \node [block] (evaluation) [below=of gardeModifie] {En évaluation \textbar~$n > 0$};
    \node [block] (attente) [below=of horsGarde] {En attente \textbar~$n > 0$};

    \draw [-stealth, thick] (-3, 0) to (horsGarde) ;

    \draw [-stealth, thick, orange] (horsGarde) to [out=300,in=135] node [below] {Garde neutre $\blacktriangleright~n := n+1$} (evaluation) ;
    \draw [-stealth, thick, orange] (evaluation) to [loop below] node [below] {Garde neutre $\blacktriangleright~n := n+1$} (evaluation) ;
    \draw [-stealth, thick, orange] (gardeModifie) to [loop above] node [above] {Garde neutre $\blacktriangleright~n := n+1$} (gardeModifie) ;

    \draw [-stealth, thick, OliveGreen] (evaluation) to [out=120,in=315] node [above] {Garde vraie $\blacktriangleright~n := 0$} (horsGarde) ;
    \draw [-stealth, thick, OliveGreen] (horsGarde) to [loop above] node [above] {Garde vraie $\blacktriangleright~n := 0$} (horsGarde) ;
    \draw [-stealth, thick, OliveGreen] (gardeModifie) to [out=175,in=5] node [above] {Garde vraie $\blacktriangleright~n := 0$} (horsGarde) ;

    \draw [-stealth, thick, blue] (evaluation) to node [below] {waitForChange} (attente) ;
    \draw [-stealth, thick, blue] (gardeModifie) to [out=185,in=355] node [below] {waitForChange} (horsGarde) ;

    \draw [-stealth, thick, Maroon] (attente) to node [left] {guardDidChange} (horsGarde) ;
    \draw [-stealth, thick, Maroon] (evaluation) to node [right] {guardDidChange} (gardeModifie) ;
  \end{tikzpicture}
  \caption{Graphe d'état des gardes d'une tâche}
  \labelFigure{commandeGardee}
  \ligne
\end{figure}


%!TEX encoding = UTF-8 Unicode
%!TEX root = ../doc-plm.tex





\chapterLabel{Configuration d'une cible}{chapitreConfCible}

\sectionLabel{Déclaration \texttt{newUnsignedRepresentation}}{DefNewUnsignedRepresentation}
\index{newUnsignedRepresentation@\plm=newUnsignedRepresentation=!Definition@Définition}

La déclaration \plm=newUnsignedRepresentation= permet de définir une représentation entière non signée. Sa syntaxe est la suivante :


\begin{PLM}
newUnsignedRepresentation @nom "type-c" taille
\end{PLM}

Où :
\begin{itemize}
  \item \plm=@nom= est le nom donné à la représentation ;
  \item \plm="type-c"= est le nom du type C qui sera utilisé lors de la génération de code ;
  \item \plm=taille= est le nombre de bits de cette représentation.
\end{itemize}

Par exemple, la représentation des entiers non signés habituels peut être définie par :
\begin{PLM}
newUnsignedRepresentation @unsigned8  "uint8_t"   8
newUnsignedRepresentation @unsigned16 "uint16_t" 16
newUnsignedRepresentation @unsigned32 "uint32_t" 32
newUnsignedRepresentation @unsigned64 "uint64_t" 64
\end{PLM}








\sectionLabel{Déclaration \texttt{newSignedRepresentation}}{DefNewSignedRepresentation}
\index{newSignedRepresentation@\plm=newSignedRepresentation=!Definition@Définition}

La déclaration \plm=newSignedRepresentation= permet de définir une représentation entière signée. Sa syntaxe est la suivante :


\begin{PLM}
newSignedRepresentation @representation "type-c" taille
\end{PLM}

Où :
\begin{itemize}
  \item \plm=@representation= est le nom donné à la représentation ;
  \item \plm="type-c"= est le nom du type C qui sera utilisé lors de la génération de code ;
  \item \plm=taille= est le nombre de bits de cette représentation.
\end{itemize}

Par exemple, la représentation des entiers signés habituels peut être définie par :
\begin{PLM}
newSignedRepresentation @signed8  "int8_t"   8
newSignedRepresentation @signed16 "int16_t" 16
newSignedRepresentation @signed32 "int32_t" 32
newSignedRepresentation @signed64 "int64_t" 64
\end{PLM}











\sectionLabel{Déclaration \texttt{newIntegerType}}{DefNewIntegerType}
\index{newIntegerType@\plm=newIntegerType=!Definition@Définition}

La déclaration \plm=newIntegerType= permet de définir un nouveau type entier signé ou non signé. Elle a la syntaxe suivante :
\begin{PLM}
newIntegerType NomDeType @representation
\end{PLM}
Où :
\begin{itemize}
  \item \plm=NomDeType= est le nom donné au type entier.
  \item \plm=@representation= est le nom de la représentation, qui doit avoir été défini soit par une déclaration \plm=newUnsignedRepresentation= (\refSectionPage{DefNewUnsignedRepresentation})\index{newUnsignedRepresentation@\plm=newUnsignedRepresentation=}, soit par une déclaration \plm=newSignedRepresentation= (\refSectionPage{DefNewSignedRepresentation}\index{newSignedRepresentation@\plm=newSignedRepresentation=}).
\end{itemize}

Le type entier ainsi déclaré est non signé si la représentation est non signée (c'est-à-dire déclarée avec \plm=newUnsignedRepresentation=), et signé si la représentation est signée (c'est-à-dire déclarée avec \plm=newSignedRepresentation=). Les types usuels peuvent ainsi être déclarés :
\begin{PLM}
newIntegerType UInt8  @unsigned8
newIntegerType UInt16 @unsigned16
newIntegerType UInt32 @unsigned32
newIntegerType UInt64 @unsigned64
newIntegerType Int8  @signed8
newIntegerType Int16 @signed16
newIntegerType Int32 @signed32
newIntegerType Int64 @signed64
\end{PLM}





%%!TEX encoding = UTF-8 Unicode
%!TEX root = ../doc-plm.tex





\chapterLabel{Compilation LLVM}{chapitreCompilationLLVM}

Ce chapitre contient différentes notes sur la compilation en LLVM des sources PLM. Sa lecture n'est pas nécessaire pour comprendre l'utilisation de PLM. 


Le point le plus délicat est l'analyse sémantique et la génération de code des constructions implicant \plm=self=, l'accès aux propriétés, aux éléments de tableaux, à l'appel de fonction. Ces constructions sont :
\begin{itemize}
  \item l'expression primaire (qui peut apparaître par exemple en partie droite de l'instruction d'affectation) ;
  \item la cible d'affection (partie gauche de l'instruction d'affectation) ;
  \item l'instruction d'appel de procédure (c'est-à-dire l'appel d'une fonction qui ne retourne aucune valeur).
\end{itemize}

\section{Diagrammes syntaxiques}

\tikzset{
  nonterminal/.style={
    % The shape:
    rectangle,
    % The size:
    minimum size=6mm,
    % The border:
    very thick,
    draw=red!50!black!50,         % 50% red and 50% black,
                                  % and that mixed with 50% white
    % The filling:
    top color=white,              % a shading that is white at the top...
    bottom color=red!50!black!20, % and something else at the bottom
    % Font
    font=\itshape,
    text height=1.5ex,
    text depth=.25ex
  },
  terminal/.style={
    % The shape:
    rounded rectangle,
    minimum size=6mm,
    % The rest
    very thick,draw=black!50,
    top color=white,bottom color=black!20,
    font=\ttfamily,
    text height=1.5ex,
    text depth=.25ex
  },
  point/.style={coordinate, >=stealth', thick, draw=black!50},
  tip/.style={->, shorten >=0.007pt,every join/.style={rounded corners}},
  hv path/.style={to path={-| (\tikztotarget)}},
  vh path/.style={to path={|- (\tikztotarget)}},
%  skip loop/.style={to path={-- ++(0,#1) -| (\tikztotarget)}}
}

\subsection{Expression primaire}

\begin{tikzpicture}
  \matrix[column sep=4mm, row sep=1mm] {
  % rows:
    & & & & & & & & \node (loopleft) [point] {}; & \node (startloopup2) [point] {}; & & \\
    & & & & & & & & & & & \node (startloopup) [point] {}; & \\
    & & & & & & & & \node (parouv) [terminal] {(}; & \node (args) [nonterminal] {arguments}; & \node (parfer) [terminal] {)};  & \\
    & & & & & & & & \node (croouv) [terminal] {\verb+[+}; & \node (exparray) [nonterminal] {expression}; & \node (crofer) [terminal] {\verb+]+};  & \\
    & & \node (self) [terminal] {self}; & \node (selfdot) [terminal] {.}; & & & & &  \node (propdot) [terminal] {.}; & \node (prop) [terminal] {idf};  & \\
    \node (pstart) [point] {}; & \node (selfstart) [point] {}; & & & \node (selfend) [point] {}; & \node (idf) [terminal] {idf}; & \node (optionloop) [point] {};  & \node (optionstart) [point] {}; & & & & & \node (pend) [point] {};\\
  };

  { [start chain]
    \chainin (pstart);
    \chainin (selfstart)     [join];
    { [start branch=selfstart]
      \chainin (self)  [join=by {vh path,tip}];
      \chainin (selfdot)  [join=by tip];
      \chainin (selfend)  [join=by {hv path,tip}];
    }
    \chainin (selfend)     [join];
    \chainin (idf)         [join=by tip];
    \chainin (optionloop) [join];
    \chainin (optionstart)  [join];
    { [start branch=optionstart]
      \chainin (propdot)  [join=by {vh path,tip}];
      \chainin (prop)     [join=by tip];
      \chainin (startloopup)  [join=by {hv path,tip}];
    }
    \chainin (pend)        [join=by tip];
  }
  { [start chain]
    \chainin (startloopup) ;
    \chainin (startloopup2) [join=by {vh path}];
    \chainin (loopleft)     [join] ;
    \chainin (optionloop) [join=by {hv path}];
  }
  { [start chain]
    \chainin (optionstart) ;
    \chainin (croouv)     [join=by {vh path, tip}];
    \chainin (exparray)   [join=by tip] ;
    \chainin (crofer)     [join=by tip];
      \chainin (startloopup) [join=by {hv path}];
  }
  { [start chain]
    \chainin (optionstart) ;
    \chainin (parouv)  [join=by {vh path, tip}];
    \chainin (args)   [join=by tip] ;
    \chainin (parfer)     [join=by tip];
    \chainin (startloopup) [join=by {hv path}];
  }
\end{tikzpicture}

Le préfixe \plm=self.= est obligatoire pour accéder à une propriété ou fonction locale.

\subsection{Cible d'affectation}

\begin{tikzpicture}
  \matrix[column sep=4mm, row sep=1mm] {
  % rows:
    & & & & & & & & \node (loopleft) [point] {}; & \node (startloopup2) [point] {}; & & \\
    & & & & & & & & & & & \node (startloopup) [point] {}; & \\
    & & & & & & & & \node (croouv) [terminal] {\verb+[+}; & \node (exparray) [nonterminal] {expression}; & \node (crofer) [terminal] {\verb+]+};  & \\
    & & \node (self) [terminal] {self}; & \node (selfdot) [terminal] {.}; & & & & &  \node (propdot) [terminal] {.}; & \node (prop) [terminal] {idf};  & \\
    \node (pstart) [point] {}; & \node (selfstart) [point] {}; & & & \node (selfend) [point] {}; & \node (idf) [terminal] {idf}; & \node (optionloop) [point] {};  & \node (optionstart) [point] {}; & & & & & \node (pend) [point] {};\\
  };

  { [start chain]
    \chainin (pstart);
    \chainin (selfstart)     [join];
    { [start branch=selfstart]
      \chainin (self)  [join=by {vh path,tip}];
      \chainin (selfdot)  [join=by tip];
      \chainin (selfend)  [join=by {hv path,tip}];
    }
    \chainin (selfend)     [join];
    \chainin (idf)         [join=by tip];
    \chainin (optionloop) [join];
    \chainin (optionstart)  [join];
    { [start branch=optionstart]
      \chainin (propdot)  [join=by {vh path,tip}];
      \chainin (prop)     [join=by tip];
      \chainin (startloopup)  [join=by {hv path,tip}];
    }
    \chainin (pend)        [join=by tip];
  }
  { [start chain]
    \chainin (startloopup) ;
    \chainin (startloopup2) [join=by {vh path}];
    \chainin (loopleft)     [join] ;
    \chainin (optionloop) [join=by {hv path}];
  }
  { [start chain]
    \chainin (optionstart) ;
    \chainin (croouv)     [join=by {vh path, tip}];
    \chainin (exparray)   [join=by tip] ;
    \chainin (crofer)     [join=by tip];
      \chainin (startloopup) [join=by {hv path}];
  }
\end{tikzpicture}



\subsection{Appel de procédure}

\begin{tikzpicture}
  \matrix[column sep=4mm, row sep=1mm] {
  % rows:
    & & & & & & & & \node (loopleft) [point] {}; & \node (startloopup2) [point] {}; & & \\
    & & & & & & & & & & & \node (startloopup) [point] {}; & \\
    & & & & & & & & \node (croouv) [terminal] {\verb+[+}; & \node (exparray) [nonterminal] {expression}; & \node (crofer) [terminal] {\verb+]+};  & \\
    & & \node (self) [terminal] {self}; & \node (selfdot) [terminal] {.}; & & & & &  \node (propdot) [terminal] {.}; & \node (prop) [terminal] {idf};  & \\
    \node (pstart) [point] {}; & \node (selfstart) [point] {}; & & & \node (selfend) [point] {}; & \node (idf) [terminal] {idf}; & \node (optionloop) [point] {};  & \node (optionstart) [point] {}; & \node (parouv) [terminal] {(}; & \node (args) [nonterminal] {arguments}; & \node (parfer) [terminal] {)}; & \node (pend) [point] {};\\
  };

  { [start chain]
    \chainin (pstart);
    \chainin (selfstart)     [join];
    { [start branch=selfstart]
      \chainin (self)  [join=by {vh path,tip}];
      \chainin (selfdot)  [join=by tip];
      \chainin (selfend)  [join=by {hv path,tip}];
    }
    \chainin (selfend)     [join];
    \chainin (idf)         [join=by tip];
    \chainin (optionloop) [join];
    \chainin (optionstart)  [join];
    { [start branch=optionstart]
      \chainin (propdot)  [join=by {vh path,tip}];
      \chainin (prop)     [join=by tip];
      \chainin (startloopup)  [join=by {hv path,tip}];
    }
    \chainin (parouv)      [join=by tip];
    \chainin (args)        [join=by tip];
    \chainin (parfer)      [join=by tip];
    \chainin (pend)        [join=by tip];
  }
  { [start chain]
    \chainin (startloopup) ;
    \chainin (startloopup2) [join=by {vh path}];
    \chainin (loopleft)     [join] ;
    \chainin (optionloop) [join=by {hv path}];
  }
  { [start chain]
    \chainin (optionstart) ;
    \chainin (croouv)      [join=by {vh path, tip}];
    \chainin (exparray)    [join=by tip] ;
    \chainin (crofer)      [join=by tip];
    \chainin (startloopup) [join=by {hv path}];
  }
\end{tikzpicture}



%!TEX encoding = UTF-8 Unicode
%!TEX root = ../doc-plm.tex





\chapter{Grammaire}

Le langage PLM définit deux grammaires :
\begin{itemize}
\item la grammaire des sources des programmes PLM (\refSectionPage{grammairePLM}) ;
\item la grammaire de description d'un cible (\refSectionPage{grammaireCible}).
\end{itemize}

Ce chapitre liste l'ensemble des règles de production de ces deux grammaires.


{

\tikzset{
  nonterminal/.style={
    % The shape:
    rectangle,
    % The size:
    minimum size=6mm,
    % The border:
    very thick,
    draw=red!50!black!50,         % 50% red and 50% black,
                                  % and that mixed with 50% white
    % The filling:
    top color=white,              % a shading that is white at the top...
    bottom color=red!50!black!20, % and something else at the bottom
    % Font
    font=\itshape\footnotesize
  },
  terminal/.style={
    % The shape:
    rounded rectangle,
    minimum size=6mm,
    % The rest
    very thick,draw=black!50,
    top color=white,bottom color=black!20,
    font=\ttfamily\footnotesize
  },
  firstPoint/.style={circle,>=stealth',thick,draw=black!50},
  point/.style={coordinate,>=stealth',thick,draw=black!50},
  tip/.style={->,shorten >=0.007pt},
  lastPoint/.style={rectangle,>=stealth',thick,draw=black!50},
  every join/.style={rounded corners}
}

\newcommand\ruleSubsection[3]{} % \subsection{Component \texttt{#1}, in file \texttt{#2}, line #3}}
\newcommand\ruleMatrixColumnSeparation{2mm}
\newcommand\ruleMatrixRowSeparation{1.5mm}
\newcommand\nonTerminalSummarySeparator{, }
\newcommand\nonTerminalSummaryEnd{.\\}
\newcommand\nonTerminalSummaryStart{Voici la liste alphabétique des non terminaux: }

\newcommand\nonTerminalSection[2]{\subsection{Non terminal \texttt{\it#1}}\label{nt:#2}}
\newcommand\nonTerminalSummary[2]{\hyperref[nt:#2]{#1}}
\newcommand\nonTerminalSymbol[2]{\hyperref[nt:#2]{#1}}
\newcommand\startSymbol[2]{L'axiome de la grammaire est \hyperref[nt:#2]{#1}.}

\sectionLabel{Grammaire du langage \texttt{PLM}}{grammairePLM}

\startSymbol{start\_symbol}{6}

\nonTerminalSummaryStart \nonTerminalSummary{assignment\_operator}{33}\nonTerminalSummarySeparator \nonTerminalSummary{assignment\_target}{37}\nonTerminalSummarySeparator \nonTerminalSummary{declaration}{7}\nonTerminalSummarySeparator \nonTerminalSummary{declaration\_init}{12}\nonTerminalSummarySeparator \nonTerminalSummary{declaration\_struct\_var}{9}\nonTerminalSummarySeparator \nonTerminalSummary{declaration\_type}{8}\nonTerminalSummarySeparator \nonTerminalSummary{effective\_parameters}{30}\nonTerminalSummarySeparator \nonTerminalSummary{expression}{16}\nonTerminalSummarySeparator \nonTerminalSummary{expression\_1}{28}\nonTerminalSummarySeparator \nonTerminalSummary{expression\_10}{19}\nonTerminalSummarySeparator \nonTerminalSummary{expression\_11}{18}\nonTerminalSummarySeparator \nonTerminalSummary{expression\_12}{17}\nonTerminalSummarySeparator \nonTerminalSummary{expression\_2}{27}\nonTerminalSummarySeparator \nonTerminalSummary{expression\_3}{26}\nonTerminalSummarySeparator \nonTerminalSummary{expression\_4}{25}\nonTerminalSummarySeparator \nonTerminalSummary{expression\_5}{24}\nonTerminalSummarySeparator \nonTerminalSummary{expression\_6}{23}\nonTerminalSummarySeparator \nonTerminalSummary{expression\_7}{22}\nonTerminalSummarySeparator \nonTerminalSummary{expression\_8}{21}\nonTerminalSummarySeparator \nonTerminalSummary{expression\_9}{20}\nonTerminalSummarySeparator \nonTerminalSummary{global\_variable\_declaration}{10}\nonTerminalSummarySeparator \nonTerminalSummary{guard}{15}\nonTerminalSummarySeparator \nonTerminalSummary{guarded\_command}{35}\nonTerminalSummarySeparator \nonTerminalSummary{if\_instruction}{34}\nonTerminalSummarySeparator \nonTerminalSummary{import\_file}{5}\nonTerminalSummarySeparator \nonTerminalSummary{instruction}{32}\nonTerminalSummarySeparator \nonTerminalSummary{instructionList}{31}\nonTerminalSummarySeparator \nonTerminalSummary{isr}{4}\nonTerminalSummarySeparator \nonTerminalSummary{module\_variable}{11}\nonTerminalSummarySeparator \nonTerminalSummary{primary}{29}\nonTerminalSummarySeparator \nonTerminalSummary{primitive}{3}\nonTerminalSummarySeparator \nonTerminalSummary{procedure}{0}\nonTerminalSummarySeparator \nonTerminalSummary{procedure\_call}{36}\nonTerminalSummarySeparator \nonTerminalSummary{procedure\_formal\_arguments}{14}\nonTerminalSummarySeparator \nonTerminalSummary{procedure\_header}{13}\nonTerminalSummarySeparator \nonTerminalSummary{section}{1}\nonTerminalSummarySeparator \nonTerminalSummary{service}{2}\nonTerminalSummarySeparator \nonTerminalSummary{start\_symbol}{6}\nonTerminalSummaryEnd \nonTerminalSection{assignment\_operator}{33}

\ruleSubsection{plm\_syntax}{instruction-assignment-operator}{37}

\begin{tikzpicture}
  \matrix[column sep=\ruleMatrixColumnSeparation, row sep=\ruleMatrixRowSeparation] {
    & & & \node (p8-3) [terminal] {*\verb=%==}; & \\
    & & & \node (p7-3) [terminal] {*=}; & \\
    & & & \node (p6-3) [terminal] {-\verb=%==}; & \\
    & & & \node (p5-3) [terminal] {-=}; & \\
    & & & \node (p4-3) [terminal] {+\verb=%==}; & \\
    & & & \node (p3-3) [terminal] {+=}; & \\
    & & & \node (p2-3) [terminal] {\verb=^==}; & \\
    & & & \node (p1-3) [terminal] {\&=}; & \\
    \node (P0start) [firstPoint] {}; & & \node (p0-2) [point] {}; & \node (p0-3) [terminal] {|=}; & \node (p0-4) [point] {}; & \node (p0-5) [lastPoint] {}; & \\
  };
  \draw[->] (P0start) -- (p0-3) ;
  \draw[->] (p0-2) |- (p1-3) ;
  \draw[->] (p0-2) |- (p2-3) ;
  \draw[->] (p0-2) |- (p3-3) ;
  \draw[->] (p0-2) |- (p4-3) ;
  \draw[->] (p0-2) |- (p5-3) ;
  \draw[->] (p0-2) |- (p6-3) ;
  \draw[->] (p0-2) |- (p7-3) ;
  \draw[->] (p0-2) |- (p8-3) ;
  \draw (p0-3) -- (p0-4) ;
  \draw[->] (p1-3) -| (p0-4) ;
  \draw[->] (p2-3) -| (p0-4) ;
  \draw[->] (p3-3) -| (p0-4) ;
  \draw[->] (p4-3) -| (p0-4) ;
  \draw[->] (p5-3) -| (p0-4) ;
  \draw[->] (p6-3) -| (p0-4) ;
  \draw[->] (p7-3) -| (p0-4) ;
  \draw[->] (p8-3) -| (p0-4) ;
  \draw[->] (p0-4) -- (p0-5) ;
\end{tikzpicture}

\nonTerminalSection{assignment\_target}{37}

\ruleSubsection{plm\_syntax}{assignment-target}{39}

\begin{tikzpicture}
  \matrix[column sep=\ruleMatrixColumnSeparation, row sep=\ruleMatrixRowSeparation] {
    & & & & & & & & & & & & & \node (p3-13) [point] {}; & \\
    & & & & & & & & & & \node (p2-10) [terminal] {[}; & \node (p2-11) [nonterminal] {\nonTerminalSymbol{expression}{16}}; & \node (p2-12) [terminal] {]}; & \\
    & & & \node (p1-3) [terminal] {self}; & \node (p1-4) [terminal] {.}; & & & & & & \node (p1-10) [terminal] {.}; & \node (p1-11) [terminal] {identifier}; & \\
    \node (P0start) [firstPoint] {}; & & \node (p0-2) [point] {}; & \node (p0-3) [point] {}; & & \node (p0-5) [point] {}; & \node (p0-6) [terminal] {identifier}; & \node (p0-7) [point] {}; & \node (p0-8) [point] {}; & \node (p0-9) [point] {}; & & & & & \node (p0-14) [lastPoint] {}; & \\
  };
  \draw (P0start) -- (p0-3) ;
  \draw[->] (p0-2) |- (p1-3) ;
  \draw[->] (p1-3) -- (p1-4) ;
  \draw (p0-3) -- (p0-5) ;
  \draw[->] (p1-4) -| (p0-5) ;
  \draw[->] (p0-5) -- (p0-6) ;
  \draw (p0-6) -- (p0-8) ;
  \draw[->] (p0-9) |- (p1-10) ;
  \draw[->] (p1-10) -- (p1-11) ;
  \draw[->] (p0-9) |- (p2-10) ;
  \draw[->] (p2-10) -- (p2-11) ;
  \draw[->] (p2-11) -- (p2-12) ;
  \draw[->] (p3-13) -| (p0-7) ;
  \draw[->] (p1-11) -| (p3-13) ;
  \draw[->] (p2-12) -| (p3-13) ;
  \draw[->] (p0-8) -- (p0-14) ;
\end{tikzpicture}

\nonTerminalSection{declaration}{7}

\ruleSubsection{plm\_syntax}{declaration-type}{9}

\begin{tikzpicture}
  \matrix[column sep=\ruleMatrixColumnSeparation, row sep=\ruleMatrixRowSeparation] {
    \node (P0start) [firstPoint] {}; & & \node (p0-2) [terminal] {type}; & \node (p0-3) [terminal] {\$type}; & \node (p0-4) [terminal] {:}; & \node (p0-5) [nonterminal] {\nonTerminalSymbol{declaration\_type}{8}}; & \node (p0-6) [lastPoint] {}; & \\
  };
  \draw[->] (P0start) -- (p0-2) ;
  \draw[->] (p0-2) -- (p0-3) ;
  \draw[->] (p0-3) -- (p0-4) ;
  \draw[->] (p0-4) -- (p0-5) ;
  \draw[->] (p0-5) -- (p0-6) ;
\end{tikzpicture}

\ruleSubsection{plm\_syntax}{type-enumeration-declaration}{25}

\begin{tikzpicture}
  \matrix[column sep=\ruleMatrixColumnSeparation, row sep=\ruleMatrixRowSeparation] {
    & & & & & & & & & & \node (p2-10) [point] {}; & \\
    & & & & & & & & & \node (p1-9) [point] {}; & \\
    \node (P0start) [firstPoint] {}; & & \node (p0-2) [terminal] {enum}; & \node (p0-3) [terminal] {\$type}; & \node (p0-4) [terminal] {\{}; & \node (p0-5) [point] {}; & \node (p0-6) [terminal] {case}; & \node (p0-7) [terminal] {identifier}; & \node (p0-8) [point] {}; & & & \node (p0-11) [terminal] {\}}; & \node (p0-12) [lastPoint] {}; & \\
  };
  \draw[->] (P0start) -- (p0-2) ;
  \draw[->] (p0-2) -- (p0-3) ;
  \draw[->] (p0-3) -- (p0-4) ;
  \draw[->] (p0-4) -- (p0-6) ;
  \draw[->] (p0-6) -- (p0-7) ;
  \draw (p0-8) |- (p1-9) ;
  \draw[->] (p2-10) -| (p0-5) ;
  \draw[->] (p1-9) -| (p2-10) ;
  \draw[->] (p0-7) -- (p0-11) ;
  \draw[->] (p0-11) -- (p0-12) ;
\end{tikzpicture}

\ruleSubsection{plm\_syntax}{type-structure-declaration}{74}

\begin{tikzpicture}
  \matrix[column sep=\ruleMatrixColumnSeparation, row sep=\ruleMatrixRowSeparation] {
    & & & & & & & & & & & & & & \node (p8-14) [point] {}; & \\
    & & & & & & & & & & & & & \node (p7-13) [terminal] {;}; & \\
    & & & & & & & & & & & & & \node (p6-13) [nonterminal] {\nonTerminalSymbol{primitive}{3}}; & \\
    & & & & & & & & & & & & & \node (p5-13) [nonterminal] {\nonTerminalSymbol{guard}{15}}; & \\
    & & & & & & & & & & & & & \node (p4-13) [nonterminal] {\nonTerminalSymbol{service}{2}}; & \\
    & & & & & & & & & & & & & \node (p3-13) [nonterminal] {\nonTerminalSymbol{section}{1}}; & \\
    & & & & & & & & \node (p2-8) [point] {}; & & & & & \node (p2-13) [nonterminal] {\nonTerminalSymbol{procedure}{0}}; & \\
    & & & & & & & \node (p1-7) [terminal] {@attribute}; & & & & & & \node (p1-13) [nonterminal] {\nonTerminalSymbol{declaration\_struct\_var}{9}}; & \\
    \node (P0start) [firstPoint] {}; & & \node (p0-2) [terminal] {struct}; & \node (p0-3) [terminal] {\$type}; & \node (p0-4) [point] {}; & \node (p0-5) [point] {}; & \node (p0-6) [point] {}; & & & \node (p0-9) [terminal] {\{}; & \node (p0-10) [point] {}; & \node (p0-11) [point] {}; & \node (p0-12) [point] {}; & & & \node (p0-15) [terminal] {\}}; & \node (p0-16) [lastPoint] {}; & \\
  };
  \draw[->] (P0start) -- (p0-2) ;
  \draw[->] (p0-2) -- (p0-3) ;
  \draw (p0-3) -- (p0-5) ;
  \draw[->] (p0-6) |- (p1-7) ;
  \draw[->] (p2-8) -| (p0-4) ;
  \draw[->] (p1-7) -| (p2-8) ;
  \draw[->] (p0-5) -- (p0-9) ;
  \draw (p0-9) -- (p0-11) ;
  \draw[->] (p0-12) |- (p1-13) ;
  \draw[->] (p0-12) |- (p2-13) ;
  \draw[->] (p0-12) |- (p3-13) ;
  \draw[->] (p0-12) |- (p4-13) ;
  \draw[->] (p0-12) |- (p5-13) ;
  \draw[->] (p0-12) |- (p6-13) ;
  \draw[->] (p0-12) |- (p7-13) ;
  \draw[->] (p8-14) -| (p0-10) ;
  \draw[->] (p1-13) -| (p8-14) ;
  \draw[->] (p2-13) -| (p8-14) ;
  \draw[->] (p3-13) -| (p8-14) ;
  \draw[->] (p4-13) -| (p8-14) ;
  \draw[->] (p5-13) -| (p8-14) ;
  \draw[->] (p6-13) -| (p8-14) ;
  \draw[->] (p7-13) -| (p8-14) ;
  \draw[->] (p0-11) -- (p0-15) ;
  \draw[->] (p0-15) -- (p0-16) ;
\end{tikzpicture}

\ruleSubsection{plm\_syntax}{type-extension-declaration}{23}

\begin{tikzpicture}
  \matrix[column sep=\ruleMatrixColumnSeparation, row sep=\ruleMatrixRowSeparation] {
    & & & & & & & & & & & & & & & & \node (p9-16) [point] {}; & \\
    & & & & & & & & \node (p8-8) [terminal] {;}; & \\
    & & & & & & & & \node (p7-8) [nonterminal] {\nonTerminalSymbol{primitive}{3}}; & \\
    & & & & & & & & \node (p6-8) [nonterminal] {\nonTerminalSymbol{guard}{15}}; & \\
    & & & & & & & & \node (p5-8) [nonterminal] {\nonTerminalSymbol{service}{2}}; & \\
    & & & & & & & & \node (p4-8) [nonterminal] {\nonTerminalSymbol{section}{1}}; & \\
    & & & & & & & & \node (p3-8) [nonterminal] {\nonTerminalSymbol{procedure}{0}}; & \\
    & & & & & & & & & \node (p2-9) [terminal] {public}; & \\
    & & & & & & & & \node (p1-8) [point] {}; & \node (p1-9) [point] {}; & \node (p1-10) [point] {}; & \node (p1-11) [terminal] {var}; & \node (p1-12) [terminal] {identifier}; & \node (p1-13) [terminal] {\$type}; & \node (p1-14) [terminal] {=}; & \node (p1-15) [nonterminal] {\nonTerminalSymbol{expression}{16}}; & \\
    \node (P0start) [firstPoint] {}; & & \node (p0-2) [terminal] {extension}; & \node (p0-3) [terminal] {\$type}; & \node (p0-4) [terminal] {\{}; & \node (p0-5) [point] {}; & \node (p0-6) [point] {}; & \node (p0-7) [point] {}; & & & & & & & & & & \node (p0-17) [terminal] {\}}; & \node (p0-18) [lastPoint] {}; & \\
  };
  \draw[->] (P0start) -- (p0-2) ;
  \draw[->] (p0-2) -- (p0-3) ;
  \draw[->] (p0-3) -- (p0-4) ;
  \draw (p0-4) -- (p0-6) ;
  \draw (p0-7) |- (p1-9) ;
  \draw[->] (p1-8) |- (p2-9) ;
  \draw (p1-9) -- (p1-10) ;
  \draw[->] (p2-9) -| (p1-10) ;
  \draw[->] (p1-10) -- (p1-11) ;
  \draw[->] (p1-11) -- (p1-12) ;
  \draw[->] (p1-12) -- (p1-13) ;
  \draw[->] (p1-13) -- (p1-14) ;
  \draw[->] (p1-14) -- (p1-15) ;
  \draw[->] (p0-7) |- (p3-8) ;
  \draw[->] (p0-7) |- (p4-8) ;
  \draw[->] (p0-7) |- (p5-8) ;
  \draw[->] (p0-7) |- (p6-8) ;
  \draw[->] (p0-7) |- (p7-8) ;
  \draw[->] (p0-7) |- (p8-8) ;
  \draw[->] (p9-16) -| (p0-5) ;
  \draw[->] (p1-15) -| (p9-16) ;
  \draw[->] (p3-8) -| (p9-16) ;
  \draw[->] (p4-8) -| (p9-16) ;
  \draw[->] (p5-8) -| (p9-16) ;
  \draw[->] (p6-8) -| (p9-16) ;
  \draw[->] (p7-8) -| (p9-16) ;
  \draw[->] (p8-8) -| (p9-16) ;
  \draw[->] (p0-6) -- (p0-17) ;
  \draw[->] (p0-17) -- (p0-18) ;
\end{tikzpicture}

\ruleSubsection{plm\_syntax}{declaration-control-register}{54}

\begin{tikzpicture}
  \matrix[column sep=\ruleMatrixColumnSeparation, row sep=\ruleMatrixRowSeparation] {
    & & & & & & & & & & & & & & & & & & & & & & & & & & & & & & & & & & & & \node (p4-36) [point] {}; & \\
    & & & & & & & & & & & & & & & & & & & & & \node (p3-21) [point] {}; & & & & & & & & \node (p3-29) [terminal] {[}; & \node (p3-30) [terminal] {integer}; & \node (p3-31) [terminal] {]}; & \\
    & & & & & & & & & \node (p2-9) [point] {}; & & & & & & & & & & & & & & & & & & \node (p2-27) [terminal] {identifier}; & \node (p2-28) [point] {}; & \node (p2-29) [point] {}; & & & \node (p2-32) [point] {}; & & & \node (p2-35) [terminal] {,}; & \\
    & & & & & & & & \node (p1-8) [terminal] {@attribute}; & & & \node (p1-11) [terminal] {[}; & \node (p1-12) [nonterminal] {\nonTerminalSymbol{expression}{16}}; & \node (p1-13) [terminal] {]}; & \node (p1-14) [terminal] {at}; & \node (p1-15) [nonterminal] {\nonTerminalSymbol{expression}{16}}; & \node (p1-16) [terminal] {:}; & \node (p1-17) [nonterminal] {\nonTerminalSymbol{expression}{16}}; & & & \node (p1-20) [point] {}; & & & & \node (p1-24) [terminal] {\{}; & \node (p1-25) [point] {}; & \node (p1-26) [point] {}; & \node (p1-27) [terminal] {integer}; & & & & & & \node (p1-33) [point] {}; & \node (p1-34) [point] {}; & & & \node (p1-37) [terminal] {\}}; & \\
    \node (P0start) [firstPoint] {}; & & \node (p0-2) [terminal] {register}; & \node (p0-3) [point] {}; & \node (p0-4) [terminal] {identifier}; & \node (p0-5) [point] {}; & \node (p0-6) [point] {}; & \node (p0-7) [point] {}; & & & \node (p0-10) [point] {}; & \node (p0-11) [terminal] {at}; & \node (p0-12) [nonterminal] {\nonTerminalSymbol{expression}{16}}; & & & & & & \node (p0-18) [point] {}; & \node (p0-19) [point] {}; & & & \node (p0-22) [terminal] {\$type}; & \node (p0-23) [point] {}; & \node (p0-24) [point] {}; & & & & & & & & & & & & & & \node (p0-38) [point] {}; & \node (p0-39) [lastPoint] {}; & \\
  };
  \draw[->] (P0start) -- (p0-2) ;
  \draw[->] (p0-2) -- (p0-4) ;
  \draw (p0-4) -- (p0-6) ;
  \draw[->] (p0-7) |- (p1-8) ;
  \draw[->] (p2-9) -| (p0-5) ;
  \draw[->] (p1-8) -| (p2-9) ;
  \draw[->] (p0-6) -- (p0-11) ;
  \draw[->] (p0-11) -- (p0-12) ;
  \draw[->] (p0-10) |- (p1-11) ;
  \draw[->] (p1-11) -- (p1-12) ;
  \draw[->] (p1-12) -- (p1-13) ;
  \draw[->] (p1-13) -- (p1-14) ;
  \draw[->] (p1-14) -- (p1-15) ;
  \draw[->] (p1-15) -- (p1-16) ;
  \draw[->] (p1-16) -- (p1-17) ;
  \draw (p0-12) -- (p0-18) ;
  \draw[->] (p1-17) -| (p0-18) ;
  \draw (p0-19) |- (p1-20) ;
  \draw[->] (p3-21) -| (p0-3) ;
  \draw[->] (p1-20) -| (p3-21) ;
  \draw[->] (p0-18) -- (p0-22) ;
  \draw (p0-22) -- (p0-24) ;
  \draw[->] (p0-23) |- (p1-24) ;
  \draw[->] (p1-24) -- (p1-27) ;
  \draw[->] (p1-26) |- (p2-27) ;
  \draw (p2-27) -- (p2-29) ;
  \draw[->] (p2-28) |- (p3-29) ;
  \draw[->] (p3-29) -- (p3-30) ;
  \draw[->] (p3-30) -- (p3-31) ;
  \draw (p2-29) -- (p2-32) ;
  \draw[->] (p3-31) -| (p2-32) ;
  \draw (p1-27) -- (p1-33) ;
  \draw[->] (p2-32) -| (p1-33) ;
  \draw[->] (p1-34) |- (p2-35) ;
  \draw[->] (p4-36) -| (p1-25) ;
  \draw[->] (p2-35) -| (p4-36) ;
  \draw[->] (p1-33) -- (p1-37) ;
  \draw (p0-24) -- (p0-38) ;
  \draw[->] (p1-37) -| (p0-38) ;
  \draw[->] (p0-38) -- (p0-39) ;
\end{tikzpicture}

\ruleSubsection{plm\_syntax}{declaration-global-constant}{25}

\begin{tikzpicture}
  \matrix[column sep=\ruleMatrixColumnSeparation, row sep=\ruleMatrixRowSeparation] {
    & & & & & \node (p1-5) [terminal] {\$type}; & \\
    \node (P0start) [firstPoint] {}; & & \node (p0-2) [terminal] {let}; & \node (p0-3) [terminal] {identifier}; & \node (p0-4) [point] {}; & \node (p0-5) [point] {}; & \node (p0-6) [point] {}; & \node (p0-7) [terminal] {=}; & \node (p0-8) [nonterminal] {\nonTerminalSymbol{expression}{16}}; & \node (p0-9) [lastPoint] {}; & \\
  };
  \draw[->] (P0start) -- (p0-2) ;
  \draw[->] (p0-2) -- (p0-3) ;
  \draw (p0-3) -- (p0-5) ;
  \draw[->] (p0-4) |- (p1-5) ;
  \draw (p0-5) -- (p0-6) ;
  \draw[->] (p1-5) -| (p0-6) ;
  \draw[->] (p0-6) -- (p0-7) ;
  \draw[->] (p0-7) -- (p0-8) ;
  \draw[->] (p0-8) -- (p0-9) ;
\end{tikzpicture}

\ruleSubsection{plm\_syntax}{declaration-global-variable}{71}

\begin{tikzpicture}
  \matrix[column sep=\ruleMatrixColumnSeparation, row sep=\ruleMatrixRowSeparation] {
    \node (P0start) [firstPoint] {}; & & \node (p0-2) [nonterminal] {\nonTerminalSymbol{global\_variable\_declaration}{10}}; & \node (p0-3) [lastPoint] {}; & \\
  };
  \draw[->] (P0start) -- (p0-2) ;
  \draw[->] (p0-2) -- (p0-3) ;
\end{tikzpicture}

\ruleSubsection{plm\_syntax}{declaration-module}{33}

\begin{tikzpicture}
  \matrix[column sep=\ruleMatrixColumnSeparation, row sep=\ruleMatrixRowSeparation] {
    & & & & & & & & & \node (p10-9) [point] {}; & \\
    & & & & & & & & \node (p9-8) [terminal] {;}; & \\
    & & & & & & & & \node (p8-8) [nonterminal] {\nonTerminalSymbol{primitive}{3}}; & \\
    & & & & & & & & \node (p7-8) [nonterminal] {\nonTerminalSymbol{guard}{15}}; & \\
    & & & & & & & & \node (p6-8) [nonterminal] {\nonTerminalSymbol{section}{1}}; & \\
    & & & & & & & & \node (p5-8) [nonterminal] {\nonTerminalSymbol{service}{2}}; & \\
    & & & & & & & & \node (p4-8) [nonterminal] {\nonTerminalSymbol{procedure}{0}}; & \\
    & & & & & & & & \node (p3-8) [nonterminal] {\nonTerminalSymbol{module\_variable}{11}}; & \\
    & & & & & & & & \node (p2-8) [nonterminal] {\nonTerminalSymbol{isr}{4}}; & \\
    & & & & & & & & \node (p1-8) [nonterminal] {\nonTerminalSymbol{declaration\_init}{12}}; & \\
    \node (P0start) [firstPoint] {}; & & \node (p0-2) [terminal] {module}; & \node (p0-3) [terminal] {identifier}; & \node (p0-4) [terminal] {\{}; & \node (p0-5) [point] {}; & \node (p0-6) [point] {}; & \node (p0-7) [point] {}; & & & \node (p0-10) [terminal] {\}}; & \node (p0-11) [lastPoint] {}; & \\
  };
  \draw[->] (P0start) -- (p0-2) ;
  \draw[->] (p0-2) -- (p0-3) ;
  \draw[->] (p0-3) -- (p0-4) ;
  \draw (p0-4) -- (p0-6) ;
  \draw[->] (p0-7) |- (p1-8) ;
  \draw[->] (p0-7) |- (p2-8) ;
  \draw[->] (p0-7) |- (p3-8) ;
  \draw[->] (p0-7) |- (p4-8) ;
  \draw[->] (p0-7) |- (p5-8) ;
  \draw[->] (p0-7) |- (p6-8) ;
  \draw[->] (p0-7) |- (p7-8) ;
  \draw[->] (p0-7) |- (p8-8) ;
  \draw[->] (p0-7) |- (p9-8) ;
  \draw[->] (p10-9) -| (p0-5) ;
  \draw[->] (p1-8) -| (p10-9) ;
  \draw[->] (p2-8) -| (p10-9) ;
  \draw[->] (p3-8) -| (p10-9) ;
  \draw[->] (p4-8) -| (p10-9) ;
  \draw[->] (p5-8) -| (p10-9) ;
  \draw[->] (p6-8) -| (p10-9) ;
  \draw[->] (p7-8) -| (p10-9) ;
  \draw[->] (p8-8) -| (p10-9) ;
  \draw[->] (p9-8) -| (p10-9) ;
  \draw[->] (p0-6) -- (p0-10) ;
  \draw[->] (p0-10) -- (p0-11) ;
\end{tikzpicture}

\ruleSubsection{plm\_syntax}{declaration-task}{51}

\begin{tikzpicture}
  \matrix[column sep=\ruleMatrixColumnSeparation, row sep=\ruleMatrixRowSeparation] {
    & & & & & & & & & & & & & & & & & & & & & & \node (p7-22) [point] {}; & \\
    & & & & & & & & & & & & \node (p6-12) [nonterminal] {\nonTerminalSymbol{guarded\_command}{35}}; & \node (p6-13) [terminal] {\{}; & \node (p6-14) [nonterminal] {\nonTerminalSymbol{instructionList}{31}}; & \node (p6-15) [terminal] {\}}; & \\
    & & & & & & & & & & & & \node (p5-12) [terminal] {init}; & \node (p5-13) [terminal] {integer}; & \node (p5-14) [terminal] {\{}; & \node (p5-15) [nonterminal] {\nonTerminalSymbol{instructionList}{31}}; & \node (p5-16) [terminal] {\}}; & \\
    & & & & & & & & & & & & & & & & \node (p4-16) [terminal] {->}; & \node (p4-17) [terminal] {\$type}; & \\
    & & & & & & & & & & & & \node (p3-12) [terminal] {func}; & \node (p3-13) [terminal] {identifier}; & \node (p3-14) [nonterminal] {\nonTerminalSymbol{procedure\_formal\_arguments}{14}}; & \node (p3-15) [point] {}; & \node (p3-16) [point] {}; & & \node (p3-18) [point] {}; & \node (p3-19) [terminal] {\{}; & \node (p3-20) [nonterminal] {\nonTerminalSymbol{instructionList}{31}}; & \node (p3-21) [terminal] {\}}; & \\
    & & & & & & & & & & & & & & & \node (p2-15) [terminal] {\$type}; & \\
    & & & & & & & & & & & & \node (p1-12) [terminal] {var}; & \node (p1-13) [terminal] {identifier}; & \node (p1-14) [point] {}; & \node (p1-15) [point] {}; & \node (p1-16) [point] {}; & \node (p1-17) [terminal] {=}; & \node (p1-18) [nonterminal] {\nonTerminalSymbol{expression}{16}}; & \\
    \node (P0start) [firstPoint] {}; & & \node (p0-2) [terminal] {task}; & \node (p0-3) [terminal] {identifier}; & \node (p0-4) [terminal] {priority}; & \node (p0-5) [terminal] {integer}; & \node (p0-6) [terminal] {stackSize}; & \node (p0-7) [terminal] {integer}; & \node (p0-8) [terminal] {\{}; & \node (p0-9) [point] {}; & \node (p0-10) [point] {}; & \node (p0-11) [point] {}; & & & & & & & & & & & & \node (p0-23) [terminal] {\}}; & \node (p0-24) [lastPoint] {}; & \\
  };
  \draw[->] (P0start) -- (p0-2) ;
  \draw[->] (p0-2) -- (p0-3) ;
  \draw[->] (p0-3) -- (p0-4) ;
  \draw[->] (p0-4) -- (p0-5) ;
  \draw[->] (p0-5) -- (p0-6) ;
  \draw[->] (p0-6) -- (p0-7) ;
  \draw[->] (p0-7) -- (p0-8) ;
  \draw (p0-8) -- (p0-10) ;
  \draw[->] (p0-11) |- (p1-12) ;
  \draw[->] (p1-12) -- (p1-13) ;
  \draw (p1-13) -- (p1-15) ;
  \draw[->] (p1-14) |- (p2-15) ;
  \draw (p1-15) -- (p1-16) ;
  \draw[->] (p2-15) -| (p1-16) ;
  \draw[->] (p1-16) -- (p1-17) ;
  \draw[->] (p1-17) -- (p1-18) ;
  \draw[->] (p0-11) |- (p3-12) ;
  \draw[->] (p3-12) -- (p3-13) ;
  \draw[->] (p3-13) -- (p3-14) ;
  \draw (p3-14) -- (p3-16) ;
  \draw[->] (p3-15) |- (p4-16) ;
  \draw[->] (p4-16) -- (p4-17) ;
  \draw (p3-16) -- (p3-18) ;
  \draw[->] (p4-17) -| (p3-18) ;
  \draw[->] (p3-18) -- (p3-19) ;
  \draw[->] (p3-19) -- (p3-20) ;
  \draw[->] (p3-20) -- (p3-21) ;
  \draw[->] (p0-11) |- (p5-12) ;
  \draw[->] (p5-12) -- (p5-13) ;
  \draw[->] (p5-13) -- (p5-14) ;
  \draw[->] (p5-14) -- (p5-15) ;
  \draw[->] (p5-15) -- (p5-16) ;
  \draw[->] (p0-11) |- (p6-12) ;
  \draw[->] (p6-12) -- (p6-13) ;
  \draw[->] (p6-13) -- (p6-14) ;
  \draw[->] (p6-14) -- (p6-15) ;
  \draw[->] (p7-22) -| (p0-9) ;
  \draw[->] (p1-18) -| (p7-22) ;
  \draw[->] (p3-21) -| (p7-22) ;
  \draw[->] (p5-16) -| (p7-22) ;
  \draw[->] (p6-15) -| (p7-22) ;
  \draw[->] (p0-10) -- (p0-23) ;
  \draw[->] (p0-23) -- (p0-24) ;
\end{tikzpicture}

\ruleSubsection{plm\_syntax}{panic}{23}

\begin{tikzpicture}
  \matrix[column sep=\ruleMatrixColumnSeparation, row sep=\ruleMatrixRowSeparation] {
    \node (P0start) [firstPoint] {}; & & \node (p0-2) [terminal] {panic}; & \node (p0-3) [terminal] {func}; & \node (p0-4) [terminal] {identifier}; & \node (p0-5) [terminal] {integer}; & \node (p0-6) [terminal] {\{}; & \node (p0-7) [nonterminal] {\nonTerminalSymbol{instructionList}{31}}; & \node (p0-8) [terminal] {\}}; & \node (p0-9) [lastPoint] {}; & \\
  };
  \draw[->] (P0start) -- (p0-2) ;
  \draw[->] (p0-2) -- (p0-3) ;
  \draw[->] (p0-3) -- (p0-4) ;
  \draw[->] (p0-4) -- (p0-5) ;
  \draw[->] (p0-5) -- (p0-6) ;
  \draw[->] (p0-6) -- (p0-7) ;
  \draw[->] (p0-7) -- (p0-8) ;
  \draw[->] (p0-8) -- (p0-9) ;
\end{tikzpicture}

\ruleSubsection{plm\_syntax}{declaration-boot}{23}

\begin{tikzpicture}
  \matrix[column sep=\ruleMatrixColumnSeparation, row sep=\ruleMatrixRowSeparation] {
    \node (P0start) [firstPoint] {}; & & \node (p0-2) [terminal] {boot}; & \node (p0-3) [terminal] {integer}; & \node (p0-4) [terminal] {\{}; & \node (p0-5) [nonterminal] {\nonTerminalSymbol{instructionList}{31}}; & \node (p0-6) [terminal] {\}}; & \node (p0-7) [lastPoint] {}; & \\
  };
  \draw[->] (P0start) -- (p0-2) ;
  \draw[->] (p0-2) -- (p0-3) ;
  \draw[->] (p0-3) -- (p0-4) ;
  \draw[->] (p0-4) -- (p0-5) ;
  \draw[->] (p0-5) -- (p0-6) ;
  \draw[->] (p0-6) -- (p0-7) ;
\end{tikzpicture}

\ruleSubsection{plm\_syntax}{declaration-init}{25}

\begin{tikzpicture}
  \matrix[column sep=\ruleMatrixColumnSeparation, row sep=\ruleMatrixRowSeparation] {
    \node (P0start) [firstPoint] {}; & & \node (p0-2) [nonterminal] {\nonTerminalSymbol{declaration\_init}{12}}; & \node (p0-3) [lastPoint] {}; & \\
  };
  \draw[->] (P0start) -- (p0-2) ;
  \draw[->] (p0-2) -- (p0-3) ;
\end{tikzpicture}

\ruleSubsection{plm\_syntax}{declaration-required-proc}{20}

\begin{tikzpicture}
  \matrix[column sep=\ruleMatrixColumnSeparation, row sep=\ruleMatrixRowSeparation] {
    \node (P0start) [firstPoint] {}; & & \node (p0-2) [terminal] {required}; & \node (p0-3) [nonterminal] {\nonTerminalSymbol{procedure\_header}{13}}; & \node (p0-4) [lastPoint] {}; & \\
  };
  \draw[->] (P0start) -- (p0-2) ;
  \draw[->] (p0-2) -- (p0-3) ;
  \draw[->] (p0-3) -- (p0-4) ;
\end{tikzpicture}

\ruleSubsection{plm\_syntax}{declaration-extern-proc}{22}

\begin{tikzpicture}
  \matrix[column sep=\ruleMatrixColumnSeparation, row sep=\ruleMatrixRowSeparation] {
    & & & & & \node (p1-5) [terminal] {->}; & \node (p1-6) [terminal] {\$type}; & \\
    \node (P0start) [firstPoint] {}; & & \node (p0-2) [terminal] {extern}; & \node (p0-3) [nonterminal] {\nonTerminalSymbol{procedure\_header}{13}}; & \node (p0-4) [point] {}; & \node (p0-5) [point] {}; & & \node (p0-7) [point] {}; & \node (p0-8) [terminal] {:}; & \node (p0-9) [terminal] {"string"}; & \node (p0-10) [lastPoint] {}; & \\
  };
  \draw[->] (P0start) -- (p0-2) ;
  \draw[->] (p0-2) -- (p0-3) ;
  \draw (p0-3) -- (p0-5) ;
  \draw[->] (p0-4) |- (p1-5) ;
  \draw[->] (p1-5) -- (p1-6) ;
  \draw (p0-5) -- (p0-7) ;
  \draw[->] (p1-6) -| (p0-7) ;
  \draw[->] (p0-7) -- (p0-8) ;
  \draw[->] (p0-8) -- (p0-9) ;
  \draw[->] (p0-9) -- (p0-10) ;
\end{tikzpicture}

\ruleSubsection{plm\_syntax}{declaration-guard}{70}

\begin{tikzpicture}
  \matrix[column sep=\ruleMatrixColumnSeparation, row sep=\ruleMatrixRowSeparation] {
    \node (P0start) [firstPoint] {}; & & \node (p0-2) [nonterminal] {\nonTerminalSymbol{guard}{15}}; & \node (p0-3) [lastPoint] {}; & \\
  };
  \draw[->] (P0start) -- (p0-2) ;
  \draw[->] (p0-2) -- (p0-3) ;
\end{tikzpicture}

\ruleSubsection{plm\_syntax}{target-generation}{9}

\begin{tikzpicture}
  \matrix[column sep=\ruleMatrixColumnSeparation, row sep=\ruleMatrixRowSeparation] {
    \node (P0start) [firstPoint] {}; & & \node (p0-2) [terminal] {target}; & \node (p0-3) [terminal] {"string"}; & \node (p0-4) [lastPoint] {}; & \\
  };
  \draw[->] (P0start) -- (p0-2) ;
  \draw[->] (p0-2) -- (p0-3) ;
  \draw[->] (p0-3) -- (p0-4) ;
\end{tikzpicture}

\nonTerminalSection{declaration\_init}{12}

\ruleSubsection{plm\_syntax}{declaration-init}{31}

\begin{tikzpicture}
  \matrix[column sep=\ruleMatrixColumnSeparation, row sep=\ruleMatrixRowSeparation] {
    \node (P0start) [firstPoint] {}; & & \node (p0-2) [terminal] {init}; & \node (p0-3) [terminal] {integer}; & \node (p0-4) [terminal] {\{}; & \node (p0-5) [nonterminal] {\nonTerminalSymbol{instructionList}{31}}; & \node (p0-6) [terminal] {\}}; & \node (p0-7) [lastPoint] {}; & \\
  };
  \draw[->] (P0start) -- (p0-2) ;
  \draw[->] (p0-2) -- (p0-3) ;
  \draw[->] (p0-3) -- (p0-4) ;
  \draw[->] (p0-4) -- (p0-5) ;
  \draw[->] (p0-5) -- (p0-6) ;
  \draw[->] (p0-6) -- (p0-7) ;
\end{tikzpicture}

\nonTerminalSection{declaration\_struct\_var}{9}

\ruleSubsection{plm\_syntax}{type-structure-declaration}{46}

\begin{tikzpicture}
  \matrix[column sep=\ruleMatrixColumnSeparation, row sep=\ruleMatrixRowSeparation] {
    & & & & & & & & \node (p2-8) [terminal] {=}; & \node (p2-9) [nonterminal] {\nonTerminalSymbol{expression}{16}}; & \\
    & & & \node (p1-3) [terminal] {public}; & & & & & & & \node (p1-10) [terminal] {=}; & \node (p1-11) [nonterminal] {\nonTerminalSymbol{expression}{16}}; & \\
    \node (P0start) [firstPoint] {}; & & \node (p0-2) [point] {}; & \node (p0-3) [point] {}; & \node (p0-4) [point] {}; & \node (p0-5) [terminal] {var}; & \node (p0-6) [terminal] {identifier}; & \node (p0-7) [point] {}; & \node (p0-8) [terminal] {\$type}; & \node (p0-9) [point] {}; & \node (p0-10) [point] {}; & & \node (p0-12) [point] {}; & \node (p0-13) [point] {}; & \node (p0-14) [lastPoint] {}; & \\
  };
  \draw (P0start) -- (p0-3) ;
  \draw[->] (p0-2) |- (p1-3) ;
  \draw (p0-3) -- (p0-4) ;
  \draw[->] (p1-3) -| (p0-4) ;
  \draw[->] (p0-4) -- (p0-5) ;
  \draw[->] (p0-5) -- (p0-6) ;
  \draw[->] (p0-6) -- (p0-8) ;
  \draw (p0-8) -- (p0-10) ;
  \draw[->] (p0-9) |- (p1-10) ;
  \draw[->] (p1-10) -- (p1-11) ;
  \draw (p0-10) -- (p0-12) ;
  \draw[->] (p1-11) -| (p0-12) ;
  \draw[->] (p0-7) |- (p2-8) ;
  \draw[->] (p2-8) -- (p2-9) ;
  \draw (p0-12) -- (p0-13) ;
  \draw[->] (p2-9) -| (p0-13) ;
  \draw[->] (p0-13) -- (p0-14) ;
\end{tikzpicture}

\nonTerminalSection{declaration\_type}{8}

\ruleSubsection{plm\_syntax}{type-array}{26}

\begin{tikzpicture}
  \matrix[column sep=\ruleMatrixColumnSeparation, row sep=\ruleMatrixRowSeparation] {
    \node (P0start) [firstPoint] {}; & & \node (p0-2) [terminal] {\$type}; & \node (p0-3) [terminal] {[}; & \node (p0-4) [nonterminal] {\nonTerminalSymbol{expression}{16}}; & \node (p0-5) [terminal] {]}; & \node (p0-6) [lastPoint] {}; & \\
  };
  \draw[->] (P0start) -- (p0-2) ;
  \draw[->] (p0-2) -- (p0-3) ;
  \draw[->] (p0-3) -- (p0-4) ;
  \draw[->] (p0-4) -- (p0-5) ;
  \draw[->] (p0-5) -- (p0-6) ;
\end{tikzpicture}

\ruleSubsection{plm\_syntax}{type-alias}{24}

\begin{tikzpicture}
  \matrix[column sep=\ruleMatrixColumnSeparation, row sep=\ruleMatrixRowSeparation] {
    \node (P0start) [firstPoint] {}; & & \node (p0-2) [terminal] {\$type}; & \node (p0-3) [lastPoint] {}; & \\
  };
  \draw[->] (P0start) -- (p0-2) ;
  \draw[->] (p0-2) -- (p0-3) ;
\end{tikzpicture}

\ruleSubsection{plm\_syntax}{type-opaque-declaration}{26}

\begin{tikzpicture}
  \matrix[column sep=\ruleMatrixColumnSeparation, row sep=\ruleMatrixRowSeparation] {
    & & & & & & & & & & & \node (p2-11) [point] {}; & \\
    & & & & & & & & & & \node (p1-10) [terminal] {@attribute}; & \\
    \node (P0start) [firstPoint] {}; & & \node (p0-2) [terminal] {(}; & \node (p0-3) [terminal] {(}; & \node (p0-4) [nonterminal] {\nonTerminalSymbol{expression}{16}}; & \node (p0-5) [terminal] {)}; & \node (p0-6) [terminal] {)}; & \node (p0-7) [point] {}; & \node (p0-8) [point] {}; & \node (p0-9) [point] {}; & & & \node (p0-12) [lastPoint] {}; & \\
  };
  \draw[->] (P0start) -- (p0-2) ;
  \draw[->] (p0-2) -- (p0-3) ;
  \draw[->] (p0-3) -- (p0-4) ;
  \draw[->] (p0-4) -- (p0-5) ;
  \draw[->] (p0-5) -- (p0-6) ;
  \draw (p0-6) -- (p0-8) ;
  \draw[->] (p0-9) |- (p1-10) ;
  \draw[->] (p2-11) -| (p0-7) ;
  \draw[->] (p1-10) -| (p2-11) ;
  \draw[->] (p0-8) -- (p0-12) ;
\end{tikzpicture}

\nonTerminalSection{effective\_parameters}{30}

\ruleSubsection{plm\_syntax}{expression-primary}{104}

\begin{tikzpicture}
  \matrix[column sep=\ruleMatrixColumnSeparation, row sep=\ruleMatrixRowSeparation] {
    & & & & & & & & & & & & & & \node (p7-14) [point] {}; & \\
    & & & & & & & & \node (p6-8) [terminal] {let}; & & & & \node (p6-12) [terminal] {\$type}; & \\
    & & & & & & \node (p5-6) [terminal] {?}; & \node (p5-7) [point] {}; & \node (p5-8) [terminal] {var}; & \node (p5-9) [point] {}; & \node (p5-10) [terminal] {identifier}; & \node (p5-11) [point] {}; & \node (p5-12) [point] {}; & \node (p5-13) [point] {}; & \\
    & & & & & & \node (p4-6) [terminal] {?}; & \node (p4-7) [terminal] {identifier}; & \\
    & & & & & & \node (p3-6) [terminal] {!?}; & \node (p3-7) [terminal] {self}; & \node (p3-8) [terminal] {.}; & \node (p3-9) [terminal] {identifier}; & \\
    & & & & & & \node (p2-6) [terminal] {!?}; & \node (p2-7) [terminal] {identifier}; & \\
    & & & & & & \node (p1-6) [terminal] {!}; & \node (p1-7) [nonterminal] {\nonTerminalSymbol{expression}{16}}; & \\
    \node (P0start) [firstPoint] {}; & & \node (p0-2) [terminal] {(}; & \node (p0-3) [point] {}; & \node (p0-4) [point] {}; & \node (p0-5) [point] {}; & & & & & & & & & & \node (p0-15) [terminal] {)}; & \node (p0-16) [lastPoint] {}; & \\
  };
  \draw[->] (P0start) -- (p0-2) ;
  \draw (p0-2) -- (p0-4) ;
  \draw[->] (p0-5) |- (p1-6) ;
  \draw[->] (p1-6) -- (p1-7) ;
  \draw[->] (p0-5) |- (p2-6) ;
  \draw[->] (p2-6) -- (p2-7) ;
  \draw[->] (p0-5) |- (p3-6) ;
  \draw[->] (p3-6) -- (p3-7) ;
  \draw[->] (p3-7) -- (p3-8) ;
  \draw[->] (p3-8) -- (p3-9) ;
  \draw[->] (p0-5) |- (p4-6) ;
  \draw[->] (p4-6) -- (p4-7) ;
  \draw[->] (p0-5) |- (p5-6) ;
  \draw[->] (p5-6) -- (p5-8) ;
  \draw[->] (p5-7) |- (p6-8) ;
  \draw (p5-8) -- (p5-9) ;
  \draw[->] (p6-8) -| (p5-9) ;
  \draw[->] (p5-9) -- (p5-10) ;
  \draw (p5-10) -- (p5-12) ;
  \draw[->] (p5-11) |- (p6-12) ;
  \draw (p5-12) -- (p5-13) ;
  \draw[->] (p6-12) -| (p5-13) ;
  \draw[->] (p7-14) -| (p0-3) ;
  \draw[->] (p1-7) -| (p7-14) ;
  \draw[->] (p2-7) -| (p7-14) ;
  \draw[->] (p3-9) -| (p7-14) ;
  \draw[->] (p4-7) -| (p7-14) ;
  \draw[->] (p5-13) -| (p7-14) ;
  \draw[->] (p0-4) -- (p0-15) ;
  \draw[->] (p0-15) -- (p0-16) ;
\end{tikzpicture}

\nonTerminalSection{expression}{16}

\ruleSubsection{plm\_syntax}{expression-operator-priority}{16}

\begin{tikzpicture}
  \matrix[column sep=\ruleMatrixColumnSeparation, row sep=\ruleMatrixRowSeparation] {
    \node (P0start) [firstPoint] {}; & & \node (p0-2) [nonterminal] {\nonTerminalSymbol{expression\_12}{17}}; & \node (p0-3) [lastPoint] {}; & \\
  };
  \draw[->] (P0start) -- (p0-2) ;
  \draw[->] (p0-2) -- (p0-3) ;
\end{tikzpicture}

\nonTerminalSection{expression\_1}{28}

\ruleSubsection{plm\_syntax}{expression-operator-priority}{354}

\begin{tikzpicture}
  \matrix[column sep=\ruleMatrixColumnSeparation, row sep=\ruleMatrixRowSeparation] {
    \node (P0start) [firstPoint] {}; & & \node (p0-2) [nonterminal] {\nonTerminalSymbol{primary}{29}}; & \node (p0-3) [lastPoint] {}; & \\
  };
  \draw[->] (P0start) -- (p0-2) ;
  \draw[->] (p0-2) -- (p0-3) ;
\end{tikzpicture}

\nonTerminalSection{expression\_10}{19}

\ruleSubsection{plm\_syntax}{expression-operator-priority}{63}

\begin{tikzpicture}
  \matrix[column sep=\ruleMatrixColumnSeparation, row sep=\ruleMatrixRowSeparation] {
    & & & & & & & & \node (p2-8) [point] {}; & \\
    & & & & & & \node (p1-6) [terminal] {and}; & \node (p1-7) [nonterminal] {\nonTerminalSymbol{expression\_9}{20}}; & \\
    \node (P0start) [firstPoint] {}; & & \node (p0-2) [nonterminal] {\nonTerminalSymbol{expression\_9}{20}}; & \node (p0-3) [point] {}; & \node (p0-4) [point] {}; & \node (p0-5) [point] {}; & & & & \node (p0-9) [lastPoint] {}; & \\
  };
  \draw[->] (P0start) -- (p0-2) ;
  \draw (p0-2) -- (p0-4) ;
  \draw[->] (p0-5) |- (p1-6) ;
  \draw[->] (p1-6) -- (p1-7) ;
  \draw[->] (p2-8) -| (p0-3) ;
  \draw[->] (p1-7) -| (p2-8) ;
  \draw[->] (p0-4) -- (p0-9) ;
\end{tikzpicture}

\nonTerminalSection{expression\_11}{18}

\ruleSubsection{plm\_syntax}{expression-operator-priority}{45}

\begin{tikzpicture}
  \matrix[column sep=\ruleMatrixColumnSeparation, row sep=\ruleMatrixRowSeparation] {
    & & & & & & & & \node (p2-8) [point] {}; & \\
    & & & & & & \node (p1-6) [terminal] {xor}; & \node (p1-7) [nonterminal] {\nonTerminalSymbol{expression\_10}{19}}; & \\
    \node (P0start) [firstPoint] {}; & & \node (p0-2) [nonterminal] {\nonTerminalSymbol{expression\_10}{19}}; & \node (p0-3) [point] {}; & \node (p0-4) [point] {}; & \node (p0-5) [point] {}; & & & & \node (p0-9) [lastPoint] {}; & \\
  };
  \draw[->] (P0start) -- (p0-2) ;
  \draw (p0-2) -- (p0-4) ;
  \draw[->] (p0-5) |- (p1-6) ;
  \draw[->] (p1-6) -- (p1-7) ;
  \draw[->] (p2-8) -| (p0-3) ;
  \draw[->] (p1-7) -| (p2-8) ;
  \draw[->] (p0-4) -- (p0-9) ;
\end{tikzpicture}

\nonTerminalSection{expression\_12}{17}

\ruleSubsection{plm\_syntax}{expression-operator-priority}{22}

\begin{tikzpicture}
  \matrix[column sep=\ruleMatrixColumnSeparation, row sep=\ruleMatrixRowSeparation] {
    & & & & & & & & \node (p2-8) [point] {}; & \\
    & & & & & & \node (p1-6) [terminal] {or}; & \node (p1-7) [nonterminal] {\nonTerminalSymbol{expression\_11}{18}}; & \\
    \node (P0start) [firstPoint] {}; & & \node (p0-2) [nonterminal] {\nonTerminalSymbol{expression\_11}{18}}; & \node (p0-3) [point] {}; & \node (p0-4) [point] {}; & \node (p0-5) [point] {}; & & & & \node (p0-9) [lastPoint] {}; & \\
  };
  \draw[->] (P0start) -- (p0-2) ;
  \draw (p0-2) -- (p0-4) ;
  \draw[->] (p0-5) |- (p1-6) ;
  \draw[->] (p1-6) -- (p1-7) ;
  \draw[->] (p2-8) -| (p0-3) ;
  \draw[->] (p1-7) -| (p2-8) ;
  \draw[->] (p0-4) -- (p0-9) ;
\end{tikzpicture}

\nonTerminalSection{expression\_2}{27}

\ruleSubsection{plm\_syntax}{expression-operator-priority}{286}

\begin{tikzpicture}
  \matrix[column sep=\ruleMatrixColumnSeparation, row sep=\ruleMatrixRowSeparation] {
    & & & & & & & & \node (p7-8) [point] {}; & \\
    & & & & & & \node (p6-6) [terminal] {!/}; & \node (p6-7) [nonterminal] {\nonTerminalSymbol{expression\_1}{28}}; & \\
    & & & & & & \node (p5-6) [terminal] {/}; & \node (p5-7) [nonterminal] {\nonTerminalSymbol{expression\_1}{28}}; & \\
    & & & & & & \node (p4-6) [terminal] {!\verb=%=}; & \node (p4-7) [nonterminal] {\nonTerminalSymbol{expression\_1}{28}}; & \\
    & & & & & & \node (p3-6) [terminal] {\verb=%=}; & \node (p3-7) [nonterminal] {\nonTerminalSymbol{expression\_1}{28}}; & \\
    & & & & & & \node (p2-6) [terminal] {*\verb=%=}; & \node (p2-7) [nonterminal] {\nonTerminalSymbol{expression\_1}{28}}; & \\
    & & & & & & \node (p1-6) [terminal] {*}; & \node (p1-7) [nonterminal] {\nonTerminalSymbol{expression\_1}{28}}; & \\
    \node (P0start) [firstPoint] {}; & & \node (p0-2) [nonterminal] {\nonTerminalSymbol{expression\_1}{28}}; & \node (p0-3) [point] {}; & \node (p0-4) [point] {}; & \node (p0-5) [point] {}; & & & & \node (p0-9) [lastPoint] {}; & \\
  };
  \draw[->] (P0start) -- (p0-2) ;
  \draw (p0-2) -- (p0-4) ;
  \draw[->] (p0-5) |- (p1-6) ;
  \draw[->] (p1-6) -- (p1-7) ;
  \draw[->] (p0-5) |- (p2-6) ;
  \draw[->] (p2-6) -- (p2-7) ;
  \draw[->] (p0-5) |- (p3-6) ;
  \draw[->] (p3-6) -- (p3-7) ;
  \draw[->] (p0-5) |- (p4-6) ;
  \draw[->] (p4-6) -- (p4-7) ;
  \draw[->] (p0-5) |- (p5-6) ;
  \draw[->] (p5-6) -- (p5-7) ;
  \draw[->] (p0-5) |- (p6-6) ;
  \draw[->] (p6-6) -- (p6-7) ;
  \draw[->] (p7-8) -| (p0-3) ;
  \draw[->] (p1-7) -| (p7-8) ;
  \draw[->] (p2-7) -| (p7-8) ;
  \draw[->] (p3-7) -| (p7-8) ;
  \draw[->] (p4-7) -| (p7-8) ;
  \draw[->] (p5-7) -| (p7-8) ;
  \draw[->] (p6-7) -| (p7-8) ;
  \draw[->] (p0-4) -- (p0-9) ;
\end{tikzpicture}

\nonTerminalSection{expression\_3}{26}

\ruleSubsection{plm\_syntax}{expression-operator-priority}{238}

\begin{tikzpicture}
  \matrix[column sep=\ruleMatrixColumnSeparation, row sep=\ruleMatrixRowSeparation] {
    & & & & & & & & \node (p5-8) [point] {}; & \\
    & & & & & & \node (p4-6) [terminal] {-\verb=%=}; & \node (p4-7) [nonterminal] {\nonTerminalSymbol{expression\_2}{27}}; & \\
    & & & & & & \node (p3-6) [terminal] {-}; & \node (p3-7) [nonterminal] {\nonTerminalSymbol{expression\_2}{27}}; & \\
    & & & & & & \node (p2-6) [terminal] {+\verb=%=}; & \node (p2-7) [nonterminal] {\nonTerminalSymbol{expression\_2}{27}}; & \\
    & & & & & & \node (p1-6) [terminal] {+}; & \node (p1-7) [nonterminal] {\nonTerminalSymbol{expression\_2}{27}}; & \\
    \node (P0start) [firstPoint] {}; & & \node (p0-2) [nonterminal] {\nonTerminalSymbol{expression\_2}{27}}; & \node (p0-3) [point] {}; & \node (p0-4) [point] {}; & \node (p0-5) [point] {}; & & & & \node (p0-9) [lastPoint] {}; & \\
  };
  \draw[->] (P0start) -- (p0-2) ;
  \draw (p0-2) -- (p0-4) ;
  \draw[->] (p0-5) |- (p1-6) ;
  \draw[->] (p1-6) -- (p1-7) ;
  \draw[->] (p0-5) |- (p2-6) ;
  \draw[->] (p2-6) -- (p2-7) ;
  \draw[->] (p0-5) |- (p3-6) ;
  \draw[->] (p3-6) -- (p3-7) ;
  \draw[->] (p0-5) |- (p4-6) ;
  \draw[->] (p4-6) -- (p4-7) ;
  \draw[->] (p5-8) -| (p0-3) ;
  \draw[->] (p1-7) -| (p5-8) ;
  \draw[->] (p2-7) -| (p5-8) ;
  \draw[->] (p3-7) -| (p5-8) ;
  \draw[->] (p4-7) -| (p5-8) ;
  \draw[->] (p0-4) -- (p0-9) ;
\end{tikzpicture}

\nonTerminalSection{expression\_4}{25}

\ruleSubsection{plm\_syntax}{expression-operator-priority}{210}

\begin{tikzpicture}
  \matrix[column sep=\ruleMatrixColumnSeparation, row sep=\ruleMatrixRowSeparation] {
    & & & & & & & & \node (p3-8) [point] {}; & \\
    & & & & & & \node (p2-6) [terminal] {>>}; & \node (p2-7) [nonterminal] {\nonTerminalSymbol{expression\_3}{26}}; & \\
    & & & & & & \node (p1-6) [terminal] {<<}; & \node (p1-7) [nonterminal] {\nonTerminalSymbol{expression\_3}{26}}; & \\
    \node (P0start) [firstPoint] {}; & & \node (p0-2) [nonterminal] {\nonTerminalSymbol{expression\_3}{26}}; & \node (p0-3) [point] {}; & \node (p0-4) [point] {}; & \node (p0-5) [point] {}; & & & & \node (p0-9) [lastPoint] {}; & \\
  };
  \draw[->] (P0start) -- (p0-2) ;
  \draw (p0-2) -- (p0-4) ;
  \draw[->] (p0-5) |- (p1-6) ;
  \draw[->] (p1-6) -- (p1-7) ;
  \draw[->] (p0-5) |- (p2-6) ;
  \draw[->] (p2-6) -- (p2-7) ;
  \draw[->] (p3-8) -| (p0-3) ;
  \draw[->] (p1-7) -| (p3-8) ;
  \draw[->] (p2-7) -| (p3-8) ;
  \draw[->] (p0-4) -- (p0-9) ;
\end{tikzpicture}

\nonTerminalSection{expression\_5}{24}

\ruleSubsection{plm\_syntax}{expression-operator-priority}{162}

\begin{tikzpicture}
  \matrix[column sep=\ruleMatrixColumnSeparation, row sep=\ruleMatrixRowSeparation] {
    & & & & \node (p4-4) [terminal] {>}; & \node (p4-5) [nonterminal] {\nonTerminalSymbol{expression\_4}{25}}; & \\
    & & & & \node (p3-4) [terminal] {<}; & \node (p3-5) [nonterminal] {\nonTerminalSymbol{expression\_4}{25}}; & \\
    & & & & \node (p2-4) [terminal] {>=}; & \node (p2-5) [nonterminal] {\nonTerminalSymbol{expression\_4}{25}}; & \\
    & & & & \node (p1-4) [terminal] {<=}; & \node (p1-5) [nonterminal] {\nonTerminalSymbol{expression\_4}{25}}; & \\
    \node (P0start) [firstPoint] {}; & & \node (p0-2) [nonterminal] {\nonTerminalSymbol{expression\_4}{25}}; & \node (p0-3) [point] {}; & \node (p0-4) [point] {}; & & \node (p0-6) [point] {}; & \node (p0-7) [lastPoint] {}; & \\
  };
  \draw[->] (P0start) -- (p0-2) ;
  \draw (p0-2) -- (p0-4) ;
  \draw[->] (p0-3) |- (p1-4) ;
  \draw[->] (p1-4) -- (p1-5) ;
  \draw[->] (p0-3) |- (p2-4) ;
  \draw[->] (p2-4) -- (p2-5) ;
  \draw[->] (p0-3) |- (p3-4) ;
  \draw[->] (p3-4) -- (p3-5) ;
  \draw[->] (p0-3) |- (p4-4) ;
  \draw[->] (p4-4) -- (p4-5) ;
  \draw (p0-4) -- (p0-6) ;
  \draw[->] (p1-5) -| (p0-6) ;
  \draw[->] (p2-5) -| (p0-6) ;
  \draw[->] (p3-5) -| (p0-6) ;
  \draw[->] (p4-5) -| (p0-6) ;
  \draw[->] (p0-6) -- (p0-7) ;
\end{tikzpicture}

\nonTerminalSection{expression\_6}{23}

\ruleSubsection{plm\_syntax}{expression-operator-priority}{134}

\begin{tikzpicture}
  \matrix[column sep=\ruleMatrixColumnSeparation, row sep=\ruleMatrixRowSeparation] {
    & & & & \node (p2-4) [terminal] {!=}; & \node (p2-5) [nonterminal] {\nonTerminalSymbol{expression\_5}{24}}; & \\
    & & & & \node (p1-4) [terminal] {==}; & \node (p1-5) [nonterminal] {\nonTerminalSymbol{expression\_5}{24}}; & \\
    \node (P0start) [firstPoint] {}; & & \node (p0-2) [nonterminal] {\nonTerminalSymbol{expression\_5}{24}}; & \node (p0-3) [point] {}; & \node (p0-4) [point] {}; & & \node (p0-6) [point] {}; & \node (p0-7) [lastPoint] {}; & \\
  };
  \draw[->] (P0start) -- (p0-2) ;
  \draw (p0-2) -- (p0-4) ;
  \draw[->] (p0-3) |- (p1-4) ;
  \draw[->] (p1-4) -- (p1-5) ;
  \draw[->] (p0-3) |- (p2-4) ;
  \draw[->] (p2-4) -- (p2-5) ;
  \draw (p0-4) -- (p0-6) ;
  \draw[->] (p1-5) -| (p0-6) ;
  \draw[->] (p2-5) -| (p0-6) ;
  \draw[->] (p0-6) -- (p0-7) ;
\end{tikzpicture}

\nonTerminalSection{expression\_7}{22}

\ruleSubsection{plm\_syntax}{expression-operator-priority}{116}

\begin{tikzpicture}
  \matrix[column sep=\ruleMatrixColumnSeparation, row sep=\ruleMatrixRowSeparation] {
    & & & & & & & & \node (p2-8) [point] {}; & \\
    & & & & & & \node (p1-6) [terminal] {\&}; & \node (p1-7) [nonterminal] {\nonTerminalSymbol{expression\_6}{23}}; & \\
    \node (P0start) [firstPoint] {}; & & \node (p0-2) [nonterminal] {\nonTerminalSymbol{expression\_6}{23}}; & \node (p0-3) [point] {}; & \node (p0-4) [point] {}; & \node (p0-5) [point] {}; & & & & \node (p0-9) [lastPoint] {}; & \\
  };
  \draw[->] (P0start) -- (p0-2) ;
  \draw (p0-2) -- (p0-4) ;
  \draw[->] (p0-5) |- (p1-6) ;
  \draw[->] (p1-6) -- (p1-7) ;
  \draw[->] (p2-8) -| (p0-3) ;
  \draw[->] (p1-7) -| (p2-8) ;
  \draw[->] (p0-4) -- (p0-9) ;
\end{tikzpicture}

\nonTerminalSection{expression\_8}{21}

\ruleSubsection{plm\_syntax}{expression-operator-priority}{98}

\begin{tikzpicture}
  \matrix[column sep=\ruleMatrixColumnSeparation, row sep=\ruleMatrixRowSeparation] {
    & & & & & & & & \node (p2-8) [point] {}; & \\
    & & & & & & \node (p1-6) [terminal] {\verb=^=}; & \node (p1-7) [nonterminal] {\nonTerminalSymbol{expression\_7}{22}}; & \\
    \node (P0start) [firstPoint] {}; & & \node (p0-2) [nonterminal] {\nonTerminalSymbol{expression\_7}{22}}; & \node (p0-3) [point] {}; & \node (p0-4) [point] {}; & \node (p0-5) [point] {}; & & & & \node (p0-9) [lastPoint] {}; & \\
  };
  \draw[->] (P0start) -- (p0-2) ;
  \draw (p0-2) -- (p0-4) ;
  \draw[->] (p0-5) |- (p1-6) ;
  \draw[->] (p1-6) -- (p1-7) ;
  \draw[->] (p2-8) -| (p0-3) ;
  \draw[->] (p1-7) -| (p2-8) ;
  \draw[->] (p0-4) -- (p0-9) ;
\end{tikzpicture}

\nonTerminalSection{expression\_9}{20}

\ruleSubsection{plm\_syntax}{expression-operator-priority}{80}

\begin{tikzpicture}
  \matrix[column sep=\ruleMatrixColumnSeparation, row sep=\ruleMatrixRowSeparation] {
    & & & & & & & & \node (p2-8) [point] {}; & \\
    & & & & & & \node (p1-6) [terminal] {|}; & \node (p1-7) [nonterminal] {\nonTerminalSymbol{expression\_8}{21}}; & \\
    \node (P0start) [firstPoint] {}; & & \node (p0-2) [nonterminal] {\nonTerminalSymbol{expression\_8}{21}}; & \node (p0-3) [point] {}; & \node (p0-4) [point] {}; & \node (p0-5) [point] {}; & & & & \node (p0-9) [lastPoint] {}; & \\
  };
  \draw[->] (P0start) -- (p0-2) ;
  \draw (p0-2) -- (p0-4) ;
  \draw[->] (p0-5) |- (p1-6) ;
  \draw[->] (p1-6) -- (p1-7) ;
  \draw[->] (p2-8) -| (p0-3) ;
  \draw[->] (p1-7) -| (p2-8) ;
  \draw[->] (p0-4) -- (p0-9) ;
\end{tikzpicture}

\nonTerminalSection{global\_variable\_declaration}{10}

\ruleSubsection{plm\_syntax}{declaration-global-variable}{77}

\begin{tikzpicture}
  \matrix[column sep=\ruleMatrixColumnSeparation, row sep=\ruleMatrixRowSeparation] {
    & & & & & & & & & & & & & & & & & & & & & & & & & \node (p11-25) [point] {}; & \\
    & & & & & & & & & & & & & & & & \node (p10-16) [terminal] {panic}; & \node (p10-17) [terminal] {func}; & \node (p10-18) [terminal] {identifier}; & \node (p10-19) [terminal] {integer}; & \\
    & & & & & & & & & & & & & & & & \node (p9-16) [terminal] {task}; & \node (p9-17) [terminal] {identifier}; & \\
    & & & & & & & & & & & & & & & & \node (p8-16) [terminal] {isr}; & \node (p8-17) [terminal] {identifier}; & \\
    & & & & & & & & & & & & & & & & \node (p7-16) [terminal] {init}; & \node (p7-17) [terminal] {integer}; & \\
    & & & & & & & & & & & & & & & & \node (p6-16) [terminal] {func}; & \node (p6-17) [terminal] {\$type}; & \node (p6-18) [terminal] {.}; & \node (p6-19) [terminal] {identifier}; & \\
    & & & & & & & & & & & & & & & & \node (p5-16) [terminal] {func}; & \node (p5-17) [terminal] {identifier}; & \\
    & & & & & & & & & & & & & & & & & & \node (p4-18) [terminal] {\$type}; & \node (p4-19) [terminal] {.}; & \\
    & & & & & & & & & & & & & & & & \node (p3-16) [terminal] {section}; & \node (p3-17) [point] {}; & \node (p3-18) [point] {}; & & \node (p3-20) [point] {}; & \node (p3-21) [terminal] {identifier}; & \\
    & & & & & & & & & & & & & \node (p2-13) [terminal] {@attribute}; & & & & & \node (p2-18) [terminal] {\$type}; & \node (p2-19) [terminal] {.}; & & & & & \node (p2-24) [point] {}; & \\
    & & & & & \node (p1-5) [terminal] {\$type}; & & & & & \node (p1-10) [terminal] {\{}; & \node (p1-11) [point] {}; & \node (p1-12) [point] {}; & \node (p1-13) [point] {}; & \node (p1-14) [point] {}; & \node (p1-15) [point] {}; & \node (p1-16) [terminal] {guard}; & \node (p1-17) [point] {}; & \node (p1-18) [point] {}; & & \node (p1-20) [point] {}; & \node (p1-21) [terminal] {identifier}; & \node (p1-22) [point] {}; & \node (p1-23) [point] {}; & & & \node (p1-26) [terminal] {\}}; & \\
    \node (P0start) [firstPoint] {}; & & \node (p0-2) [terminal] {var}; & \node (p0-3) [terminal] {identifier}; & \node (p0-4) [point] {}; & \node (p0-5) [point] {}; & \node (p0-6) [point] {}; & \node (p0-7) [terminal] {=}; & \node (p0-8) [nonterminal] {\nonTerminalSymbol{expression}{16}}; & \node (p0-9) [point] {}; & \node (p0-10) [point] {}; & & & & & & & & & & & & & & & & & \node (p0-27) [point] {}; & \node (p0-28) [lastPoint] {}; & \\
  };
  \draw[->] (P0start) -- (p0-2) ;
  \draw[->] (p0-2) -- (p0-3) ;
  \draw (p0-3) -- (p0-5) ;
  \draw[->] (p0-4) |- (p1-5) ;
  \draw (p0-5) -- (p0-6) ;
  \draw[->] (p1-5) -| (p0-6) ;
  \draw[->] (p0-6) -- (p0-7) ;
  \draw[->] (p0-7) -- (p0-8) ;
  \draw (p0-8) -- (p0-10) ;
  \draw[->] (p0-9) |- (p1-10) ;
  \draw (p1-10) -- (p1-13) ;
  \draw[->] (p1-12) |- (p2-13) ;
  \draw (p1-13) -- (p1-14) ;
  \draw[->] (p2-13) -| (p1-14) ;
  \draw[->] (p1-14) -- (p1-16) ;
  \draw (p1-16) -- (p1-18) ;
  \draw[->] (p1-17) |- (p2-18) ;
  \draw[->] (p2-18) -- (p2-19) ;
  \draw (p1-18) -- (p1-20) ;
  \draw[->] (p2-19) -| (p1-20) ;
  \draw[->] (p1-20) -- (p1-21) ;
  \draw[->] (p1-15) |- (p3-16) ;
  \draw (p3-16) -- (p3-18) ;
  \draw[->] (p3-17) |- (p4-18) ;
  \draw[->] (p4-18) -- (p4-19) ;
  \draw (p3-18) -- (p3-20) ;
  \draw[->] (p4-19) -| (p3-20) ;
  \draw[->] (p3-20) -- (p3-21) ;
  \draw[->] (p1-15) |- (p5-16) ;
  \draw[->] (p5-16) -- (p5-17) ;
  \draw[->] (p1-15) |- (p6-16) ;
  \draw[->] (p6-16) -- (p6-17) ;
  \draw[->] (p6-17) -- (p6-18) ;
  \draw[->] (p6-18) -- (p6-19) ;
  \draw[->] (p1-15) |- (p7-16) ;
  \draw[->] (p7-16) -- (p7-17) ;
  \draw[->] (p1-15) |- (p8-16) ;
  \draw[->] (p8-16) -- (p8-17) ;
  \draw[->] (p1-15) |- (p9-16) ;
  \draw[->] (p9-16) -- (p9-17) ;
  \draw[->] (p1-15) |- (p10-16) ;
  \draw[->] (p10-16) -- (p10-17) ;
  \draw[->] (p10-17) -- (p10-18) ;
  \draw[->] (p10-18) -- (p10-19) ;
  \draw (p1-21) -- (p1-22) ;
  \draw[->] (p3-21) -| (p1-22) ;
  \draw[->] (p5-17) -| (p1-22) ;
  \draw[->] (p6-19) -| (p1-22) ;
  \draw[->] (p7-17) -| (p1-22) ;
  \draw[->] (p8-17) -| (p1-22) ;
  \draw[->] (p9-17) -| (p1-22) ;
  \draw[->] (p10-19) -| (p1-22) ;
  \draw (p1-23) |- (p2-24) ;
  \draw[->] (p11-25) -| (p1-11) ;
  \draw[->] (p2-24) -| (p11-25) ;
  \draw[->] (p1-22) -- (p1-26) ;
  \draw (p0-10) -- (p0-27) ;
  \draw[->] (p1-26) -| (p0-27) ;
  \draw[->] (p0-27) -- (p0-28) ;
\end{tikzpicture}

\nonTerminalSection{guard}{15}

\ruleSubsection{plm\_syntax}{declaration-guard}{30}

\begin{tikzpicture}
  \matrix[column sep=\ruleMatrixColumnSeparation, row sep=\ruleMatrixRowSeparation] {
    & & & & & & & & & & & \node (p2-11) [point] {}; & \\
    & & & \node (p1-3) [terminal] {public}; & & & & & & & \node (p1-10) [terminal] {@attribute}; & & & & \node (p1-14) [terminal] {:}; & \node (p1-15) [nonterminal] {\nonTerminalSymbol{procedure\_call}{36}}; & \\
    \node (P0start) [firstPoint] {}; & & \node (p0-2) [point] {}; & \node (p0-3) [point] {}; & \node (p0-4) [point] {}; & \node (p0-5) [terminal] {guard}; & \node (p0-6) [terminal] {identifier}; & \node (p0-7) [point] {}; & \node (p0-8) [point] {}; & \node (p0-9) [point] {}; & & & \node (p0-12) [nonterminal] {\nonTerminalSymbol{procedure\_formal\_arguments}{14}}; & \node (p0-13) [point] {}; & \node (p0-14) [point] {}; & & \node (p0-16) [point] {}; & \node (p0-17) [terminal] {\{}; & \node (p0-18) [nonterminal] {\nonTerminalSymbol{instructionList}{31}}; & \node (p0-19) [terminal] {\}}; & \node (p0-20) [lastPoint] {}; & \\
  };
  \draw (P0start) -- (p0-3) ;
  \draw[->] (p0-2) |- (p1-3) ;
  \draw (p0-3) -- (p0-4) ;
  \draw[->] (p1-3) -| (p0-4) ;
  \draw[->] (p0-4) -- (p0-5) ;
  \draw[->] (p0-5) -- (p0-6) ;
  \draw (p0-6) -- (p0-8) ;
  \draw[->] (p0-9) |- (p1-10) ;
  \draw[->] (p2-11) -| (p0-7) ;
  \draw[->] (p1-10) -| (p2-11) ;
  \draw[->] (p0-8) -- (p0-12) ;
  \draw (p0-12) -- (p0-14) ;
  \draw[->] (p0-13) |- (p1-14) ;
  \draw[->] (p1-14) -- (p1-15) ;
  \draw (p0-14) -- (p0-16) ;
  \draw[->] (p1-15) -| (p0-16) ;
  \draw[->] (p0-16) -- (p0-17) ;
  \draw[->] (p0-17) -- (p0-18) ;
  \draw[->] (p0-18) -- (p0-19) ;
  \draw[->] (p0-19) -- (p0-20) ;
\end{tikzpicture}

\nonTerminalSection{guarded\_command}{35}

\ruleSubsection{plm\_syntax}{instruction-sync}{47}

\begin{tikzpicture}
  \matrix[column sep=\ruleMatrixColumnSeparation, row sep=\ruleMatrixRowSeparation] {
    & & & & \node (p3-4) [terminal] {while}; & & & & \node (p3-8) [terminal] {.}; & \node (p3-9) [terminal] {identifier}; & \node (p3-10) [nonterminal] {\nonTerminalSymbol{effective\_parameters}{30}}; & \\
    & & & \node (p2-3) [point] {}; & \node (p2-4) [terminal] {until}; & \node (p2-5) [point] {}; & \node (p2-6) [terminal] {identifier}; & \node (p2-7) [point] {}; & \node (p2-8) [nonterminal] {\nonTerminalSymbol{effective\_parameters}{30}}; & & & \node (p2-11) [point] {}; & \\
    & & & & & & \node (p1-6) [terminal] {while}; & & & & \node (p1-10) [terminal] {.}; & \node (p1-11) [terminal] {identifier}; & \node (p1-12) [nonterminal] {\nonTerminalSymbol{effective\_parameters}{30}}; & \\
    \node (P0start) [firstPoint] {}; & & \node (p0-2) [point] {}; & \node (p0-3) [terminal] {when}; & \node (p0-4) [nonterminal] {\nonTerminalSymbol{expression}{16}}; & \node (p0-5) [point] {}; & \node (p0-6) [terminal] {until}; & \node (p0-7) [point] {}; & \node (p0-8) [terminal] {identifier}; & \node (p0-9) [point] {}; & \node (p0-10) [nonterminal] {\nonTerminalSymbol{effective\_parameters}{30}}; & & & \node (p0-13) [point] {}; & \node (p0-14) [point] {}; & \node (p0-15) [lastPoint] {}; & \\
  };
  \draw[->] (P0start) -- (p0-3) ;
  \draw[->] (p0-3) -- (p0-4) ;
  \draw[->] (p0-4) -- (p0-6) ;
  \draw[->] (p0-5) |- (p1-6) ;
  \draw (p0-6) -- (p0-7) ;
  \draw[->] (p1-6) -| (p0-7) ;
  \draw[->] (p0-7) -- (p0-8) ;
  \draw[->] (p0-8) -- (p0-10) ;
  \draw[->] (p0-9) |- (p1-10) ;
  \draw[->] (p1-10) -- (p1-11) ;
  \draw[->] (p1-11) -- (p1-12) ;
  \draw (p0-10) -- (p0-13) ;
  \draw[->] (p1-12) -| (p0-13) ;
  \draw[->] (p0-2) |- (p2-4) ;
  \draw[->] (p2-3) |- (p3-4) ;
  \draw (p2-4) -- (p2-5) ;
  \draw[->] (p3-4) -| (p2-5) ;
  \draw[->] (p2-5) -- (p2-6) ;
  \draw[->] (p2-6) -- (p2-8) ;
  \draw[->] (p2-7) |- (p3-8) ;
  \draw[->] (p3-8) -- (p3-9) ;
  \draw[->] (p3-9) -- (p3-10) ;
  \draw (p2-8) -- (p2-11) ;
  \draw[->] (p3-10) -| (p2-11) ;
  \draw (p0-13) -- (p0-14) ;
  \draw[->] (p2-11) -| (p0-14) ;
  \draw[->] (p0-14) -- (p0-15) ;
\end{tikzpicture}

\nonTerminalSection{if\_instruction}{34}

\ruleSubsection{plm\_syntax}{instruction-if}{49}

\begin{tikzpicture}
  \matrix[column sep=\ruleMatrixColumnSeparation, row sep=\ruleMatrixRowSeparation] {
    & & & & & & & & & & \node (p4-10) [terminal] {@attribute}; & \\
    & & & & & & & \node (p3-7) [terminal] {else}; & \node (p3-8) [terminal] {if}; & \node (p3-9) [point] {}; & \node (p3-10) [point] {}; & \node (p3-11) [point] {}; & \node (p3-12) [nonterminal] {\nonTerminalSymbol{if\_instruction}{34}}; & \\
    & & & & & & & & & & \node (p2-10) [terminal] {@attribute}; & \\
    & & & & & & & \node (p1-7) [terminal] {else}; & \node (p1-8) [terminal] {\{}; & \node (p1-9) [point] {}; & \node (p1-10) [point] {}; & \node (p1-11) [point] {}; & \node (p1-12) [nonterminal] {\nonTerminalSymbol{instructionList}{31}}; & \node (p1-13) [terminal] {\}}; & \\
    \node (P0start) [firstPoint] {}; & & \node (p0-2) [nonterminal] {\nonTerminalSymbol{expression}{16}}; & \node (p0-3) [terminal] {\{}; & \node (p0-4) [nonterminal] {\nonTerminalSymbol{instructionList}{31}}; & \node (p0-5) [terminal] {\}}; & \node (p0-6) [point] {}; & \node (p0-7) [point] {}; & & & & & & & \node (p0-14) [point] {}; & \node (p0-15) [lastPoint] {}; & \\
  };
  \draw[->] (P0start) -- (p0-2) ;
  \draw[->] (p0-2) -- (p0-3) ;
  \draw[->] (p0-3) -- (p0-4) ;
  \draw[->] (p0-4) -- (p0-5) ;
  \draw (p0-5) -- (p0-7) ;
  \draw[->] (p0-6) |- (p1-7) ;
  \draw[->] (p1-7) -- (p1-8) ;
  \draw (p1-8) -- (p1-10) ;
  \draw[->] (p1-9) |- (p2-10) ;
  \draw (p1-10) -- (p1-11) ;
  \draw[->] (p2-10) -| (p1-11) ;
  \draw[->] (p1-11) -- (p1-12) ;
  \draw[->] (p1-12) -- (p1-13) ;
  \draw[->] (p0-6) |- (p3-7) ;
  \draw[->] (p3-7) -- (p3-8) ;
  \draw (p3-8) -- (p3-10) ;
  \draw[->] (p3-9) |- (p4-10) ;
  \draw (p3-10) -- (p3-11) ;
  \draw[->] (p4-10) -| (p3-11) ;
  \draw[->] (p3-11) -- (p3-12) ;
  \draw (p0-7) -- (p0-14) ;
  \draw[->] (p1-13) -| (p0-14) ;
  \draw[->] (p3-12) -| (p0-14) ;
  \draw[->] (p0-14) -- (p0-15) ;
\end{tikzpicture}

\nonTerminalSection{import\_file}{5}

\ruleSubsection{plm\_syntax}{syntax-grammar}{19}

\begin{tikzpicture}
  \matrix[column sep=\ruleMatrixColumnSeparation, row sep=\ruleMatrixRowSeparation] {
    \node (P0start) [firstPoint] {}; & & \node (p0-2) [terminal] {import}; & \node (p0-3) [terminal] {"string"}; & \node (p0-4) [lastPoint] {}; & \\
  };
  \draw[->] (P0start) -- (p0-2) ;
  \draw[->] (p0-2) -- (p0-3) ;
  \draw[->] (p0-3) -- (p0-4) ;
\end{tikzpicture}

\nonTerminalSection{instruction}{32}

\ruleSubsection{plm\_syntax}{directive-check}{18}

\begin{tikzpicture}
  \matrix[column sep=\ruleMatrixColumnSeparation, row sep=\ruleMatrixRowSeparation] {
    \node (P0start) [firstPoint] {}; & & \node (p0-2) [terminal] {check}; & \node (p0-3) [nonterminal] {\nonTerminalSymbol{expression}{16}}; & \node (p0-4) [lastPoint] {}; & \\
  };
  \draw[->] (P0start) -- (p0-2) ;
  \draw[->] (p0-2) -- (p0-3) ;
  \draw[->] (p0-3) -- (p0-4) ;
\end{tikzpicture}

\ruleSubsection{plm\_syntax}{instruction-assignment}{18}

\begin{tikzpicture}
  \matrix[column sep=\ruleMatrixColumnSeparation, row sep=\ruleMatrixRowSeparation] {
    \node (P0start) [firstPoint] {}; & & \node (p0-2) [nonterminal] {\nonTerminalSymbol{assignment\_target}{37}}; & \node (p0-3) [terminal] {=}; & \node (p0-4) [nonterminal] {\nonTerminalSymbol{expression}{16}}; & \node (p0-5) [lastPoint] {}; & \\
  };
  \draw[->] (P0start) -- (p0-2) ;
  \draw[->] (p0-2) -- (p0-3) ;
  \draw[->] (p0-3) -- (p0-4) ;
  \draw[->] (p0-4) -- (p0-5) ;
\end{tikzpicture}

\ruleSubsection{plm\_syntax}{instruction-assignment-operator}{70}

\begin{tikzpicture}
  \matrix[column sep=\ruleMatrixColumnSeparation, row sep=\ruleMatrixRowSeparation] {
    \node (P0start) [firstPoint] {}; & & \node (p0-2) [nonterminal] {\nonTerminalSymbol{assignment\_target}{37}}; & \node (p0-3) [nonterminal] {\nonTerminalSymbol{assignment\_operator}{33}}; & \node (p0-4) [nonterminal] {\nonTerminalSymbol{expression}{16}}; & \node (p0-5) [lastPoint] {}; & \\
  };
  \draw[->] (P0start) -- (p0-2) ;
  \draw[->] (p0-2) -- (p0-3) ;
  \draw[->] (p0-3) -- (p0-4) ;
  \draw[->] (p0-4) -- (p0-5) ;
\end{tikzpicture}

\ruleSubsection{plm\_syntax}{instruction-var}{26}

\begin{tikzpicture}
  \matrix[column sep=\ruleMatrixColumnSeparation, row sep=\ruleMatrixRowSeparation] {
    & & & & & \node (p1-5) [terminal] {\$type}; & \\
    \node (P0start) [firstPoint] {}; & & \node (p0-2) [terminal] {var}; & \node (p0-3) [terminal] {identifier}; & \node (p0-4) [point] {}; & \node (p0-5) [point] {}; & \node (p0-6) [point] {}; & \node (p0-7) [terminal] {=}; & \node (p0-8) [nonterminal] {\nonTerminalSymbol{expression}{16}}; & \node (p0-9) [lastPoint] {}; & \\
  };
  \draw[->] (P0start) -- (p0-2) ;
  \draw[->] (p0-2) -- (p0-3) ;
  \draw (p0-3) -- (p0-5) ;
  \draw[->] (p0-4) |- (p1-5) ;
  \draw (p0-5) -- (p0-6) ;
  \draw[->] (p1-5) -| (p0-6) ;
  \draw[->] (p0-6) -- (p0-7) ;
  \draw[->] (p0-7) -- (p0-8) ;
  \draw[->] (p0-8) -- (p0-9) ;
\end{tikzpicture}

\ruleSubsection{plm\_syntax}{instruction-var}{46}

\begin{tikzpicture}
  \matrix[column sep=\ruleMatrixColumnSeparation, row sep=\ruleMatrixRowSeparation] {
    \node (P0start) [firstPoint] {}; & & \node (p0-2) [terminal] {var}; & \node (p0-3) [terminal] {identifier}; & \node (p0-4) [terminal] {\$type}; & \node (p0-5) [lastPoint] {}; & \\
  };
  \draw[->] (P0start) -- (p0-2) ;
  \draw[->] (p0-2) -- (p0-3) ;
  \draw[->] (p0-3) -- (p0-4) ;
  \draw[->] (p0-4) -- (p0-5) ;
\end{tikzpicture}

\ruleSubsection{plm\_syntax}{instruction-let}{19}

\begin{tikzpicture}
  \matrix[column sep=\ruleMatrixColumnSeparation, row sep=\ruleMatrixRowSeparation] {
    & & & & & \node (p1-5) [terminal] {\$type}; & \\
    \node (P0start) [firstPoint] {}; & & \node (p0-2) [terminal] {let}; & \node (p0-3) [terminal] {identifier}; & \node (p0-4) [point] {}; & \node (p0-5) [point] {}; & \node (p0-6) [point] {}; & \node (p0-7) [terminal] {=}; & \node (p0-8) [nonterminal] {\nonTerminalSymbol{expression}{16}}; & \node (p0-9) [lastPoint] {}; & \\
  };
  \draw[->] (P0start) -- (p0-2) ;
  \draw[->] (p0-2) -- (p0-3) ;
  \draw (p0-3) -- (p0-5) ;
  \draw[->] (p0-4) |- (p1-5) ;
  \draw (p0-5) -- (p0-6) ;
  \draw[->] (p1-5) -| (p0-6) ;
  \draw[->] (p0-6) -- (p0-7) ;
  \draw[->] (p0-7) -- (p0-8) ;
  \draw[->] (p0-8) -- (p0-9) ;
\end{tikzpicture}

\ruleSubsection{plm\_syntax}{instruction-assert}{18}

\begin{tikzpicture}
  \matrix[column sep=\ruleMatrixColumnSeparation, row sep=\ruleMatrixRowSeparation] {
    \node (P0start) [firstPoint] {}; & & \node (p0-2) [terminal] {assert}; & \node (p0-3) [nonterminal] {\nonTerminalSymbol{expression}{16}}; & \node (p0-4) [lastPoint] {}; & \\
  };
  \draw[->] (P0start) -- (p0-2) ;
  \draw[->] (p0-2) -- (p0-3) ;
  \draw[->] (p0-3) -- (p0-4) ;
\end{tikzpicture}

\ruleSubsection{plm\_syntax}{instruction-panic}{18}

\begin{tikzpicture}
  \matrix[column sep=\ruleMatrixColumnSeparation, row sep=\ruleMatrixRowSeparation] {
    \node (P0start) [firstPoint] {}; & & \node (p0-2) [terminal] {panic}; & \node (p0-3) [nonterminal] {\nonTerminalSymbol{expression}{16}}; & \node (p0-4) [lastPoint] {}; & \\
  };
  \draw[->] (P0start) -- (p0-2) ;
  \draw[->] (p0-2) -- (p0-3) ;
  \draw[->] (p0-3) -- (p0-4) ;
\end{tikzpicture}

\ruleSubsection{plm\_syntax}{instruction-if}{23}

\begin{tikzpicture}
  \matrix[column sep=\ruleMatrixColumnSeparation, row sep=\ruleMatrixRowSeparation] {
    & & & & \node (p1-4) [terminal] {@attribute}; & & & & \node (p1-8) [terminal] {@attribute}; & \\
    \node (P0start) [firstPoint] {}; & & \node (p0-2) [terminal] {if}; & \node (p0-3) [point] {}; & \node (p0-4) [point] {}; & \node (p0-5) [point] {}; & \node (p0-6) [nonterminal] {\nonTerminalSymbol{if\_instruction}{34}}; & \node (p0-7) [point] {}; & \node (p0-8) [point] {}; & \node (p0-9) [point] {}; & \node (p0-10) [lastPoint] {}; & \\
  };
  \draw[->] (P0start) -- (p0-2) ;
  \draw (p0-2) -- (p0-4) ;
  \draw[->] (p0-3) |- (p1-4) ;
  \draw (p0-4) -- (p0-5) ;
  \draw[->] (p1-4) -| (p0-5) ;
  \draw[->] (p0-5) -- (p0-6) ;
  \draw (p0-6) -- (p0-8) ;
  \draw[->] (p0-7) |- (p1-8) ;
  \draw (p0-8) -- (p0-9) ;
  \draw[->] (p1-8) -| (p0-9) ;
  \draw[->] (p0-9) -- (p0-10) ;
\end{tikzpicture}

\ruleSubsection{plm\_syntax}{instruction-sync}{122}

\begin{tikzpicture}
  \matrix[column sep=\ruleMatrixColumnSeparation, row sep=\ruleMatrixRowSeparation] {
    & & & & & & & & & & & & & \node (p2-13) [point] {}; & \\
    & & & & & \node (p1-5) [terminal] {@attribute}; & & & & & & & \node (p1-12) [point] {}; & & & & \node (p1-16) [terminal] {@attribute}; & \\
    \node (P0start) [firstPoint] {}; & & \node (p0-2) [terminal] {sync}; & \node (p0-3) [terminal] {\{}; & \node (p0-4) [point] {}; & \node (p0-5) [point] {}; & \node (p0-6) [point] {}; & \node (p0-7) [point] {}; & \node (p0-8) [nonterminal] {\nonTerminalSymbol{guarded\_command}{35}}; & \node (p0-9) [terminal] {:}; & \node (p0-10) [nonterminal] {\nonTerminalSymbol{instructionList}{31}}; & \node (p0-11) [point] {}; & & & \node (p0-14) [terminal] {\}}; & \node (p0-15) [point] {}; & \node (p0-16) [point] {}; & \node (p0-17) [point] {}; & \node (p0-18) [lastPoint] {}; & \\
  };
  \draw[->] (P0start) -- (p0-2) ;
  \draw[->] (p0-2) -- (p0-3) ;
  \draw (p0-3) -- (p0-5) ;
  \draw[->] (p0-4) |- (p1-5) ;
  \draw (p0-5) -- (p0-6) ;
  \draw[->] (p1-5) -| (p0-6) ;
  \draw[->] (p0-6) -- (p0-8) ;
  \draw[->] (p0-8) -- (p0-9) ;
  \draw[->] (p0-9) -- (p0-10) ;
  \draw (p0-11) |- (p1-12) ;
  \draw[->] (p2-13) -| (p0-7) ;
  \draw[->] (p1-12) -| (p2-13) ;
  \draw[->] (p0-10) -- (p0-14) ;
  \draw (p0-14) -- (p0-16) ;
  \draw[->] (p0-15) |- (p1-16) ;
  \draw (p0-16) -- (p0-17) ;
  \draw[->] (p1-16) -| (p0-17) ;
  \draw[->] (p0-17) -- (p0-18) ;
\end{tikzpicture}

\ruleSubsection{plm\_syntax}{instruction-while}{20}

\begin{tikzpicture}
  \matrix[column sep=\ruleMatrixColumnSeparation, row sep=\ruleMatrixRowSeparation] {
    & & & & & \node (p1-5) [terminal] {@attribute}; & & & & & & & \node (p1-12) [terminal] {@attribute}; & \\
    \node (P0start) [firstPoint] {}; & & \node (p0-2) [terminal] {do}; & \node (p0-3) [terminal] {while}; & \node (p0-4) [point] {}; & \node (p0-5) [point] {}; & \node (p0-6) [point] {}; & \node (p0-7) [nonterminal] {\nonTerminalSymbol{expression}{16}}; & \node (p0-8) [terminal] {\{}; & \node (p0-9) [nonterminal] {\nonTerminalSymbol{instructionList}{31}}; & \node (p0-10) [terminal] {\}}; & \node (p0-11) [point] {}; & \node (p0-12) [point] {}; & \node (p0-13) [point] {}; & \node (p0-14) [lastPoint] {}; & \\
  };
  \draw[->] (P0start) -- (p0-2) ;
  \draw[->] (p0-2) -- (p0-3) ;
  \draw (p0-3) -- (p0-5) ;
  \draw[->] (p0-4) |- (p1-5) ;
  \draw (p0-5) -- (p0-6) ;
  \draw[->] (p1-5) -| (p0-6) ;
  \draw[->] (p0-6) -- (p0-7) ;
  \draw[->] (p0-7) -- (p0-8) ;
  \draw[->] (p0-8) -- (p0-9) ;
  \draw[->] (p0-9) -- (p0-10) ;
  \draw (p0-10) -- (p0-12) ;
  \draw[->] (p0-11) |- (p1-12) ;
  \draw (p0-12) -- (p0-13) ;
  \draw[->] (p1-12) -| (p0-13) ;
  \draw[->] (p0-13) -- (p0-14) ;
\end{tikzpicture}

\ruleSubsection{plm\_syntax}{instruction-for-in-do}{21}

\begin{tikzpicture}
  \matrix[column sep=\ruleMatrixColumnSeparation, row sep=\ruleMatrixRowSeparation] {
    \node (P0start) [firstPoint] {}; & & \node (p0-2) [terminal] {for}; & \node (p0-3) [terminal] {identifier}; & \node (p0-4) [terminal] {in}; & \node (p0-5) [nonterminal] {\nonTerminalSymbol{expression}{16}}; & \node (p0-6) [terminal] {\{}; & \node (p0-7) [nonterminal] {\nonTerminalSymbol{instructionList}{31}}; & \node (p0-8) [terminal] {\}}; & \node (p0-9) [lastPoint] {}; & \\
  };
  \draw[->] (P0start) -- (p0-2) ;
  \draw[->] (p0-2) -- (p0-3) ;
  \draw[->] (p0-3) -- (p0-4) ;
  \draw[->] (p0-4) -- (p0-5) ;
  \draw[->] (p0-5) -- (p0-6) ;
  \draw[->] (p0-6) -- (p0-7) ;
  \draw[->] (p0-7) -- (p0-8) ;
  \draw[->] (p0-8) -- (p0-9) ;
\end{tikzpicture}

\ruleSubsection{plm\_syntax}{instruction-for-in-lower-upper-bounds}{24}

\begin{tikzpicture}
  \matrix[column sep=\ruleMatrixColumnSeparation, row sep=\ruleMatrixRowSeparation] {
    \node (P0start) [firstPoint] {}; & & \node (p0-2) [terminal] {for}; & \node (p0-3) [terminal] {identifier}; & \node (p0-4) [terminal] {\$type}; & \node (p0-5) [terminal] {in}; & \node (p0-6) [nonterminal] {\nonTerminalSymbol{expression}{16}}; & \node (p0-7) [terminal] {..<}; & \node (p0-8) [nonterminal] {\nonTerminalSymbol{expression}{16}}; & \node (p0-9) [terminal] {\{}; & \node (p0-10) [nonterminal] {\nonTerminalSymbol{instructionList}{31}}; & \node (p0-11) [terminal] {\}}; & \node (p0-12) [lastPoint] {}; & \\
  };
  \draw[->] (P0start) -- (p0-2) ;
  \draw[->] (p0-2) -- (p0-3) ;
  \draw[->] (p0-3) -- (p0-4) ;
  \draw[->] (p0-4) -- (p0-5) ;
  \draw[->] (p0-5) -- (p0-6) ;
  \draw[->] (p0-6) -- (p0-7) ;
  \draw[->] (p0-7) -- (p0-8) ;
  \draw[->] (p0-8) -- (p0-9) ;
  \draw[->] (p0-9) -- (p0-10) ;
  \draw[->] (p0-10) -- (p0-11) ;
  \draw[->] (p0-11) -- (p0-12) ;
\end{tikzpicture}

\ruleSubsection{plm\_syntax}{instruction-procedure-call}{19}

\begin{tikzpicture}
  \matrix[column sep=\ruleMatrixColumnSeparation, row sep=\ruleMatrixRowSeparation] {
    \node (P0start) [firstPoint] {}; & & \node (p0-2) [nonterminal] {\nonTerminalSymbol{procedure\_call}{36}}; & \node (p0-3) [lastPoint] {}; & \\
  };
  \draw[->] (P0start) -- (p0-2) ;
  \draw[->] (p0-2) -- (p0-3) ;
\end{tikzpicture}

\nonTerminalSection{instructionList}{31}

\ruleSubsection{plm\_syntax}{instructionList}{23}

\begin{tikzpicture}
  \matrix[column sep=\ruleMatrixColumnSeparation, row sep=\ruleMatrixRowSeparation] {
    & & & & & & \node (p3-6) [point] {}; & \\
    & & & & & \node (p2-5) [terminal] {;}; & \\
    & & & & & \node (p1-5) [nonterminal] {\nonTerminalSymbol{instruction}{32}}; & \\
    \node (P0start) [firstPoint] {}; & & \node (p0-2) [point] {}; & \node (p0-3) [point] {}; & \node (p0-4) [point] {}; & & & \node (p0-7) [lastPoint] {}; & \\
  };
  \draw (P0start) -- (p0-3) ;
  \draw[->] (p0-4) |- (p1-5) ;
  \draw[->] (p0-4) |- (p2-5) ;
  \draw[->] (p3-6) -| (p0-2) ;
  \draw[->] (p1-5) -| (p3-6) ;
  \draw[->] (p2-5) -| (p3-6) ;
  \draw[->] (p0-3) -- (p0-7) ;
\end{tikzpicture}

\nonTerminalSection{isr}{4}

\ruleSubsection{plm\_syntax}{declaration-isr}{22}

\begin{tikzpicture}
  \matrix[column sep=\ruleMatrixColumnSeparation, row sep=\ruleMatrixRowSeparation] {
    & & & & & & & & \node (p2-8) [point] {}; & \\
    & & & & & & & \node (p1-7) [terminal] {@attribute}; & \\
    \node (P0start) [firstPoint] {}; & & \node (p0-2) [terminal] {isr}; & \node (p0-3) [terminal] {identifier}; & \node (p0-4) [point] {}; & \node (p0-5) [point] {}; & \node (p0-6) [point] {}; & & & \node (p0-9) [terminal] {\{}; & \node (p0-10) [nonterminal] {\nonTerminalSymbol{instructionList}{31}}; & \node (p0-11) [terminal] {\}}; & \node (p0-12) [lastPoint] {}; & \\
  };
  \draw[->] (P0start) -- (p0-2) ;
  \draw[->] (p0-2) -- (p0-3) ;
  \draw (p0-3) -- (p0-5) ;
  \draw[->] (p0-6) |- (p1-7) ;
  \draw[->] (p2-8) -| (p0-4) ;
  \draw[->] (p1-7) -| (p2-8) ;
  \draw[->] (p0-5) -- (p0-9) ;
  \draw[->] (p0-9) -- (p0-10) ;
  \draw[->] (p0-10) -- (p0-11) ;
  \draw[->] (p0-11) -- (p0-12) ;
\end{tikzpicture}

\nonTerminalSection{module\_variable}{11}

\ruleSubsection{plm\_syntax}{declaration-module}{17}

\begin{tikzpicture}
  \matrix[column sep=\ruleMatrixColumnSeparation, row sep=\ruleMatrixRowSeparation] {
    & & & & & \node (p1-5) [point] {}; & \\
    \node (P0start) [firstPoint] {}; & & \node (p0-2) [terminal] {var}; & \node (p0-3) [terminal] {identifier}; & \node (p0-4) [point] {}; & \node (p0-5) [terminal] {\$type}; & \node (p0-6) [point] {}; & \node (p0-7) [terminal] {=}; & \node (p0-8) [nonterminal] {\nonTerminalSymbol{expression}{16}}; & \node (p0-9) [lastPoint] {}; & \\
  };
  \draw[->] (P0start) -- (p0-2) ;
  \draw[->] (p0-2) -- (p0-3) ;
  \draw[->] (p0-3) -- (p0-5) ;
  \draw (p0-4) |- (p1-5) ;
  \draw (p0-5) -- (p0-6) ;
  \draw[->] (p1-5) -| (p0-6) ;
  \draw[->] (p0-6) -- (p0-7) ;
  \draw[->] (p0-7) -- (p0-8) ;
  \draw[->] (p0-8) -- (p0-9) ;
\end{tikzpicture}

\nonTerminalSection{primary}{29}

\ruleSubsection{plm\_syntax}{expression-operator-priority}{360}

\begin{tikzpicture}
  \matrix[column sep=\ruleMatrixColumnSeparation, row sep=\ruleMatrixRowSeparation] {
    \node (P0start) [firstPoint] {}; & & \node (p0-2) [terminal] {$\sim$}; & \node (p0-3) [nonterminal] {\nonTerminalSymbol{primary}{29}}; & \node (p0-4) [lastPoint] {}; & \\
  };
  \draw[->] (P0start) -- (p0-2) ;
  \draw[->] (p0-2) -- (p0-3) ;
  \draw[->] (p0-3) -- (p0-4) ;
\end{tikzpicture}

\ruleSubsection{plm\_syntax}{expression-operator-priority}{373}

\begin{tikzpicture}
  \matrix[column sep=\ruleMatrixColumnSeparation, row sep=\ruleMatrixRowSeparation] {
    \node (P0start) [firstPoint] {}; & & \node (p0-2) [terminal] {not}; & \node (p0-3) [nonterminal] {\nonTerminalSymbol{primary}{29}}; & \node (p0-4) [lastPoint] {}; & \\
  };
  \draw[->] (P0start) -- (p0-2) ;
  \draw[->] (p0-2) -- (p0-3) ;
  \draw[->] (p0-3) -- (p0-4) ;
\end{tikzpicture}

\ruleSubsection{plm\_syntax}{expression-operator-priority}{386}

\begin{tikzpicture}
  \matrix[column sep=\ruleMatrixColumnSeparation, row sep=\ruleMatrixRowSeparation] {
    \node (P0start) [firstPoint] {}; & & \node (p0-2) [terminal] {-}; & \node (p0-3) [nonterminal] {\nonTerminalSymbol{primary}{29}}; & \node (p0-4) [lastPoint] {}; & \\
  };
  \draw[->] (P0start) -- (p0-2) ;
  \draw[->] (p0-2) -- (p0-3) ;
  \draw[->] (p0-3) -- (p0-4) ;
\end{tikzpicture}

\ruleSubsection{plm\_syntax}{expression-operator-priority}{399}

\begin{tikzpicture}
  \matrix[column sep=\ruleMatrixColumnSeparation, row sep=\ruleMatrixRowSeparation] {
    \node (P0start) [firstPoint] {}; & & \node (p0-2) [terminal] {-\verb=%=}; & \node (p0-3) [nonterminal] {\nonTerminalSymbol{primary}{29}}; & \node (p0-4) [lastPoint] {}; & \\
  };
  \draw[->] (P0start) -- (p0-2) ;
  \draw[->] (p0-2) -- (p0-3) ;
  \draw[->] (p0-3) -- (p0-4) ;
\end{tikzpicture}

\ruleSubsection{plm\_syntax}{expression-operator-priority}{412}

\begin{tikzpicture}
  \matrix[column sep=\ruleMatrixColumnSeparation, row sep=\ruleMatrixRowSeparation] {
    \node (P0start) [firstPoint] {}; & & \node (p0-2) [terminal] {(}; & \node (p0-3) [nonterminal] {\nonTerminalSymbol{expression}{16}}; & \node (p0-4) [terminal] {)}; & \node (p0-5) [lastPoint] {}; & \\
  };
  \draw[->] (P0start) -- (p0-2) ;
  \draw[->] (p0-2) -- (p0-3) ;
  \draw[->] (p0-3) -- (p0-4) ;
  \draw[->] (p0-4) -- (p0-5) ;
\end{tikzpicture}

\ruleSubsection{plm\_syntax}{expression-convert}{19}

\begin{tikzpicture}
  \matrix[column sep=\ruleMatrixColumnSeparation, row sep=\ruleMatrixRowSeparation] {
    \node (P0start) [firstPoint] {}; & & \node (p0-2) [terminal] {convert}; & \node (p0-3) [nonterminal] {\nonTerminalSymbol{expression}{16}}; & \node (p0-4) [terminal] {:}; & \node (p0-5) [terminal] {\$type}; & \node (p0-6) [lastPoint] {}; & \\
  };
  \draw[->] (P0start) -- (p0-2) ;
  \draw[->] (p0-2) -- (p0-3) ;
  \draw[->] (p0-3) -- (p0-4) ;
  \draw[->] (p0-4) -- (p0-5) ;
  \draw[->] (p0-5) -- (p0-6) ;
\end{tikzpicture}

\ruleSubsection{plm\_syntax}{expression-extend}{19}

\begin{tikzpicture}
  \matrix[column sep=\ruleMatrixColumnSeparation, row sep=\ruleMatrixRowSeparation] {
    \node (P0start) [firstPoint] {}; & & \node (p0-2) [terminal] {extend}; & \node (p0-3) [nonterminal] {\nonTerminalSymbol{expression}{16}}; & \node (p0-4) [terminal] {:}; & \node (p0-5) [terminal] {\$type}; & \node (p0-6) [lastPoint] {}; & \\
  };
  \draw[->] (P0start) -- (p0-2) ;
  \draw[->] (p0-2) -- (p0-3) ;
  \draw[->] (p0-3) -- (p0-4) ;
  \draw[->] (p0-4) -- (p0-5) ;
  \draw[->] (p0-5) -- (p0-6) ;
\end{tikzpicture}

\ruleSubsection{plm\_syntax}{expression-truncate}{19}

\begin{tikzpicture}
  \matrix[column sep=\ruleMatrixColumnSeparation, row sep=\ruleMatrixRowSeparation] {
    \node (P0start) [firstPoint] {}; & & \node (p0-2) [terminal] {truncate}; & \node (p0-3) [nonterminal] {\nonTerminalSymbol{expression}{16}}; & \node (p0-4) [terminal] {:}; & \node (p0-5) [terminal] {\$type}; & \node (p0-6) [lastPoint] {}; & \\
  };
  \draw[->] (P0start) -- (p0-2) ;
  \draw[->] (p0-2) -- (p0-3) ;
  \draw[->] (p0-3) -- (p0-4) ;
  \draw[->] (p0-4) -- (p0-5) ;
  \draw[->] (p0-5) -- (p0-6) ;
\end{tikzpicture}

\ruleSubsection{plm\_syntax}{expression-constructor-call}{24}

\begin{tikzpicture}
  \matrix[column sep=\ruleMatrixColumnSeparation, row sep=\ruleMatrixRowSeparation] {
    & & & & & & & & & \node (p2-9) [point] {}; & \\
    & & & & & & & \node (p1-7) [terminal] {!}; & \node (p1-8) [nonterminal] {\nonTerminalSymbol{expression}{16}}; & \\
    \node (P0start) [firstPoint] {}; & & \node (p0-2) [terminal] {\$type}; & \node (p0-3) [terminal] {(}; & \node (p0-4) [point] {}; & \node (p0-5) [point] {}; & \node (p0-6) [point] {}; & & & & \node (p0-10) [terminal] {)}; & \node (p0-11) [lastPoint] {}; & \\
  };
  \draw[->] (P0start) -- (p0-2) ;
  \draw[->] (p0-2) -- (p0-3) ;
  \draw (p0-3) -- (p0-5) ;
  \draw[->] (p0-6) |- (p1-7) ;
  \draw[->] (p1-7) -- (p1-8) ;
  \draw[->] (p2-9) -| (p0-4) ;
  \draw[->] (p1-8) -| (p2-9) ;
  \draw[->] (p0-5) -- (p0-10) ;
  \draw[->] (p0-10) -- (p0-11) ;
\end{tikzpicture}

\ruleSubsection{plm\_syntax}{expression-typed-constant}{18}

\begin{tikzpicture}
  \matrix[column sep=\ruleMatrixColumnSeparation, row sep=\ruleMatrixRowSeparation] {
    & & & \node (p1-3) [terminal] {\$type}; & \\
    \node (P0start) [firstPoint] {}; & & \node (p0-2) [point] {}; & \node (p0-3) [point] {}; & \node (p0-4) [point] {}; & \node (p0-5) [terminal] {.}; & \node (p0-6) [terminal] {identifier}; & \node (p0-7) [lastPoint] {}; & \\
  };
  \draw (P0start) -- (p0-3) ;
  \draw[->] (p0-2) |- (p1-3) ;
  \draw (p0-3) -- (p0-4) ;
  \draw[->] (p1-3) -| (p0-4) ;
  \draw[->] (p0-4) -- (p0-5) ;
  \draw[->] (p0-5) -- (p0-6) ;
  \draw[->] (p0-6) -- (p0-7) ;
\end{tikzpicture}

\ruleSubsection{plm\_syntax}{expression-if}{22}

\begin{tikzpicture}
  \matrix[column sep=\ruleMatrixColumnSeparation, row sep=\ruleMatrixRowSeparation] {
    \node (P0start) [firstPoint] {}; & & \node (p0-2) [terminal] {if}; & \node (p0-3) [nonterminal] {\nonTerminalSymbol{expression}{16}}; & \node (p0-4) [terminal] {\{}; & \node (p0-5) [nonterminal] {\nonTerminalSymbol{expression}{16}}; & \node (p0-6) [terminal] {\}}; & \node (p0-7) [terminal] {else}; & \node (p0-8) [terminal] {\{}; & \node (p0-9) [nonterminal] {\nonTerminalSymbol{expression}{16}}; & \node (p0-10) [terminal] {\}}; & \node (p0-11) [lastPoint] {}; & \\
  };
  \draw[->] (P0start) -- (p0-2) ;
  \draw[->] (p0-2) -- (p0-3) ;
  \draw[->] (p0-3) -- (p0-4) ;
  \draw[->] (p0-4) -- (p0-5) ;
  \draw[->] (p0-5) -- (p0-6) ;
  \draw[->] (p0-6) -- (p0-7) ;
  \draw[->] (p0-7) -- (p0-8) ;
  \draw[->] (p0-8) -- (p0-9) ;
  \draw[->] (p0-9) -- (p0-10) ;
  \draw[->] (p0-10) -- (p0-11) ;
\end{tikzpicture}

\ruleSubsection{plm\_syntax}{expression-literal-integer}{17}

\begin{tikzpicture}
  \matrix[column sep=\ruleMatrixColumnSeparation, row sep=\ruleMatrixRowSeparation] {
    \node (P0start) [firstPoint] {}; & & \node (p0-2) [terminal] {integer}; & \node (p0-3) [lastPoint] {}; & \\
  };
  \draw[->] (P0start) -- (p0-2) ;
  \draw[->] (p0-2) -- (p0-3) ;
\end{tikzpicture}

\ruleSubsection{plm\_syntax}{expression-literal-string}{17}

\begin{tikzpicture}
  \matrix[column sep=\ruleMatrixColumnSeparation, row sep=\ruleMatrixRowSeparation] {
    \node (P0start) [firstPoint] {}; & & \node (p0-2) [terminal] {"string"}; & \node (p0-3) [lastPoint] {}; & \\
  };
  \draw[->] (P0start) -- (p0-2) ;
  \draw[->] (p0-2) -- (p0-3) ;
\end{tikzpicture}

\ruleSubsection{plm\_syntax}{expression-true-false}{17}

\begin{tikzpicture}
  \matrix[column sep=\ruleMatrixColumnSeparation, row sep=\ruleMatrixRowSeparation] {
    \node (P0start) [firstPoint] {}; & & \node (p0-2) [terminal] {true}; & \node (p0-3) [lastPoint] {}; & \\
  };
  \draw[->] (P0start) -- (p0-2) ;
  \draw[->] (p0-2) -- (p0-3) ;
\end{tikzpicture}

\ruleSubsection{plm\_syntax}{expression-true-false}{24}

\begin{tikzpicture}
  \matrix[column sep=\ruleMatrixColumnSeparation, row sep=\ruleMatrixRowSeparation] {
    \node (P0start) [firstPoint] {}; & & \node (p0-2) [terminal] {false}; & \node (p0-3) [lastPoint] {}; & \\
  };
  \draw[->] (P0start) -- (p0-2) ;
  \draw[->] (p0-2) -- (p0-3) ;
\end{tikzpicture}

\ruleSubsection{plm\_syntax}{expression-cst-registre}{27}

\begin{tikzpicture}
  \matrix[column sep=\ruleMatrixColumnSeparation, row sep=\ruleMatrixRowSeparation] {
    & & & & & & & & & & & & \node (p2-12) [point] {}; & \\
    & & & & & & & \node (p1-7) [terminal] {:}; & \node (p1-8) [nonterminal] {\nonTerminalSymbol{expression}{16}}; & & & \node (p1-11) [terminal] {,}; & \\
    \node (P0start) [firstPoint] {}; & & \node (p0-2) [terminal] {\$type}; & \node (p0-3) [terminal] {\{}; & \node (p0-4) [point] {}; & \node (p0-5) [terminal] {identifier}; & \node (p0-6) [point] {}; & \node (p0-7) [point] {}; & & \node (p0-9) [point] {}; & \node (p0-10) [point] {}; & & & \node (p0-13) [terminal] {\}}; & \node (p0-14) [lastPoint] {}; & \\
  };
  \draw[->] (P0start) -- (p0-2) ;
  \draw[->] (p0-2) -- (p0-3) ;
  \draw[->] (p0-3) -- (p0-5) ;
  \draw (p0-5) -- (p0-7) ;
  \draw[->] (p0-6) |- (p1-7) ;
  \draw[->] (p1-7) -- (p1-8) ;
  \draw (p0-7) -- (p0-9) ;
  \draw[->] (p1-8) -| (p0-9) ;
  \draw[->] (p0-10) |- (p1-11) ;
  \draw[->] (p2-12) -| (p0-4) ;
  \draw[->] (p1-11) -| (p2-12) ;
  \draw[->] (p0-9) -- (p0-13) ;
  \draw[->] (p0-13) -- (p0-14) ;
\end{tikzpicture}

\ruleSubsection{plm\_syntax}{expression-primary}{69}

\begin{tikzpicture}
  \matrix[column sep=\ruleMatrixColumnSeparation, row sep=\ruleMatrixRowSeparation] {
    & & & & & & & & & & & & & \node (p4-13) [point] {}; & \\
    & & & & & & & & & & \node (p3-10) [nonterminal] {\nonTerminalSymbol{effective\_parameters}{30}}; & \\
    & & & & & & & & & & \node (p2-10) [terminal] {[}; & \node (p2-11) [nonterminal] {\nonTerminalSymbol{expression}{16}}; & \node (p2-12) [terminal] {]}; & \\
    & & & \node (p1-3) [terminal] {self}; & \node (p1-4) [terminal] {.}; & & & & & & \node (p1-10) [terminal] {.}; & \node (p1-11) [terminal] {identifier}; & \\
    \node (P0start) [firstPoint] {}; & & \node (p0-2) [point] {}; & \node (p0-3) [point] {}; & & \node (p0-5) [point] {}; & \node (p0-6) [terminal] {identifier}; & \node (p0-7) [point] {}; & \node (p0-8) [point] {}; & \node (p0-9) [point] {}; & & & & & \node (p0-14) [lastPoint] {}; & \\
  };
  \draw (P0start) -- (p0-3) ;
  \draw[->] (p0-2) |- (p1-3) ;
  \draw[->] (p1-3) -- (p1-4) ;
  \draw (p0-3) -- (p0-5) ;
  \draw[->] (p1-4) -| (p0-5) ;
  \draw[->] (p0-5) -- (p0-6) ;
  \draw (p0-6) -- (p0-8) ;
  \draw[->] (p0-9) |- (p1-10) ;
  \draw[->] (p1-10) -- (p1-11) ;
  \draw[->] (p0-9) |- (p2-10) ;
  \draw[->] (p2-10) -- (p2-11) ;
  \draw[->] (p2-11) -- (p2-12) ;
  \draw[->] (p0-9) |- (p3-10) ;
  \draw[->] (p4-13) -| (p0-7) ;
  \draw[->] (p1-11) -| (p4-13) ;
  \draw[->] (p2-12) -| (p4-13) ;
  \draw[->] (p3-10) -| (p4-13) ;
  \draw[->] (p0-8) -- (p0-14) ;
\end{tikzpicture}

\nonTerminalSection{primitive}{3}

\ruleSubsection{plm\_syntax}{declaration-primitive}{23}

\begin{tikzpicture}
  \matrix[column sep=\ruleMatrixColumnSeparation, row sep=\ruleMatrixRowSeparation] {
    & & & & & & & & & & & \node (p2-11) [point] {}; & \\
    & & & \node (p1-3) [terminal] {public}; & & & & & & & \node (p1-10) [terminal] {@attribute}; & & & & \node (p1-14) [terminal] {->}; & \node (p1-15) [terminal] {\$type}; & \\
    \node (P0start) [firstPoint] {}; & & \node (p0-2) [point] {}; & \node (p0-3) [point] {}; & \node (p0-4) [point] {}; & \node (p0-5) [terminal] {primitive}; & \node (p0-6) [terminal] {identifier}; & \node (p0-7) [point] {}; & \node (p0-8) [point] {}; & \node (p0-9) [point] {}; & & & \node (p0-12) [nonterminal] {\nonTerminalSymbol{procedure\_formal\_arguments}{14}}; & \node (p0-13) [point] {}; & \node (p0-14) [point] {}; & & \node (p0-16) [point] {}; & \node (p0-17) [terminal] {\{}; & \node (p0-18) [nonterminal] {\nonTerminalSymbol{instructionList}{31}}; & \node (p0-19) [terminal] {\}}; & \node (p0-20) [lastPoint] {}; & \\
  };
  \draw (P0start) -- (p0-3) ;
  \draw[->] (p0-2) |- (p1-3) ;
  \draw (p0-3) -- (p0-4) ;
  \draw[->] (p1-3) -| (p0-4) ;
  \draw[->] (p0-4) -- (p0-5) ;
  \draw[->] (p0-5) -- (p0-6) ;
  \draw (p0-6) -- (p0-8) ;
  \draw[->] (p0-9) |- (p1-10) ;
  \draw[->] (p2-11) -| (p0-7) ;
  \draw[->] (p1-10) -| (p2-11) ;
  \draw[->] (p0-8) -- (p0-12) ;
  \draw (p0-12) -- (p0-14) ;
  \draw[->] (p0-13) |- (p1-14) ;
  \draw[->] (p1-14) -- (p1-15) ;
  \draw (p0-14) -- (p0-16) ;
  \draw[->] (p1-15) -| (p0-16) ;
  \draw[->] (p0-16) -- (p0-17) ;
  \draw[->] (p0-17) -- (p0-18) ;
  \draw[->] (p0-18) -- (p0-19) ;
  \draw[->] (p0-19) -- (p0-20) ;
\end{tikzpicture}

\nonTerminalSection{procedure}{0}

\ruleSubsection{plm\_syntax}{declaration-func}{70}

\begin{tikzpicture}
  \matrix[column sep=\ruleMatrixColumnSeparation, row sep=\ruleMatrixRowSeparation] {
    & & & & \node (p1-4) [terminal] {->}; & \node (p1-5) [terminal] {\$type}; & \\
    \node (P0start) [firstPoint] {}; & & \node (p0-2) [nonterminal] {\nonTerminalSymbol{procedure\_header}{13}}; & \node (p0-3) [point] {}; & \node (p0-4) [point] {}; & & \node (p0-6) [point] {}; & \node (p0-7) [terminal] {\{}; & \node (p0-8) [nonterminal] {\nonTerminalSymbol{instructionList}{31}}; & \node (p0-9) [terminal] {\}}; & \node (p0-10) [lastPoint] {}; & \\
  };
  \draw[->] (P0start) -- (p0-2) ;
  \draw (p0-2) -- (p0-4) ;
  \draw[->] (p0-3) |- (p1-4) ;
  \draw[->] (p1-4) -- (p1-5) ;
  \draw (p0-4) -- (p0-6) ;
  \draw[->] (p1-5) -| (p0-6) ;
  \draw[->] (p0-6) -- (p0-7) ;
  \draw[->] (p0-7) -- (p0-8) ;
  \draw[->] (p0-8) -- (p0-9) ;
  \draw[->] (p0-9) -- (p0-10) ;
\end{tikzpicture}

\nonTerminalSection{procedure\_call}{36}

\ruleSubsection{plm\_syntax}{instruction-procedure-call}{26}

\begin{tikzpicture}
  \matrix[column sep=\ruleMatrixColumnSeparation, row sep=\ruleMatrixRowSeparation] {
    \node (P0start) [firstPoint] {}; & & \node (p0-2) [nonterminal] {\nonTerminalSymbol{assignment\_target}{37}}; & \node (p0-3) [nonterminal] {\nonTerminalSymbol{effective\_parameters}{30}}; & \node (p0-4) [lastPoint] {}; & \\
  };
  \draw[->] (P0start) -- (p0-2) ;
  \draw[->] (p0-2) -- (p0-3) ;
  \draw[->] (p0-3) -- (p0-4) ;
\end{tikzpicture}

\nonTerminalSection{procedure\_formal\_arguments}{14}

\ruleSubsection{plm\_syntax}{declaration-func}{132}

\begin{tikzpicture}
  \matrix[column sep=\ruleMatrixColumnSeparation, row sep=\ruleMatrixRowSeparation] {
    & & & & & & & & & \node (p4-9) [point] {}; & \\
    & & & & & & \node (p3-6) [terminal] {?}; & \node (p3-7) [terminal] {identifier}; & \node (p3-8) [terminal] {\$type}; & \\
    & & & & & & \node (p2-6) [terminal] {?!}; & \node (p2-7) [terminal] {identifier}; & \node (p2-8) [terminal] {\$type}; & \\
    & & & & & & \node (p1-6) [terminal] {!}; & \node (p1-7) [terminal] {identifier}; & \node (p1-8) [terminal] {\$type}; & \\
    \node (P0start) [firstPoint] {}; & & \node (p0-2) [terminal] {(}; & \node (p0-3) [point] {}; & \node (p0-4) [point] {}; & \node (p0-5) [point] {}; & & & & & \node (p0-10) [terminal] {)}; & \node (p0-11) [lastPoint] {}; & \\
  };
  \draw[->] (P0start) -- (p0-2) ;
  \draw (p0-2) -- (p0-4) ;
  \draw[->] (p0-5) |- (p1-6) ;
  \draw[->] (p1-6) -- (p1-7) ;
  \draw[->] (p1-7) -- (p1-8) ;
  \draw[->] (p0-5) |- (p2-6) ;
  \draw[->] (p2-6) -- (p2-7) ;
  \draw[->] (p2-7) -- (p2-8) ;
  \draw[->] (p0-5) |- (p3-6) ;
  \draw[->] (p3-6) -- (p3-7) ;
  \draw[->] (p3-7) -- (p3-8) ;
  \draw[->] (p4-9) -| (p0-3) ;
  \draw[->] (p1-8) -| (p4-9) ;
  \draw[->] (p2-8) -| (p4-9) ;
  \draw[->] (p3-8) -| (p4-9) ;
  \draw[->] (p0-4) -- (p0-10) ;
  \draw[->] (p0-10) -- (p0-11) ;
\end{tikzpicture}

\nonTerminalSection{procedure\_header}{13}

\ruleSubsection{plm\_syntax}{declaration-func}{100}

\begin{tikzpicture}
  \matrix[column sep=\ruleMatrixColumnSeparation, row sep=\ruleMatrixRowSeparation] {
    & & & & & & & & & & \node (p2-10) [point] {}; & & & & & & \node (p2-16) [point] {}; & \\
    & & & \node (p1-3) [terminal] {public}; & & & & & & \node (p1-9) [point] {}; & & & & & & \node (p1-15) [terminal] {@attribute}; & \\
    \node (P0start) [firstPoint] {}; & & \node (p0-2) [point] {}; & \node (p0-3) [point] {}; & \node (p0-4) [point] {}; & \node (p0-5) [terminal] {func}; & \node (p0-6) [point] {}; & \node (p0-7) [terminal] {`mode}; & \node (p0-8) [point] {}; & & & \node (p0-11) [terminal] {identifier}; & \node (p0-12) [point] {}; & \node (p0-13) [point] {}; & \node (p0-14) [point] {}; & & & \node (p0-17) [nonterminal] {\nonTerminalSymbol{procedure\_formal\_arguments}{14}}; & \node (p0-18) [lastPoint] {}; & \\
  };
  \draw (P0start) -- (p0-3) ;
  \draw[->] (p0-2) |- (p1-3) ;
  \draw (p0-3) -- (p0-4) ;
  \draw[->] (p1-3) -| (p0-4) ;
  \draw[->] (p0-4) -- (p0-5) ;
  \draw[->] (p0-5) -- (p0-7) ;
  \draw (p0-8) |- (p1-9) ;
  \draw[->] (p2-10) -| (p0-6) ;
  \draw[->] (p1-9) -| (p2-10) ;
  \draw[->] (p0-7) -- (p0-11) ;
  \draw (p0-11) -- (p0-13) ;
  \draw[->] (p0-14) |- (p1-15) ;
  \draw[->] (p2-16) -| (p0-12) ;
  \draw[->] (p1-15) -| (p2-16) ;
  \draw[->] (p0-13) -- (p0-17) ;
  \draw[->] (p0-17) -- (p0-18) ;
\end{tikzpicture}

\nonTerminalSection{section}{1}

\ruleSubsection{plm\_syntax}{declaration-section}{23}

\begin{tikzpicture}
  \matrix[column sep=\ruleMatrixColumnSeparation, row sep=\ruleMatrixRowSeparation] {
    & & & & & & & & & & & \node (p2-11) [point] {}; & \\
    & & & \node (p1-3) [terminal] {public}; & & & & & & & \node (p1-10) [terminal] {@attribute}; & & & & \node (p1-14) [terminal] {->}; & \node (p1-15) [terminal] {\$type}; & \\
    \node (P0start) [firstPoint] {}; & & \node (p0-2) [point] {}; & \node (p0-3) [point] {}; & \node (p0-4) [point] {}; & \node (p0-5) [terminal] {section}; & \node (p0-6) [terminal] {identifier}; & \node (p0-7) [point] {}; & \node (p0-8) [point] {}; & \node (p0-9) [point] {}; & & & \node (p0-12) [nonterminal] {\nonTerminalSymbol{procedure\_formal\_arguments}{14}}; & \node (p0-13) [point] {}; & \node (p0-14) [point] {}; & & \node (p0-16) [point] {}; & \node (p0-17) [terminal] {\{}; & \node (p0-18) [nonterminal] {\nonTerminalSymbol{instructionList}{31}}; & \node (p0-19) [terminal] {\}}; & \node (p0-20) [lastPoint] {}; & \\
  };
  \draw (P0start) -- (p0-3) ;
  \draw[->] (p0-2) |- (p1-3) ;
  \draw (p0-3) -- (p0-4) ;
  \draw[->] (p1-3) -| (p0-4) ;
  \draw[->] (p0-4) -- (p0-5) ;
  \draw[->] (p0-5) -- (p0-6) ;
  \draw (p0-6) -- (p0-8) ;
  \draw[->] (p0-9) |- (p1-10) ;
  \draw[->] (p2-11) -| (p0-7) ;
  \draw[->] (p1-10) -| (p2-11) ;
  \draw[->] (p0-8) -- (p0-12) ;
  \draw (p0-12) -- (p0-14) ;
  \draw[->] (p0-13) |- (p1-14) ;
  \draw[->] (p1-14) -- (p1-15) ;
  \draw (p0-14) -- (p0-16) ;
  \draw[->] (p1-15) -| (p0-16) ;
  \draw[->] (p0-16) -- (p0-17) ;
  \draw[->] (p0-17) -- (p0-18) ;
  \draw[->] (p0-18) -- (p0-19) ;
  \draw[->] (p0-19) -- (p0-20) ;
\end{tikzpicture}

\nonTerminalSection{service}{2}

\ruleSubsection{plm\_syntax}{declaration-service}{23}

\begin{tikzpicture}
  \matrix[column sep=\ruleMatrixColumnSeparation, row sep=\ruleMatrixRowSeparation] {
    & & & & & & & & & & & \node (p2-11) [point] {}; & \\
    & & & \node (p1-3) [terminal] {public}; & & & & & & & \node (p1-10) [terminal] {@attribute}; & & & & \node (p1-14) [terminal] {->}; & \node (p1-15) [terminal] {\$type}; & \\
    \node (P0start) [firstPoint] {}; & & \node (p0-2) [point] {}; & \node (p0-3) [point] {}; & \node (p0-4) [point] {}; & \node (p0-5) [terminal] {service}; & \node (p0-6) [terminal] {identifier}; & \node (p0-7) [point] {}; & \node (p0-8) [point] {}; & \node (p0-9) [point] {}; & & & \node (p0-12) [nonterminal] {\nonTerminalSymbol{procedure\_formal\_arguments}{14}}; & \node (p0-13) [point] {}; & \node (p0-14) [point] {}; & & \node (p0-16) [point] {}; & \node (p0-17) [terminal] {\{}; & \node (p0-18) [nonterminal] {\nonTerminalSymbol{instructionList}{31}}; & \node (p0-19) [terminal] {\}}; & \node (p0-20) [lastPoint] {}; & \\
  };
  \draw (P0start) -- (p0-3) ;
  \draw[->] (p0-2) |- (p1-3) ;
  \draw (p0-3) -- (p0-4) ;
  \draw[->] (p1-3) -| (p0-4) ;
  \draw[->] (p0-4) -- (p0-5) ;
  \draw[->] (p0-5) -- (p0-6) ;
  \draw (p0-6) -- (p0-8) ;
  \draw[->] (p0-9) |- (p1-10) ;
  \draw[->] (p2-11) -| (p0-7) ;
  \draw[->] (p1-10) -| (p2-11) ;
  \draw[->] (p0-8) -- (p0-12) ;
  \draw (p0-12) -- (p0-14) ;
  \draw[->] (p0-13) |- (p1-14) ;
  \draw[->] (p1-14) -- (p1-15) ;
  \draw (p0-14) -- (p0-16) ;
  \draw[->] (p1-15) -| (p0-16) ;
  \draw[->] (p0-16) -- (p0-17) ;
  \draw[->] (p0-17) -- (p0-18) ;
  \draw[->] (p0-18) -- (p0-19) ;
  \draw[->] (p0-19) -- (p0-20) ;
\end{tikzpicture}

\nonTerminalSection{start\_symbol}{6}

\ruleSubsection{plm\_syntax}{syntax-grammar}{31}

\begin{tikzpicture}
  \matrix[column sep=\ruleMatrixColumnSeparation, row sep=\ruleMatrixRowSeparation] {
    & & & & & & \node (p8-6) [point] {}; & \\
    & & & & & \node (p7-5) [nonterminal] {\nonTerminalSymbol{import\_file}{5}}; & \\
    & & & & & \node (p6-5) [nonterminal] {\nonTerminalSymbol{isr}{4}}; & \\
    & & & & & \node (p5-5) [nonterminal] {\nonTerminalSymbol{primitive}{3}}; & \\
    & & & & & \node (p4-5) [nonterminal] {\nonTerminalSymbol{service}{2}}; & \\
    & & & & & \node (p3-5) [nonterminal] {\nonTerminalSymbol{section}{1}}; & \\
    & & & & & \node (p2-5) [nonterminal] {\nonTerminalSymbol{procedure}{0}}; & \\
    & & & & & \node (p1-5) [nonterminal] {\nonTerminalSymbol{declaration}{7}}; & \\
    \node (P0start) [firstPoint] {}; & & \node (p0-2) [point] {}; & \node (p0-3) [point] {}; & \node (p0-4) [point] {}; & & & \node (p0-7) [lastPoint] {}; & \\
  };
  \draw (P0start) -- (p0-3) ;
  \draw[->] (p0-4) |- (p1-5) ;
  \draw[->] (p0-4) |- (p2-5) ;
  \draw[->] (p0-4) |- (p3-5) ;
  \draw[->] (p0-4) |- (p4-5) ;
  \draw[->] (p0-4) |- (p5-5) ;
  \draw[->] (p0-4) |- (p6-5) ;
  \draw[->] (p0-4) |- (p7-5) ;
  \draw[->] (p8-6) -| (p0-2) ;
  \draw[->] (p1-5) -| (p8-6) ;
  \draw[->] (p2-5) -| (p8-6) ;
  \draw[->] (p3-5) -| (p8-6) ;
  \draw[->] (p4-5) -| (p8-6) ;
  \draw[->] (p5-5) -| (p8-6) ;
  \draw[->] (p6-5) -| (p8-6) ;
  \draw[->] (p7-5) -| (p8-6) ;
  \draw[->] (p0-3) -- (p0-7) ;
\end{tikzpicture}




\renewcommand\nonTerminalSection[2]{\subsection{Non terminal \texttt{\it#1}}\label{nt1:#2}}
\renewcommand\nonTerminalSummary[2]{\hyperref[nt1:#2]{#1}}
\renewcommand\nonTerminalSymbol[2]{\hyperref[nt1:#2]{#1}}
\renewcommand\startSymbol[2]{L'axiome de la grammaire est \hyperref[nt1:#2]{#1}.}

\sectionLabel{Grammaire du langage de description de cible}{grammaireCible}

\startSymbol{configuration\_start\_symbol}{2}

\nonTerminalSummaryStart \nonTerminalSummary{configuration}{0}\nonTerminalSummarySeparator \nonTerminalSummary{configuration\_start\_symbol}{2}\nonTerminalSummarySeparator \nonTerminalSummary{import\_file}{1}\nonTerminalSummarySeparator \nonTerminalSummary{interruptConfigList}{3}\nonTerminalSummaryEnd \nonTerminalSection{configuration}{0}

\ruleSubsection{plm\_target\_specific\_syntax}{configuration}{67}

\begin{tikzpicture}
  \matrix[column sep=\ruleMatrixColumnSeparation, row sep=\ruleMatrixRowSeparation] {
    \node (P0start) [firstPoint] {}; & & \node (p18-2) [terminal] {configuration}; & \\
    & & \node (p17-2) [terminal] {\$type}; & \\
    & & \node (p16-2) [terminal] {:}; & \\
    & & \node (p15-2) [terminal] {\$type}; & \\
    & & \node (p14-2) [terminal] {:}; & \\
    & & \node (p13-2) [terminal] {\$type}; & \\
    & & \node (p12-2) [terminal] {:}; & \\
    & & \node (p11-2) [terminal] {integer}; & \\
    & & \node (p10-2) [terminal] {:}; & \\
    & & \node (p9-2) [terminal] {integer}; & \\
    & & \node (p8-2) [terminal] {:}; & \\
    & & \node (p7-2) [terminal] {integer}; & \\
    & & \node (p6-2) [terminal] {:}; & \\
    & & \node (p5-2) [terminal] {integer}; & \\
    & & \node (p4-2) [terminal] {:}; & \\
    & & \node (p3-2) [terminal] {integer}; & \\
    & & \node (p2-2) [terminal] {:}; & \\
    & & \node (p1-2) [terminal] {"string"}; & \\
    & & \node (p0-2) [nonterminal] {\nonTerminalSymbol{interruptConfigList}{3}}; & \node (p0-3) [lastPoint] {}; & \\
  };
  \draw[->] (P0start) -- (p18-2) ;
  \draw[->] (p18-2) -- (p17-2) ;
  \draw[->] (p17-2) -- (p16-2) ;
  \draw[->] (p16-2) -- (p15-2) ;
  \draw[->] (p15-2) -- (p14-2) ;
  \draw[->] (p14-2) -- (p13-2) ;
  \draw[->] (p13-2) -- (p12-2) ;
  \draw[->] (p12-2) -- (p11-2) ;
  \draw[->] (p11-2) -- (p10-2) ;
  \draw[->] (p10-2) -- (p9-2) ;
  \draw[->] (p9-2) -- (p8-2) ;
  \draw[->] (p8-2) -- (p7-2) ;
  \draw[->] (p7-2) -- (p6-2) ;
  \draw[->] (p6-2) -- (p5-2) ;
  \draw[->] (p5-2) -- (p4-2) ;
  \draw[->] (p4-2) -- (p3-2) ;
  \draw[->] (p3-2) -- (p2-2) ;
  \draw[->] (p2-2) -- (p1-2) ;
  \draw[->] (p1-2) -- (p0-2) ;
  \draw[->] (p0-2) -- (p0-3) ;
\end{tikzpicture}

\nonTerminalSection{configuration\_start\_symbol}{2}

\ruleSubsection{plm\_target\_specific\_syntax}{syntax-grammar}{69}

\begin{tikzpicture}
  \matrix[column sep=\ruleMatrixColumnSeparation, row sep=\ruleMatrixRowSeparation] {
    & & & & & & & \node (p2-7) [point] {}; & \\
    & & & & & & \node (p1-6) [nonterminal] {\nonTerminalSymbol{import\_file}{1}}; & \\
    \node (P0start) [firstPoint] {}; & & \node (p0-2) [nonterminal] {\nonTerminalSymbol{configuration}{0}}; & \node (p0-3) [point] {}; & \node (p0-4) [point] {}; & \node (p0-5) [point] {}; & & & \node (p0-8) [lastPoint] {}; & \\
  };
  \draw[->] (P0start) -- (p0-2) ;
  \draw (p0-2) -- (p0-4) ;
  \draw[->] (p0-5) |- (p1-6) ;
  \draw[->] (p2-7) -| (p0-3) ;
  \draw[->] (p1-6) -| (p2-7) ;
  \draw[->] (p0-4) -- (p0-8) ;
\end{tikzpicture}

\nonTerminalSection{import\_file}{1}

\ruleSubsection{plm\_target\_specific\_syntax}{syntax-grammar}{58}

\begin{tikzpicture}
  \matrix[column sep=\ruleMatrixColumnSeparation, row sep=\ruleMatrixRowSeparation] {
    \node (P0start) [firstPoint] {}; & & \node (p0-2) [terminal] {import}; & \node (p0-3) [terminal] {"string"}; & \node (p0-4) [lastPoint] {}; & \\
  };
  \draw[->] (P0start) -- (p0-2) ;
  \draw[->] (p0-2) -- (p0-3) ;
  \draw[->] (p0-3) -- (p0-4) ;
\end{tikzpicture}

\nonTerminalSection{interruptConfigList}{3}

\ruleSubsection{plm\_target\_specific\_syntax}{configuration}{46}

\begin{tikzpicture}
  \matrix[column sep=\ruleMatrixColumnSeparation, row sep=\ruleMatrixRowSeparation] {
    & & & & & & & & & & & \node (p3-11) [point] {}; & \\
    & & & & & & & & \node (p2-8) [terminal] {:}; & \node (p2-9) [terminal] {integer}; & \\
    & & & & & & \node (p1-6) [terminal] {identifier}; & \node (p1-7) [point] {}; & \node (p1-8) [point] {}; & & \node (p1-10) [point] {}; & \\
    \node (P0start) [firstPoint] {}; & & \node (p0-2) [terminal] {\{}; & \node (p0-3) [point] {}; & \node (p0-4) [point] {}; & \node (p0-5) [point] {}; & & & & & & & \node (p0-12) [terminal] {\}}; & \node (p0-13) [lastPoint] {}; & \\
  };
  \draw[->] (P0start) -- (p0-2) ;
  \draw (p0-2) -- (p0-4) ;
  \draw[->] (p0-5) |- (p1-6) ;
  \draw (p1-6) -- (p1-8) ;
  \draw[->] (p1-7) |- (p2-8) ;
  \draw[->] (p2-8) -- (p2-9) ;
  \draw (p1-8) -- (p1-10) ;
  \draw[->] (p2-9) -| (p1-10) ;
  \draw[->] (p3-11) -| (p0-3) ;
  \draw[->] (p1-10) -| (p3-11) ;
  \draw[->] (p0-4) -- (p0-12) ;
  \draw[->] (p0-12) -- (p0-13) ;
\end{tikzpicture}


}














%-----------------------------------------------------------------------------------------------------------------------*
%                                                                                                                       *
%   B I B L I O G R A P H I E                                                                                           *
%                                                                                                                       *
%-----------------------------------------------------------------------------------------------------------------------*

%\input{bibliographie/bibliographie.tex}

%-----------------------------------------------------------------------------------------------------------------------*
%                                                                                                                       *
%   I N D E X                                                                                                           *
%                                                                                                                       *
%-----------------------------------------------------------------------------------------------------------------------*

\phantomsection  % Pour faire correctement pointer l'hyperlien dans la table des matières

%--- Redéfinir le style pour les pages d'index
%    Le style à redéfinir *doit* être le style "plain" (http://www.tug.org/pipermail/pdftex/2004-April/004942.html)
\fancypagestyle{plain}{
  \fancyhead{} % clear all header fields
  \fancyfoot{} % clear all footer fields
  %--- Numéro de page : à gauche pages paires, à droite pages impaires
%  \fancyhead[EL,OR]{\thepage}
  %--- Nom de chapitre : à droite page paires
%  \fancyhead[ER]{Index}
  %--- Nom de l'entrée courante : à gauche page impaires (???)
  %\fancyhead[OL]{\topxmark}
  %--- Filet de 0 pt
  \renewcommand{\headrulewidth}{0 pt}
}


%--- Écrire l'index
\printindex
%--- Appliquer le style
\pagestyle{plain}

%-----------------------------------------------------------------------------------------------------------------------*
%                                                                                                                       *
%   F I N    D U    D O C U M E N T                                                                                     *
%                                                                                                                       *
%-----------------------------------------------------------------------------------------------------------------------*

\end{document}
