%!TEX encoding = UTF-8 Unicode
%!TEX root = ../doc-plm.tex





\chapter{Déclaration des constantes globales}

Les constantes peuvent être déclarées en deux endroits :
\begin{itemize}
  \item en dehors de toute routine : c'est une constante globale (voir ci-après) ;
  \item parmi les instructions d'une routine : c'est une constante locale à la routine (voir \refSectionPage{declarationConstanteLocale}).
\end{itemize}




\sectionLabel{Déclaration d'une constante globale}{declarationConstanteGlobale}\index{Constante!globale}

La déclaration d'une constante globale est la suivante :

\begin{PLM}
let nom : Type = expression_statique
\end{PLM}

Où :
\begin{itemize}
  \item \plm=nom= est le nom de la constante globale ;
  \item \plm=Type= est le nom du type de la constante globale ;
  \item \plm=expression_statique= est l'expression qui fournit la valeur de cette constante ; cette expression est calculée lors de la compilation.
\end{itemize}

La portée d'une constante globale est le programme dans son intégralité : peut importe le fichier et la ligne où elle est déclarée.

L'\plm=expression_statique= ne peut pas nommer une autre constante globale, ni une variable globale. 

