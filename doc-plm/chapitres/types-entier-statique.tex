%!TEX encoding = UTF-8 Unicode
%!TEX root = ../doc-plm.tex





\chapterLabel{Le type \emph{entier statique}}{chapitreTypeEntierStatique}

Ce chapitre est consacré au type \emph{entier statique} \plm!LiteralInt!, qui est le type de toute constante entière littérale. Mais ce type est aussi applicable à des constantes \emph{entières statiques}, c'est-à-dire dont la valeur est calculée à la compilation.

Le type \plm!LiteralInt! n'est pas applicable à une variable : une variable entière doit être déclarée du type \plm!UIntN! (entier non signé de $N$ bits) ou \plm!IntN! (entier signé de $N$ bits), décrit au \refChapterPage{chapitreTypesEntiers}.


Le langage accepte des constantes littérales entières d'une valeur quelconque. Une constante littérale entière a pour type \plm!LiteralInt!.

Il est valide de déclarer des constantes de type \plm!LiteralInt! :
\begin{PLM}
let N LiteralInt = 1_000_000 // Ok, N a pour type LiteralInt
\end{PLM}

L'annotation de type peut être omise :
\begin{PLM}
let N = 1_000_000 // Ok, N a pour type LiteralInt
\end{PLM}



Il est possible d'utiliser une constante entière statique pour définir une autre constante :
\begin{PLM}
let P = N + 1 // Ok, P a pour type LiteralInt, et vaut 1_000_001
\end{PLM}

Le type \plm!LiteralInt! n'est pas acceptable pour une variable, aussi la déclaration suivante provoque une erreur de compilation :
\begin{PLM}
var N = 1_000_000 // Erreur, LiteralInt invalide pour une variable
\end{PLM}

Il faut une annotation de type qui nomme un type \plm!UIntN! (entier non signé de $N$ bits) ou \plm!IntN! (entier signé de $N$ bits) :
\begin{PLM}
var v : UInt32 = 1_000_000 // Ok
\end{PLM}

Une constante entière statique de type \plm!LiteralInt! est silencieusement convertie en un type entier \plm!UIntN! ou \plm!IntN!, en vérifiant si la conversion est possible ; par exemple, l'écriture suivante déclenche une erreur de compilation :
\begin{PLM}
var w UInt8 = 1_000 // Erreur, 1_000 ne peut pas être représenté
                     // par un entier non signé de 8 bits
\end{PLM}

Comme les constantes entières statiques sont calculées à la compilation, l'écriture suivante est correcte~:
\begin{PLM}
let A = 1_000_000_000_000_000_000_000_000_000_000_000_000_000
let B = 1_000_000_000_000_000_000_000_000_000_000_000_000_001
var z UInt1 = B - A
\end{PLM}

En effet, la valeur initiale de \plm!z! est $1$, représentable par un entier non signé de $1$ bit.


