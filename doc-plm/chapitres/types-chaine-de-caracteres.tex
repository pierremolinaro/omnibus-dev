%!TEX encoding = UTF-8 Unicode
%!TEX root = ../doc-plm.tex





\chapter{Les types chaîne de caractères}

La prise en charge des chaînes de caractères est très partielle : uniquement le type \plm+LiteralString+ est défini.

\section{Constante littérale chaîne de caractères}

Une constante littérale chaîne de caractères est délimitée par des « " ». Comme les sources PLM sont des textes UTF-8, la constante chaîne de caractères accepte tout caractère UTF-8. Par Exemple : \plm+"Hello !"+. % , \plm+"// œuf"+.

Une constante littérale chaîne de caractères a pour type \plm+LiteralString+.


\section{Le type \texttt{LiteralString}}

\subsection{Déclaration d'un objet de type \texttt{LiteralString}}

Le type \plm+LiteralString+ représente une référence vers une chaîne statique. La sémantique du langage garantit que cette référence toujours valide\footnote{On peut faire un parallèle avec le langage C, où un pointeur vers une chaîne littérale est du type «~\texttt{const char *}~» ; cependant, en C, ce pointeur peut recevoir la valeur \texttt{NULL}. En PLM, la sémantique interdit cette situation : une référence est toujours valide.}. Une utilisation typique est la déclaration d'une constante :
\begin{PLM}
let x LiteralString = "Hello !"
\end{PLM}

L'annotation de type peut être omise :
\begin{PLM}
let x = "Hello !"
\end{PLM}

La chaîne référencée ne peut pas être modifiée, cepandant une référence déclarée par \plm!var! peut figurer comme cible d'une affectation :
\begin{PLM}
var x = "Hello !"
x = "Bonjour !"
\end{PLM}
 
\subsection{Énumération d'une chaîne}

L'instruction \plm!for! permet d'énumérer un objet de type \plm+LiteralString+ :
\begin{PLM}
for c in x {
  ...
}
\end{PLM}

\plm!c! est une constante locale à l'instruction \plm!for!, de type \plm!UInt8!. Pour le moment, la seule opération que l'on peut faire à l'intérieur de la boucle est d'afficher \plm!c! sous la forme d'un caractère sur l'afficheur LCD :
\begin{PLM}
for c in x {
  lcd.writeData_inUserMode (!c)
}
\end{PLM}


