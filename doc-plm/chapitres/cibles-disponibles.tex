%!TEX encoding = UTF-8 Unicode
%!TEX root = ../doc-plm.tex





\chapterLabel{Cibles}{chapitreCiblesDefinies}

Dans l'état actuel de PLM, une seule cible est définie : \texttt{teensy-3-1-it}. L'objet de ce chapitre est de décrire son utilisation.

Il est possible de définir sa propre cible (\refChapterPage{chapitreConfCible}).

\section{Cible \texttt{teensy-3-1-it}}\index{Cible!\texttt{teensy-3-1-it}}

Cette cible est une carte \emph{Teensy 3.1}. Elle permet une programmation séquentielle avec interruptions activées. L'interruption \texttt{systick} est programmée pour se déclencher chaque milliseconde. 




















\subsection{Organigramme d'exécution}

La \refFigure{}{sequenceDemarrageTeensySequentialSystick} définit l'organigramme d'exécution d'un programme.

Le micro-contrôleur démarre sur une horloge interne, la mémoire vive n'étant pas initialisée. Il est dans le mode \emph{thread, priviliged access}, avec une seule pile. La configuration conservera cette pile unique, qui servira donc pour les routines de fond et les routines d'interruption.

La première étape est de configurer les horloges internes du micro-contrôleur : c'est le rôle des routines \plm=boot=. À ce stade, la mémoire vive n'est toujours pas initialisée, aussi les routines \plm=boot= n'y accèdent pas (le compilateur l'assure).

La deuxième étape est d'initialiser les \emph{variables globales}, c'est-à-dire mettre à zéro la zone \texttt{bss}, et de recopier à partir de la flash les valeurs initiales des variables initialisées.

La troisième étape est l'exécution des routines \plm=init=. À partir de cette étape et pour les suivantes, les variables globales sont initialisées, et donc leur emploi est autorisé. Le rôle des routines \plm=init= est de configurer les entrées/sorties du micro-contrôleur. En particulier, la routine \plm=init 0= de cette cible configure le \emph{SysTick Timer} pour qu'il engendre une interruption toutes les millisecondes (\refSubsectionPage{fichierTarget}).

Ensuite, le micro-contrôleur est passé en mode \emph{thread, unpriviliged access}, ce qui correspond au mode \plm+`user+ de PLM pour cette cible.

La routine \plm+setup+ est exécutée une fois, puis \plm+loop+ est exécutée indéfiniment.


\begin{figure}[t]
  \centering
  \small
  \begin{tikzpicture}[
      cloud/.style ={draw=red, thick, ellipse,fill=red!20, minimum height=2em},
      block/.style ={rectangle, draw=blue, thick, fill=green!20, align=center},
      decision/.style={chamfered rectangle, draw=blue, thick, fill=green!20},
      node distance=5mm
    ]
    \node [cloud] (start) {\textsc{Démarrage}} ;
    \node [block] (confDepart) [below=of start] {Mode \emph{thread}, accès priviliégié, une seule pile} ;
    \node [block] (boot) [below=of confDepart] {Routines \bf\texttt{boot}} ;
    \node [block] (raz) [below=of boot] {Initialisation des variables globales} ;
    \node [block] (init) [below=of raz] {Routines \bf\texttt{init}} ;
    \node [block] (user) [below=of init] {Passage en mode \emph{thread}, accès non priviliégié} ;
    \node [block] (setup) [below=of user] {Procédure \texttt{setup}} ;
    \node [block] (loop) [below=of setup] {Procédure \texttt{loop}} ;

    \draw [-stealth, thick] (start) -- (confDepart) ;
    \draw [-stealth, thick] (confDepart) -- (boot) ;
    \draw [-stealth, thick] (boot) -- (raz) ;
    \draw [-stealth, thick] (raz) -- (init) ;
    \draw [-stealth, thick] (init) -- (user) ;
    \draw [-stealth, thick] (user) -- (setup) ;
    \draw [-stealth, thick] (setup) -- (loop) ;
    \draw [-stealth, thick] (loop.south) -- +(0, -.25) -- +(1.7, -.25) -- +(1.7, 0.85)-- +(0, 0.85) ;
  \end{tikzpicture}
  \caption{Organigramme d'exécution de la cible \texttt{teensy-3-1-it}}
  \labelFigure{sequenceDemarrageTeensySequentialSystick}
  \ligne
\end{figure}










\subsection{Les routines d'interruption}

Les \refTableau{tableItTeensySequentialSystick1}, \refTableau{tableItTeensySequentialSystick2} et \refTableau{tableItTeensySequentialSystick3} listent les interruptions définies par le processeur qui équipe la carte \emph{Teensy 3.1}. L'utilisateur peut définir une routine d'interruption pour chacune d'entre elles, à l'exception de l'interruption n°1 (remise à zéro), et de la n°15 (\texttt{SysTick}), qui est prise en charge de façon particulière (voir §§).

\begin{table}[!t]
  \centering
  \begin{tabular}{llllll}
    \textbf{Numéro}& \textbf{Nom routine \texttt{@weak}} \\
    1  & --\\
    2  & \texttt{NMIHandler}\\
    3  & \texttt{HardFaultHandler}\\
    4  & \texttt{MemManageHandler}\\
    5  & \texttt{BusFaultHandler}\\
    6  & \texttt{UsageFaultHandler}\\
    11 & \texttt{svcHandler}\\
    12 & \texttt{DebugMonitorHandler}\\
    14 & \texttt{PendSVHandler}\\
    15 & \texttt{userSystickHandler}, voir texte \\
  \end{tabular}
  \caption{Table des interruptions 1 à 15 de la cible \texttt{teensy-3-1-it}}
  \labelTableau{tableItTeensySequentialSystick1}
  \ligne
\end{table}

\begin{table}[!t]
  \centering
  \begin{tabular}{llllll}
    \textbf{Numéro} & \textbf{Nom routine} \\
    16  & \texttt{DMAChannel0TranfertCompleteHandler}\\
    17  & \texttt{DMAChannel1TranfertCompleteHandler}\\
    18  & \texttt{DMAChannel2TranfertCompleteHandler}\\
    19  & \texttt{DMAChannel3TranfertCompleteHandler}\\
    20  & \texttt{DMAChannel4TranfertCompleteHandler}\\
    21  & \texttt{DMAChannel5TranfertCompleteHandler}\\
    22  & \texttt{DMAChannel6TranfertCompleteHandler}\\
    23  & \texttt{DMAChannel7TranfertCompleteHandler}\\
    24  & \texttt{DMAChannel8TranfertCompleteHandler}\\
    25  & \texttt{DMAChannel9TranfertCompleteHandler}\\
    26  & \texttt{DMAChannel10TranfertCompleteHandler}\\
    27  & \texttt{DMAChannel11TranfertCompleteHandler}\\
    28  & \texttt{DMAChannel12TranfertCompleteHandler}\\
    29  & \texttt{DMAChannel13TranfertCompleteHandler}\\
    30  & \texttt{DMAChannel14TranfertCompleteHandler}\\
    31  & \texttt{DMAChannel15TranfertCompleteHandler}\\
    32  & \texttt{DMAErrorHandler}\\
    34  & \texttt{flashMemoryCommandCompleteHandler}\\
    35  & \texttt{flashMemoryReadCollisionHandler}\\
    36  & \texttt{modeControllerHandler}\\
    37  & \texttt{LLWUHandler}\\
    38  & \texttt{WDOGEWMHandler}\\
    40  & \texttt{I2C0Handler}\\
    41  & \texttt{I2C1Handler}\\
    42  & \texttt{SPI0Handler}\\
    43  & \texttt{SPI1Handler}\\
    45  & \texttt{CAN0MessageBufferHandler}\\
    46  & \texttt{CAN0BusOffHandler}\\
    47  & \texttt{CAN0ErrorHandler}\\
    48  & \texttt{CAN0TransmitWarningHandler}\\
    49  & \texttt{CAN0ReceiveWarningHandler}\\
    50  & \texttt{CAN0WakeUpHandler}\\
    51  & \texttt{I2S0TransmitHandler}\\
    52  & \texttt{I2S0ReceiveHandler}\\
  \end{tabular}
  \caption{Table des interruptions 16 à 52 de la cible \texttt{teensy-3-1-it}}
  \labelTableau{tableItTeensySequentialSystick2}
  \ligne
\end{table}

\begin{table}[!t]
  \centering
  \begin{tabular}{llllll}
    \textbf{Numéro}& \textbf{Nom routine \texttt{@weak}} \\
    60  & \texttt{UART0LONHandler}\\
    61  & \texttt{UART0StatusHandler}\\
    62  & \texttt{UART0ErrorHandler}\\
    63  & \texttt{UART1StatusHandler}\\
    64  & \texttt{UART1ErrorHandler}\\
    65  & \texttt{UART2StatusHandler}\\
    66  & \texttt{UART2ErrorHandler}\\
    73  & \texttt{ADC0Handler}\\
    74  & \texttt{ADC1Handler}\\
    75  & \texttt{CMP0Handler}\\
    76  & \texttt{CMP1Handler}\\
    77  & \texttt{CMP2Handler}\\
    78  & \texttt{FMT0Handler}\\
    79  & \texttt{FMT1Handler}\\
    80  & \texttt{FMT2Handler}\\
    81  & \texttt{CMTHandler}\\
    82  & \texttt{RTCAlarmHandler}\\
    83  & \texttt{RTCSecondHandler}\\
    84  & \texttt{PITChannel0Handler}\\
    85  & \texttt{PITChannel1Handler}\\
    86  & \texttt{PITChannel2Handler}\\
    87  & \texttt{PITChannel3Handler}\\
    88  & \texttt{PDBHandler}\\
    89  & \texttt{USBOTGHandler}\\
    90  & \texttt{USBChargerDetectHandler}\\
    97  & \texttt{DAC0Handler}\\
    99  & \texttt{TSIHandler}\\
    100  & \texttt{MCGHandler}\\
    101  & \texttt{lowPowerTimerHandler}\\
    103  & \texttt{pinDetectPortAHandler}\\
    104  & \texttt{pinDetectPortBHandler}\\
    105  & \texttt{pinDetectPortCHandler}\\
    106  & \texttt{pinDetectPortDHandler}\\
    107  & \texttt{pinDetectPortEHandler}\\
    110  & \texttt{softwareInterruptHandler}
  \end{tabular}
  \caption{Table des interruptions 60 à 110 de la cible \texttt{teensy-3-1-it}}
  \labelTableau{tableItTeensySequentialSystick3}
  \ligne
\end{table}


\subsubsection{Routines d'interruption par défaut, exceptions activées}

Quand les exceptions sont activées, une routine par défaut est prédéfinie pour chaque interruption (sauf la n°1 et la n°15). Celle-ci exécute l'instruction \plm+panic+, dont l'argument est le numéro de l'interruption. Par exemple, pour l'interruption n°2 (\texttt{NMI}), la routine prédéfinie est :
\begin{PLM}[1]
proc NMIHandler `isr @nullOnNoException @weak () {
  panic 2
}
\end{PLM}


\subsubsection{Routines d'interruption par défaut, exceptions inactivées}

Quand les exceptions sont inactivées, l'attribut \plm+@nullOnNoException+ associé à chaque routine d'interruption provoque la suppression de cette routine d'interruption, et son remplacement de son adresse dans la table des vecteurs d'interruption par la valeur $0$.



\subsubsection{Routines d'interruption définies par l'utilisateur}

L'attribut \plm+@weak+ associé à chaque routine d'interruption permet sa redéfinition par l'utilisateur. La routine par défaut est alors ignorée, que les exceptions soient activées ou non.

Par exemple, l'utilisateur peut définir la routine associée à \texttt{NMI} par :
\begin{PLM}[1]
proc NMIHandler `isr {
  ...
}
\end{PLM}

Il est important de conserver le nom. Si celui-ci est mal orthographié et ne correspond à aucune interruption :
\begin{itemize}
  \item la routine d'interruption par défaut est conservée (execeptions activées) ou supprimée (exception inactivées) ;
  \item un \emph{warning} est déclenché, signalant que la routine utilisateur est inutilisée.
\end{itemize}



\subsubsection{Routine associée à l'interruption \texttt{SysTick}}

L'interruption \texttt{SysTick} est particulière. Elle est programmé par la routine \plm+init 0+ de façon à engendrer une interruption chaque milliseconde. Cette interruption se déclenche après la fin de l'exécution la routine \plm+init 0+, y compris éventuellement dans les routines \plm+init+ qui suivent. La routine d'interruption associée, inaccessible à l'utilisateur :
\begin{itemize}
  \item incrémente une variable globale de comptage du temps ;
  \item appelle la routine \plm+userSystickHandler+.
\end{itemize}

La routine \plm+userSystickHandler+ est définie par :
\begin{PLM}[1]
proc userSystickHandler `isr @weak () {
}
\end{PLM}

L'attribut \plm+@weak+ permet sa redéfinition par l'utilisateur.



