%!TEX encoding = UTF-8 Unicode
%!TEX root = ../doc-plm.tex





\chapter{Structures de contrôle}









\section{Instruction \texttt{if}}

L'instruction \plm=if= a une structure classique, où \plm=condition= est une expression booléenne :
\begin{PLM}
if condition then
  instructions_then
else
  instructions_else
end
\end{PLM}

Le compilateur vérifie que la \plm=condition= n'est pas une expression statique : une erreur de compilation est émise si elle l'est.

La branche \plm=else= est optionnelle :
\begin{PLM}
if condition then
  instructions_then
end
\end{PLM}


Une ou plusieurs branches \plm=elsif= peuvent être ajoutées :
\begin{PLM}
if condition then
  instructions_then
elsif condition2 then
  instructions_elsif
end
\end{PLM}






\section{Instruction \texttt{while}}

L'instruction \plm=while= permet d'exprimer une répétition, où la \plm=condition= est testée avant l'exécution des instructions de la boucle :
\begin{PLM}
while condition do
  instructions_while
end
\end{PLM}

\plm=condition= est une expression booléenne, qui ne doit pas être statique : une erreur de compilation est émise si elle l'est.

