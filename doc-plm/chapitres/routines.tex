%!TEX encoding = UTF-8 Unicode
%!TEX root = ../doc-plm.tex





\chapter{Procédures et fonctions}\index{Procedure@Procédure}\index{Fonction}

Le langage définit plusieurs natures de sous-programmes :
\begin{itemize}
  \item les \emph{procédures}, dont l'appel est une instruction ;
  \item les \emph{fonctions}, dont l'appel apparaît dans une expression ;
  \item les \emph{boot routines}, exécutées une fois avant l'initialisation des variables globales (\refSectionPage{bootRoutine}) ;
  \item les \emph{init routines}, exécutées une fois après l'initialisation des variables globales (\refSectionPage{initRoutine}) ;
  \item les \emph{routines d'exception}, exécutées lors d'une exception (\refSectionPage{routineException}).
\end{itemize}

Ce chapitre décrit les \emph{procédures} et les \emph{fonctions}. Les \emph{routines}\index{Routine} se distinguent des \emph{procédures} par le fait qu'elles n'ont pas d'arguments formels explicites, et qu'elles ne peuvent pas être appelées explicitement. Elles sont présentées dans des sections particulières, citées ci-dessus.

\section{Arguments formels, paramètres effectifs, sélecteurs}
\index{Argument formel}
\index{Parametre effectif@Paramètre effectif}
\index{Selecteur@Sélecteur}

Une \emph{procédure} déclare zéro, un ou plusieurs arguments formels qui peuvent être en \emph{entrée}, en \emph{sortie} ou en \emph{entrée/sortie}. Une fonction déclare zéro, un ou plusieurs arguments formels en \emph{entrée}.

La syntaxe des différents arguments formels et de leur paramètre effectif est résumée dans le \refTableau{argumentsFormelsParametresEffectifs}.


\begin{table}[t]
  \centering
  \begin{tabular}{llll}
    \textbf{Argument formel} & \textbf{Sélecteur} & \textbf{Paramètre effectif} & \textbf{Sélecteur} \\
    Entrée & \plm+?+         & Sortie & \plm+!expression+ \\
           & \plm+?selecteur:+ & & \plm+!selecteur:expression+ \\
    Sortie & \plm+!+         & Entrée & \plm+?variable+ \\
           & \plm+!selecteur:+ & & \plm+?selecteur:variable+ \\
    Entrée/sortie & \plm+?!+         & Sortie/entrée & \plm+!?variable+ \\
           & \plm+?!selecteur:+ & & \plm+!?selecteur:variable+ \\
  \end{tabular}
  \caption{Argument formel et paramètre effectif}
  \labelTableau{argumentsFormelsParametresEffectifs}
  \ligne
\end{table}


\section{Déclaration d'une procédure}

La déclaration d'une procédure est la suivante :
\begin{PLM}
proc nom $mode @attribut (arguments_formels) {
  liste_instructions
}
\end{PLM}
Où :
\begin{itemize}
  \item \plm=nom= est le nom de la procédure ;
  \item \plm=$mode= est la liste non vide de l'ensemble des modes où la procédure peut être appelée ;
  \item \plm=@attribut= est une liste éventuellement vide d'attributs associés à la procédure.
\end{itemize}

\subsection{Exemples}

\begin{PLM}
proc setup $user () {
}
\end{PLM}

Ceci définit la procédure \plm=setup=, sans argument, sans attribut, appelable uniquement en mode \plm=$user=.

\begin{PLM}
proc goto $user @noWarningIfUnused (?line:inLine : UInt32 ?column:inColumn : UInt8) {
}
\end{PLM}


\section{Procédures requises}

\colorbox{red}{Doc à écrire.}


\sectionLabel{Attribut \texttt{@weak}}{attributWeak}\index{"@weak}

\colorbox{red}{Doc à écrire.}


\sectionLabel{Attribut \texttt{@noWarningIfUnused}}{attributNoWarningIfUnused}\index{"@noWarningIfUnused}

\colorbox{red}{Doc à écrire.}


