%!TEX encoding = UTF-8 Unicode
%!TEX root = ../doc-plm.tex


\chapter{Tutorial}

%--- Pour supprimer tout en-tête et pied de page sur la 1re page d'un chapitre
\thispagestyle{empty}

\colorbox{red}{À revoir}


\emph{PLM} est un langage destiné aux systèmes embarqués. Particularités :
\begin{itemize}
  \item détection des débordements arithmétiques ;
  \item mode d'exécution d'une routine ;
  \item routines de démarrage, d'initialisation, de panique ;
  \item arithmétique sur une taille quelconque ;
  \item tâches ;
  \item synchronisations définissables par le langage ;
  \item commandes gardées définissables par le langage ;
\end{itemize}


\sectionLabel{Premier exemple « blinkled »}{exempleBlinkled}

\subsection{Programmes d'exemple embarqués}

Le compilateur PLM embarque un certain nombre de fichiers d'exemple, prêts à l'emploi. Deux options (voir \refSectionPage{optionsExemplesEmbarques}) permettent de lister les fichiers d'exemple et de les extraire.

Pour afficher la liste des fichiers d'exemple, entrez :

\begin{SHELL}
{\bfseries plm -l}\\
Embedded sample code:\\ 
\hspace*{1.2em}/teensy-3-1-it/01-blink-led.plm\\
\hspace*{1.2em}...
\end{SHELL}

Les exemples sont triés par cible, ainsi, pour la première ligne, la cible est \texttt{teensy-3-1-it}, et le fichier d'exemple \texttt{01-blink-led.plm}. Les noms des fichiers d'exemple commencent par des chiffres, simplement pour qu'ils apparaissent dans l'ordre souhaité. Vous pouvez nommer vos fichiers PLM comme vous le voulez.

Comme son nom le suggère, l'exemple \texttt{01-blink-led.plm} fait clignoter une led, celle embarquée sur la carte \emph{Teensy 3.1}.

\subsection{Extraction d'un fichier d'exemple}

Pour extraire le fichier d'exemple \texttt{01-blink-led.plm}, exécuter :

\begin{SHELL}
\bfseries plm -x=/teensy-3-1-it/01-blink-led.plm
\end{SHELL}


Ceci écrit le fichier \texttt{01-blink-led.plm} dans le répertoire courant. Ce fichier est prêt à être compilé.

\subsection{Compilation et flashage}

Pour compiler, entrer\footnote{L'option « \texttt{-{}-Oz} » ordonne une optimisation avec réduction de la taille du code engendré ; pour compiler et flasher, ajouter l'option « \texttt{-f} » (\refSectionPage{optionCodeEngendre}) ; pour compiler avec des messages détaillés, ajouter l'option « \texttt{-v} » (\refSectionPage{optionsGenerales}). Les options peuvent apparaître sur la ligne de commande avant ou après le nom du fichier compilé.} :
\begin{SHELL}
{\bfseries plm --Oz 01-blink-led.plm}\\
Downloading compiler tool chain\\
...........................................................\\
\textcolor{blue}{Making "objects" directory}\\
\textcolor{blue}{Compiling src.c}\\
\textcolor{blue}{Assembling src.s}\\
\textcolor{blue}{LLVM Link sources/src.ll objects/src.c.ll}\\
\textcolor{blue}{Optimizing all.ll}\\
\textcolor{blue}{Compiling objects/opt.all.ll}\\
\textcolor{blue}{Assembling objects/opt.all.ll.s}\\
\textcolor{blue}{Stack requirements}\\
\textcolor{blue}{Check stacks}\\
\textcolor{blue}{Linking product/product-linker.elf}\\
\textcolor{blue}{Hexing product/product.ihex}\\
Code:~~~~~~~~4472 bytes\\
ROM data:~~~~0 bytes\\
RAM + STACK:~1660 bytes\\
\end{SHELL}

À la première compilation, le compilateur C va être automatiquement téléchargé\footnote{Les outils de compilation C sont installés en \texttt{$\sim$/plm-tools/}.}, cela peut prendre plusieurs minutes.

Ensuite, la compilation s'effectue, et range tous ses fichiers de production dans un sous répertoire de \texttt{$\sim$/plm-products/}, dont le nom est le  chemin absolu du fichier source (sans son extension), dans lequel tous les « \texttt{/} » ont été remplacés par des « \texttt{+} ».

Si l'option « \texttt{-f} » était présente, la carte \emph{Teensy 3.1} est prête à être flashée : appuyer sur le bouton \emph{reset} de la carte pour démarrer le flashage\footnote{Si vous ne voulez pas flasher le micro-contrôleur, entrez simplement « \texttt{ctrl~C} ».}. L'affichage sur le terminal devient :
\begin{SHELL}
Found HalfKay Bootloader\\
Read "product/product.ihex": 1472 bytes, 1.1\% usage\\
Programming..\\
Booting\\
Success
\end{SHELL}

Le programme d'exemple a été flashé (« \emph{Programming} »), puis le micro-contrôleur a été redémarré (« \emph{Booting} ») : la led de la carte clignote.

\subsection{Texte source de \texttt{01-blink-led.plm}}

Maintenant, nous allons examiner le code source PLM du fichier \texttt{01-blink-led.plm} :

\begin{PLM}[1]
target "teensy-3-1-tp"

task T1 priority 1 stackSize 512 {
  var compteur UInt32 = 0

  while time.waitUntilMS (!deadline:self.compteur) {
    leds.on (!LED_L0)
    self.compteur +%= 500
    time.waitUntilMS (!deadline:self.compteur)
    leds.off (!LED_L0)
    self.compteur +%= 500
    lcd.goto (!line:0 !column:0)
    lcd.printUnsigned (!time.millis ())
  }
}
\end{PLM}

Examinons le code ligne par ligne.

\begin{PLM}[1]
target "teensy-3-1-tp"
\end{PLM}

La première ligne \plm+target "teensy-3-1-it"+ nomme la cible, en référençant un fichier embarqué dans le compilateur. Un fichier de définition de cible définit types, routines, registres de contrôle, … Un fichier de définition de cible ne peut être compilé séparement, il doit être cité dans un programme utilisateur (ici \texttt{01-blink-led.plm}).

\begin{PLM}[3]
task T1 priority 1 stackSize 512 {
\end{PLM}

Cette ligne déclare une tâche nommée \plm+T1+, de priorité \plm=1=, avec une pile de \plm=512= octets. Ici, le nom de la tâche n'a pas d'importance, ni sa priorité (c'est la seule tâche). Les tâches sont démarrées automatiquement, après la phase d'initialisation.

\begin{PLM}[4]
  var compteur UInt32 = 0
\end{PLM}

Déclaration d'une variable \plm=compteur=, de type \plm=UInt32= (entier non signé sur 32 bits), initialisée à 0. Les variables déclarées dans une tâche sont locales à la tâche, aucune autre tâche ne peut y accéder. PLM prédéfinit les types entiers de taille quelconque (jusqu'à 2048 bits), signés (\plm=IntXX=), et non signés (\plm=UIntXX=). Il est donc parfaitement légal d'utiliser des types comme par exemple \plm=Int8=, \plm=Int13=, ou encore \plm=UInt2000=.


\begin{PLM}[6]
  while time.waitUntilMS (!deadline:self.compteur) {
\end{PLM}

Appel de la commande gardée \plm=waitUntilMS= du pilote \plm=time=, avec un argument, la variable de tâche \plm=compteur=. L'utilisation de \plm=self= est obligatoire pour accéder aux variables de tâche\footnote{Sans \texttt{self}, ce serait un accès à une variable locale à la fonction, ou à une constante globale.}. Cette appel effectue l'attente jusqu'à la date exprimée par l'argument \plm=self.compteur=. Cette date est comptée en millisecondes depuis le démarrage du micro-contrôleur.



\begin{PLM}[7]
    leds.on (!LED_L0)
\end{PLM}

Appel de la fonction \plm=on= du pilote \plm=leds=, avec un argument, la constante \plm=LED_L0=. L'effet de l'exécution de cet appel est d'allumer la led n°0.



\begin{PLM}[8]
    self.compteur +%= 500
\end{PLM}

Incrémente la variable de tâche \plm=compteur= de \plm=500=. L'opérateur \plm!+%=! est l'addition sans test de débordement. Si on voulait le tester, et exécuter un code de panique en cas de débordement, on utiliserait l'opérateur \plm!+=!. Comme \plm=compteur= est utilisé pour l'attente d'échéance qui s'exprimme en millisecondes, \plm=500= représente la date de la prochaine échéance, qui est $500~ms$ plus tard.


\begin{PLM}[9]
    time.waitUntilMS (!deadline:self.compteur)
    leds.off (!LED_L0)
    self.compteur +%= 500
\end{PLM}

Attente de la nouvelle échéance, extinction de la led n°0, incrémentation de l'échéance de $500~ms$.


\begin{PLM}[12]
    lcd.goto (!line:0 !column:0)
\end{PLM}

Appel de la fonction \plm=goto= du pilote \plm=lcd=, avec deux arguments. L'effet de l'exécution de cet appel est de positionner le curseur de l'afficheur LCD au début de la première ligne.




\begin{PLM}[13]
    lcd.printUnsigned (!time.millis ())
\end{PLM}

Cette ligne affiche sur l'afficheur LCD à la position du curseur le nombre de millisecondes écoulés depuis le démarrage du micro-contrôleur.






\begin{PLM}[14]
  }
\end{PLM}

Cette ligne marque la fin de la commande gardée qui commence à la ligne $6$. Comme cette commande gardée commence le mot réservé \plm=while=, elle est re-exécutée.


%\sectionLabel{Deuxième exemple : clignotement avec routine d'interruption}{deuxiemeExemple}
%
%La cible \texttt{teensy-3-1-it} programme le décompteur \texttt{SysTick} de façon que l'interruption correspondante se déclenche toutes les millisecondes, et appelle une routine d'interruption. Par défaut, cette routine ne comporte aucune instruction et n'a donc aucun effet. 
%
%Le programme d'exemple \texttt{02-blinkled-systick.plm} redéfinit cette routine d'interruption et l'utilise pour compter le temps.
%
%Pour extraire le programme d'exemple :
%\begin{SHELL}
%\bfseries plm -x=/teensy-3-1-it/02-blinkled-systick.plm
%\end{SHELL}
%
%Ensuite, vous pouvez le compiler et effectuer son flashage sur la carte \emph{Teensy 3.1} :
%\begin{SHELL}
%\bfseries plm 02-blinkled-systick.plm
%\end{SHELL}
%
%La led sur la carte \emph{Teensy 3.1} clignote à la fréquence de `1`~Hz.
%
%Voyons maintenant le contenu du fichier d'exemple \texttt{02-blinkled-systick.plm} :
%
%\begin{PLM}[1]
%target "teensy-3-1-it"
%
%proc setup `user () {
%  PORTC_PCR5 = PORTC_PCR5::mux (1)
%  GPIOC_PDDR |= (1 << 5)
%}
%
%var gUpTimeMS UInt32 = 0 {
%  @rw proc systickHandler ()
%  proc wait (?ms: inDuration UInt32)
%}
%
%proc systickHandler `isr () {
%  gUpTimeMS += 1
%}
%
%proc wait `user (?ms: inDuration UInt32) {
%  let deadline = gUpTimeMS + inDuration
%  while deadline > gUpTimeMS do
%  end
%}
%
%proc loop `user () {
%  wait (!ms:500)
%  GPIOC_PSOR = 1 << 5 // Allumer la led
%  wait (!ms:500)
%  GPIOC_PCOR = 1 << 5  // Éteindre la led
%}
%\end{PLM}
%
%La référence à la cible \plm+target "teensy-3-1-it"+ et la définition de la routine \plm+setup+ sont identiques à celles écrites dans le premier exemple.
%
%La définition de la variable globale \plm+gUpTimeMS+ :
%
%\begin{PLM}[8]
%var gUpTimeMS UInt32 = 0 {
%  @rw proc systickHandler ()
%  proc wait (?ms: inDuration UInt32)
%}
%\end{PLM}
%
%Cette variable compte le temps exprimé en millisecondes. Deux routines y accèdent, \plm+systickHandler+ en lecture / écriture, et \plm+wait+, uniquement en lecture. Remarquons que \plm+wait+ présente un argument formel en entrée, \plm+inDuration+, dont la notation sera présentée plus loin.
%
%La routine d'interruption \plm+systickHandler+ :
%
%\begin{PLM}[13]
%proc systickHandler `isr () {
%  gUpTimeMS += 1
%}
%\end{PLM}
%
%Le mode associé à une routine d'interruption est toujours \plm+`isr+. L'instruction d'appel de routine (\refSectionPage{instructionAppelProc}) vérifie que les modes de la routine appelée soient un sous-ensemble des modes de la routine appelante : ainsi, par exemple, il n'est pas possible d'appeler \plm+systickHandler+ à partir de la routine \plm+loop+.
%
%La routine d'interruption \plm+systickHandler+ est appelée toutes les millisecondes, la variable globale \plm+gUpTimeMS+ contient donc le temps absolu depuis le démarrage, exprimé en millisecondes.
%
%L'attente durant un certain délai est effectué par la routine \plm+wait+ :
%\begin{PLM}[17]
%proc wait `user (?ms: inDuration UInt32) {
%  let deadline = gUpTimeMS + inDuration
%  while deadline > gUpTimeMS do
%  end
%}
%\end{PLM}
%
%Cette routine présente un paramètre formel : \plm+inDuration+, de type \plm+UInt32+. Le \emph{sélecteur} \plm+?ms:+ indique que \plm+inDuration+ est un paramètre en entrée (son sélecteur commence par « \texttt{?} »), et que le nom de ce sélecteur est « \texttt{ms} »\footnote{La liste de tous les arguments formels et des paramètres effectifs correspondants est présentée à la \refSectionPage{argumentFormel}.}.
%
%Le code de cette routine est simple : après avoir calculé la constante \plm+deadline+\footnote{Une constante locale à une routine est déclarée par le mot réservé \texttt{\bfseries let} (voir \refSectionPage{declarationConstanteLocale}).}, la boucle \plm+while+ attend le délai imparti.
%
%Il reste à commenter le code ce la routine \plm+loop+ :
%\begin{PLM}[23]
%proc loop `user () {
%  wait (!ms:500)
%  GPIOC_PSOR = 1 << 5 // Allumer la led
%  wait (!ms:500)
%  GPIOC_PCOR = 1 << 5  // Éteindre la led
%}
%\end{PLM}
%
%Intéressons-nous à l'appel de la routine \plm+wait+ ; celle-ci présente un argument formel en entrée, dont le sélecteur est \plm+?ms:+. L'instruction d'appel devra donc comporter un paramètre effectif en sortie (donc son sélecteur commence par « \texttt{!} »), et de même nom : le sélecteur du paramètre effectif est donc \plm+!ms:+. En ce qui concerne les modes, la routine appelée \plm+wait+ est déclarée avec le mode \plm+`user+, comme la routine appelante \plm+loop+ : l'appel est donc valide.
%


\section{Fichiers produits par la compilation}

À titre d'information, on décrit l'organisation des fichiers produits par la compilation d'un fichier \texttt{f{}ichier.plm}. Cette organisation est résumée par la \refFigure{}{compilationProgrammePLM}.

\begin{figure}[ht]
  \centering
  \scriptsize
  \begin{tikzpicture}[
      block/.style ={rectangle, draw=black, thick, fill=white, align=center, minimum height=0.5cm, minimum width=1.4cm},
      block2/.style ={rectangle, draw=black, thick, fill=white, align=center, minimum height=0.5cm, minimum width=1.6cm},
      node distance=0.5cm and 1.4cm
    ]
    \node [block] (plm) at (0, 0) {\tt f{}ichier.plm} ;
    \node (ws) [right=of plm, minimum height=0.6cm, minimum width=1.4cm] {} ;
    \node [block] (plmc) [right=of ws] {\tt src.ll} ;
    \node [block] (make) [below=of plmc] {\tt make.py};
    \node [block] (makeplm) [below=of make] {\tt plm.py};
    \node [block] (plms) [below=of makeplm] {\tt src.s} ;
    \node [block] (plmco) [right=of plmc] {\tt opt.src.ll} ;
    \node [block] (plmcos) [below=of plmco] {\tt opt.src.ll.s} ;
    \node [block] (plmco2) [below=of plmcos] {\tt opt.src.ll.s.o} ;
    \node [block] (plmso) [right=of plms] {\tt src.s.o} ;
    \node [block] (link) [below=of plms] {\tt link.ld} ;
    \node [block] (elf) [right=of plmso] {\tt plm.elf} ;
    \node (elf2) [above=of elf, minimum height=0.6cm] {};
    \begin{scope}[node distance=0.1cm and 1.6cm]
      \node [block2] (build) at (3, -0.5) {\tt build.py};
      \node [block2] (clean) [below=of build] {\tt clean.py};
      \node [block2] (run) [below=of clean] {\tt run.py};
      \node [block2] (buildas) [below=of run] {\tt build-as.py};
      \node [block2] (buildverbose) [below=of buildas] {\tt build-verbose.py};
      \node [block2] (objsize) [below=of buildverbose] {\tt objsize.py};
      \node [block2] (objdump) [below=of objsize] {\tt objdump.py};
    \end{scope}

    \draw [-stealth, thick] (plm) to node [above] {\tt plm} (plmc) ;
    \draw [-stealth, thick] (ws.east) -- +(0.7, 0) |- (plms.west) ;
    \draw [-stealth, thick] (ws.east) -- +(0.7, 0) |- (link) ;
    \draw [-stealth, thick] (ws.east) -- +(0.7, 0) |- (make) ;
    \draw [-stealth, thick] (ws.east) -- +(0.7, 0) |- (makeplm) ;
    \draw [-stealth, thick] (plm.east) -- +(0.3, 0) |- (build.west) ;
    \draw [-stealth, thick] (plm.east) -- +(0.3, 0) |- (run.west) ;
    \draw [-stealth, thick] (plm.east) -- +(0.3, 0) |- (clean.west) ;
    \draw [-stealth, thick] (plm.east) -- +(0.3, 0) |- (buildas.west) ;
    \draw [-stealth, thick] (plm.east) -- +(0.3, 0) |- (buildverbose.west) ;
    \draw [-stealth, thick] (plm.east) -- +(0.3, 0) |- (objsize.west) ;
    \draw [-stealth, thick] (plm.east) -- +(0.3, 0) |- (objdump.west) ;
    \draw [-stealth, thick] (plms) to node[above] {\tt as} (plmso) ;
    \draw [-stealth, thick] (plmso) to node[above right] {\tt ld} (elf) ;
    \draw [-stealth, thick] (plmc) to node[above] {\tt opt} (plmco) ;
    \draw [thick] (elf.west) -- +(-0.7, 0) |- (plmco2.east) ;
    \draw [thick] (elf.west) -- +(-0.7, 0) |- (link.east) ;
    \draw [-stealth, thick] (plmco) to node[right] {\tt llc} (plmcos) ;
    \draw [-stealth, thick] (plmcos) to node[right] {\tt as} (plmco2) ;
    
    \begin{scope}[node distance=5mm]
    \node (repws) [above=of ws] {\tt ws/} ;
    \node (source) [above=of plmc] {\tt ws/sources/} ;
    \node (objects) [above=of plmco] {\tt ws/objects/} ;
    \node (product) [right=of objects] {\tt ws/product/} ;
    \node (titre) [above=of plm] {Répertoires :} ;
    \end{scope}
  \end{tikzpicture}
  \caption{Fichiers produits par la compilation d'un programme PLM}
  \labelFigure{compilationProgrammePLM}
  \ligne
\end{figure}



Les fichiers produits par la compilation sont regroupés dans un répertoire noté \texttt{ws} dans la \refFigure{}{compilationProgrammePLM}, et obtenu de la façon suivante :
\begin{itemize}
  \item on considère le chemin absolu du fichier source ; par exemple, pour {\tt f{}ichier.plm} situé dans le répertoire courant \texttt{/chemin/absolu} : \colorbox{gray!25}{\tt/chemin/absolu/f{}ichier.plm} ;
  \item on retire l'extension du fichier source ; \colorbox{gray!25}{\tt/chemin/absolu/f{}ichier} ;
  \item on remplace chaque « \texttt{/} » par un « \texttt{+} » : \colorbox{gray!25}{\tt+chemin+absolu+f{}ichier} ;
  \item le chemin vers le répertoire \texttt{ws} est obtenu en préfixant par \texttt{$\sim$/plm-product/}, où $\sim$ est le répertoire \emph{home} de l'utilisateur : \colorbox{gray!25}{\tt$\sim$/plm-product/+chemin+absolu+f{}ichier}.
\end{itemize}

\begin{table}[t!]
\centering
\begin{tabular}{p{2cm}p{11cm}}
  \textbf{Nom} & \textbf{Commentaire} \\
  \texttt{build.py} & Effectue la compilation de \texttt{plm.c}, l'assemblage de \texttt{plm.s}, et l'édition de liens afin d'obtenir l'exécutable \texttt{plm.elf}, comme décrit à la \refFigure{}{compilationProgrammePLM}. \\
  \texttt{build-as.py} & Effectue la compilation de \texttt{plm.c} jusqu'à la génération d'un fichier assembleur \texttt{plm.c.s} qui est placé dans \texttt{ws/objects}. Cette opération permet de connaître la traduction en assembleur du source C.\\
  \texttt{build-verbose.py} & Effectue les mêmes opérations que \texttt{build.py}, mais affiche les lignes de commande.\\
  \texttt{run.py} & Construit l'exécutable \texttt{plm.elf} (comme le fait \texttt{build.py}), et lance l'exécution sur la cible. Équivalent à l'option \texttt{-f} de la ligne de commande. \\
  \texttt{clean.py} & Supprime les répertoires \texttt{ws/objects} et \texttt{ws/product}. \\
  \texttt{objdump.py} & Effectue les mêmes opérations que \texttt{build.py}, puis affiche le code machine obtenu désassemblé. \\
  \texttt{objsize.py} & Effectue les mêmes opérations que \texttt{build.py}, puis affiche la taille du code engendré. \\
\end{tabular}
\caption{Scripts Python engendrés dans le répertoire \texttt{ws}}
\labelTableau{scriptsPythonEngendres}
\ligne
\end{table}

\begin{table}[t!]
\centering
\begin{tabular}{p{2cm}p{11cm}}
  \textbf{Nom} & \textbf{Commentaire} \\
  \texttt{src.ll} & Contient tout le code LLVM produit par la compilation du fichier source {\tt f{}ichier.plm}. \\
  \texttt{src.s} & Contient tout le code assembleur produit par la compilation du fichier source {\tt f{}ichier.plm}. En fonction de la cible, ce fichier peut être vide.\\
  \texttt{make.py} & Makefile générique écrit en Python.\\
  \texttt{plm.py} & Configuration du makefile générique pour la compilation PLM.\\
  \texttt{link.ld} & Script de l'édition de liens.\\
\end{tabular}
\caption{Fichiers engendrés dans le répertoire \texttt{ws/sources}}
\labelTableau{fichiersSourcesEngendres}
\ligne
\end{table}

La compilation place dans le répertoire \texttt{ws} les scripts \emph{python} décrits par le \refTableau{scriptsPythonEngendres}. Ces scripts peuvent être appelés directement à partir de la ligne de commande.


Le répertoire \texttt{ws/sources} contient les fichiers décrits dans le \refTableau{fichiersSourcesEngendres}, \texttt{ws/objects} contient les fichiers objets issus de la compilation et l'assemblage, et \texttt{ws/product} le fichier exécutable nommé \texttt{plm.elf}. Suivant la cible, il peut aussi contenir l'exécutable \texttt{plm.ihex} sous le format \emph{hex} d'Intel.
