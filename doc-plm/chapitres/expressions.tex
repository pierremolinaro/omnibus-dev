%!TEX encoding = UTF-8 Unicode
%!TEX root = ../doc-plm.tex





\chapter{Expressions}


\begin{table}[ht]
\centering
\begin{tabular}{llll}
  \textbf{Priorité} & \textbf{Opérateur} & \textbf{Commentaire} & \textbf{Lien} \\
   0 (plus prioritaire) & \plm+-+, \plm+%-+ & \emph{moins} unaire & \\
   0 & \plm+~+, \plm+not+ & \emph{non} binaire et \emph{non} logique & \\
   1 & \plm=convert= & Conversion &\\
   2 & \plm+*+, \plm+%*+, \plm+/+, \plm+%/+, \plm-%-, \plm-&%- & Multiplication, division, modulo & \\
   3 & \plm-+-, \plm-%+-, \plm+-+, \plm+%-+ & Addition, soustraction & \\
   4 & \plm+<<+, \plm+>>+ & Décalage à gauche et à droite & \\
   5 & \plm+<=+, \plm+<+, \plm+>=+, \plm+>+ & Comparaison & \\
   6 & \plm+==+, \plm+!=+ & Test d'égalité, d'inégalité & \\
   7 & \plm+&+ & \emph{et} binaire & \\
   8 & \plm+^+ & \emph{ou exclusif} binaire & \\
   9 & \plm+|+ & \emph{ou} binaire & \\
   10 & \plm+and+ & \emph{et} logique & \\
   11 & \plm+xor+ & \emph{ou exclusif} logique & \\
   12 (moins prioritaire) & \plm+or+ & \emph{ou} logique & \\
\end{tabular}
\caption{Priorité des opérateurs}\index{Operateur@Opérateur!Priorite@Priorité}
\labelTableau{tableauPrioriteOperateurs}
\end{table}
