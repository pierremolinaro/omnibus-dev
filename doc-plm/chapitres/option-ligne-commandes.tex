%!TEX encoding = UTF-8 Unicode
%!TEX root = ../doc-plm.tex





\chapter{Options de la ligne de commande}

Les options de la ligne de commande commencent toutes par un caractère « \texttt{-} ». La forme courte début par un simple « \texttt{-} », la forme longue par un double « \texttt{-} ». Certaines options acceptent au choix les deux formes (par exemple \texttt{-{}-verbose} ou \texttt{-v}). Une option ne commençant par un « \texttt{-} » est considérée comme un nom de fichier source. Les deux seules extensions autorisées pour les fichiers sources sont « \texttt{.plm} » (fichier source PLM) et  « \texttt{.plms} » (fichier de description d'une cible). Par exemple, pour compiler un fichier source \texttt{source.plm} avec l'option \emph{verbose} :

\texttt{plm -v source.plm}

Les options peuvent apparaître avant ou après le nom du fichier source, elles sont toujours examinées avant que la compilation ne commence.


\section{Options générales}

{\ttfamily\bfseries-{}-help} Affiche l'aide.


{\ttfamily\bfseries-{}-version} Affiche la version du compilateur PLM.



{\ttfamily\bfseries-{}-log-file-read} Affiche l'accès en lecture à tout fichier.

{\ttfamily\bfseries-{}-max-errors=$n$} Arrête la compilation si le nombre de $n$ erreurs est atteint. Par défaut, la borne est égale à $100$.

{\ttfamily\bfseries-{}-max-warnings=$n$} Arrête la compilation si le nombre de $n$ alertes est atteint. Par défaut, la borne est égale à $100$.

{\ttfamily\bfseries-{}-Werror} Transforme toute alerte en erreur.

{\ttfamily\bfseries-{}-verbose}, {\ttfamily\bfseries-v} Affiche des messages indiquant la progression de la compilation.








\sectionLabel{Options affectant le code engendré}{optionCodeEngendre}

{\ttfamily\bfseries-{}-no-exception-generation} Inhibe la génération du code des exceptions.



{\ttfamily\bfseries-{}-no-file-generation} Inhibe l'écriture de fichiers par le compilateur.


{\ttfamily\bfseries-{}-routine-invocation-graph}, {\ttfamily\bfseries-i} Engendre un fichier au format \texttt{dot} pouvant être ouvert par \texttt{graphviz}\index{graphviz} contenant le graphe d'invocation des routines.

{\ttfamily\bfseries-{}-no-c-compilation}, {\ttfamily\bfseries-c} N'effectue ni la compilation du code C engendré, ni le flashage du micro-contrôleur cible.


{\ttfamily\bfseries-{}-no-flash}, {\ttfamily\bfseries-f} N'effectue pas le flashage du micro-contrôleur cible.


{\ttfamily\bfseries-{}-do-not-detect-recursive-calls}, {\ttfamily\bfseries-r} N'effectue pas la détection des routines récursives. Par défaut le compilateur affiche un message d'erreur si il trouve des routines récursives.\index{Routines!recursives@récursives}






\section{Options de débogage du compilateur}


{\ttfamily\bfseries-{}-output-concrete-syntax-tree} Engendre un fichier au format \texttt{dot} pouvant être ouvert par \texttt{graphviz}\index{graphviz} contenant l'arbre syntaxique concret du texte source.


{\ttfamily\bfseries-{}-mode=$nom$}, où \emph{nom} peut prendre pour valeur :
\begin{itemize}
  \item « \emph{vide} » : fonctionnement nominal, le compilateur effectue toutes les phases : analyse lexicale, analyse syntaxique, analyse sémantique, et génération de code ;
  \item \texttt{lexical-only} : le compilateur s'arrête après l'analyse lexicale, et affiche la séquence des symboles terminaux obtenue ;
  \item \texttt{syntax-only} : le compilateur s'arrête après l'analyse syntaxique, et affiche l'arbre de dérivation.
  \item \texttt{latex} : le compilateur s'arrête après l'analyse lexicale, et engendre un fichier latex contenant le texte source.
\end{itemize}

Écrire l'option \texttt{-{}-mode=} est équivalent à l'absence de cette option.






\section{Options d'accès aux fichiers d'exemple embarqués}

{\ttfamily\bfseries-{}-list-embedded-samples}, {\ttfamily\bfseries-l} Affiche la liste des fichiers d'exemple embarqués dans le compilateur.

{\ttfamily\bfseries-{}-extract-embedded-sample-code=$nom$}, {\ttfamily\bfseries-x=$nom$} Extrait le fichier d'exemple $nom$ et l'écrit dans le répertoire courant.

Voici comment on utilise ces deux options pour obtenir un fichier d'exemple. D'abord, appeler l'option \texttt{-l} :

\begin{SHELL}
 plm -l
\end{SHELL}


Cette commande affiche la liste de tous les exemples embarqués dans le compilateur, triés par cible :

\begin{SHELL}
Embedded sample code:\\ 
\hspace*{1.2em}/teensy-3-1-sequential-systick/blinkled.plm\\
\hspace*{1.2em}...
\end{SHELL}


Pour extraire le fichier d'exemple ci-dessus, exécuter :

\begin{SHELL}
plm -x=/teensy-3-1-sequential-systick/blinkled.plm
\end{SHELL}


Ceci écrit le fichier \texttt{blinkled.plm} dans le répertoire courant.

