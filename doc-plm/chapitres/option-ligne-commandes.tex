%!TEX encoding = UTF-8 Unicode
%!TEX root = ../doc-plm.tex


\newcommand\OPTION[1]{\colorbox{gray!15}{\ttfamily\bfseries #1}}


\chapter{Options de la ligne de commande}

Les options de la ligne de commande commencent toutes par un caractère « \texttt{-} ». La forme courte débute par un simple « \texttt{-} », la forme longue par un double « \texttt{-} ». Certaines options acceptent au choix les deux formes (par exemple \texttt{-{}-verbose} ou \texttt{-v}). Une option ne commençant par un « \texttt{-} » est considérée comme un nom de fichier source. Les deux seules extensions autorisées pour les fichiers sources sont « \texttt{.plm} » (fichier source PLM) et  « \texttt{.plm-target} » (fichier de description d'une cible). Par exemple, pour compiler un fichier source \texttt{source.plm} avec l'option \emph{verbose} :

\begin{SHELL}
\bfseries plm -v source.plm
\end{SHELL}

Les options peuvent apparaître avant ou après le nom du fichier source, elles sont toujours examinées avant que la compilation ne commence.


\sectionLabel{Options générales}{optionsGenerales}

\OPTION{-{}-help} Affiche l'aide.


\OPTION{-{}-version} Affiche la version du compilateur PLM.



\OPTION{-{}-log-file-read} Affiche l'accès en lecture à tout fichier.

\OPTION{-{}-max-errors=$n$} Arrête la compilation si le nombre de $n$ erreurs est atteint. Par défaut, la borne est égale à $100$.

\OPTION{-{}-max-warnings=$n$} Arrête la compilation si le nombre de $n$ alertes est atteint. Par défaut, la borne est égale à $100$.

\OPTION{-{}-Werror} Transforme toute alerte en erreur.

\OPTION{-{}-verbose}, \OPTION{-v} Affiche des messages indiquant la progression de la compilation.








\sectionLabel{Options affectant le code engendré}{optionCodeEngendre}

\OPTION{-{}-no-panic-generation} Inhibe la génération du code de la panique organisée.



\OPTION{-{}-no-file-generation} Inhibe l'écriture de fichiers par le compilateur.


\OPTION{-{}-routine-invocation-graph}, \OPTION{-i} Engendre un fichier au format \texttt{dot} pouvant être ouvert par \texttt{graphviz}\index{graphviz} contenant le graphe d'invocation des routines.

\OPTION{-{}-no-c-compilation}, \OPTION{-c} N'effectue ni la compilation du code C engendré, ni le flashage du micro-contrôleur cible.


\OPTION{-{}-flash-target}, \OPTION{-f} Après une compilation sans erreur, effectue le flashage du micro-contrôleur cible.


\OPTION{-{}-do-not-detect-recursive-calls}, \OPTION{-r} N'effectue pas la détection des routines récursives. Par défaut le compilateur affiche un message d'erreur si il trouve des routines récursives.\index{Routines!recursives@récursives}






\section{Options de débogage du compilateur}


\OPTION{-{}-output-concrete-syntax-tree} Engendre un fichier au format \texttt{dot} pouvant être ouvert par \texttt{graphviz}\index{graphviz} contenant l'arbre syntaxique concret du texte source.


\OPTION{-{}-mode=$nom$}, où \emph{nom} peut prendre pour valeur :
\begin{itemize}
  \item « \emph{vide} » : fonctionnement nominal, le compilateur effectue toutes les phases : analyse lexicale, analyse syntaxique, analyse sémantique, et génération de code ;
  \item \texttt{lexical-only} : le compilateur s'arrête après l'analyse lexicale, et affiche la séquence des symboles terminaux obtenue ;
  \item \texttt{syntax-only} : le compilateur s'arrête après l'analyse syntaxique, et affiche l'arbre de dérivation.
  \item \texttt{latex} : le compilateur s'arrête après l'analyse lexicale, et engendre un fichier latex contenant le texte source.
\end{itemize}

Écrire l'option \texttt{-{}-mode=} est équivalent à l'absence de cette option.






\sectionLabel{Options d'accès aux fichiers d'exemple embarqués}{optionsExemplesEmbarques}

\OPTION{-{}-list-embedded-samples}, \OPTION{-l} Affiche la liste des fichiers d'exemple embarqués dans le compilateur.

\OPTION{-{}-extract-embedded-sample-code=$nom$}, \OPTION{-x=$nom$} Extrait le fichier d'exemple $nom$ et l'écrit dans le répertoire courant.

L'utilisation de ces deux options est illustrée à la \refSectionPage{exempleBlinkled}.






\sectionLabel{Options d'accès aux cibles embarquées}{optionsCiblesEmbarquees}

\OPTION{-{}-list-embedded-targets}, \OPTION{-L} Affiche la liste des cibles embarquées dans le compilateur.

\OPTION{-{}-extract-embedded-targets=$repertoire$}, \OPTION{-X=$repertoire$} Extrait les fichiers de définition des cibles embarquées dans le compilateur et les écrit dans le répertoire $repertoire$.

\OPTION{-{}-use-target-dir=$repertoire$}, \OPTION{-T=$repertoire$} N'utilise pas les cibles embarquées, au profit des cibles définies dans le répertoire $repertoire$.

L'utilisation de ces options est illustrée à la \refSectionPage{exempleDefinitionCible}.
