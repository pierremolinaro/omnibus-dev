%!TEX encoding = UTF-8 Unicode
%!TEX root = ../doc-plm.tex





\chapterLabel{Définition d'une cible}{chapitreConfCible}

Dans ce chapitre, nous allons voir comment est définie une cible, en prenant pour exemple la cible \texttt{teensy-3-1-it}. 

Si vous voulez définir votre propre cible, la lecture de ce chapitre est indispensable. 

Le compilateur recherche la cible citée :
\begin{itemize}
  \item par défaut, parmi les fichiers qu'il embarque ;
  \item si l'option \texttt{-T=$cibles$} est présente dans la ligne de commande, dans les fichiers du répertoire $cibles$.
\end{itemize}

Pour explorer les fichiers indispensables à la définition d'une cible, il est nécessaire de les placer dans le système de fichiers. La première section de ce chapitre (\refSectionPage{cibleDansSystemeFichiers}) va donc indiquer comment extraire de l'exécutable les fichiers de définition des cibles, et comment les exploiter.

%\minitoc

\sectionLabel{Utilisation d'une cible dans le système de fichiers}{cibleDansSystemeFichiers}

Une cible est définie par une arborescence de fichiers, soit embarquée dans le compilateur, soit situé dans le système de fichiers. On va voir dans cette section comment est configurée une cible dans le système de fichiers, permettant ainsi de créer ses propres cibles. Pour cela, on commence par extraire de l'exécutable du compilateur les cibles qu'il embarque.

\subsection{Liste des cibles embarquées}
À titre d'information, on peut appeler l'option \texttt{-L} pour obtenir la liste des cibles embarquées :
\begin{SHELL}
{\bfseries plm -L}\\
Embedded targets:\\ 
\hspace*{1.2em}teensy-3-1-it\\
\hspace*{1.2em}...
\end{SHELL}

\subsection{Extraction des cibles embarquées}
L'option \texttt{-X=cibles} permet d'extraire de l'exécutable du compilateur tous les fichiers de définition des cibles et de les placer dans le répertoire \texttt{cibles}\footnote{Ici, le répertoire destination \texttt{cibles} est un répertoire relatif au répertoire courant, mais un répertoire absolu peut aussi être utilisé.} :
\begin{SHELL}
{\bfseries plm -X=cibles}\\
\hspace*{1.2em}cibles/makef{}ile.py\\
\hspace*{1.2em}cibles/teensy-3-1-it.plm-target\\
\hspace*{1.2em}cibles/files/boot-teensy-3-1-it.plm\\
\hspace*{1.2em}...
\end{SHELL}

\subsection{Extraction d'un fichier d'exemple}
On extrait maintenant un fichier d'exemple \texttt{01-blinkled.plm} pour la cible \texttt{teensy-3-1-it} :
\begin{SHELL}
\bfseries plm -x=/teensy-3-1-it/01-blinkled.plm
\end{SHELL}

\subsection{Compilation du fichier exemple}
Le fichier \texttt{01-blinkled.plm} est écrit dans le répertoire local. Maintenant, pour compiler ce fichier, on peut faire référence aux cibles que l'on vient d'extraire au moyen de l'option \texttt{-T} :
\begin{SHELL}
\bfseries plm -T=cibles 01-blinkled.plm
\end{SHELL}


\subsection{Changement du nom de la cible dans le système de fichiers}

Pour se convaincre que la cible nommée dans le fichier est bien celle située dans le répertoire \texttt{cibles}, nous allons changer son nom. Pour cela, nous devons modifier :
\begin{itemize}
  \item la référence à la cible dans le fichier \texttt{01-blinkled.plm} ;
  \item le nom d'un fichier et d'un répertoire dans le répertoire \texttt{cibles}. 
\end{itemize}

Commençons par le fichier \texttt{01-blinkled.plm}. Sa première ligne est :

\begin{PLM}[1]
target "teensy-3-1"
\end{PLM}

Cette ligne signifie que la cible s'appelle \texttt{teensy-3-1-it}. Nous allons la renommer en \texttt{myTeensy}. La première ligne du fichier \texttt{01-blinkled.plm} devient donc :

\begin{PLM}[1]
target "myTeensy"
\end{PLM}

Maintenant, si on essaie de compiler le fichier \texttt{01-blinkled.plm}, une erreur se déclenche : la cible \texttt{myTeensy} n'est pas définie. Pour la définir, listons le contenu le répertoire \texttt{cibles} :
\begin{SHELL}
{\bfseries ls -1F cibles}\\
f{}iles/\\
makef{}ile.py\\
...\\
teensy-3-1-it/\\
teensy-3-1-it.plm-target\\
...
\end{SHELL}

Le répertoire \texttt{f{}iles} et le fichier \texttt{makef{}ile.py} seront présentés plus loin. Chaque cible est définie par un répertoire et un fichier, dont l'extension est \texttt{plm-target}. Nous allons donc effectuer les renommages suivants :
\begin{itemize}
  \item le répertoire \texttt{teensy-3-1-it} est renommé en \texttt{myTeensy} ;
  \item le fichier \texttt{teensy-3-1-it.plm-target} en \texttt{myTeensy.plm-target}.
\end{itemize}

À présent, la compilation s'effectue avec succès :
\begin{SHELL}
\bfseries plm -T=cibles 01-blinkled.plm
\end{SHELL}

Si on essaie de compiler en utilisant les cibles embarquées dans le compilateur :
\begin{SHELL}
\bfseries plm 01-blinkled.plm
\end{SHELL}

Une erreur se déclenche, la cible \texttt{myTeensy} n'étant pas définie dans l'exécutable du compilateur.






\sectionLabel{Comment est définie une cible}{exempleDefinitionCible}

Nous allons décrire comment est définie une cible dans le système de fichiers. Nous prenons comme exemple la cible \texttt{myTeensy} du répertoire \texttt{cibles} (voir la \refSectionPage{cibleDansSystemeFichiers}).

Pour la cible \texttt{myTeensy} :
\begin{itemize}
  \item le fichier \texttt{myTeensy.plm-target} permet de paramétrer l'analyse sémantique ;
  \item le répertoire \texttt{myTeensy} permet de configurer la génération de code.
\end{itemize}

On va explorer successivement le fichier \texttt{myTeensy.plm-target} puis le répertoire \texttt{myTeensy}.











\subsectionLabel{Le fichier \texttt{myTeensy.plm-target}}{fichierTarget}

Voici le contenu du fichier \texttt{myTeensy.plm-target} :

\begin{PLM}[1]
configuration
  $int32 // Panic code type
  :$uint32 // Panic line type
  :$uint32 // Unsigned integer type
  :32 // Pointer bit count
  :1024 // System stack size
  :32 // User stacked registers byte count
  :12 // Service handler system stack byte count footprint 
  :8 // Section handler system stack byte count footprint
  :"udfcoded" // Section invocation scheme
{ // Interruption vectors
  NMI : 2
  MemManage : 4
  BusFault : 5
  UsageFault : 6
  DebugMonitor : 12
  PendSV : 14
  systick : 15
  DMAChannel0TranfertComplete : 16
  ...
  softwareInterrupt : 110
}

import "files/registers-mk20dx256.plm"
import "files/teensy-3-1-boot.plm"
import "files/teensy-3-1-xtr.plm"
import "files/teensy-3-1-digital-io.plm"
import "files/teensy-3-1-lcd.plm"
import "files/teensy-3-1-panic.plm"
import "files/semaphore.plm"
\end{PLM}

Le fichier \texttt{myTeensy.plm-target} contient la directive \plm=configuration= et une séquence d'importation, qui liste tous les fichiers qui seront automatiquement inclus lors de la compilation d'un projet.

{\bf Panic code type.} C'est le type utilisé pour le code de panique.

{\bf Panic line type.} Type utilisé pour coder le numéro de la ligne source du fichier contenant l'instruction qui peut engendrer une panique.

{\bf Unsigned integer type.}

{\bf Pointer bit count.} Le nombre de bits d'un pointeur. Le Cortex-M4 étant une machine 32 bits, sa valeur est 32.

{\bf System stack size.} La taille en octets allouée à la pile système.

{\bf User stacked registers byte count.} Le nombre d'octets empilés dans la pile utilisateur lors d'une exception. Le Cortex-M4 empilant 8 registres, cette valeur est 32. Elle est utilisée pour les calculs de pile.

{\bf Service handler system stack byte count footprint.} Le \emph{service handler}\footnote{Utilisé pour implémenter les appels système par \texttt{SVC} et les interruptions.} utilise le nombre indiqué d'octets dans la pile. Cette valeur est utilisée pour les calculs de pile.

{\bf Section handler system stack byte count footprint.} Le \emph{section handler}\footnote{Utilisé pour implémenter les sections.} utilise le nombre indiqué d'octets dans la pile. Cette valeur est utilisée pour les calculs de pile.

{\bf Section invocation scheme.} Cette chaîne définit le schéma qui est utilisé pour implémenter les appels de section (\refSubsectionPage{schemaAppelSections}).

{\bf Interruption vectors.} La liste de toutes les interruptions, chacune d'elle étant accompagnée de son numéro.






\subsectionLabel{Schéma d'appel des sections}{schemaAppelSections}

Quand le code d'une tâche appelle une section, celui-ci doit s'effectuer à travers un \emph{system call}\footnote{Quand une section est appelée à partir d'un mode privilégié, l'appel est un simple appel de routine.}.

L'implémentation de cette opération est paramètrable grâce à quatre fichiers assembleurs, placés dans le répertoire \texttt{myTeensy}. Le nom de ces fichiers est préfixé par la chaîne spécifiée dans le champ \emph{Section invocation scheme} de la \plm=configuration= (voir \refSubsectionPage{fichierTarget}), ce qui autorise de disposer de plusieurs implémentations possibles. Pour le schéma nommé \plm="udfcoded"=, ces quatre fichiers sont :
\begin{itemize}
  \item \texttt{udfcoded-section-dispatcher-code.s} : implémentation du code exécuté par l'appel système invoqué par l'appel à une section ;
  \item \texttt{udfcoded-section-dispatcher-entry.s} : entrée du tableau définissant le nom de la fonction implémentant la section ;
  \item \texttt{udfcoded-section-dispatcher-header.s} : en-tête du tableau contenant les noms de la fonction implémentant la section ;
  \item \texttt{udfcoded-section-invocation.s} : implémentation de l'appel à une section. 
\end{itemize}

Trois schémas sont proposés, \plm="udfcoded"=, \plm="r12idx"= et \plm="r12direct"=, et sont décris dans les sections suivantes. Le \refTableauPage{dureeSchemasServices} résume les caractéristiques de chacun. Les durées sont estimées grâce au projet \texttt{04-section-service-duration.plm}. 

\begin{table}[t]
\centering
\begin{tabular}{llll}
  \textbf{Nom du schéma} & \textbf{Par entrée} & \textbf{\texttt{as\_section\_handler}} & \textbf{Durée estimée} \\
   & (octets) & (octets) & (en cycles d'horloge) \\
  \plm="udfcoded"= & 4 + 4 & 40 & 64 \\
  \plm="r12idx"= & 8 + 4 & 36 & 65 \\
  \plm="r12direct"= & 12 + 0 & 24 & 61 \\
\end{tabular}
\caption{Caractèristiques de différents schémas d'appel de section}
\labelTableau{dureeSchemasServices}
\ligne
\end{table}


\subsubsection{Schéma \texttt{udfcoded} d'appel des sections}

Dans ce schéma, l'instruction indéfinie \texttt{UDF} est utilisée. Le code de cette instruction est \texttt{0xDF$xx$}, où $xx$ est une valeur que le Cortex ignore. On utilise ce champ pour coder l'indice du service appelé.

Pour coder l'appel relatif à une section, PLM va utiliser le fichier \texttt{udfcoded-section-invocation.s}. Le contenu de fichier est :
\begin{SHELL}\small
\hspace*{1.2em}.section  ".text.!USER\_ROUTINE!","ax",\%progbits\\
\hspace*{1.2em}.global !USER\_ROUTINE!\\
\hspace*{1.2em}.type  !USER\_ROUTINE!,\%function\\
\hspace*{1.2em}.align  1\\
\hspace*{1.2em}.code  16\\
\hspace*{1.2em}.thumb\_func\\
\\
!USER\_ROUTINE!:\\
\hspace*{1.2em}.fnstart\\
\hspace*{1.2em}udf !IDX!\\
\hspace*{1.2em}bx  lr\\
\\
.Lfunc\_end\_!USER\_ROUTINE!:\\
\hspace*{1.2em}.size  !USER\_ROUTINE!, .Lfunc\_end\_!USER\_ROUTINE! - !USER\_ROUTINE!\\
\hspace*{1.2em}.cantunwind\\
\hspace*{1.2em}.fnend
\end{SHELL}

Trois séquences sont remplacées par le compilateur PLM :
\begin{itemize}
\item \texttt{!USER\_ROUTINE!}, par le nom assembleur de la fonction qui implémente l'appel de la section en mode utilisateur ;
\item \texttt{!IMPLEMENTATION\_ROUTINE!}, par le nom assembleur de la fonction qui implémente la section (inutilisée dans ce shéma) ;
\item \texttt{!IDX!}, par l'indice de la section\footnote{Lorsqu'il analyse un projet, PLM numérote les sections qu'il rencontre (à partir de $0$). C'est ce numéro qui remplace la séquence \texttt{!IDX!}.}.
\end{itemize}

PLM effectue donc cette opération pour chaque section, par ordre croissant de l'indice, et accumule le texte produit dans le fichier assembleur \texttt{src.s}.

Donc, quand une instruction \texttt{UDF} est rencontrée, le Cortex exécute l'exception n°3, qui renvoie vers \texttt{as\_section\_handler}, défini par le fichier \texttt{udfcoded-section-dispatcher-code.s} :

\begin{SHELL}\small
\hspace*{1.2em}.section	".text.as\_section\_handler","ax",\%progbits\\
\hspace*{1.2em}.global as\_section\_handler\\
\hspace*{1.2em}.type as\_section\_handler, \%function\\
\\
as\_section\_handler:\\
@--------------------- Save preserved registers\\
\hspace*{1.2em}push  {r5, lr}\\
@--------------------- R5 <- thread SP\\
\hspace*{1.2em}mrs   r5, psp           @ r5 <- thread SP\\
@--------------------- LR <- Address of UDF instruction\\
\hspace*{1.2em}ldr   lr, [r5, \#24]     @ 24 : 6 stacked registers before saved PC\\
@--------------------- Set return address to instruction following UDF\\
\hspace*{1.2em}adds  lr, \#2\\
\hspace*{1.2em}str   lr, [r5, \#24]\\
@--------------------- R12 <- address of dispatcher\\
\hspace*{1.2em}ldr   r12, =\_\_udf\_dispatcher\_table\\
@--------------------- LR <- bits 0-7 of UDF instruction\\
\hspace*{1.2em}ldrb  lr, [lr, \#-2]            @ LR is service call index\\
@--------------------- r12 <- address of routine to call\\
\hspace*{1.2em}ldr   r12, [r12, lr, lsl \#2]   @ Address R12 + LR << 2\\
@--------------------- Call service routine\\
\hspace*{1.2em}blx   r12                      @ R5: thread PSP\\
@--------------------- Set return code (from R0 to R3) in stacked registers\\
\hspace*{1.2em}stmia r5!, {r0, r1, r2, r3}    @ R5 is thread SP\\
@--------------------- Restore preserved registers, return from interrupt\\
\hspace*{1.2em}pop   {r5, pc}
\end{SHELL}

Ce code va rechercher dans l'instruction \texttt{UDF} la valeur $xx$ de l'octet de poids faible, et l'utilise pour construire l'adresse d'une entrée du tableau \texttt{\_\_udf\_dispatcher\_table}. Cette entrée contient l'adresse de la routine à exécuter.

Il faut donc construire ce tableau. Le fichier \texttt{udfcoded-section-dispatcher-header.s} définit son en-tête :

\begin{SHELL}\small
\_\_udf\_dispatcher\_table:
\end{SHELL}

Ce texte est ajouté à la fin du fichier assembleur \texttt{src.s}.

Ensuite, PLM va utiliser le fichier \texttt{udfcoded-section-dispatcher-entry.s} pour définir chaque entrée du tableau :
\begin{SHELL}\small
\hspace*{1.2em}.word !IMPLEMENTATION\_ROUTINE! @ !IDX!, user routine !USER\_ROUTINE!

\end{SHELL}

PLM effectue pour chaque section les substitutions de \texttt{!IMPLEMENTATION\_ROUTINE!}, de \texttt{!USER\_ROUTINE!} et de \texttt{!IDX!}, par ordre croissant de l'indice, en ajoutant le texte produit à la fin du fichier assembleur \texttt{src.s}.










\subsubsection{Schéma \texttt{r12idx} d'appel des sections}

Dans ce schéma, l'instruction indéfinie \texttt{UDF} est utilisée.  Le code de cette instruction est \texttt{0xDF$xx$}, où $xx$ est une valeur que le Cortex ignore. Contrairement au shéma précédent, le champ $xx$ est toujours à $0$, l'indice de la section est entré dans le registre $r12$.

Pour coder l'appel relatif à une section, PLM va utiliser le fichier \texttt{r12idx-section-invocation.s}. Le contenu de fichier est :
\begin{SHELL}\small
\hspace*{1.2em}.section  ".text.!USER\_ROUTINE!","ax",\%progbits\\
\hspace*{1.2em}.global !USER\_ROUTINE!\\
\hspace*{1.2em}.type  !USER\_ROUTINE!,\%function\\
\hspace*{1.2em}.align  1\\
\hspace*{1.2em}.code  16\\
\hspace*{1.2em}.thumb\_func\\
\\
!USER\_ROUTINE!:\\
\hspace*{1.2em}.fnstart\\
\hspace*{1.2em}mov r12, !IDX!\\
\hspace*{1.2em}udf 0\\
\hspace*{1.2em}bx  lr\\
\\
.Lfunc\_end\_!USER\_ROUTINE!:\\
\hspace*{1.2em}.size  !USER\_ROUTINE!, .Lfunc\_end\_!USER\_ROUTINE! - !USER\_ROUTINE!\\
\hspace*{1.2em}.cantunwind\\
\hspace*{1.2em}.fnend
\end{SHELL}

Trois séquences sont remplacées par le compilateur PLM :
\begin{itemize}
\item \texttt{!USER\_ROUTINE!}, par le nom assembleur de la fonction qui implémente l'appel de la section en mode utilisateur ;
\item \texttt{!IMPLEMENTATION\_ROUTINE!}, par le nom assembleur de la fonction qui implémente la section (inutilisée dans ce shéma) ;
\item \texttt{!IDX!}, par l'indice de la section\footnote{Lorsqu'il analyse un projet, PLM numérote les sections qu'il rencontre (à partir de $0$). C'est ce numéro qui remplace la séquence \texttt{!IDX!}.}.
\end{itemize}

PLM effectue donc cette opération pour chaque section, par ordre croissant de l'indice, et accumule le texte produit dans le fichier assembleur \texttt{src.s}.

Donc, quand une instruction \texttt{UDF} est rencontrée, le Cortex exécute l'exception n°3, qui renvoie vers \texttt{as\_section\_handler}, défini par le fichier \texttt{r12idx-section-dispatcher-code.s} :

\begin{SHELL}\small
\hspace*{1.2em}.section	".text.as\_section\_handler","ax",\%progbits\\
\hspace*{1.2em}.global as\_section\_handler\\
\hspace*{1.2em}.type as\_section\_handler, \%function\\
\\
as\_section\_handler:\\
@--------------------- Save preserved registers\\
\hspace*{1.2em}push  {r5, lr}\\
@--------------------- R5 <- thread SP\\
\hspace*{1.2em}mrs   r5, psp           @ r5 <- thread SP\\
@--------------------- LR <- Address of UDF instruction\\
\hspace*{1.2em}ldr   lr, [r5, \#24]     @ 24 : 6 stacked registers before saved PC\\
@--------------------- Set return address to instruction following UDF\\
\hspace*{1.2em}adds  lr, \#2\\
\hspace*{1.2em}str   lr, [r5, \#24]\\
@--------------------- LR <- address of dispatcher table\\
\hspace*{1.2em}ldr   lr, =\_\_udf\_dispatcher\_table\\
@--------------------- r12 <- address of routine to call\\
\hspace*{1.2em}ldr   r12, [lr, r12, lsl \#2]   @ Address : LR + R12 << 2\\
@--------------------- Call service routine\\
\hspace*{1.2em}blx   r12                      @ R5: thread PSP\\
@--------------------- Set return code (from R0 to R3) in stacked registers\\
\hspace*{1.2em}stmia r5!, {r0, r1, r2, r3}    @ R5 is thread SP\\
@--------------------- Restore preserved registers, return from interrupt\\
\hspace*{1.2em}pop   {r5, pc}
\end{SHELL}

Ce code va utiliser la valeur \texttt{r12} pour construire l'adresse d'une entrée du tableau \texttt{\_\_udf\_dispatcher\_table}. Cette entrée contient l'adresse de la routine à exécuter.

Il faut donc construire ce tableau\footnote{Les fichiers \texttt{r12idx-section-dispatcher-header.s} et \texttt{r12idx-section-dispatcher-entry.s} sont identiques à ceux du schéma précédent \texttt{"udfcoded"}.}. Le fichier \texttt{r12idx-section-dispatcher-header.s} définit son en-tête :

\begin{SHELL}\small
\_\_udf\_dispatcher\_table:
\end{SHELL}

Ce texte est ajouté à la fin du fichier assembleur \texttt{src.s}.

Ensuite, PLM va utiliser le fichier \texttt{r12idx-section-dispatcher-entry.s} pour définir chaque entrée du tableau :
\begin{SHELL}\small
\hspace*{1.2em}.word !IMPLEMENTATION\_ROUTINE! @ !IDX!, user routine !USER\_ROUTINE!

\end{SHELL}

PLM effectue pour chaque section les substitutions de \texttt{!IMPLEMENTATION\_ROUTINE!}, de \texttt{!USER\_ROUTINE!} et de \texttt{!IDX!}, par ordre croissant de l'indice, en ajoutant le texte produit à la fin du fichier assembleur \texttt{src.s}.











\subsubsection{Schéma \texttt{r12direct} d'appel des sections}

Dans ce schéma, l'instruction indéfinie \texttt{UDF} est utilisée. L'adresse de la routine implémentant la section est entrée dans le registre $r12$.

Pour coder l'appel relatif à une section, PLM va utiliser le fichier \texttt{r12direct-section-invocation.s}. Le contenu de fichier est :
\begin{SHELL}\small
\hspace*{1.2em}.section  ".text.!USER\_ROUTINE!","ax",\%progbits\\
\hspace*{1.2em}.global !USER\_ROUTINE!\\
\hspace*{1.2em}.type  !USER\_ROUTINE!,\%function\\
\hspace*{1.2em}.align  1\\
\hspace*{1.2em}.code  16\\
\hspace*{1.2em}.thumb\_func\\
\\
!USER\_ROUTINE!:\\
\hspace*{1.2em}.fnstart\\
\hspace*{1.2em}ldr r12, =!IMPLEMENTATION\_ROUTINE!\\
\hspace*{1.2em}udf 0\\
\hspace*{1.2em}bx  lr\\
\\
.Lfunc\_end\_!USER\_ROUTINE!:\\
\hspace*{1.2em}.size  !USER\_ROUTINE!, .Lfunc\_end\_!USER\_ROUTINE! - !USER\_ROUTINE!\\
\hspace*{1.2em}.cantunwind\\
\hspace*{1.2em}.fnend
\end{SHELL}

Trois séquences sont remplacées par le compilateur PLM :
\begin{itemize}
\item \texttt{!USER\_ROUTINE!}, par le nom assembleur de la fonction qui implémente l'appel de la section en mode utilisateur ;
\item \texttt{!IMPLEMENTATION\_ROUTINE!}, par le nom assembleur de la fonction qui implémente la section ;
\item \texttt{!IDX!}, par l'indice de la section (inutilisé dans ce shéma).
\end{itemize}

PLM effectue donc cette opération pour chaque section, par ordre croissant de l'indice, et accumule le texte produit dans le fichier assembleur \texttt{src.s}.

Donc, quand une instruction \texttt{UDF} est rencontrée, le Cortex exécute l'exception n°3, qui renvoie vers \texttt{as\_section\_handler}, défini par le fichier \texttt{r12direct-section-dispatcher-code.s} :

\begin{SHELL}\small
\hspace*{1.2em}.section	".text.as\_section\_handler","ax",\%progbits\\
\hspace*{1.2em}.global as\_section\_handler\\
\hspace*{1.2em}.type as\_section\_handler, \%function\\
\\
as\_section\_handler:\\
@--------------------- Save preserved registers\\
\hspace*{1.2em}push  {r5, lr}\\
@--------------------- R5 <- thread SP\\
\hspace*{1.2em}mrs   r5, psp           @ r5 <- thread SP\\
@--------------------- LR <- Address of UDF instruction\\
\hspace*{1.2em}ldr   lr, [r5, \#24]     @ 24 : 6 stacked registers before saved PC\\
@--------------------- Set return address to instruction following UDF\\
\hspace*{1.2em}adds  lr, \#2\\
\hspace*{1.2em}str   lr, [r5, \#24]\\
@--------------------- Call service routine\\
\hspace*{1.2em}blx   r12                      @ R5: thread PSP\\
@--------------------- Set return code (from R0 to R3) in stacked registers\\
\hspace*{1.2em}stmia r5!, {r0, r1, r2, r3}    @ R5 is thread SP\\
@--------------------- Restore preserved registers, return from interrupt\\
\hspace*{1.2em}pop   {r5, pc}
\end{SHELL}

Ce code utilise directement la valeur \texttt{r12}, c'est l'adresse de la routine à exécuter.

Les fichiers \texttt{r12direct-section-dispatcher-header.s} et \texttt{r12direct-section-dispatcher-entry.s} sont inutiles pour ce schéma, et sont donc vides.










\subsection{Le répertoire \texttt{myTeensy}}

Texte à définir…




\sectionLabel{La directive \texttt{check target}}{directiveCheckTarget}\index{check target}

La directive \plm=check target= permet de vérifier l'identité de la cible. Elle est utile pour s'assurer qu'un fichier est acceptable par une cible. À la compilation, toute directive \plm=check target= incorrecte entraîne une erreur.

Par exemple, compiler pour la cible \plm+"teensy-3-1"+ est spécifié par :

\begin{PLM}
target "teensy-3-1"
\end{PLM}

Si la définition du projet implique d'importer le fichier \plm+"monFichier.plm"+ :

\begin{PLM}
target "teensy-3-1"
import "monFichier.plm"
\end{PLM}

Si le contenu de ce fichier commence par :
\begin{PLM}
check target "lpc2294"
\end{PLM}

Alors une erreur de compilation apparaît. La directive \plm=check target= doit être :
\begin{PLM}
check target "teensy-3-1"
\end{PLM}

