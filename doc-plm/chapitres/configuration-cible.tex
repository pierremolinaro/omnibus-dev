%!TEX encoding = UTF-8 Unicode
%!TEX root = ../doc-plm.tex





\chapterLabel{Configuration d'une cible}{chapitreConfCible}

\sectionLabel{Déclaration \texttt{newUnsignedRepresentation}}{DefNewUnsignedRepresentation}
\index{newUnsignedRepresentation@\plm=newUnsignedRepresentation=!Definition@Définition}

La déclaration \plm=newUnsignedRepresentation= permet de définir une représentation entière non signée. Sa syntaxe est la suivante :


\begin{PLM}
newUnsignedRepresentation @nom "type-c" taille
\end{PLM}

Où :
\begin{itemize}
  \item \plm=@nom= est le nom donné à la représentation ;
  \item \plm="type-c"= est le nom du type C qui sera utilisé lors de la génération de code ;
  \item \plm=taille= est le nombre de bits de cette représentation.
\end{itemize}

Par exemple, la représentation des entiers non signés habituels peut être définie par :
\begin{PLM}
newUnsignedRepresentation @unsigned8  "uint8_t"   8
newUnsignedRepresentation @unsigned16 "uint16_t" 16
newUnsignedRepresentation @unsigned32 "uint32_t" 32
newUnsignedRepresentation @unsigned64 "uint64_t" 64
\end{PLM}








\sectionLabel{Déclaration \texttt{newSignedRepresentation}}{DefNewSignedRepresentation}
\index{newSignedRepresentation@\plm=newSignedRepresentation=!Definition@Définition}

La déclaration \plm=newSignedRepresentation= permet de définir une représentation entière signée. Sa syntaxe est la suivante :


\begin{PLM}
newSignedRepresentation @representation "type-c" taille
\end{PLM}

Où :
\begin{itemize}
  \item \plm=@representation= est le nom donné à la représentation ;
  \item \plm="type-c"= est le nom du type C qui sera utilisé lors de la génération de code ;
  \item \plm=taille= est le nombre de bits de cette représentation.
\end{itemize}

Par exemple, la représentation des entiers signés habituels peut être définie par :
\begin{PLM}
newSignedRepresentation @signed8  "int8_t"   8
newSignedRepresentation @signed16 "int16_t" 16
newSignedRepresentation @signed32 "int32_t" 32
newSignedRepresentation @signed64 "int64_t" 64
\end{PLM}











\sectionLabel{Déclaration \texttt{newIntegerType}}{DefNewIntegerType}
\index{newIntegerType@\plm=newIntegerType=!Definition@Définition}

La déclaration \plm=newIntegerType= permet de définir un nouveau type entier signé ou non signé. Elle a la syntaxe suivante :
\begin{PLM}
newIntegerType NomDeType @representation
\end{PLM}
Où :
\begin{itemize}
  \item \plm=NomDeType= est le nom donné au type entier.
  \item \plm=@representation= est le nom de la représentation, qui doit avoir été défini soit par une déclaration \plm=newUnsignedRepresentation= (\refSectionPage{DefNewUnsignedRepresentation})\index{newUnsignedRepresentation@\plm=newUnsignedRepresentation=}, soit par une déclaration \plm=newSignedRepresentation= (\refSectionPage{DefNewSignedRepresentation}\index{newSignedRepresentation@\plm=newSignedRepresentation=}).
\end{itemize}

Le type entier ainsi déclaré est non signé si la représentation est non signée (c'est-à-dire déclarée avec \plm=newUnsignedRepresentation=), et signé si la représentation est signée (c'est-à-dire déclarée avec \plm=newSignedRepresentation=). Les types usuels peuvent ainsi être déclarés :
\begin{PLM}
newIntegerType UInt8  @unsigned8
newIntegerType UInt16 @unsigned16
newIntegerType UInt32 @unsigned32
newIntegerType UInt64 @unsigned64
newIntegerType Int8  @signed8
newIntegerType Int16 @signed16
newIntegerType Int32 @signed32
newIntegerType Int64 @signed64
\end{PLM}




