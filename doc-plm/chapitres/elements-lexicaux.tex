%!TEX encoding = UTF-8 Unicode
%!TEX root = ../doc-plm.tex



\chapter{Élements lexicaux}

%--- Pour supprimer tout en-tête et pied de page sur la 1re page d'un chapitre
\thispagestyle{empty}





\section{Identificateurs}
Un identificateur commence par une lettre Unicode, qui est suivie par zéro, un ou plusieurs lettres Unicode, chiffres décimaux, caractères «~\texttt{\_}~».

La casse est significative pour les identificateurs. Comme toute lettre Unicode est acceptée, les lettres accentuées, les lettres grecques, cyrilliques, ... sont autorisées :

\begin{PLM}
let entréeValidée Bool = yes
let α UInt4 = 2
let постоянная UInt8 = 200
\end{PLM}



\section{Mots réservés}\index{Mots réservés}

%Certains identificateurs sont réservés (\refSubsectionPage{motsReservesLangage}) .
%
%\subsectionLabel{Mots réservés correspondants aux éléments du langage}{motsReservesLangage}

Les mots réservés correspondant aux éléments du langage sont listés dans le \refTableau{motReservesLangage}.

\begin{table}[htbp]
  \centering
  \begin{tabular}{llllll}
      \plm!and!  &  \plm!assert!  &  \plm!at!  &  \plm!boot!  &  \plm!case!   \\
  \plm!check!  &  \plm!convert!  &  \plm!do!  &  \plm!else!  &  \plm!enum!   \\
  \plm!extend!  &  \plm!extension!  &  \plm!extern!  &  \plm!for!  &  \plm!func!   \\
  \plm!guard!  &  \plm!if!  &  \plm!import!  &  \plm!in!  &  \plm!init!   \\
  \plm!isr!  &  \plm!let!  &  \plm!loop!  &  \plm!module!  &  \plm!no!   \\
  \plm!nop!  &  \plm!not!  &  \plm!or!  &  \plm!panic!  &  \plm!primitive!   \\
  \plm!priority!  &  \plm!public!  &  \plm!register!  &  \plm!required!  &  \plm!safe!   \\
  \plm!section!  &  \plm!self!  &  \plm!service!  &  \plm!setup!  &  \plm!stackSize!   \\
  \plm!staticArray!  &  \plm!struct!  &  \plm!switch!  &  \plm!sync!  &  \plm!system!   \\
  \plm!target!  &  \plm!task!  &  \plm!truncate!  &  \plm!type!  &  \plm!until!   \\
  \plm!user!  &  \plm!var!  &  \plm!when!  &  \plm!while!  &  \plm!xor!   \\
  \plm!yes!  &  &    &    &    \\

  \end{tabular}
  \caption{Mots réservés du langage PLM}
  \labelTableau{motReservesLangage}
%  \ligne
\end{table}




%\section{Nom de type}
%
%En PLM, le nom d'un type commence toujours par le préfixe «~\$~»~: \plm=Bool=, \plm=UInt8=, \plm=$monType=, \dots






%\section{Constante chaîne de caractères}
%
%Comme en C, les chaînes de caractères sont délimitées par des caractères \texttt{"}. Les séquences d’échappement suivantes sont acceptées : \texttt{\textbackslash f}, \texttt{\textbackslash n}, \texttt{\textbackslash r}, \texttt{\textbackslash v}, \texttt{\textbackslash\textbackslash}, \texttt{\textbackslash\textquotedbl}, \texttt{\textbackslash\textquotesingle}, \texttt{\textbackslash0}.
%
%\section{Constante caractère}
%
%Comme en C, les caractères sont délimités par des caractères « \texttt{\textquotesingle} ». Par exemple :\plm!'A'!, \plm!'+'!.


%Les séquences d’échappement suivantes sont acceptées : \plm!'\f'!, \plm!'\n', !\plm!'\r'!, \plm!'\v'!, \plm!'\\'!, \plm!'\''!, \plm!'\0'!.

\section{Constante entière}

Vous pouvez écrire les constantes entières en décimal, en hexadécimal ou en binaire. Une constante entière de type \plm!LiteralInt!~; ce type est décrit au \refChapterPage{chapitreTypeEntierStatique}. 

\textbf{Décimal.} Une constante entière décimale commence par un chiffre décimal, et est suivie par zéro, un ou plusieurs chiffres décimaux, ou caractères «~\texttt{\_}~».

Contrairement au C, un nombre qui commence par un zéro est un nombre écrit en décimal.

\textbf{Hexadécimal.} Un chiffre hexadécimal est soit un chiffre décimal, soit une lettre entre «~\texttt{a}~» et «~\texttt{f}~», écrite indifféremment en minuscule ou en majuscule. Une constante hexadécimale commence par la séquence «~\texttt{0x}~», et est suivie par un ou plusieurs chiffres hexadécimal, ou caractères « \texttt{\_} ».

\textbf{Binaire.} Une constante entière binaire commence par la séquence «~\texttt{0b}~» suivie par un ou plusieurs chiffres binaires, \texttt{0} ou \texttt{1}, ou caractères « \texttt{\_} ».

\textbf{Le caractère «~\_~».} Dans une constante entière, écrite en binaire, décimal ou en hexadécimal, le caractère «~\texttt{\_}~» peut servir de séparateur~; on peut ainsi écrire indifféremment : \plm!123!, \plm!1_23!, \plm!1_2_3!, \plm!1___23!, \plm!1___2__3__!, \dots

\textbf{Taille.} Une constante entière n'est pas limitée en taille : vous pouvez écrire des constantes entières avec un nombre quelconque de chiffres :

\begin{PLM}
let a = 123_456_789_123_456_789_123_456_789_123_456_789_123_456_789
\end{PLM}





\section{Délimiteurs}

PLM définit les délimiteurs listés dans le \refTableau{delimiteursLangage}.

\begin{table}[htbp]
  \centering
  \begin{tabular}{llllllllllllll}
      \plm@!%@  &  \plm@!%=@  &  \plm@!/@  &  \plm@!/=@  &  \plm@%@  &  \plm@%=@  &  \plm@&@  &  \plm@&=@  &  \plm@(@  &  \plm@)@  &  \plm@*@  &  \plm@*%@   \\
  \plm@*%=@  &  \plm@*=@  &  \plm@+@  &  \plm@+%@  &  \plm@+%=@  &  \plm@+=@  &  \plm@,@  &  \plm@-@  &  \plm@-%@  &  \plm@-%=@  &  \plm@-=@  &  \plm@->@   \\
  \plm@.@  &  \plm@...@  &  \plm@..<@  &  \plm@/@  &  \plm@/=@  &  \plm@:@  &  \plm@;@  &  \plm@<@  &  \plm@<<@  &  \plm@<<=@  &  \plm@=@  &  \plm@==@   \\
  \plm@>@  &  \plm@>>@  &  \plm@>>=@  &  \plm@[@  &  \plm@]@  &  \plm@^@  &  \plm@^=@  &  \plm@_@  &  \plm@{@  &  \plm@|@  &  \plm@|=@  &  \plm@}@   \\
  \plm@~@  &  \plm@≠@  &  \plm@≤@  &  \plm@≥@  &  &    &    &    &    &    &    &    \\

  \end{tabular}
  \caption{Délimiteurs du langage PLM}
  \labelTableau{delimiteursLangage}
%  \ligne
\end{table}

%\begin{table}[htbp]
%  \centering
%  \begin{tabular}{ccccccccccccccccc}
%    \plm!:!  & \plm!.! & \plm!,!  & \plm+!=+ & \plm!<=! & \plm!>=! & \plm!;! & \plm!==! & \plm!<! & \plm!>! & \plm![! & \plm!]! \\
%    \plm!&! & \plm!&=! & \plm!|! & \plm!|=!  & \plm!}! & \plm!}! & \plm!(!  & \plm!)!  & \plm!/!  & \plm+!/+ \\
%    \plm!-! & \plm!-%! & \plm!+!  & \plm!+%! & \plm!^!  & \plm!^=! & \plm!<<! & \plm!>>! & \plm!~!\\
%    \plm!%! & \plm+!%+ & \plm!=!\\
%    \plm!->! & \plm!::! & \plm!*! & \plm!*%!
%  \end{tabular}
%  \caption{Délimiteurs du langage PLM}
%  \labelTableau{delimiteursLangage}
%  \ligne
%\end{table}










\sectionLabel{Attributs}{attribut}


Un attribut est une séquence commençant par un \texttt{@} et suivi d'un ou plusieurs chiffres, lettres. Voici quelques attributs valides : \plm!@toto!, \plm!@truc1!, \plm!@123!. Un attribut sert à différents usages comme par exemple (liste non exhautive) :
\begin{itemize}
  \item \plm!@ro! indique qu'un registre de contrôle est accessible en lecture seule (\refSubsectionPage{attributRo})~;
  \item ajouter une étiquette à une instruction \plm!if! (\refSectionPage{instructionIF}), pour en augmenter la lisibilité.
\end{itemize}








\sectionLabel{Sélecteurs}{selecteur}

Un sélecteur spécifie le mode de passage d'un argument formel et d'un paramètre effectif. Ils se présentent sous plusieurs formes :
\begin{itemize}
  \item une forme anonyme : \plm+?+, \plm+!+, \plm+?!+, \plm+!?+~;
  \item \plm+?selecteur:+, \plm+!selecteur:+, \plm+?!selecteur:+, \plm+!?selecteur:+, où $selecteur$ est une séquence de lettres ou de chiffres.
\end{itemize}



Les sélecteurs font l'objet de la \refSectionPage{argumentFormel}.

\section{Séparateurs}

Tout caractère dont le code ASCII est compris entre \texttt{0x01} et \texttt{0x20} est considéré comme un séparateur (ceci inclut donc la tabulation horizontale \texttt{HT} (\texttt{0x09}), le passage à la ligne \texttt{LF} (\texttt{0x0A}), le retour-chariot \texttt{CR} (\texttt{0x0D}), et l’espace (\texttt{0x20}).









\section{Commentaires}

Un commentaire commence par deux barres obliques consécutives \plm+//+, et s’étend jusqu’à la fin de la ligne courante.




\section{Format}

Le format est libre, la fin de ligne n’est pas significative (sauf pour les commentaires, qui commencent par deux barres obliques consécutives \plm!//!, et s’étendent jusqu’à la fin de la ligne courante). Le compilateur accepte de manière indifférente que les fins de ligne soient codés par un caractère LF (\texttt{0x0A}), un caractère CR (\texttt{0x0D}), ou par la séquence CRLF (\texttt{0x0D}, \texttt{0x0A}).

