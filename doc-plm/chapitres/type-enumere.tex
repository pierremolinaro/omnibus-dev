%!TEX encoding = UTF-8 Unicode
%!TEX root = ../doc-plm.tex





\chapter{Les types énumérés}


\section{Déclaration d'un type énuméré}

La déclaration d'un type énuméré est introduite par le mot réservé \plm!enum! :

\begin{PLM}
enum $feu {
  case vert
  case orange
  case rouge
}
\end{PLM}

\section{Utilisation d'un type énuméré}

\subsection{Constructeurs}
La déclaration d'un type énuméré définit les constructeurs associés au type énuméré, ici \plm!$feu.vert!, \plm!$feu.orange! et \plm!$feu.rouge!.


\subsection{Constante globale et variable globale}
Les constructeurs d'un type énuméré sont \emph{statiques}, c'est-à-dire qu'ils permettent d'initialiser des variables globales et des constantes globales :

\begin{PLM}
let ROUGE $feu = $feu.rouge
\end{PLM}

L'annotation de type peut être omis une fois, c'est-à-dire que l'on peut aussi écrire :
\begin{PLM}
let ROUGE $feu = .rouge
\end{PLM}

Ou encore :
\begin{PLM}
let ROUGE = $feu.rouge
\end{PLM}


De même, on peut déclarer une variable globale appartenant à un type énuméré :
\begin{PLM}
var unFeu $feu = $feu.vert { ... }
\end{PLM}

L'annotation de type peut être omis une fois, c'est-à-dire que l'on peut aussi écrire :
\begin{PLM}
var unFeu $feu = .vert { ... }
\end{PLM}

Ou encore :
\begin{PLM}
var unFeu = $feu.vert { ... }
\end{PLM}

\subsection{Comparaison}

Les opérateurs \plm!==!, \plm+!=+, \plm!<!, \plm!<=!, \plm!>! et \plm!>=! permettent de comparer les valeurs d'un type énuméré ; la relation d'ordre est donnée par l'ordre de déclaration des constantes, c'est-à-dire que \plm!$feu.vert < $feu.orange! et \plm!$feu.orange < $feu.rouge!.


\sectionLabel{Représentation d'un type énuméré}{representation-type-enumere}

Par défaut, un type énuméré est représenté dans le code engendré par un \plm!@size!. L'attribut \plm!@size! représente un entier non signé, et doit avoir été défini par le fichier de définition de la cible\footnote{Voir par exemple \refSubsectionPage{fichierTarget}.}.

Il est possible de changer de représentation, en ajoutant une annotation dans la déclaration du type énuméré :
\begin{PLM}
enum $feu @representation {
  case vert
  case orange
  case rouge
}
\end{PLM}

Sont acceptables pour \plm!@representation! toutes les représentations déclarées dans la définition de la cible par \plm!newUnsignedRepresentation!. Ainsi, pour la définition de cible présentée à la \refSubsectionPage{fichierTarget}, on peut utiliser :
\begin{itemize}
  \item \plm!@unsigned8!, représentation sous la forme d'un octet ;
  \item \plm!@unsigned16!, représentation sous la forme de deux octets ;
  \item \plm!@unsigned32!, représentation sous la forme de quatre octets, identique pour cette cible à \plm!@size! ;
  \item \plm!@unsigned64!, représentation sous la forme de huit octets.
\end{itemize}


