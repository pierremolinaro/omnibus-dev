%!TEX encoding = UTF-8 Unicode
%!TEX root = ../doc-plm.tex





\chapter{Les types énumérés}


\section{Déclaration d'un type énuméré}

La déclaration d'un type énuméré est introduite par le mot réservé \plm!enum! :

\begin{PLM}
enum Feu {
  case vert
  case orange
  case rouge
}
\end{PLM}

\section{Utilisation d'un type énuméré}

\subsection{Constructeurs}
La déclaration d'un type énuméré définit les constructeurs associés au type énuméré, ici \plm!Feu.vert!, \plm!Feu.orange! et \plm!Feu.rouge!.


\subsection{Constante globale et variable globale}
Les constructeurs d'un type énuméré sont \emph{statiques}, c'est-à-dire qu'ils permettent d'initialiser des variables globales et des constantes globales :

\begin{PLM}
let ROUGE : Feu = Feu.rouge
\end{PLM}

L'annotation de type peut être omis une fois, c'est-à-dire que l'on peut aussi écrire :
\begin{PLM}
let ROUGE : Feu = .rouge
\end{PLM}

Ou encore :
\begin{PLM}
let ROUGE = Feu.rouge
\end{PLM}


De même, on peut déclarer une variable globale appartenant à un type énuméré :
\begin{PLM}
var unFeu : Feu = Feu.vert
\end{PLM}

L'annotation de type peut être omis une fois, c'est-à-dire que l'on peut aussi écrire :
\begin{PLM}
var unFeu : Feu = .vert
\end{PLM}

Ou encore :
\begin{PLM}
var unFeu = Feu.vert
\end{PLM}

\subsection{Comparaison}

Les opérateurs \plm!==!, \plm+≠+, \plm!<!, \plm!≤!, \plm!>! et \plm!≥! permettent de comparer les valeurs d'un type énuméré ; la relation d'ordre est donnée par l'ordre de déclaration des constantes, c'est-à-dire que \plm!Feu.vert < Feu.orange! et \plm!Feu.orange < Feu.rouge!.







\section{Accesseur}

\subsection{Accesseur \texttt{uint$N$}}

Tout type énuméré implémente un accesseur qui retourne la valeur entière non signée associée à la valeur du récepteur. Le type de la valeur retournée est \plm=UIntN=, où $N$ est le nombre de bits nécessaires pour coder la valeur de ce type.

Par exemple, avec le type énuméré~:
\begin{PLM}
enum Feu {
  case vert
  case orange
  case rouge
}
\end{PLM}

L'accesseur est \plm=uint3=, et~:
\begin{PLM}
let x = Feu.orange
let y UInt3 = x.uint3 () // 1
\end{PLM}








\sectionLabel{Représentation d'un type énuméré}{representation-type-enumere}

Un type énuméré est représenté dans le code engendré par une valeur codée sur le plus petit nombre de bits nécessaire. Par exemple, le type énuméré suivant code trois valeurs.
\begin{PLM}
enum Feu {
  case vert
  case orange
  case rouge
}
\end{PLM}

Un objet de ce type est donc codé sur deux bits, et \plm+vert+ est représenté par $0$, \plm+orange+ par $1$ et \plm=rouge= par $2$.

