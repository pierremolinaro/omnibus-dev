%!TEX encoding = UTF-8 Unicode
%!TEX root = ../doc-plm.tex





\chapterLabel{Les types entiers}{chapitreTypesEntiers}

Sont définis implicitement les types entiers signés et non signés d'une taille variant entre $1$ bit et $32768$ bits, et sont notés :
\begin{itemize}
  \item \plm!$uint1! à \plm!$uint32768! pour les types entiers non signés de $1$ à $32768$ bits ;
  \item \plm!$int1! à \plm!$int32768! pour les types entiers signés de $1$ à $32768$ bits.
\end{itemize}

PLM définit aussi le type \plm!$staticInt!, qui est le type de toute constante entière littérale. Mais ce type est aussi applicable à des constantes entières \emph{statiques}, c'est-à-dire dont la valeur est calculée à la compilation. Ce type est décrit au \refChapterPage{chapitreTypeEntierStatique}.

\section{Constante littérale entière}

Le langage accepte des constantes littérales entières d'une taille quelconque. Une constante est convertie dans le type entier requis par le contexte sémantique, et une erreur est déclenchée à la compilation en cas d'impossibilité. Par exemple :

\begin{PLM}
var v $int8 = 128  // Erreur de compilation : 128 non
                   // représentable par un entier signé 8 bits
var v $int8 = -128 // Ok
\end{PLM}

Une constante littérale entière a pour type \plm!$staticInt!, or ce type n'est pas acceptable pour une variable. Par exemple, si on écrit :
\begin{PLM}
var v = 28 // Erreur, le type $staticInt n'est pas valide pour une variable
\end{PLM}

Dans ce cas, il faut que la déclaration contienne l'annotation de type :
\begin{PLM}
var v $int32 = 28 // Ok
\end{PLM}




\section{Conversion entre objets de type entier}

Il y a trois types de conversion entre objets de type entier :
\begin{itemize}
  \item les conversions toujours possibles \plm+extend exp : $type+ (\refSubsectionPage{conversionsToujoursPossibles}) ;
  \item les conversions pouvant échouer \plm+convert exp : $type+ (\refSubsectionPage{conversionsPouvantEchouer}) ;
  \item les troncatures \plm+truncate exp : $type+  (\refSubsectionPage{conversionsTroncature}).
\end{itemize}


\subsectionLabel{Conversions toujours possibles : \texttt{extend}}{conversionsToujoursPossibles}

Les conversions qui sont toujours possibles sont exprimées par le mot réservé \plm=extend=. Par exemple :
\begin{PLM}
let v $uint8 = ...
let x $uint9 = extend v : $uint9
let y $int9 =  extend v : $int9
let z $int10 = extend y : $int10
\end{PLM}

D'une manière générale :
\begin{itemize}
\item un entier non signé peut être étendu en un entier non signé de taille strictement supérieure ;
\item un entier non signé peut être étendu en un entier signé de taille strictement supérieure ;
\item un entier signé peut être étendu en un entier signé de taille strictement supérieure.
\end{itemize}

Par contre, une conversion pouvant provoquer un débordement est rejetée à la compilation :
\begin{PLM}
let s $int8 = ...
let x $uint16 = x // Erreur de compilation
\end{PLM}




\subsectionLabel{Conversions pouvant échouer : \texttt{convert}}{conversionsPouvantEchouer}

Les conversions pouvant échouer sont exprimées par le mot réservé \plm=convert=. Par exemple :

\begin{PLM}
let s $int8 = ...
let x $uint16 = convert x : $uint16
\end{PLM}

L'opérateur \plm+convert+ engendre un code qui vérifie à l'exécution que l'expression source (ici \plm+x+) peut être convertie dans le type cible (ici \plm+$uint16+) sans débordement. En cas de débordement détecté à l'exécution, la panique dont le code est donné dans le \refTableauPage{tableauCodePanique} est déclenchée. L'opérateur \plm+convert+ est donc interdit dans les constructions où la panique ne peut être déclenchée : il faut alors utiliser l'opérateur \plm=truncate=.

L'opérateur \plm+convert+ ne peut pas apparaître dans une expression statique.

De plus, une erreur de compilation est déclenchée si l'opérateur \plm+convert+ est utilisé alors que la conversion est toujours possible :
\begin{PLM}
let v $uint8 = ...
let y = convert v : $int16 // Erreur, conversion toujours possible
\end{PLM}

\subsectionLabel{Troncatures : \texttt{truncate}}{conversionsTroncature}

L'opérateur \plm+truncate+ permet de spécifier une conversion explicite silencieuse, qui ne déclenche aucune panique. La valeur de l'expression source est tronquée en cas de débordement\footnote{L'opérateur \texttt{truncate} est équivalent au \emph{type cast} entre entiers du langage C.}. Par exemple :

\begin{PLM}
let s $int8 = -10
let x $uint16 = truncate x : $uint16
\end{PLM}

L'opérateur \plm+truncate+ ne peut pas apparaître dans une expression statique.

De plus, une erreur de compilation est déclenchée si l'opérateur \plm+truncate+ est utilisé alors qu'une conversion implicite est possible :
\begin{PLM}
let v $uint8 = ...
let y = truncate v : $int16 // Erreur, conversion toujours possible
\end{PLM}

\section{Opérateurs infixes de comparaison}

Les valeurs entières sont comparables, les six opérateurs \plm+==+, \plm+!=+, \plm+>=+, \plm+>+, \plm+<=+ et \plm+<+ sont acceptés.

La comparaison ne peut s'effectuer qu'entre objets du même type entier, ou entre un objet de type entier et une constante littérale entière.









\sectionLabel{Opérateurs infixes arithmétiques}{operateursInfixArithmétiques}


Les opérateurs infixes arithmétiques sont listés dans le \refTableau{operateursInfixesArithmetiques} avec leur signification. Ils ne peuvent opérer qu'entre objets du même type entier, ou entre un objet de type entier et une constante littérale entière.


\begin{table}[h]
\centering
\begin{tabular}{lllll}
  \textbf{Opérateur} & \textbf{Signification} \\
  \plm=+= & Addition avec détection de débordement\\
  \plm=-= & Soustraction avec détection de débordement\\
  \plm=*= & Multiplication avec détection de débordement\\
  \plm=/= & Division avec détection de débordement\\
  \plm=%= & Modulo avec détection de division par zéro\\
  \plm=+%= & Addition sans détection de débordement\\
  \plm=-%= & Soustraction sans détection de débordement\\
  \plm=*%= & Multiplication sans détection de débordement\\
  \plm=!/= & Division sans détection de débordement\\
  \plm=!%= & Modulo sans détection de division par zéro\\
\end{tabular}
\caption{Opérateurs infixes arithmétiques}
\labelTableau{operateursInfixesArithmetiques}
\ligne
\end{table}




\section{Opérateurs préfixés de négation arithmétique}

\subsectionLabel{Opérateur \texttt{-}}{negationOvf}

L'opérateur préfixé \plm=-= est la négation arithmétique avec détection de débordement. Il n'est accepté que sur les types signés. La négation de la borne inférieure d'un type signé (\plm+-128+ pour \plm+$int8+, \plm+-32768+ pour \plm+$int16+, ...) entraîne un débordement arithmétique qui déclenche une panique dont le code est donné dans le \refTableau{tableauCodePanique}.


\subsectionLabel{Opérateur \texttt{-\%}}{negationNoOvf}

L'opérateur préfixé \plm=-%= est la négation arithmétique sans détection de débordement. Il n'est accepté que sur les types signés. La négation de la borne inférieure d'un type signé (\plm+-128+ pour \plm+$int8+, \plm+-32768+ pour \plm+$int16+, ...) retourne cette même valeur. Cet opérateur ne déclenche jamais de panique.




\sectionLabel{Opérateurs infixes bit-à-bit}{operateurBitABitEntier}
\index{"\&!Entier}
\index{\textbar!Entier}
\index{\^!Entier}

Les opérateurs infixes bit-à-bit acceptent les types entiers non signés (\refTableau{operateursInfixesBitABit}).

\begin{table}[h]
\centering
\begin{tabular}{lllll}
  \textbf{Opérateur} & \textbf{Signification} \\
  \plm=|= & \emph{ou} bit-à-bit\\
  \plm=&= & \emph{et} bit-à-bit\\
  \plm=^= & \emph{ou exclusif} bit-à-bit\\
\end{tabular}
\caption{Opérateurs infixes bit-à-bit sur les entiers non signés}
\labelTableau{operateursInfixesBitABit}
\ligne
\end{table}





\section{Opérateur préfixé bit-à-bit}

L'opérateur préfixé \plm=~= retourne la complémentation bit-à-bit d'une valeur entière non signée.




\section{Opérateurs infixes de décalage}

Les opérateurs infixes \plm=<<= et \plm=>>= réalisent respectivement le décalage à gauche et à droite de l'opérande de gauche. L'amplitude du décalage est spécifiée par la valeur de l'opérande droite (\refTableau{operateursInfixesDecalage}). \plm=a= est une expression entière signée ou non signée, et l'expression renvoie une valeur de même type que \plm=a=. L'expression \plm=b= est une expression entière non signée.

\begin{table}[h]
\centering
\begin{tabular}{lllll}
  \textbf{Expression} & \textbf{Signification} \\
  \plm=a << b= & Décalage à gauche de \plm=a= d'une amplitude de \plm=b= bits\\
  \plm=a >> b= & Décalage à droite de \plm=a= d'une amplitude de \plm=b= bits\\
\end{tabular}
\caption{Opérateurs infixes de décalage sur les entiers}
\labelTableau{operateursInfixesDecalage}
\ligne
\end{table}








\sectionLabel{Opérateurs combinées avec une affectation}{operateursCombinesAffectationEntier}
\index{\&=!Entier}
\index{\textbar=!Entier}
\index{\^{}=!Entier}

Les opérateurs suivants sont définis pour les entiers.

\plm!a &= b! est équivalent à \plm!a = a & b!.

\plm!a |= b! est équivalent à \plm!a = a | b!.

\plm!a ^= b! est équivalent à \plm!a = a ^ b!.

\plm!a += b! est équivalent à \plm!a = a + b!.

\plm!a +%= b! est équivalent à \plm!a = a +% b!.

\plm!a -= b! est équivalent à \plm!a = a - b!.

\plm!a -%= b! est équivalent à \plm!a = a -% b!.

\plm!a *= b! est équivalent à \plm!a = a * b!.

\plm!a *%= b! est équivalent à \plm!a = a *% b!.

Les opérateurs infixes \plm!&=!, \plm!|=! et \plm!^=! sont décrits à la \refSectionPage{operateurBitABitEntier}.

Les opérateurs infixes \plm!+!, \plm!+%!, \plm!-!, \plm!-%!, \plm!*! et \plm!*%! sont décrits à la \refSectionPage{operateursInfixArithmétiques}.

