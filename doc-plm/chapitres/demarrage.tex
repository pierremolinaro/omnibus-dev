%!TEX encoding = UTF-8 Unicode
%!TEX root = ../doc-plm.tex

\chapterLabel{Démarrage du micro-contrôleur}{chapitreDemarrageMicro}


\sectionLabel{Séquence de démarrage}{sequenceDemarrage}

La séquence de démarrage du micro-contrôleur est illustrée par la \refFigure{}{sequenceDemarrage}.

La première étape est de configurer les horloges internes du micro-contrôleur : c'est le rôle des routines \plm=boot=. À ce stade, la mémoire vive n'est toujours pas initialisée, aussi les routines \plm=boot= n'y accèdent pas (le compilateur l'assure).

La deuxième étape est d'initialiser les \emph{variables globales}, c'est-à-dire mettre à zéro la zone «~\texttt{.bss.*}~», et de recopier à partir de la flash les valeurs initiales des variables initialisées.

La troisième étape est l'exécution des routines \plm=init=. À partir de cette étape et pour les suivantes, les variables globales sont initialisées, et donc leur emploi est autorisé. Le rôle des routines \plm=init= est de configurer les entrées/sorties du micro-contrôleur.

Ensuite, les tâches sont lancées, et exécutées en fonction de leurs priorités et synchronisations.

\begin{figure}[htbp]
  \centering
  \small
  \begin{tikzpicture}[
      cloud/.style ={draw=red, thick, chamfered rectangle,fill=red!20, minimum height=2em},
      block/.style ={rectangle, draw=blue, thick, fill=green!20, align=center},
      decision/.style={chamfered rectangle, draw=blue, thick, fill=green!20},
      node distance=5mm
    ]
    \node [cloud] (start) {\textsc{Démarrage du micro-contrôleur}} ;
    \node [block] (boot) [below=of start] {Routines \bf\texttt{boot}} ;
    \node [block] (raz) [below=of boot] {Initialisation des variables globales} ;
    \node [block] (init) [below=of raz] {Routines \bf\texttt{init}} ;
    \node [block] (setup) [below=of init] {Démarrage des tâches} ;

    \draw [-stealth, thick] (start) -- (boot) ;
    \draw [-stealth, thick] (boot) -- (raz) ;
    \draw [-stealth, thick] (raz) -- (init) ;
    \draw [-stealth, thick] (init) -- (setup) ;
  \end{tikzpicture}
  \caption{Organigramme d'exécution de la séquence de démarrage}
  \labelFigure{sequenceDemarrage}
%  \ligne
\end{figure}


\sectionLabel{\texttt{boot} routines}{bootRoutine}
\index{boot@\plm=boot=!Routine}
\index{Routine!boot@\plm=boot=}

Une routine \plm=boot= est exécutée une et une seule fois, lors du démarrage du micro-contrôleur, avant que les variables globales ne soient initialisées. Elle a la syntaxe suivante :
\begin{PLM}
boot priorité {
  liste_instructions
}
\end{PLM}
Où \plm=priorité= est la priorité de la routine. C'est une constante entière statique. Les routines \plm=boot= sont exécutées dans l'ordre des priorités croissantes. Le compilateur vérifie que deux routines \plm=boot= n'ont pas la même priorité.

Les routines \plm=boot= s'exécutent dans le mode \plm=boot=.\index{boot}

Comme les routines \plm=boot= s'exécutent avant que les variables globales soient initialisées, l'accès aux variables globales y est interdit. L'interdiction est mise en place de la façon suivante : il n'est pas possible d'associer une routine \plm=boot= à une variable globale lors de sa déclaration (\refSectionPage{declarationVariableGlobale}).

Par contre, l'accès aux registres de contrôle est autorisé dans une routine \plm=boot= (d'ailleurs, elle sert à cela : configurer le micro-contrôleur au démarrage).







\sectionLabel{\texttt{init} routines}{initRoutine}
\index{init@\plm=init=!Routine}
\index{Routine!init@\plm=init=}

Une routine \plm=init= est exécutée une et une seule fois, lors du démarrage du micro-contrôleur, après l'initialisation des variables globales. Elle a la syntaxe suivante :
\begin{PLM}
init priorité {
  liste_instructions
}
\end{PLM}
Où \plm=priorité= est la priorité de la routine. C'est une constante entière statique. Les routines \plm=init= sont exécutées dans l'ordre des priorités croissantes. Le compilateur vérifie que deux routines \plm=init= n'ont pas la même priorité.

Les routines \plm=init= s'exécutent dans le mode \plm=init=.\index{init}

Comme les routines \plm=init= s'exécutent après l'initialisation des variables globales, l'accès aux variables globales y est autorisé, du moment que la variable globale cite le mode \plm=init= dans sa déclaration (\refSectionPage{declarationVariableGlobale}).


















