%!TEX encoding = UTF-8 Unicode
%!TEX root = ../doc-plm.tex





\chapter{Les types opaque}

Un \emph{type opaque} est un type dont la définition est externe à PLM, dans le code C associé. Les objets de ce type ne peuvent pas être copiés, ni comparés, et leur contenu est inaccessible. 

\section{Déclaration d'un type opaque}

La déclaration d'un type opaque est introduite par le mot réservé \plm!opaqueType! :

\begin{PLM}
opaqueType MonTypeOpaque : 32
\end{PLM}

Le nombre associé (ici \plm!32!) est le nombre de bits nécessaires pour représenter un objet de ce type.









