%!TEX encoding = UTF-8 Unicode
%!TEX root = ../doc-plm.tex





\chapter{Le type booléen}


Le type booléen \plm=$bool=\index{bool!\plm=$bool=} est prédéfini par le langage.



\section{Les mots réservés \texttt{true} et \texttt{false}}

Les mots réservés \plm+true+ et \plm+false+ dénotent respectivement la valeur logique \emph{vraie} et la valeur logique \emph{fausse}.

\section{Les opérateurs infix de comparaison}

Les valeurs booléennes sont comparables, les six opérateurs \plm+==+, \plm+!=+, \plm+>=+, \plm+>+, \plm+<=+ et \plm+<+ sont acceptés, avec \plm=false= < \plm=true=.
 
\section{Les opérateurs infixes \texttt{and}, \texttt{or} et \texttt{xor}}

Les opérateurs infixes \plm=and=, \plm=or= et \plm=xor= implémentent respectivement le \emph{et} logique, \emph{ou} logique, \emph{ou exclusif} logique. Les deux premiers évaluent les opérandes en \emph{court-circuit}, c'est-à-dire que si la valeur de l'opérande de gauche détermine la valeur de l'expression, alors l'opérande de droite n'est pas évalué.

Noter que les opérateurs infixes \plm=&=, \plm=|= et \plm=^= sont des opérateurs bit-à-bit sur les entiers non signés, et ne peuvent pas être appliqués à des valeurs booléennes.


\section{L'opérateur préfixé \texttt{not}}

L'opérateur préfixé \plm=not= est la complémentation booléenne. Noter que l'opérateur préfixé \plm=~= effectue la complémentation bit-à-bit d'un entier non signé et ne peut pas être appliqué à une valeur booléenne.

\section{Conversion en une valeur entière}

La conversion d'une valeur booléenne en une valeur entière s'effectue par l'intermédiaire d'une expression \plm=if=. Par exemple :

\begin{PLM}
let x $bool = ...
let result $uint8 = if x : 4 else 2 end
\end{PLM}


\section{Conversion d'une valeur entière en booléen}

Il n'y a pas d'opérateur dédié à la conversion d'une valeur entière vers un booléen. Il suffit d'utiliser des opérateurs entre entiers comme \plm+==+ ou \plm+!=+ pour réaliser une conversion :

\begin{PLM}
let result $bool = x != 0 // x est une expression entière
\end{PLM}


