%!TEX encoding = UTF-8 Unicode
%!TEX root = ../doc-plm.tex





\chapter{Le type \texttt{Bool}}


\section{Les mots réservés \texttt{true} et \texttt{false}}


\section{Les opérateurs infix de comparaison}


\section{Les opérateurs infix \texttt{and}, \texttt{or} et \texttt{xor}}

\section{L'opérateur préfixé \texttt{not}}

\section{Conversion en une valeur entière}

\section{Conversion d'une valeur entière en booléen}

Il n'y a pas d'opérateur dédié à la conversion d'une valeur entière vers un booléen. Il suffit d'utiliser des opérateurs entre entiers comme \plm+==+ ou \plm+!=+ pour réaliser une conversion :

\begin{PLM}
let result : Bool = x != 0 // X est une expression entière
\end{PLM}

