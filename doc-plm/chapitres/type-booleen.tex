%!TEX encoding = UTF-8 Unicode
%!TEX root = ../doc-plm.tex





\chapter{Le type booléen}


Par défaut, aucun type booléen n'est prédéfini par le langage ; il est d'usage que ce soit la définition de la cible qui le fasse, et que ce type booléen se nomme \plm=Bool=\index{Bool!\plm=Bool=} (voir \refSectionPage{DefinitionTypeBooleen}).



\section{Les mots réservés \texttt{true} et \texttt{false}}

Les mots réservés \plm+true+ et \plm+false+ dénotent respectivement la valeur logique \emph{vraie} et la valeur logique \emph{fausse}.

\section{Les opérateurs infix de comparaison}

Les valeurs booléennes sont comparables, les six opérateurs \plm+==+, \plm+!=+, \plm+>=+, \plm+>+, \plm+<=+ et \plm+<+ sont acceptés, avec \plm=false= < \plm=true=.
 
\section{Les opérateurs infixes \texttt{and}, \texttt{or} et \texttt{xor}}

Les opérateurs infixes \plm=and=, \plm=or= et \plm=xor= implémentent respectivement le \emph{et} logique, \emph{ou} logique, \emph{ou exclusif} logique. Les deux premiers évaluent les opérandes en \emph{court-circuit}, c'est-à-dire que si la valeur de l'opérande de gauche détermine la valeur de l'expression, alors l'opérande de droite n'est pas évalué.

Noter que les opérateurs infixes \plm=&=, \plm=|= et \plm=^= sont des opérateurs bit-à-bit sur les entiers non signés, et ne peuvent pas être appliqués à des valeurs booléennes.


\section{L'opérateur préfixé \texttt{not}}

L'opérateur préfixé \plm=not= est la complémentation booléenne. Noter que l'opérateur préfixé \plm=~= effectue la complémentation bit-à-bit d'un entier non signé et ne peut pas être appliqué à une valeur booléenne.

\section{Conversion en une valeur entière}

Lors d'une conversion valeur booléenne vers valeur entière, \plm=false= est converti en la valeur entière $0$, et \plm=true= en la valeur entière $1$. Comme tous les types entiers peuvent représenter $0$ et $1$, cette conversion est toujours acceptée silencieusement. Par exemple :

\begin{PLM}
let result : UInt8 = true // result a pour valeur 1
\end{PLM}


\section{Conversion d'une valeur entière en booléen}

Il n'y a pas d'opérateur dédié à la conversion d'une valeur entière vers un booléen. Il suffit d'utiliser des opérateurs entre entiers comme \plm+==+ ou \plm+!=+ pour réaliser une conversion :

\begin{PLM}
let result : Bool = x != 0 // x est une expression entière
\end{PLM}


\sectionLabel{Définition du type booléen}{DefinitionTypeBooleen}\index{Type Booleen@Type booléen!definition@Définition}

Par défaut, aucun type booléen n'est défini par le langage. Il est d'usage que la définition de la cible le fasse, au moyen de la déclaration \plm=booleanType= :\index{booleanType@\plm=booleanType=}

\begin{PLM}
booleanType nom representation
\end{PLM}

Par exemple :
\begin{PLM}
booleanType Bool @unsigned8
\end{PLM}

Où :
\begin{itemize}
  \item \plm=Bool= est le nom donné au type booléen ;
  \item \plm=@unsigned8= est l'attribut précisant la représentation d'une valeur booléenne dans le code C engendré.
\end{itemize}

L'attribut \plm=@unsigned8= n'est pas prédéfini, mais doit être déclaré dans une déclaration \plm=newUnsignedRepresentation=\index{newUnsignedRepresentation@\plm=newUnsignedRepresentation=} (voir \refSectionPage{DefNewUnsignedRepresentation}), qui figure normalement dans la définition de la cible. Si l'auteur de cette définition est logique, \plm=@unsigned8= représente dans le code C engendré les booléens comme des entiers non signés de 8 bits.

