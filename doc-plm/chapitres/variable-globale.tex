%!TEX encoding = UTF-8 Unicode
%!TEX root = ../doc-plm.tex





\chapter{Variables globales}

Les variables peuvent être déclarées en deux endroits :
\begin{itemize}
  \item en dehors de toute routine : c'est une variable globale (voir ci-après) ;
  \item parmi les instructions d'une routine : c'est une variable locale à la routine (voir \refSectionPage{declarationVariableLocale}).
\end{itemize}





\sectionLabel{Déclaration d'une variable globale}{declarationVariableGlobale}\index{Variable!globale}

La déclaration d'une variable globale est la suivante :

\begin{PLM}
var nom : Type = expression_statique {
  en_tetes_routines_autorisees
}
\end{PLM}

Où :
\begin{itemize}
  \item \plm=nom= est le nom de la variable globale ;
  \item \plm=Type= est le nom du type de la variable globale ;
  \item \plm=expression_statique= est l'expression qui fournit la valeur initiale de cette variable ; cette expression est calculée lors de la compilation ;
  \item \plm=en_tetes_routines_autorisees= est la liste des routines qui sont autorisées à accéder à la variable globale ; par défaut, l'accès est autorisé en lecture seule ; pour avoir l'autorisation d'écrire la variable, l'attribut \plm=@rw=\index{"@rw} doit être placé avant la routine.
\end{itemize}

La portée d'une variable globale est le programme dans son intégralité : peu importe où est déclarée la variable. Toutefois, seules les routines autorisées peuvent y accéder.

L'\plm=expression_statique= peut utiliser la valeur des constantes globales : le compilateur évalue les constantes globales avant d'évaluer les valeurs initiales des variables globales. Par contre, \plm=expression_statique= ne peut pas utiliser les valeurs d'une autre variable globale.

Par exemple, voici comment on peut implémenter une variable globale \texttt{gCompteur}, incrémentée par la routine d'interruption \texttt{timerHandler}, et écrire un service d'attente de délai \texttt{wait} avec une boucle d'attente active :

\begin{PLM}
var gCompteur : UInt32 = 0 {
  @rw proc timerHandler ()
  proc waitMS (?inDuration : UInt32)
}

proc timerHandler `isr () {
  gCompteur &++
}

proc wait `user (?inDuration : UInt32) {
  let deadline = gCompteur + inDuration
  while gCompteur < deadline do
  end
}
\end{PLM}

Note : si une variable globale est accédée par des routines appartenant à plusieurs modes, le compilateur ajoute le qualificatif \texttt{volatile}\index{volatile@\texttt{volatile}} dans le code C engendré pour déclarer cette variable. C'est le cas de l'exemple ci-dessus, la variable \texttt{gCompteur} pouvant être accédée dans les modes \plm=`isr= et \plm=`user=.


Les routines d'initialisation (\refSectionPage{initRoutine})\index{init@\plm=init=!Routine} et les routines d'exception (\refSubsectionPage{routinesException})\index{Routine!exception@\plm=exception=}
 doivent de même être déclarées pour pouvoir accéder à une variable globale (et avec l'attribut \plm=@rw= si on veut l'accès en écriture). Par exemple :
\begin{PLM}
var gCompteur : UInt32 = 0 {
  @rw init 112
  @rw exception loop 179
}

proc init 112 {
  gCompteur = 10
}

proc exception loop 179 {
  gCompteur = 100
}
\end{PLM}

Comme les routines \plm+boot+ sont exécutées avant l'initialisation des variables globales, le compilateur interdit l'accès des routines \plm+boot+ aux variables globales.

\section{Restrictions d'usage des variables globales}
\index{Parametre effectif@Paramètre effectif!Variable globale}

Une variable globale ne peut pas :
\begin{itemize}
  \item apparaître comme paramètre effectif en entrée d'une procédure ;
  \item apparaître comme paramètre effectif en sortie/entrée d'une procédure.
\end{itemize}

Prenons un exemple ; la procédure \texttt{uneProcedure} présente un argument formel en sortie, et la variable globale \texttt{gGlobale} peut être accédée en lecture et écriture par la procédure \texttt{autreProcedure} :
\begin{PLM}
proc uneProcedure `user (!outValue : UInt32) {
  outValue = 5
}

var gGlobale : UInt32 {
  @rw proc autreProcedure ()
}
\end{PLM}

L'écriture suivante est correcte (écriture de la variable globale) :
\begin{PLM}
proc autreProcedure `user () {
  gGlobale = 10 // Ok
}
\end{PLM}


Par contre, l'écriture suivante est rejetée par le compilateur (passage d'une variable globale comme paramètre effectif en entrée) :
\begin{PLM}
proc autreProcedure `user () {
  uneProcedure (?gGlobale) // Erreur, variable globale comme paramètre
                           // effectif en entrée
}
\end{PLM}

