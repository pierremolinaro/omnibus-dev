%!TEX encoding = UTF-8 Unicode
%!TEX root = ../doc-plm.tex





\chapter{Cible \texttt{teensy-3-1-it}}\index{Cible!\texttt{teensy-3-1-it}}

Dans l'état actuel de PLM, une seule cible est définie : \texttt{teensy-3-1-it}.  Elle permet une programmation séquentielle avec routines d'interruption. L'interruption \texttt{systick} est programmée pour se déclencher chaque milliseconde. L'objet de ce chapitre est de décrire son utilisation.

Il est possible de définir sa propre cible (\refChapterPage{chapitreConfCible}).




















\sectionLabel{Organigramme d'exécution}{organigrammeExecutionTeensy31It}

La \refFigure{}{sequenceDemarrageTeensySequentialSystick} définit l'organigramme d'exécution d'un programme.

Le micro-contrôleur démarre sur une horloge interne, la mémoire vive n'étant pas initialisée. Il est dans le mode \emph{thread, priviliged access}, avec une seule pile. La configuration conservera cette pile unique, jusqu'au démarrage des tâches.

La première étape est de configurer les horloges internes du micro-contrôleur : c'est le rôle des routines \plm=boot= (\refSectionPage{personalisationDemarrageTeensy31it}). À ce stade, la mémoire vive n'est toujours pas initialisée, aussi les routines \plm=boot= n'y accèdent pas (le compilateur l'assure).

La deuxième étape est d'initialiser les \emph{variables globales}, c'est-à-dire mettre à zéro la zone \texttt{bss}, et de recopier à partir de la flash les valeurs initiales des variables initialisées.

La troisième étape est l'exécution des routines \plm=init= (\refSectionPage{personalisationInitTeensy31it}). À partir de cette étape et pour les suivantes, les variables globales sont initialisées, et donc leur emploi est autorisé. Le rôle des routines \plm=init= est de configurer les entrées/sorties du micro-contrôleur.

Ensuite, les tâches sont lancées, et exécutées en fonction de leurs priorités et synchronisations.


\begin{figure}[t]
  \centering
  \small
  \begin{tikzpicture}[
      cloud/.style ={draw=red, thick, ellipse,fill=red!20, minimum height=2em},
      block/.style ={rectangle, draw=blue, thick, fill=green!20, align=center},
      decision/.style={chamfered rectangle, draw=blue, thick, fill=green!20},
      node distance=5mm
    ]
    \node [cloud] (start) {\textsc{Démarrage du micro-contrôleur}} ;
    \node [block] (boot) [below=of start] {Routines \bf\texttt{boot}} ;
    \node [block] (raz) [below=of boot] {Initialisation des variables globales} ;
    \node [block] (init) [below=of raz] {Routines \bf\texttt{init}} ;
    \node [block] (setup) [below=of init] {Démarrage des tâches} ;

    \draw [-stealth, thick] (start) -- (boot) ;
    \draw [-stealth, thick] (boot) -- (raz) ;
    \draw [-stealth, thick] (raz) -- (init) ;
    \draw [-stealth, thick] (init) -- (setup) ;
  \end{tikzpicture}
  \caption{Organigramme d'exécution de la cible \texttt{teensy-3-1-it}}
  \labelFigure{sequenceDemarrageTeensySequentialSystick}
  \ligne
\end{figure}










\sectionLabel{Personalisation du démarrage}{personalisationDemarrageTeensy31it}

La cible définit la routine \plm+boot 0+ qui configure le micro-contrôleur.

Vous pouvez ajouter vos propres routines \plm+boot+. À chaque routine \plm+boot+ est associée une priorité d'exécution, qui doit être unique. Les routines \plm+boot+ sont exécutées dans l'ordre croissant des priorités, c'est-à-dire que la routine \plm+boot 0+ est exécutée la première.





\sectionLabel{Personalisation de l'initialisation}{personalisationInitTeensy31it}

La cible définit la routine \plm+init 0+ qui configure le \emph{SysTick Timer} pour qu'il engendre une interruption toutes les millisecondes.

Vous pouvez ajouter vos propres routines \plm+init+. À chaque routine \plm+init+ est associée une priorité d'exécution, qui doit être unique. Les routines \plm+init+ sont exécutées dans l'ordre croissant des priorités, c'est-à-dire que la routine \plm+init 0+ est exécutée la première.












\section{API}

L'API de la cible \texttt{teensy-3-1-it} regroupe les fonctions disponibles dans des modules et des types :
\begin{itemize}
  \item le module \plm=time=, la gestion du temps (\refSubsectionPage{moduleTime}) ;
  \item le module \plm=leds=, la gestion des leds (\refSubsectionPage{moduleLeds}) ;
  \item le module \plm=lcd=, la gestion de l'afficheur LCD (\refSubsectionPage{moduleLCD}) ;
  \item le type \plm=$semaphore=, le sémaphore de Dijkstra (\refSubsectionPage{typeSemaphore}) ;
\end{itemize}

\subsectionLabel{Module \texttt{time}}{moduleTime}

Le module \plm=time= regroupe des fonctions dédiées à la gestion du temps. Pour la cible \texttt{teensy-3-1-it}, le temps est compté en nombre de milli-secondes écoulées depuis le démarrage du micro-contrôleur. 

\subsubsection{Fonction \texttt{oneMillisecondBusyWait}}


\begin{PLM}
  public func panic init oneMillisecondBusyWait ()
\end{PLM}

Cette fonction réalise une attente active jusqu'à la prochaine milli-seconde. Elle n'est appelable que dans les modes \plm=panic= et \plm=init=. 





\subsubsection{Fonction \texttt{busyWaitingDuringMS}}

\begin{PLM}
  public func panic init busyWaitingDuringMS (?inDelay $uint32)
\end{PLM}

Cette fonction effectue une attente active en appelant \plm=inDelay= fois la fonction \plm=oneMillisecondBusyWait=. Elle n'est appelable que dans les modes \plm=panic= et \plm=init=. 





\subsubsection{Primitive \texttt{waitUntilMS}}

\begin{PLM}
  public primitive waitUntilMS (?deadline: inDate $uint32)
\end{PLM}

Cette fonction réalise une attente passive jusqu'à la date absolue \plm=inDate=. 






\subsubsection{Primitive \texttt{waitDuringMS}}

\begin{PLM}
  public primitive waitDuringMS (?delay: inDelay $uint32)
\end{PLM}

Cette fonction réalise une attente passive pendant \plm=inDelay= milli-secondes. 





\subsubsection{Garde \texttt{waitUntilMS}}

\begin{PLM}
  public guard waitUntilMS (?deadline:inDeadline $uint32)
\end{PLM}

Cette fonction exprime l'attente passive en garde pendant \plm=inDelay= milli-secondes. 









\subsectionLabel{Module \texttt{leds}}{moduleLeds}


\subsubsection{Constantes}

Les constantes suivantes sont de type \plm=$uint32= et sont dédiées aux leds : \plm=LED_L0=, \plm=LED_L1=, \plm=LED_L2=, \plm=LED_L3= et \plm=LED_L4=.


\subsubsectionLabel{Fonction \texttt{write(?off{}:)}}{routineLedOffTeensy31it}

\begin{PLM}
  public func user panic service write (?off:inLeds $uint32) 
\end{PLM}

Cette routine éteint un ensemble de leds. Pour éteindre une led, écrire :

\begin{PLM}[1]
  leds.write (!off:LED_L0)
\end{PLM}

 Pour éteindre plusieurs leds, utiliser l'opérateur \plm=|= :
 
\begin{PLM}[1]
  leds.write (!off:LED_L0 | LED_L4)
\end{PLM}



\subsubsectionLabel{Fonction \texttt{write(?on{}:)}}{routineLedOnTeensy31it}

\begin{PLM}
  public func user panic service write (?on:inLeds $uint32) 
\end{PLM}

Cette routine allume un ensemble de leds. Pour allumer une led, écrire :

\begin{PLM}[1]
  leds.write (!on:LED_L0)
\end{PLM}

 Pour allumer plusieurs leds, utiliser l'opérateur \plm=|= :
 
\begin{PLM}[1]
  leds.write (!on:LED_L0 | LED_L4)
\end{PLM}



\subsubsectionLabel{Fonction \texttt{write(?toggle{}:)}}{routineLedToggleTeensy31it}

\begin{PLM}
  public func user panic service write (?toggle:inLeds $uint32) 
\end{PLM}

Cette routine complémente un ensemble de leds.

Pour complémenter une led, écrire :

\begin{PLM}[1]
  leds.write (!toggle:LED_L0)
\end{PLM}

Pour complémenter plusieurs leds, utiliser l'opérateur \plm=|= :

\begin{PLM}[1]
  leds.write (!toggle:LED_L0 | LED_L4)
\end{PLM}






\subsectionLabel{Module \texttt{lcd}}{moduleLCD}


\subsubsection{Fonction \texttt{clearScreen}}

\begin{PLM}
  public func user clearScreen ()
\end{PLM}

Cette fonction efface l'afficheur LCD, et place le curseur au début de la première ligne.





\subsubsection{Fonction \texttt{goto}}

\begin{PLM}
public func user goto (?line:inLine $uint2
                       ?column:inColumn $uint8)
\end{PLM}

Cette fonction place le curseur à la colonne \plm=inColumn= de la ligne \plm=inLine=. L'afficheur possédant quatre lignes, l'argument \plm=inLine= est de type \plm=$uint2=.




\subsubsection{Fonction \texttt{printSpaces}}

\begin{PLM}
  public func user printSpaces (?inCount $uint32)
\end{PLM}

Cette fonction écrit \plm=inCount= caractères espace à partir de la position du curseur.





\subsubsection{Fonction \texttt{printUnsigned}}

\begin{PLM}
  public func user printUnsigned (?inValue $uint32)
\end{PLM}

Cette fonction écrit la valeur de l'argument (un entier non signé sur 32 bits) \plm=inValue= à partir de la position du curseur.






\subsubsection{Fonction \texttt{printSigned}}

\begin{PLM}
  public func user printSigned (?inValue $int32)
\end{PLM}

Cette fonction écrit la valeur de l'argument (un entier signé sur 32 bits) \plm=inValue= à partir de la position du curseur.







\subsubsection{Fonction \texttt{printString}}

\begin{PLM}
  public func user printString (?inValue $staticString)
\end{PLM}

Cette fonction écrit la valeur de l'argument (une chaîne de caractères) \plm=inValue= à partir de la position du curseur.









\subsubsection{Fonction \texttt{clearScreenInPanicMode}}

\begin{PLM}
  public func panic clearScreenInPanicMode ()
\end{PLM}

Cette fonction efface l'afficheur LCD, et place le curseur au début de la première ligne. Appelable uniquement en mode \plm=panic=.





\subsubsection{Fonction \texttt{gotoInPanicMode}}

\begin{PLM}
public func panic gotoInPanicMode (?line:inLine $uint2
                                   ?column:inColumn $uint8)
\end{PLM}

Cette fonction place le curseur à la colonne \plm=inColumn= de la ligne \plm=inLine=. L'afficheur possédant quatre lignes, l'argument \plm=inLine= est de type \plm=$uint2=. Appelable uniquement en mode \plm=panic=.




\subsubsection{Fonction \texttt{printSpacesInPanicMode}}

\begin{PLM}
public func panic printSpacesInPanicMode (?inCount $uint32)
\end{PLM}

Cette fonction écrit \plm=inCount= caractères espace à partir de la position du curseur. Appelable uniquement en mode \plm=panic=.





\subsubsection{Fonction \texttt{printUnsignedInPanicMode}}

\begin{PLM}
public func panic printUnsignedInPanicMode (?inValue $uint32)
\end{PLM}

Cette fonction écrit la valeur de l'argument (un entier non signé sur 32 bits) \plm=inValue= à partir de la position du curseur. Appelable uniquement en mode \plm=panic=.






\subsubsection{Fonction \texttt{printSignedInPanicMode}}

\begin{PLM}
  public func panic printSignedInPanicMode (?inValue $int32)
\end{PLM}

Cette fonction écrit la valeur de l'argument (un entier signé sur 32 bits) \plm=inValue= à partir de la position du curseur. Appelable uniquement en mode \plm=panic=.







\subsubsection{Fonction \texttt{printStringInPanicMode}}

\begin{PLM}
  public func panic printStringInPanicMode (?inValue $staticString)
\end{PLM}

Cette fonction écrit la valeur de l'argument (une chaîne de caractères) \plm=inValue= à partir de la position du curseur. Appelable uniquement en mode \plm=panic=.








\subsectionLabel{Type \texttt{\$semaphore}}{typeSemaphore}

Le type \plm=$semaphore= implémente le sémaphore de Dijstra. La liste des tâches bloquées est ordonnée selon leur priorité.


La déclaration d'un sémaphore mentionne sa valeur initiale, qui est un entier non signé de 32 bits :

\begin{PLM}[1]
  var s = $semaphore (!value:0)
\end{PLM}


\subsubsection{Service \texttt{signal()}}

\begin{PLM}
  public service signal ()
\end{PLM}

Ce service effectue l'opération \texttt{signal} sur le sémaphore sur lequel il s'applique. C'est un service, car il est appelable à partir d'une tâche, d'une primitive, et d'une routine d'interruption.



\subsubsection{Primitive \texttt{wait()}}

\begin{PLM}
  public primitive wait () 
\end{PLM}

Cette primitive effectue l'opération \texttt{wait} sur le sémaphore sur lequel il s'applique. C'est une primitive, car elle n'est pas appelable à partir d'une routine d'interruption.




\subsubsection{Primitive \texttt{wait(?untilDeadline{}:)}}

\begin{PLM}
  public primitive wait (?untilDeadline:inDeadline $uint32) -> $bool
\end{PLM}

Cette primitive attend d'effectuer l'opération \texttt{wait}, jusqu'à l'échéance \plm=inDeadline=. Elle renvoie \plm=true= si la primitive \texttt{P} a été effectuée, ou \plm=false= si l'échéance est arrivée.





\subsubsection{Garde \texttt{wait()}}

\begin{PLM}
  public guard wait ()
\end{PLM}

Cette primitive exprime l'opération \texttt{wait} en garde.











\subsectionLabel{Écriture des primitives, des services et des gardes}{EcriturePrimitivesServicesGardes}

PLM définit deux types opaques et neuf fonctions externes\footnote{Externes à PLM, car écrites en C.} qui permettent d'écrire outils de synchronisation et gardes.

\subsubsection{Type opaque \texttt{\$taskList}}

Le type opaque \plm=$taskList= est utilisé pour maintenir la liste des tâches bloquées sur un outil de synchronisation.


\subsubsection{Type opaque \texttt{\$guardList}}

Le type opaque \plm=$guardList= est utilisé pour maintenir la liste des tâches qui ont invoqué une garde d'un outil de synchronisation.



\subsubsection{Fonction \texttt{block(?!inList{}:)}}

\begin{PLM}
func primitive block (?!inList:ioWaitingList $taskList)
\end{PLM}

Cette fonction bloque la tâche en cours dans la liste \plm=ioWaitingList= passée en argument. Elle est appelable en mode \plm=primitive=, c'est-à-dire par les primitives.




\subsubsection{Fonction \texttt{block(?onDeadline{}:)}}

\begin{PLM}
func primitive block (?onDeadline:deadline $uint32) 
\end{PLM}

Cette fonction bloque la tâche en cours jusqu'à la date \plm=deadline=. Elle est appelable en mode \plm=primitive=, c'est-à-dire par les primitives.







\subsubsection{Fonction \texttt{block(?!inList{}:?onDeadline{}:)}}

\begin{PLM}
func primitive block (?!inList:waitingList $taskList 
                    ?onDeadline:deadline $uint32) 
\end{PLM}

Cette fonction bloque la tâche en cours dans la liste \plm=waitingList= jusqu'à la date \plm=deadline=. Elle est appelable en mode \plm=primitive=, c'est-à-dire par les primitives.








\subsubsection{Fonction \texttt{makeTaskReady(?!fromList{}:)}}

\begin{PLM}
func primitive service guard
makeTaskReady (?!fromList:waitingList $taskList) -> $bool
\end{PLM}

Si la liste \plm=waitingList= est vide, la fonction renvoie \plm=false=. Si la liste n'est pas vide, cette fonction retire de la liste la tâche de plus forte priorité, la rend prête, et la fonction renvoie \plm=true=.

Cette fonction est appelable en mode \plm=primitive=, c'est-à-dire par les primitives, et en mode \plm=service=, c'est-à-dire par les routines d'interruption.










\subsubsection{Fonction \texttt{makeTasksReady(?fromCurrentDate{}:)}}

\begin{PLM}
func service
makeTasksReady (?fromCurrentDate:currentDate $uint32)
\end{PLM}

Cette fonction rend prête toutes les tâches en attente d'échéance dont l'échéance est atteinte.

Cette fonction est appelable mode \plm=service=, c'est-à-dire par les routines d'interruption.






\subsubsection{Fonction \texttt{handleGuardedCommand}}

\begin{PLM}
func guard handleGuardedCommand (?!guard:ioGuard $guardList)
\end{PLM}

Cette fonction enregistre la tâche courante dans la liste \plm=ioGuard=. Cette fonction est appelable mode \plm=guard=, c'est-à-dire par les gardes.








\subsubsection{Fonction \texttt{handleGuardedWaitUntil}}

\begin{PLM}
func guard handleGuardedWaitUntil (?deadline:inDeadline $uint32)
\end{PLM}

Cette fonction enregistre la tâche courante pour une attente en garde jusqu'à la date \plm=inDeadline=. Cette fonction est appelable mode \plm=guard=, c'est-à-dire par les gardes.










\subsubsection{Fonction \texttt{guardDidChange}}

\begin{PLM}
func primitive service guardDidChange (?!guard:ioGuard $guardList)
\end{PLM}

Cette fonction signifie aux tâches enregistrées dans \plm=ioGuard= que les gardes doivent être re-évaluées. Cette fonction est appelable en mode \plm=primitive= et en mode  \plm=service=.











\subsubsection{Fonction \texttt{tickHandlerForGuardedWaitUntil}}

\begin{PLM}
func service
tickHandlerForGuardedWaitUntil (?currentDate:inCurrentDate $uint32)
\end{PLM}

Cette fonction signifie aux tâches en attente d'échéance en garde que l'instant \plm=inCurrentDate= est atteint, et qu'elles doivent re-évaluer leur gardes. Cette fonction est appelable en mode \plm=service=.








\section{Les routines d'interruption}

Toutes les interruptions du micro-contrôleur sont accessibles en PLM, avec quelques exceptions :
\begin{itemize}
  \item l'interruption n°$1$ (\emph{Reset}), est réservée par PLM ;
  \item l'interruption n°$3$ (\emph{HardFault}), est réservée par PLM ;
  \item l'interruption n°$11$ (\emph{svc}), est réservée par PLM ;
  \item l'interruption n°$15$ (\emph{systick}), est implémentée par la cible ; toutefois, une fonction \texttt{userSystickHandler} est disponible, voir \refSubsectionPage{SystickPourTeensy31It}.
\end{itemize}

Une routine d'interruption porte un des noms définis par la cible :
\begin{itemize}
  \item interruptions $1$ à $27$ : \refTableauPage{tableItTeensySequentialSystick1} ;
  \item interruptions $28$ à $90$ : \refTableauPage{tableItTeensySequentialSystick2} ;
  \item interruptions $91$ à $110$ : \refTableauPage{tableItTeensySequentialSystick3}.
\end{itemize}




% Les \refTableau{tableItTeensySequentialSystick1}, \refTableau{tableItTeensySequentialSystick2} et \refTableau{tableItTeensySequentialSystick3} listent les interruptions définies par le processeur qui équipe la carte \emph{Teensy 3.1}. L'utilisateur peut définir une routine d'interruption pour chacune d'entre elles, sauf l'interruption n°1 (remise à zéro), et la n°15 (\texttt{SysTick}) qui est prise en charge de façon particulière (voir la \refSubsectionTitlePage{SystickPourTeensy31It}). Celle des autres interruptions est décrite dans les sections suivantes :
%\begin{itemize}
%  \item \refSubsectionTitlePage{itsTeensy31AvecExceptions} ;
%  \item \refSubsectionTitlePage{itsTeensy31SansExceptions} ;
%  \item \refSubsectionTitlePage{itsTeensyRoutinesUtilisateur}.
%\end{itemize}


\begin{table}[!t]
  \small
  \centering
  \begin{tabular}{ll|ll}
    \textbf{Numéro}& \textbf{Nom routine} & \textbf{Numéro} & \textbf{Nom routine} \\
    1  & \emph{Reset, réservé par PLM} & 16  & \texttt{DMAChannel0TranfertComplete} \\
    2  & \texttt{NMI} & 17  & \texttt{DMAChannel1TranfertComplete}\\
    3  & \texttt{HardFault}, réservé par PLM & 18  & \texttt{DMAChannel2TranfertComplete}\\
    4  & \texttt{MemManage} & 19  & \texttt{DMAChannel3TranfertComplete}\\
    5  & \texttt{BusFault} & 20  & \texttt{DMAChannel4TranfertComplete}\\
    6  & \texttt{UsageFault} & 21  & \texttt{DMAChannel5TranfertComplete}\\
    7 à 10 & \emph{réservées par ARM} & 22  & \texttt{DMAChannel6TranfertComplete} \\
    11 & \texttt{svc}, \emph{réservé par PLM} & 23  & \texttt{DMAChannel7TranfertComplete} \\
    12 & \texttt{DebugMonitor} & 24  & \texttt{DMAChannel8TranfertComplete}\\
    13 & \emph{réservée par ARM} & 25  & \texttt{DMAChannel9TranfertComplete}\\
    14 & \texttt{PendSV} & 26  & \texttt{DMAChannel10TranfertComplete}\\
    15 & \texttt{Systick}, voir \refSubsectionPage{SystickPourTeensy31It} & 27  & \texttt{DMAChannel11TranfertComplete}\\
  \end{tabular}
  \caption{Table des interruptions 1 à 27 de la cible \texttt{teensy-3-1-it}}
  \labelTableau{tableItTeensySequentialSystick1}
  \ligne
\end{table}

\begin{table}[!t]
  \small
  \centering
  \begin{tabular}{ll|ll}
    \textbf{Numéro}& \textbf{Nom routine} & \textbf{Numéro} & \textbf{Nom routine} \\
    28  & \texttt{DMAChannel12TranfertComplete} & 60  & \texttt{UART0LON}\\
    29  & \texttt{DMAChannel13TranfertComplete} & 61  & \texttt{UART0Status}\\
    30  & \texttt{DMAChannel14TranfertComplete} & 62  & \texttt{UART0Error}\\
    31  & \texttt{DMAChannel15TranfertComplete} & 63  & \texttt{UART1Status}\\
    32  & \texttt{DMAError}                     & 64  & \texttt{UART1Error}\\
    33  & \emph{inutilisée} & 65  & \texttt{UART2Status} \\
    34  & \texttt{flashMemoryCommandComplete} & 66  & \texttt{UART2Error}\\
    35  & \texttt{flashMemoryReadCollision} & 67 à 72  & \emph{inutilisées} \\
    36  & \texttt{modeController} & 73  & \texttt{ADC0}\\
    37  & \texttt{LLWU}           & 74  & \texttt{ADC1}\\
    38  & \texttt{WDOGEWM}        & 75  & \texttt{CMP0}\\
    39  & \emph{inutilisée}       & 76  & \texttt{CMP1}\\
    40  & \texttt{I2C0}           & 77  & \texttt{CMP2}\\
    41  & \texttt{I2C1}           & 78  & \texttt{FMT0}\\
    42  & \texttt{SPI0}           & 79  & \texttt{FMT1}\\
    43  & \texttt{SPI1}           & 80  & \texttt{FMT2}\\
    44  & \emph{inutilisée}       & 81  & \texttt{CMT}\\
    45  & \texttt{CAN0MessageBuffer} & 82  & \texttt{RTCAlarm}\\
    46  & \texttt{CAN0BusOff} & 83  & \texttt{RTCSecond}\\
    47  & \texttt{CAN0Error} & 84  & \texttt{PITChannel0}\\
    48  & \texttt{CAN0TransmitWarning} & 85  & \texttt{PITChannel1}\\
    49  & \texttt{CAN0ReceiveWarning} & 86  & \texttt{PITChannel2}\\
    50  & \texttt{CAN0WakeUp} & 87  & \texttt{PITChannel3}\\
    51  & \texttt{I2S0Transmit} & 88  & \texttt{PDB}\\
    52  & \texttt{I2S0Receive} & 89  & \texttt{USBOTG}\\
    53 à 59  & \emph{inutilisées} & 90  & \texttt{USBChargerDetect} \\
  \end{tabular}
  \caption{Table des interruptions 28 à 90 de la cible \texttt{teensy-3-1-it}}
  \labelTableau{tableItTeensySequentialSystick2}
  \ligne
\end{table}

\begin{table}[!t]
  \small
  \centering
  \begin{tabular}{ll|ll}
    \textbf{Numéro}& \textbf{Nom routine} & \textbf{Numéro}& \textbf{Nom routine} \\
    91 à 96  & \emph{inutilisées} &     104  & \texttt{pinDetectPortB}\\
    97  & \texttt{DAC0}  & 105  & \texttt{pinDetectPortC} \\
    98  & \emph{inutilisée} & 106  & \texttt{pinDetectPortD}\\
    99  & \texttt{TSI} & 107  & \texttt{pinDetectPortE}\\
    100  & \texttt{MCG} & 108 et 109  & \emph{inutilisées}\\
    101  & \texttt{lowPowerTimer} & 110  & \texttt{softwareInterrupt}\\
    103  & \texttt{pinDetectPortA} & &\\
  \end{tabular}
  \caption{Table des interruptions 91 à 110 de la cible \texttt{teensy-3-1-it}}
  \labelTableau{tableItTeensySequentialSystick3}
  \ligne
\end{table}


\subsection{Définir une routine d'interruption}

Prenons l'exemple de l'interruption \texttt{PITChannel0}, qui porte le numéro $84$. La routine d'interruption est écrite comme suit :

\begin{PLM}[1]
isr PITChannel0 {
  ...
}
\end{PLM}

Une routine d'interruption s'exécute en mode \plm=section=, c'est-à-dire que les routines de l'exécutif lui sont inaccessibles. 

Si l'on veut appeler les routines de l'exécutif du mode \plm=service=, on ajoute l'attribut \plm=@xtr= :

\begin{PLM}[1]
isr PITChannel0 @xtr {
  ...
}
\end{PLM}








\subsectionLabel{Routines d'interruption par défaut, panique activée}{itsTeensy31AvecExceptions}

Quand la panique est activée, une routine par défaut est prédéfinie pour chaque interruption (sauf pour celles réservées par ARM et par PLM). Celle-ci exécute l'instruction \plm+panic+, dont l'argument est l'opposé du numéro de l'interruption. Par exemple, pour l'interruption $84$ (\texttt{PITChannel0}), la routine prédéfinie est :

\begin{PLM}[1]
isr PITChannel0 {
  panic -84
}
\end{PLM}


\subsectionLabel{Routines d'interruption par défaut, panique inactivée}{itsTeensy31SansExceptions}

Quand la panique est inactivée, aucune routine d'interruption pas défaut n'est engendrée ; dans la table des vecteurs d'interruption, l'entrée d'une routine d'interruption indéfinie contient la valeur $0$.





\subsectionLabel{Routine associée à l'interruption \texttt{SysTick}}{SystickPourTeensy31It}

L'interruption \texttt{SysTick} est particulière. Elle est programmé par la routine \plm+init 0+ de façon à engendrer une interruption chaque milliseconde. Cette interruption se déclenche après la fin de l'exécution la routine \plm+init 0+, y compris éventuellement dans les routines \plm+init+ qui suivent. La routine d'interruption associée, inaccessible à l'utilisateur :
\begin{itemize}
  \item incrémente une variable globale de comptage du temps ;
  \item appelle la routine \plm=makeTasksReadyFrom= qui contrôle les tâches bloquées en attente d'échéance ; 
  \item appelle la routine \plm=tickHandlerForGuardedWaitUntil= qui contrôle les tâches en attente d'échéance en garde ; 
  \item appelle la routine \plm+userSystickHandler+.
\end{itemize}

Cette routine \plm+userSystickHandler+ est définie par :

\begin{PLM}[1]
public func service userTickHandler @weak () {
}
\end{PLM}

Comme elle est déclarée avec l'attribut \plm+@weak+, elle peut être redéfinie par l'utilisateur. Elle s'exécute en mode \plm=service=, si bien que l'on peut appeler des services de l'exécutif qui débloquent des tâches.



