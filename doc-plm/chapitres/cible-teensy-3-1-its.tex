%!TEX encoding = UTF-8 Unicode
%!TEX root = ../doc-plm.tex





\chapter{Cible \texttt{teensy-3-1-it}}\index{Cible!\texttt{teensy-3-1-it}}

Dans l'état actuel de PLM, une seule cible est définie : \texttt{teensy-3-1-it}.  Elle permet une programmation séquentielle avec routines d'interruption. L'interruption \texttt{systick} est programmée pour se déclencher chaque milliseconde. L'objet de ce chapitre est de décrire son utilisation.

Il est possible de définir sa propre cible (\refChapterPage{chapitreConfCible}).




















\sectionLabel{Organigramme d'exécution}{organigrammeExecutionTeensy31It}

La \refFigure{}{sequenceDemarrageTeensySequentialSystick} définit l'organigramme d'exécution d'un programme.

Le micro-contrôleur démarre sur une horloge interne, la mémoire vive n'étant pas initialisée. Il est dans le mode \emph{thread, priviliged access}, avec une seule pile. La configuration conservera cette pile unique, qui servira donc pour les routines de fond et les routines d'interruption.

La première étape est de configurer les horloges internes du micro-contrôleur : c'est le rôle des routines \plm=boot= (\refSectionPage{personalisationDemarrageTeensy31it}). À ce stade, la mémoire vive n'est toujours pas initialisée, aussi les routines \plm=boot= n'y accèdent pas (le compilateur l'assure).

La deuxième étape est d'initialiser les \emph{variables globales}, c'est-à-dire mettre à zéro la zone \texttt{bss}, et de recopier à partir de la flash les valeurs initiales des variables initialisées.

La troisième étape est l'exécution des routines \plm=init= (\refSectionPage{personalisationInitTeensy31it}). À partir de cette étape et pour les suivantes, les variables globales sont initialisées, et donc leur emploi est autorisé. Le rôle des routines \plm=init= est de configurer les entrées/sorties du micro-contrôleur.

Ensuite, le micro-contrôleur est passé en mode \emph{thread, unpriviliged access}, ce qui correspond au mode \plm+`user+ de PLM pour cette cible.

La routine \plm+setup+ est exécutée une fois, puis \plm+loop+ est exécutée indéfiniment.


\begin{figure}[t]
  \centering
  \small
  \begin{tikzpicture}[
      cloud/.style ={draw=red, thick, ellipse,fill=red!20, minimum height=2em},
      block/.style ={rectangle, draw=blue, thick, fill=green!20, align=center},
      decision/.style={chamfered rectangle, draw=blue, thick, fill=green!20},
      node distance=5mm
    ]
    \node [cloud] (start) {\textsc{Démarrage}} ;
    \node [block] (confDepart) [below=of start] {Mode \emph{thread}, accès priviliégié, une seule pile} ;
    \node [block] (boot) [below=of confDepart] {Routines \bf\texttt{boot}} ;
    \node [block] (raz) [below=of boot] {Initialisation des variables globales} ;
    \node [block] (init) [below=of raz] {Routines \bf\texttt{init}} ;
    \node [block] (user) [below=of init] {Passage en mode \emph{thread}, accès non priviliégié} ;
    \node [block] (setup) [below=of user] {Procédure \texttt{setup}} ;
    \node [block] (loop) [below=of setup] {Procédure \texttt{loop}} ;

    \draw [-stealth, thick] (start) -- (confDepart) ;
    \draw [-stealth, thick] (confDepart) -- (boot) ;
    \draw [-stealth, thick] (boot) -- (raz) ;
    \draw [-stealth, thick] (raz) -- (init) ;
    \draw [-stealth, thick] (init) -- (user) ;
    \draw [-stealth, thick] (user) -- (setup) ;
    \draw [-stealth, thick] (setup) -- (loop) ;
    \draw [-stealth, thick] (loop.south) -- +(0, -.25) -- +(1.7, -.25) -- +(1.7, 0.85)-- +(0, 0.85) ;
  \end{tikzpicture}
  \caption{Organigramme d'exécution de la cible \texttt{teensy-3-1-it}}
  \labelFigure{sequenceDemarrageTeensySequentialSystick}
  \ligne
\end{figure}










\sectionLabel{Personalisation du démarrage}{personalisationDemarrageTeensy31it}

La cible définit la routine \plm+boot 0+ qui configure le micro-contrôleur.

Vous pouvez ajouter vos propres routines \plm+boot+. À chaque routine \plm+boot+ est associée une priorité d'exécution, qui doit être unique. Les routines \plm+boot+ sont exécutées dans l'ordre croissant des priorités, c'est-à-dire que la routine \plm+boot 0+ est exécutée la première.





\sectionLabel{Personalisation de l'initialisation}{personalisationInitTeensy31it}

La cible définit la routine \plm+init 0+ qui configure le \emph{SysTick Timer} pour qu'il engendre une interruption toutes les millisecondes.

Vous pouvez ajouter vos propres routines \plm+init+. À chaque routine \plm+init+ est associée une priorité d'exécution, qui doit être unique. Les routines \plm+init+ sont exécutées dans l'ordre croissant des priorités, c'est-à-dire que la routine \plm+init 0+ est exécutée la première.












\section{API}

L'API de la cible \texttt{teensy-3-1-it} partagent les routines en trois groupes :
\begin{itemize}
  \item les routines appelables dans tous les modes ;
  \item les routines appelables uniquement dans le mode \plm+`user+ (\refSubsectionPage{RoutinesTousModeTeensy31}) ;
  \item les routines appelables uniquement dans le mode \plm+`panic+ ;
\end{itemize}

\subsectionLabel{Routines appelables dans tous les modes}{RoutinesTousModeTeensy31}

Ces routines sont appelables dans les modes \plm+`user+, \plm+`isr+ et \plm+`panic+ :
\begin{itemize}
\item \texttt{ledOff} : \refSubsubsectionTitlePage{routineLedOffTeensy31it} ;
\item \texttt{ledOn} : \refSubsubsectionTitlePage{routineLedOnTeensy31it}.
\end{itemize}


\subsubsectionLabel{Routine \texttt{ledOff}}{routineLedOffTeensy31it}

\begin{PLM}
proc ledOff `user `panic `isr (?inLeds $uint32)
\end{PLM}

\subsubsectionLabel{Routine \texttt{ledOn}}{routineLedOnTeensy31it}

\begin{PLM}
proc ledOn `user `panic `isr (?inLeds $uint32)
\end{PLM}


\section{Les routines d'interruption}

Les \refTableau{tableItTeensySequentialSystick1}, \refTableau{tableItTeensySequentialSystick2} et \refTableau{tableItTeensySequentialSystick3} listent les interruptions définies par le processeur qui équipe la carte \emph{Teensy 3.1}. L'utilisateur peut définir une routine d'interruption pour chacune d'entre elles, sauf l'interruption n°1 (remise à zéro), et la n°15 (\texttt{SysTick}) qui est prise en charge de façon particulière (voir la \refSubsectionTitlePage{SystickPourTeensy31It}). Celle des autres interruptions est décrite dans les sections suivantes :
\begin{itemize}
  \item \refSubsectionTitlePage{itsTeensy31AvecExceptions} ;
  \item \refSubsectionTitlePage{itsTeensy31SansExceptions} ;
  \item \refSubsectionTitlePage{itsTeensyRoutinesUtilisateur}.
\end{itemize}


\begin{table}[!t]
  \centering
  \begin{tabular}{llllll}
    \textbf{Numéro}& \textbf{Nom routine} \\
    1  & \emph{ResetHandler, réservé par PLM} \\
    2  & \texttt{NMIHandler}\\
    3  & \texttt{HardFaultHandler}\\
    4  & \texttt{MemManageHandler}\\
    5  & \texttt{BusFaultHandler}\\
    6  & \texttt{UsageFaultHandler}\\
    7 à 10 & \emph{réservées par ARM} \\
    11 & \texttt{svcHandler}\\
    12 & \texttt{DebugMonitorHandler}\\
    13 & \emph{réservée par ARM} \\
    14 & \texttt{PendSVHandler}\\
    15 & \texttt{userSystickHandler}, voir \refSubsectionPage{SystickPourTeensy31It} \\
  \end{tabular}
  \caption{Table des interruptions 1 à 15 de la cible \texttt{teensy-3-1-it}}
  \labelTableau{tableItTeensySequentialSystick1}
  \ligne
\end{table}

\begin{table}[!t]
  \centering
  \begin{tabular}{llllll}
    \textbf{Numéro} & \textbf{Nom routine} \\
    16  & \texttt{DMAChannel0TranfertCompleteHandler}\\
    17  & \texttt{DMAChannel1TranfertCompleteHandler}\\
    18  & \texttt{DMAChannel2TranfertCompleteHandler}\\
    19  & \texttt{DMAChannel3TranfertCompleteHandler}\\
    20  & \texttt{DMAChannel4TranfertCompleteHandler}\\
    21  & \texttt{DMAChannel5TranfertCompleteHandler}\\
    22  & \texttt{DMAChannel6TranfertCompleteHandler}\\
    23  & \texttt{DMAChannel7TranfertCompleteHandler}\\
    24  & \texttt{DMAChannel8TranfertCompleteHandler}\\
    25  & \texttt{DMAChannel9TranfertCompleteHandler}\\
    26  & \texttt{DMAChannel10TranfertCompleteHandler}\\
    27  & \texttt{DMAChannel11TranfertCompleteHandler}\\
    28  & \texttt{DMAChannel12TranfertCompleteHandler}\\
    29  & \texttt{DMAChannel13TranfertCompleteHandler}\\
    30  & \texttt{DMAChannel14TranfertCompleteHandler}\\
    31  & \texttt{DMAChannel15TranfertCompleteHandler}\\
    32  & \texttt{DMAErrorHandler}\\
    33  & \emph{inutilisée} \\
    34  & \texttt{flashMemoryCommandCompleteHandler}\\
    35  & \texttt{flashMemoryReadCollisionHandler}\\
    36  & \texttt{modeControllerHandler}\\
    37  & \texttt{LLWUHandler}\\
    38  & \texttt{WDOGEWMHandler}\\
    39  & \emph{inutilisée} \\
    40  & \texttt{I2C0Handler}\\
    41  & \texttt{I2C1Handler}\\
    42  & \texttt{SPI0Handler}\\
    43  & \texttt{SPI1Handler}\\
    44  & \emph{inutilisée} \\
    45  & \texttt{CAN0MessageBufferHandler}\\
    46  & \texttt{CAN0BusOffHandler}\\
    47  & \texttt{CAN0ErrorHandler}\\
    48  & \texttt{CAN0TransmitWarningHandler}\\
    49  & \texttt{CAN0ReceiveWarningHandler}\\
    50  & \texttt{CAN0WakeUpHandler}\\
    51  & \texttt{I2S0TransmitHandler}\\
    52  & \texttt{I2S0ReceiveHandler}\\
    53 à 59  & \emph{inutilisées} \\
  \end{tabular}
  \caption{Table des interruptions 16 à 59 de la cible \texttt{teensy-3-1-it}}
  \labelTableau{tableItTeensySequentialSystick2}
  \ligne
\end{table}

\begin{table}[!t]
  \centering
  \begin{tabular}{llllll}
    \textbf{Numéro}& \textbf{Nom routine} \\
    60  & \texttt{UART0LONHandler}\\
    61  & \texttt{UART0StatusHandler}\\
    62  & \texttt{UART0ErrorHandler}\\
    63  & \texttt{UART1StatusHandler}\\
    64  & \texttt{UART1ErrorHandler}\\
    65  & \texttt{UART2StatusHandler}\\
    66  & \texttt{UART2ErrorHandler}\\
    67 à 72  & \emph{inutilisées} \\
    73  & \texttt{ADC0Handler}\\
    74  & \texttt{ADC1Handler}\\
    75  & \texttt{CMP0Handler}\\
    76  & \texttt{CMP1Handler}\\
    77  & \texttt{CMP2Handler}\\
    78  & \texttt{FMT0Handler}\\
    79  & \texttt{FMT1Handler}\\
    80  & \texttt{FMT2Handler}\\
    81  & \texttt{CMTHandler}\\
    82  & \texttt{RTCAlarmHandler}\\
    83  & \texttt{RTCSecondHandler}\\
    84  & \texttt{PITChannel0Handler}\\
    85  & \texttt{PITChannel1Handler}\\
    86  & \texttt{PITChannel2Handler}\\
    87  & \texttt{PITChannel3Handler}\\
    88  & \texttt{PDBHandler}\\
    89  & \texttt{USBOTGHandler}\\
    90  & \texttt{USBChargerDetectHandler}\\
    91 à 96  & \emph{inutilisées} \\
    97  & \texttt{DAC0Handler}\\
    98  & \emph{inutilisée} \\
    99  & \texttt{TSIHandler}\\
    100  & \texttt{MCGHandler}\\
    101  & \texttt{lowPowerTimerHandler}\\
    103  & \texttt{pinDetectPortAHandler}\\
    104  & \texttt{pinDetectPortBHandler}\\
    105  & \texttt{pinDetectPortCHandler}\\
    106  & \texttt{pinDetectPortDHandler}\\
    107  & \texttt{pinDetectPortEHandler}\\
    108 et 109  & \emph{inutilisées} \\
    110  & \texttt{softwareInterruptHandler}
  \end{tabular}
  \caption{Table des interruptions 60 à 110 de la cible \texttt{teensy-3-1-it}}
  \labelTableau{tableItTeensySequentialSystick3}
  \ligne
\end{table}


\subsectionLabel{Routines d'interruption par défaut, panique activée}{itsTeensy31AvecExceptions}

Quand la panique est activée, une routine par défaut est prédéfinie pour chaque interruption (sauf la n°1 et la n°15). Celle-ci exécute l'instruction \plm+panic+, dont l'argument est le numéro de l'interruption. Par exemple, pour l'interruption n°2 (\texttt{NMI}), la routine prédéfinie est :
\begin{PLM}[1]
proc NMIHandler `isr @nullWhenPanicDisabled @weak () {
  panic 2
}
\end{PLM}


\subsectionLabel{Routines d'interruption par défaut, panique inactivée}{itsTeensy31SansExceptions}

Quand la panique est inactivée, l'attribut \plm+@nullWhenPanicDisabled+ associé à chaque routine d'interruption provoque la suppression de cette routine d'interruption, et son remplacement de son adresse dans la table des vecteurs d'interruption par la valeur $0$.



\subsectionLabel{Routines d'interruption définies par l'utilisateur}{itsTeensyRoutinesUtilisateur}

L'attribut \plm+@weak+ associé à chaque routine d'interruption permet sa redéfinition par l'utilisateur. La routine par défaut est alors ignorée, que la panique soit activée ou non.

Par exemple, l'utilisateur peut définir la routine associée à \texttt{NMI} par :
\begin{PLM}[1]
proc NMIHandler `isr {
  ...
}
\end{PLM}

Il est important de conserver le nom. Si celui-ci est mal orthographié et ne correspond à aucune interruption :
\begin{itemize}
  \item la routine d'interruption par défaut est conservée (panique activée) ou supprimée (panique inactivée) ;
  \item un \emph{warning} est déclenché, signalant que la routine utilisateur est inutilisée.
\end{itemize}



\subsectionLabel{Routine associée à l'interruption \texttt{SysTick}}{SystickPourTeensy31It}

L'interruption \texttt{SysTick} est particulière. Elle est programmé par la routine \plm+init 0+ de façon à engendrer une interruption chaque milliseconde. Cette interruption se déclenche après la fin de l'exécution la routine \plm+init 0+, y compris éventuellement dans les routines \plm+init+ qui suivent. La routine d'interruption associée, inaccessible à l'utilisateur :
\begin{itemize}
  \item incrémente une variable globale de comptage du temps ;
  \item appelle la routine \plm+userSystickHandler+.
\end{itemize}

La routine \plm+userSystickHandler+ est définie par :
\begin{PLM}[1]
proc userSystickHandler `isr @weak () {
}
\end{PLM}

L'attribut \plm+@weak+ permet sa redéfinition par l'utilisateur.



