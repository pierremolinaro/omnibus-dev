%!TEX encoding = UTF-8 Unicode
%!TEX root = ../doc-plm.tex





\chapter{Déclaration des variables}

Les variables peuvent être déclarées en deux endroits :
\begin{itemize}
  \item en dehors de toute routine : c'est une variable globale ;
  \item parmi les instructions d'une routine : c'est une variable locale à la routine.
\end{itemize}





\sectionLabel{Déclaration d'une variable globale}{declarationVariableGlobale}\index{Variable!globale}

La déclaration d'une variable globale est la suivante :

\begin{PLM}
var nom $mode1 $mode2 : Type = expression_statique
\end{PLM}

Où :
\begin{itemize}
  \item \plm=nom= est le nom de la variable globale ;
  \item \plm=$mode1=, \plm=$mode2= est la liste (non vide) des modes où la variable peut être accédé ;
  \item \plm=Type= est le nom du type de la variable globale ;
  \item \plm=expression_statique= est l'expression qui fournit la valeur initiale de cette variable ; cette expression est calculée lors de la compilation.
\end{itemize}

La portée d'une variable globale est le programme dans son intégralité : peut importe où est déclarée la variable.

Note : si plusieurs modes sont mentionnés, le compilateur ajoute le qualificatif \texttt{volatile} dans le code C engendré.




\sectionLabel{Déclaration d'une variable locale}{declarationVariableLocale}\index{Variable!locale}

La déclaration d'une variable locale peut prendre plusieurs formes, suivant que la variable est initialisée ou non.

\subsection{Déclaration d'une variable locale initialisée}

\begin{PLM}
var nom : Type = expression
\end{PLM}

Où :
\begin{itemize}
  \item \plm=nom= est le nom de la variable locale ;
  \item \plm=Type= est le nom du type de la variable globale ;
  \item \plm=expression= est l'expression qui fournit la valeur initiale de cette variable ; cette expression peut être statique ou non.
\end{itemize}

L'annotation de type peut être omise ; la variable a alors le type de l'expression qui l'initialise :
\begin{PLM}
var nom = expression
\end{PLM}


\subsection{Déclaration d'une variable locale non initialisée}
\begin{PLM}
var nom : Type
\end{PLM}
Où :
\begin{itemize}
  \item \plm=nom= est le nom de la variable locale ;
  \item \plm=Type= est le nom du type de la variable globale.
\end{itemize}

Le compilateur garantit qu'aucune lecture n'est fait avant que la variable reçoive une valeur.
