%!TEX encoding = UTF-8 Unicode
%!TEX root = ../doc-plm.tex





\chapter{Tableaux statiques constants}

PLM permet de construire des tableaux statiques constants, en séparant déclaration du tableau et constitution. Cette caractéristique s'appuie sur les trois constructions suivantes :
\begin{itemize}
  \item déclaration du tableau statique (\refSection{DecTypeTableauStatique}) ;
  \item ajout d'un élément au tableau statique (\refSection{ajoutElementTableauStatique}) ;
  \item parcours de'un tableau statique (\refSection{parcoursTableauStatique}).
\end{itemize}











\sectionLabel{Déclaration}{DecTypeTableauStatique}

La déclaration d'un tableau statique est réalisée par la construction \plm=staticArray= :

\begin{PLM}
staticArray maListeStatique {
  let a $uint32
  let b $uint32
}
\end{PLM}

Cette construction déclare la constante \plm=maListeStatique= comme tableau constant vide. Pour le remplir, utiliser la construction \plm=extend staticArray= (\refSection{ajoutElementTableauStatique}).

La composition de chaque élément est spécifiée par la liste des propriétés, chacune d'elles étant définie par son nom et son type.









\sectionLabel{Ajout d'un élément au tableau}{ajoutElementTableauStatique}

Un tableau statique est construit élément par élément : 

\begin{PLM}
extend staticArray maListeStatique (5, 9)
\end{PLM}

Cette déclaration ajoute un élément au tableau statique, élément dont toutes les propriétés doivent être initialisées par une expression statique. 

\fbox{\begin{minipage}{1.0\textwidth}
   {\bf Attention !} L'ordre des éléments ne peut pas être spécifié. Il peut varier d'une façon imprévisible d'une compilation à une autre. Aussi, il faut veiller que les opérations réalisées soient indépendantes de l'ordre dans lequel les éléments sont placés dans le tableau statique.
\end{minipage}}





\sectionLabel{Parcours d'un tableau statique}{parcoursTableauStatique}

L'instruction \plm=for= (\refSectionPage{instructionFor}) est la seule qui accède à un tableau statique. Elle permet de parcourir tous les éléments du tableau :

\begin{PLM}
var total $uint32 = 0
for élément in maListeStatique {
  total += élément.a
  total += élément.b
}
\end{PLM}

L'élément courant est désigné par la constante \plm=élément=, et on accède aux propriétés par la notation pointée habituelle. 





