%!TEX encoding = UTF-8 Unicode
%!TEX root = ../doc-plm.tex





\chapter{Directives}



\section{Directive \texttt{target}}\index{target@\plm=target=}





\section{Directive \texttt{import}}\index{import@\plm=import=}


\sectionLabel{Directive \texttt{check}}{directiveCheck}\index{check@\plm=check=}

La directive \plm=check= apparaît dans une liste d'instructions et a la syntaxe suivante :
\begin{PLM}
check expression
\end{PLM}

L'\plm=expression= est une expression booléenne calculée statiquement.

Contrairement à l'instruction \plm=assert= (\refSectionPage{instructionAssert}) qui évalue l'expression booléenne à l'exécution, la directive \plm=check= est toujours évaluée à la compilation. Elle permet d'exprimer des assertions qui sont évaluées lors de la compilation.

Aucun code n'est engendré. La directive \plm=check= peut donc apparaître dans des listes d'instructions où les exceptions sont interdites.

Exemples :
\begin{PLM}
check true  // Ok
check false // Erreur, expression fausse
\end{PLM}


