%!TEX encoding = UTF-8 Unicode
%!TEX root = ../doc-plm.tex





\chapter{Directives}



\sectionLabel{Directive \texttt{target}}{directiveTarget}\index{target@\plm=target=}

La directive \plm=target= fixe la cible pour laquelle le code source est compilé. Sa syntaxe est la suivante :
\begin{PLM}
target "cible.plms"
\end{PLM} 
Où \plm="cible.plms"= est le nom du descriptif de la cible.

L'extension \texttt{.plms} doit être mentionnée dans le descriptif.



\sectionLabel{Directive \texttt{import}}{directiveImport}\index{import@\plm=import=}


La directive \plm=import= permet d'ajouter les définitions contenues dans le fichier texte nommé. Sa syntaxe est la suivante :
\begin{PLM}
import "chemin.plm"
\end{PLM} 
Où \plm="chemin.plm"= est un chemin (absolu, relatif) vers le fichier à importer.

Importer plusieurs fois le même fichier n'est pas une erreur. Le compilateur garde trace des importations déjà effectuées et ignore les importations des fichiers déjà importés.

L'extension \texttt{.plm} doit être mentionnée dans le chemin.