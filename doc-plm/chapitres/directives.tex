%!TEX encoding = UTF-8 Unicode
%!TEX root = ../doc-plm.tex





\chapter{Directives}


\sectionLabel{Directive \texttt{target}}{directiveTarget}\index{target@\plm=target=}

La directive \plm=target= fixe la cible pour laquelle le code source est compilé. Sa syntaxe est la suivante :
\begin{PLM}
target "cible"
\end{PLM} 
Où \plm="cible"= est le nom du descriptif de la cible.





\sectionLabel{Directive \texttt{import}}{directiveImport}\index{import@\plm=import=}


La directive \plm=import= permet d'ajouter les définitions contenues dans le fichier texte nommé. Sa syntaxe est la suivante :
\begin{PLM}
import "chemin.plm"
\end{PLM} 
Où \plm="chemin.plm"= est un chemin (absolu, relatif) vers le fichier à importer.

Importer plusieurs fois le même fichier n'est pas une erreur. Le compilateur garde trace des importations déjà effectuées et ignore les importations des fichiers déjà importés.

L'extension \texttt{.plm} doit être mentionnée dans le chemin.


\sectionLabel{La directive \texttt{check target}}{directiveCheckTarget}\index{check target}

La directive \plm=check target= permet de vérifier l'identité de la cible. Elle est utile pour s'assurer qu'un fichier est acceptable par une cible. À la compilation, toute directive \plm=check target= incorrecte entraîne une erreur.

Par exemple, compiler pour la cible \plm+"teensy-3-1"+ est spécifié par~:

\begin{PLM}
target "teensy-3-1"
\end{PLM}

Si la définition du projet implique d'importer le fichier \plm+"monFichier.plm"+~:

\begin{PLM}
target "teensy-3-1"
import "monFichier.plm"
\end{PLM}

Si le contenu de ce fichier commence par~:
\begin{PLM}
check target "lpc2294"
\end{PLM}

Alors une erreur de compilation apparaît. La directive \plm=check target= doit être~:
\begin{PLM}
check target "teensy-3-1"
\end{PLM}

