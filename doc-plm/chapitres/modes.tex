%!TEX encoding = UTF-8 Unicode
%!TEX root = ../doc-plm.tex





\chapterLabel{Modes d'exécution}{chapitreModesExecution}

En PLM, à toute routine est associé un ou plusieurs modes \emph{modes d'exécution}\index{Mode@Mode d'exécution}.

Un \emph{mode d'exécution} est une annotation qui permet au compilateur de vérifier la cohérence des appels des routines, et qui n'ajoute aucun code supplémentaire à l'exécution.



Syntaxiquement, un mode d'exécution est un nom précédé du caractère « \texttt{\$} » : par exemple, \plm+$boot+. 


\section{Des exemples}

Considérons le code suivant :
\begin{PLM}
proc uneRoutine $user () {
}

proc autreRoutine $user () {
  uneRoutine ()
}
\end{PLM}


\section{Modes prédéfinis}

Le compilateur prédéfinit trois modes d'exécution (\refTableauPage{ModeExecutionPredefinis}).





\begin{table}[h]
\centering
\ligne\\
\begin{tabular}{lll}
  \textbf{Mode} & \textbf{Commentaire} & \textbf{Lien} \\
  \hline
  \plm=$boot= & Routines exécutées au démarrage & \refSectionPage{bootRoutine} \\
  \plm=$init= & Routines d'initialisation & \refSectionPage{initRoutine} \\
  \plm=$exception= & Routines d'exception & \refSectionPage{routineException} \\
\end{tabular}
\caption{Modes d'exécution prédéfinis}
\labelTableau{ModeExecutionPredefinis}
\ligne
\end{table}
